% $Id$

\documentclass[a4paper,10pt]{article}

\setlength{\parindent}{0mm}
\setlength{\parskip}{7.5mm}

\usepackage{amssymb}
\usepackage[usenames]{color}
\usepackage[arc,poly,all]{xy}
\usepackage{ulem}
\usepackage{listings}
\usepackage{verbatim}
\usepackage{fancyvrb}
\usepackage[left=2cm,right=2cm,top=4cm,bottom=4cm]{geometry}
\usepackage{graphicx}
\usepackage{url}


\newcommand{\primarykey}[1]{\mbox{{\color{NavyBlue}{$#1$}}}}
\newcommand{\foreignkey}[1]{{\mbox{\color{Emerald}{$#1$}}}}


\begin{document}

\title{4BA8 \\ Assignment}

\author{Conall O'Brien \\ 01734351 \\ conallob@maths.tcd.ie}

\maketitle

\section{Introduction}

Given the broad guidelines relating to the implementation of this 
assignment,  programming language and Web Services, I have decided to 
use three technologies, Perl, MySQL and XML-RPC.


I have chosen to use Perl for multiple reasons, primarily it's
popular use developing server-side applications and dynamic web content.
In addition, the extensive library of modules available to Perl ensures
a complete implementation of numerous web service technologies.


I have chosen to use XML-RPC for the exchanging of RPC messages due to
it's lightweight and uncomplicated definition. It is my belief that the 
use of XML-RPC over compeditors such as SOAP is appropariate for an
application of this size and complexity.


For backend database storage, I have opted to use MySQL, the open source
SQL complient database over other Web Services backend solutions, such 
as UDDI standard, who's definition is incomplete. I have documented the
functional dependency diagrams of each table in my database design in
Section \ref{sec:sql} 

\section{Architectural Design}

My application design is simple, based upon the client - server
architecture. A Perl script on the primary server acts as a translator
between XML-RPC methodss and the MySQL backend. An additional feature of the server script not implemented by the 
assignment of the deadline allows two server scripts to communicate
between each other, allowing for backend replication between multiple,
resielient hosts using XML-RPC.


The client interface is a PHP script making XML-RPC calls to the CGI
script. It allows a user to do multiple actions, such as
tourism service searchs (queries on the database), comparison of tourism 
services (comparison of query results) and creation of new service
instances (insert queries on the backend).

\section{Server API}

\input api.tex

\pagebreak

\section{SQL Database Design}

\label{sec:sql}

\begin{figure}[hp]
\begin{tabular}{c}
Legend: \\
\primarykey{Primary Key} \\ 
\\
\foreignkey{Foreign Key} \\ 
\end{tabular}
\end{figure}

\vspace{15mm}
\begin{figure}[hp]

\xy<1cm,0cm>:
(15,2) *=(3,1)!UC\txt{\primarykey{TypeID}} *\frm{-} ,
(15,0) *=(3,1)!UC\txt{TypeName} *\frm{-} ,
(15,0) ; (15,1) **\dir{-}  *\dir{>} ,
\endxy

\caption{Table servicetype}

\end{figure}

\vspace{15mm}

\begin{figure}[hp]

\xy<1cm,0cm>:
(15,2) *=(3,1)!UC\txt{\primarykey{ServiceID}} *\frm{-} ,
(10,2) *=(3,1)!UC\txt{Opening} *\frm{-} ,
(20,2) *=(3,1)!UC\txt{Closing} *\frm{-} ,
(15,0) *=(3,1)!UC\txt{ServiceName} *\frm{-} ,
(15,4) *=(3,1)!UC\txt{\foreignkey{ServiceType}} *\frm{-} ,
(15,0) ; (15,1) **\dir{-}  *\dir{>} ,
(15,3) ; (15,2) **\dir{-}  *\dir{>} ,
(11.5,1.5) ; (13.5,1.5) **\dir{-}  *\dir{>} ,
(18.5,1.5) ; (16.5,1.5) **\dir{-}  *\dir{>} ,
\endxy

\caption{Table service}

\end{figure}

\vspace{15mm}
\begin{figure}[hp]

\xy<1cm,0cm>:
(15,2) *=(3,1)!UC\txt{\primarykey{UniqueID}} *\frm{-} ,
(10,2) *=(3,1)!UC\txt{ServiceDate} *\frm{-} ,
(20,2) *=(3,1)!UC\txt{ServiceName} *\frm{-} ,
(20,0) *=(3,1)!UC\txt{Destination} *\frm{-} ,
(15,4) *=(3,1)!UC\txt{\foreignkey{ServiceID}} *\frm{-} ,
(10,0) *=(3,1)!UC\txt{Source} *\frm{-} ,
(10,4) *=(3,1)!UC\txt{Capacity} *\frm{-} ,
(20,4) *=(3,1)!UC\txt{Details} *\frm{-} ,
(15,3) ; (15,2) **\dir{-}  *\dir{>} ,
(11.5,1.5) ; (13.5,1.5) **\dir{-}  *\dir{>} ,
(18.5,1.5) ; (16.5,1.5) **\dir{-}  *\dir{>} ,
(18.5,3.5) ; (16.5,2) **\dir{-}  *\dir{>} ,
(11.5,3.5) ; (13.5,2) **\dir{-}  *\dir{>} ,
(11.5,3.5) ; (13.5,2) **\dir{-}  *\dir{>} ,
(18.5,0) ; (16.5,1) **\dir{-}  *\dir{>} ,
(11.5,0) ; (13.5,1) **\dir{-}  *\dir{>} ,
\endxy

\caption{Table instance}

\end{figure}

\vspace{15mm}
\begin{figure}[hp]

\xy<1cm,0cm>:
(15,2) *=(3,1)!UC\txt{\primarykey{BookingID}} *\frm{-} ,
(20,0) *=(3,1)!UC\txt{Confirmed} *\frm{-} ,
(15,4) *=(3,1)!UC\txt{\foreignkey{UniqueID}} *\frm{-} ,
(10,0) *=(3,1)!UC\txt{Cancelled} *\frm{-} ,
(10,4) *=(3,1)!UC\txt{Booked} *\frm{-} ,
(20,4) *=(3,1)!UC\txt{Flexible} *\frm{-} ,
(15,3) ; (15,2) **\dir{-}  *\dir{>} ,
(18.5,3.5) ; (16.5,2) **\dir{-}  *\dir{>} ,
(11.5,3.5) ; (13.5,2) **\dir{-}  *\dir{>} ,
(11.5,3.5) ; (13.5,2) **\dir{-}  *\dir{>} ,
(18.5,0) ; (16.5,1) **\dir{-}  *\dir{>} ,
(11.5,0) ; (13.5,1) **\dir{-}  *\dir{>} ,
\endxy

\caption{Table booking}

\end{figure}

\pagebreak
\section{Code}

\subsection{Database}

\lstset{breaklines=true,tabsize=3,basicstyle=\ttfamily}
\lstinputlisting{backend.sql}

\subsection{Server}

\lstset{breaklines=true,tabsize=3,basicstyle=\ttfamily}
\lstinputlisting{xmlrpc.cgi}[language=perl]

\subsection{Client}

\lstset{breaklines=true,tabsize=3,basicstyle=\ttfamily}
\lstinputlisting{4ba8.cgi}[language=perl]

\section{Demonstration}

A partially working demonstration of my server and client are available online.


My server configuration is setup on a server situated in Tampa, Florida.
The URL of the CGI script is
\url{http://master-4ba8.conall.ie/cgi-bin/xmlrpc.cgi}


My client script is hosted on a server situated in Dublin, Ireland. The
URL for the client Perl CGI script is
\url{http://client-4ba8.conall.ie}. Unfortunately, I was unable to
complete a fully functioning HTML interface for a client script. My
client scripts are merely invoking indivudual XML-RPC calls from my
server and displaying the output. This basic client does not fully
satisfy the assignment requirement of a fully functioning online bookign
system, but it does demonstrate that my defined API is fully working.


Both servers are powered by Debian GNU/Linux, secured by restrictive, 
stateful iptables firewall, making only the necessary ports available.

\section{Technical References}

\begin{tabular}{lll}
Title	&	Author(s)	&	Publisher	\\
Edition	&	ISBN	&	Publish Date 	\\
			&			&						\\
			&			&						\\
Programming Web Services with Perl	&	Pavel Kulchenko,  Randy J. Ray	&
O'Reilly	\\
1st&	0-596-00206-8 & December 2002	\\
			&			&						\\
Programming Web Services with XML-RPC & Edd Dumbill, Joe Johnston, Simon
St. Laurent &	O'Reilly	\\
3rd & 0-596-00119-3	&	June 2001	\\
			&			&						\\
Real World Web Services	&	Will Iverson	&	O'Reilly	\\
1st & 0-596-00642-X	&	October 2004	\\
			&			&						\\
Learning Perl	&	Randal L. Schwartz, Tom Phoenix	&	O'Reilly	\\
3rd  		&	0-596-00132-0	&	July 2001	\\
			&			&						\\
Programming Perl	&	Larry Wall, Tom Christiansen, Jon Orwant	&
O'Reilly	\\
3rd		&	0-596-00027-8	&	July 2000	\\
\end{tabular}

\end{document}
