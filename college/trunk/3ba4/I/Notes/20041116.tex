% $Id: 20041108.tex 353 2004-11-12 00:21:25Z conall $

\documentclass[a4paper,12pt]{article}
\usepackage{amssymb}

\setlength{\parindent}{0mm}
\setlength{\parskip}{7.5mm}

\begin{document}

\title{Course 3BA4: Computer Architecture II \\ Lecture Notes \\ $16^{th}$ November 2004}

\maketitle

\section*{Page Fault Handling}

OS must solve page faults by performing one or more of the following
actions:

\begin{itemize}

\item Allocate a page of physical memory (from the OS free list) for use
as a secondary page table.

\item Allocate a page of physical memory for the referenced page

\item Update assocciated page table entries.

\item Read in page data from disk (contect swithc to another process)

\item Signalling an access violation (eg if writing to cache area) -
protection level error

\item Restart (or confining) the faulting instruction.

\end{itemize}

\section*{Translation Look Aside Buffer (TLB)}

\$word an internal cache (TLB) each virtual $\to$ physical address
translation requires $1$ memory access for each level of page table. 
($2$ accesses for $2$ level scheme).

MMU controls an m-entry on-chip translation cache (TLB) which provides
direct mappings for the $m$ most recently accessed virtual pages.

When a virtual address is sent to the TLB, the virtual page number is
compared with \textbf{all} $m$ tag entries in the TLB in parallel (fill
assocciative cache)

If a match is found (TLB Hit), the corresponding cached secondary page
table entry is output by the TLB to provide the physical address.

If a matchnot found (TLB Miss), page tables are walked my MMU

\indent - least recently used TLB entry is replaced with the new
mapping.

\indent how in the hardware find the LRN entry simply.

Remember that the page tables are just data structures held in main
memory and can be walked by a CPU using ordinary instructions.

\indent This approach is taken by many RISCs

\indent a TLB miss generates an interrupt and the CPU walks the page
table using ordinary instructions (TLB Miss $\equiv$ page fault)

\indent Organisation of page table structure set by software and is
\textbf{not} hard-wired into CPU/MMU.

\indent Need a CPU instruction to replace LRU TLB cache.

TLBs are normally small - typically $64$ entry fully assocciative TLB
should achieve a hit rate $> 90\%$.

A CPU may have more than one MMU/TLB - Pentium has one for instruction
and one for data accesses.

\end{document}
