% $Id$

\documentclass[a4paper,12pt]{article}
\usepackage{amssymb}

\setlength{\parindent}{0mm}
\setlength{\parskip}{7.5mm}

\begin{document}

\title{Course 3BA4: Computer Architecture II \\ Lecture Notes \\ $2^{nd}$ November 2004}

\maketitle

\section{Pipeline Forwarding} 

ALU results from \emph{previous} two instructions can be forwarded to the ALU input 
from the $\verb!ALU!_{OUT 0}$ of the ALU pipeline register before the results are 
written back to the system file. Tag $\verb!ALU!_{OUT 0}$ and
$\verb!ALU!_{OUT 1}$ with the destination register, $EX$ stage checks for source 
register in the order $\verb!ALU!_{OUT 0}$, $\verb!ALU!_{OUT 1}$ and
then $A/B$.


Two Phase Clocking Pipeline Forwarding Example: 

Reads $\verb!r1!$ from $\verb!ALU!_{OUT 0}$ 

\begin{eqnarray*}
\verb!r1! 	&	\leftarrow	& \verb!r2 + r3!					\\ 
\verb!r4! 	&	\leftarrow	& \verb!r1 - r5!					\\ 
\verb!r6! 	&	\leftarrow	& \verb!r1 slr r7!				\\ 
\verb!r8! 	&	\leftarrow	& \verb!r1! \otimes \verb!r9!	\\ 
\verb!r8! 	&	\leftarrow	& \verb!r1 + r9!					 
\end{eqnarray*}

\begin{tabular}{|c|c|c|c|c|c|c|c|c|}
\hline
$IF$	&	$ID$	&	$EX$	&	$MA$		&	$WB$		&			&			&			&			\\
		&	$IF$	&	$ID$	&	$EX^{1}$	&	$MA$		&	$WB$	&			&			&			\\
		&			&	$IF$	&	$ID$		&	$EX^{2}$	&	$MA$	&	$WB$	&			&			\\
		&			&			&	$IF$		&	$IR^{3}$	&	$EX$	&	$MA$	&	$WB$	&			\\
		&			&			&				&	$IF$		&	$ID$	&	$EX$	&	$MA$	&	$WB$	\\
\hline
\end{tabular}

$^{1}$: Reads $\verb!r1!$ from $\verb!ALU!_{OUT 0}$ \\
$^{2}$: Reads $\verb!r1!$ from $\verb!ALU!_{OUT 1}$ \\
$^{3}$: $ID$ stage reads correct value of $\verb!r1!$ due to two-phase clocking/access 
to the register file 

\section*{Load Hazards}

Consider the following instruction secuence:

\begin{eqnarray*}
\verb!r1! 	& 	\leftarrow 	&	\verb!M[a]! 	\\
\verb!r4! 	& 	\leftarrow	&	\verb!r1 + r7! \\
\verb!r5!	&	\leftarrow	&	\verb!r1 - r8! \\
\verb!r6!	&	\leftarrow	&	\verb!r2 & r7!	\\
\end{eqnarray*}

\begin{tabular}{l|l|l|l|l|l|l|l|l|l}
\verb!r1! $\leftarrow$ \verb!M[a]!	& IF	& ID	& EX	& MA			& WB	& 		&		&		&	\\
\verb!r4!									&		& IF	& ID	& --stall--	& EX	& MA	& WB	&		&	\\
												&		&		& IF	& --stall-- & ID	& EX	& MA	& WB	&	\\
												&		&		&		& --stall--	& IF	& ID	& EX	& MA	& WB \\
\end{tabular}

Dependancy between \verb!load! and \verb!add! results in a one cycle
pipeline call.


The pipeline interlock occurs when a load hazard is detected, resulting
in a pipeline stall.

Loaded data has still to be forwarded to the \verb!add! instruction.


\subsection*{Instruction Scheduling Example}

Consider the following where \verb!a ... f! are memory locations.

\begin{eqnarray*}
\verb!a!	& 	\leftarrow 	&	\verb!b + c! \\
\verb!d! &	\leftarrow	&	\verb!e - f! \\
\end{eqnarray*}

Compiler generated scheduled code might look like this:

\begin{eqnarray*}
\verb!r2!	& \leftarrow	&	\verb!M[b]!	\\
\verb!r3!	& \leftarrow	&	\verb!M[c]!	\\
\verb!r5!	& \leftarrow	&	\verb!M[e] // swapped with add to avoid stall!	\\
\verb!r1!	& \leftarrow	&	\verb!r2 + r3!	\\
\verb!r6!	& \leftarrow	&	\verb!M[f]!	\\
\verb!M[a]!	& \leftarrow	&	\verb!r1!	\\
\verb!r4!	& \leftarrow	&	\verb!r5 - r6 // load/store swapped to avoid stall in sub!	\\
\verb!M[b]!	& \leftarrow	&	\verb!r4!		\\
\end{eqnarray*}


Use of different registers critical for a legal schedule

Pipeline scheduling \emph{generally increases} register use.

\section*{Control Hazards}

A simple DLX branch implementation results in a 5 cycle stall per branch
instruction.

\indent New PC not know until end of UA
\indent 3 cycle penalty whether branch is taken \textbf{or not}.

\begin{tabular}{l|l|l|l|l|l|l|l|l|l|l|l}
branch	&	IF	&	ID	&	MA			&	WB			&		&		&		&		&		&		&	\\
i + 1		&		&	IF	& --stall--	& --stall--	& IF	& ID	& EX	& MA	& WB	&		&	\\
i + 2		&		&		&				& 				&		& IF	& ID	& EX	& MA	& WB	&	\\	
i + 3		&		&		&				& 				&		& 		& IF	& ID	& EX	& MA	& WB	\\	
\end{tabular}

Subsequent instructions may be fetched twice (don't know if that
innstruction is going to be a branch until ID stage)

A 30\% branch frequency and a 3 cycle stall results in \textbf{only}
$\approx$ 50\% of the total potential pipeline up (consider $1000$
instructions, none pipelined $500$, perfectly pipelined $100$, 3 cycle
branch stall $30 * 4 + 70 = 190$) % = ???


Need to

\indent Find whether branch is taken or not \textbf{earlier} in
pipeline.

\indent Compute target address \textbf{earlier} in pipeline.

\section*{DLX Branches}

DLX doesn't have a conventional CCR.

Uses instruction set conditional instruction traunch zero combination.

Set a register with $0$ or $1$ depending on the comparison of two source
operands.

eg

\begin{verbatim}
slt r2,r3,r1    ; r1 = (r2 < r3) ? 0(if):1(else)
beqz r1,L       ; branch if (r1 == 0)
\end{verbatim}

Also \verb!sgt! ($>$), \verb!sle! ($<=$), \verb!sge! ($=>$), \verb!seq!
($==$), \verb!sne! ($!=$) and \verb!bnez! (branch not equal to)

Needs additional instructions compared with the implicit/explicit
setting of conditional codes

DLX uses additional hardware to resolve branches during the IID stage
(ie to test if a register == $0$ and to add offset to PC)

\begin{tabular}{lll}
IF	&		&	\verb!IR! $\leftarrow$ \verb!M[PC]!; \verb!PC! $\leftarrow$ \verb!PC + 4!\\
ID	&		&	if(Rsrc1 $== 1$,$!=0$); \verb!PC! $\leftarrow$ \verb!PC + Offset! \\
EX	&		&	idle \\
MA	&		&	idle \\
WB	&		&	idle \\
\end{tabular}

\end{document}
