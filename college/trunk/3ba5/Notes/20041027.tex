% $Id: 20041012.tex 336 2004-10-27 12:47:38Z conall $

\documentclass[a4paper,12pt]{article}
\usepackage{amssymb}
\usepackage{amsfonts}
\usepackage{amsmath}
\usepackage[all]{xy}
\usepackage{tabularx}


\setlength{\parindent}{0mm}
\setlength{\parskip}{7.5mm}

\begin{document}

\title{Course 3BA5: Computer Engineering \\ Lecture Notes \\ $27^{th}$ October 2004}

\maketitle

\section*{Non Flynn Classifiable Parallel Systems}

There are interesting alternative architectures still at the
experimental stage.

\subsection*{Artificial Neural Networks}

They use a processing element which implements a threshold function.

% xy-pic Diagram to do

Compute:

\[ S = \sum^{n}_{i = 1} a_{i} x_{i} = [arithmetic] \]

$a_{i}$ - Vector of weights

$x = \left\{
\begin{array}{cc}
p if	& 	s < 1	\\
1 if	& 	s \geq 1	\\
\end{array}
\right\} = [\mbox{Logic If}]$


Programming consists of selecting the weight $a$, for each $PE_{i}$ and
designing the instructions between them.

Example:

$\underleftarrow{x} = $ pixel array of character scan \\

$\underleftarrow{y} = \left( y_{1} \cdots y_{26} \right) = $ the
identidy of $\underleftarrow{x}$

$x = z$ gives $y_{26} = 1$, $y_{1} = 0 \forall 0 < i < 26$

\section*{Fuzzy Logic}

Deductive logic uses $\left\{ \textbf{TRUE}, \textbf{FALSE} \right\}$

Fuzzy logic uses $\left\{ \textbf{TRUE}, \textbf{FALSE}, \textbf{PROBABLY},
\textbf{POSSIBLY}, \textbf{UNLIKELY}, \textbf{FALSE} \right\}$ and defines a set
of functions over these.

Like Artificial Neural Networks, the object is to deal with noise and
uncertainty.

\section*{Processor System Performance}

Let

\begin{eqnarray*}
IF_{i}			&	=	&	\mbox{Frequency of instruction } I_{i}					\\
CP_{i}			&	=	&	\mbox{Clock cycles required for }I_{i}					\\ 
T					&	=	&	\mbox{Clock period in } ns									\\
\overline{CP}	&	=	&	\mbox{Average number of cycles per instruction}		\\
\overline{T}	&	=	&	\mbox{Average time (in $ns$) per instruction} 		\\
MIPS				&	=	&	\mbox{Average millions of instructions per second} \\
\end{eqnarray*}

Then $\overline{CP} = \sum_{i} IF_{i} \times CP_{i}$

$\overline{T} = \overline{CP} \times T $

$MIPS = \frac{1}{\overline{T} \times 10^{-9}} \times \frac{1}{10^{6}} =
\frac{10^{3}}{\overline{T}}$


High performance MISD or MIMD systems have pipelines for floating point
arithmetic and are rated in the millions of floating point operations
per second (FLOPs)

\subsubsection*{Exmaple}

A vector processor with $T = 5ns$ has seperate floating point add and
multiply pipelines, which can accept new input operands each clock
period. What is the maximum MFLOPs?

FP ops per sec per pipe = $\frac{1 s}{5ns} = \frac{1}{5 x 10^{-9}} = 200 \times 10^{6}$.

So MFLOPs per pipe is $200$ MFLOPs.

On certain computations, eq Horner's Scheme Vector inner production
matrix multiplication, we can keep both pipes busy.

Hence MFLOPs$_{\mbox{max}} = 400$ MFLOPs.

\subsection*{Other Performance Measures}

\begin{tabularx}{\linewidth}{llX}
Throughput	&	&	No of tasks completed per unit time									\\
Utilisation	&	&	$\frac{\mbox{Busy Time}}{\mbox{Total Time}} \times 100 \%$	\\
Response Time	&	&	Time between a request and completion of the task			\\
Memory Bandwidth	&	&	Number of memory words accessible per second				\\
Memory Access Time	&	&	Time to access a word in $ns$								\\
Memory Size		&	&	Size of primary memory in Megabytes								\\
Latency			&	&	Time between a request and start of the response			\\
\end{tabularx}

\end{document}
