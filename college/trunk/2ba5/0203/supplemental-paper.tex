% $Id: supplemental-paper.tex,v 1.1 2004/05/30 14:26:26 conall Exp $

\documentclass[a4paper,12pt]{article}

\setlength{\parindent}{0mm}
\setlength{\parskip}{7.5mm}

\begin{document}

\title{2Ba5 - Supplemental Paper 2003}

\maketitle

\section*{A}

\subsection{1}

(a) Explain what is meant by a \emph{degenerate semiconductor}. [6 marks]


(b) Given that the number of electrons $n$ in the conduction band and the
number of holes $p$ in the velence band of a semiconductor is given by:

\begin{eqnarray*}
n & = & N_{C}\exp^{\frac{-(E_{C} - E_{F})}{kT}} \\
p & = & N_{V}\exp^{\frac{-(E_{F} - E_{V})}{kT}}
\end{eqnarray*}

Derive an expression for the fermi level $E_{F}$ where:

\begin{eqnarray*}
N_{C} & = & \frac{2(2m_{n} \times \pi kT)^{\frac{3}{2}}}{h^{2}} \\
N_{V} & = & \frac{2(2m_{p} \times \pi kT)^{\frac{3}{2}}}{h^{2}}
\end{eqnarray*}

All the symbols have their usual meanings. [10 marks]


(c) Use the information in 1 (b) to derive an expression for intrinsic
carrier concentration $\eta_{i}$ at $600^{o} C$. [4 marks]

\subsection*{2}

(a) Sketch minority carrier concentration diagrams for a $PN$ junction under
conditions of:

(i) Equilibruim \\
(ii) Forward Bias \\
(iii) reverse Bias


(b) Show that the built-in potential $V_{B}$ of a $PN$ junction is given by:

\[ V_{B} = \frac{kT}{q} \ln[\frac{N_{A} N_{D}}{\eta_{i}^{2}} \]

[10 marks]

(c) Calculate $V_{B}$ at room temperature for a $PN$ junction where the
acceptor and donor concentrations are $10^{15} cm^{-3}$ and $10^{16}
cm^{-3}$ respectively. [4 marks]

\subsection*{3}

Describe the operation of an $NPN$ bipolar transistor taking into
account the various contributions due to drift and diffusion. sketch and
carefully label a minority carrier distribution diagram. [20 marks]

\subsubsection*{4}

(a) Explain what is meant by \emph{solid solubility} in terms of diffusion
in a semiconductor. [5 marks]


(b) A $PN$ junction is to be formed by diffusing boron into an n-type wafer
with a background concentration $C_{B} = 10^{15} cm^{-3}$.
Predisposition takes place at $1000C$ for $15 min$ and this is followed
by drive-in lasting $2$ hours at $1100C$. Determine the final junction
depth and the surface concentration following drive-in. [15 marks]

\subsection*{5}

(a) To ensure that a chip works correctly first time, a design flow is
followed. Give the design flow diagram, clearly outlining each of the
steps in the flow. [10 marks]


(b) Discuss the advantages/disadvantages of syncronous versus asyncornous
design. [6 marks]


(c) The following verilog code gives the outline of a simple positive edged
D-type flip flop. Fill in the missing code.


Module flipflop (d, q, clk, reset);


	input d, clk, reset;
	output q;

	reg q;

	-- missing code

end module


[2 marks]

Give a suitable timing diagram to describe your code. [2 marks]

\section*{B}

\subsection*{6}

Distinguish between the average and the rms value of a periodic
function. Deduce the average and the rms values of current flowing in a
resistance of $50 \Omega$ wneh a constant potential of $100V$ in series
with an alternating potential of peak value $250V$ is applied to it.
calculate the mean power dissipated in the resistor under these
conditions.

\subsection*{7}

derive an expression for the growth of a current in a series $R-L$
circuit when a direct voltage is \emph{suddenly} applied to it. Derive
the time constant for such a circuit.

A resistance of $20 \Omega$ in series with an inductance of $0.2 H$ has
a potential difference of $250V$ suddenly applied to it. Find the
voltage drop across the inductance $0.01 sec$ after application. What is
the value of the total number of flux linkages at this instant?

\subsection*{8}

What is meant by a standing wave on a transmission line? The primary
constants per loop kilometre of a loaded telephone cable are:
resistantce $30 \Omega$, inductance $20 mH$, capacitance $0.06 \mu F$,
leakage negligible. For an engular frequency of $5000 rad/sec$,
calculate (a) $Z_{0}$ (b) the attenuation coefficient and (c) the
velocity of propogation.

\end{document}
