% $Id: alan-moore-mt.tex,v 1.5 2004/05/13 20:28:04 conall Exp $

\documentclass[a4paper,12pt]{article}

\usepackage[curve,arc,poly,all]{xy}
\usepackage{pstcol,pst-plot}
\usepackage[basic]{circ}

\setlength{\parindent}{0mm}
\setlength{\parskip}{7.5mm}

\begin{document}

%\chapter{Alan Moore}


% 2003-10-06

\section{2BA5 Objectives}

\begin{itemize}

\item
Charge Carriers

\item
Energy Band Diagrams

\item
PN Junction

\item
NPN Bipolar Teransistor

\item
Field Effect Transistor

\end{itemize}

\begin{itemize}

\item
Device Physics To Understand How They Work

\item
Processing Techniques To Understand How They Are Made

\item
Hardware Description Languages To Unsersatand How They Are Designed

\end{itemize}

\section{Commonly Used Symbols}

\begin{itemize}

\item
$N_{D}$ : Donor Concentration 

\item
$N_{A}$ : Acceptor Concentration

\item
$\eta_{i}$ : Intrinsic Carrier Concentration

\end{itemize}

\begin{table}[hbtp]

\xy <1cm,0cm>:
0 *=(4,0.5)!DL\txt{$n$ region} *\frm{-} ,
0 *=(4,1)!UL\txt{$p$ region} *\frm{-}
\endxy

\caption{For Silicon, $\eta_{i} = 1.5 x 10^{15} cm^{-3}$.}

\end{table}

\[ N_{D} = 10^{21} cm^{-3} \]

\[ p = 10^{16} cm^{-3} \]

Ratio $\frac{10^{21}}{10^{16}} = 1:10^{5}$

\section{Space Charge Neutrality}

Exact relationship among electron, hole, donor and acceptor
concentrations are found by considering the requirements for Space
Charge Neutrality. For a material to remain electrostatically neutral,
the sum of positive charges (holes and ionised donors) must balance the
sum of the negative charges (electrons and ionised acceptor atoms).

\[ P_{o} + N_{D}^{+} = \eta_{o} + N_{A}^{-} \cdots (1) \]

For an n-type material $\eta_{o} = P_{o} + (N_{D} - N_{A})$ and if
$\eta_{o}$ is far greater than $P_{o}$, then $\eta_{o} \approx N_{D} -
N_{A}$.

Since the intrinsic semiconductor is electrostatically neutral
and the doping atoms we add are also neutral, the requirement of
Equation (1) must be maintained at equilibrium.


% 2003-10-08

\subsection{Low and High Doped Material}

In practice, it is necessary sometimes to use very low doping
concentrations. N-type, then represented with $n^{-}$ or $\psi$ while 
p-type is represented by $p^{-}$ or $\pi$. Thermally generated carrier
pairs can no longer be ignored in comparison to impurity concentrations.
In the case of $\nu$ material, with $N_{A} = 0$, we get

\[ \eta_{o} = N_{D} + P_{o} = N_{D} + \frac{\eta{i}^{2}}{\eta{o}} \]

(Which is a quadratic equation for $\eta_{o}$.)

\subsection{Degenetate Semiconductors}

So far, we have assummed imputities are "few and far in between" in a
semiconductor. When concentrations are high, interactions between
closely packed impurity atoms can no longer beignored. When donor
concentrations approach $10^{20} cm^{-3}$, donor states form bonds which
may overlap with the overlap of the conduction band. In this case, the
Fermi Level, $E_{F}$, no longer lies within the bandgap, but within the
conduction basd. When this occurs, the material is said to be
degenerate. The same applies for a p-type material.

$E_{F}$ - Chemical potential, conditions across the junction
at equilibrium.

\begin{table}[hbtp]

\xy <1cm,0cm>:
0 ; (4,0) **\dir{-} *+!CL{E_C} ,
(0,-0.35) ; (4,-.35) **\dir{-} *+!CL{E_F} ,
(0,-1) ; (4,-1) **\dir{--} *+!CL{E_i}  ,
(0,-2) ; (4,-2) **\dir{-} *+!CL{E_V}
\endxy

\caption{n-type}

\end{table}

\begin{table}[hbtp]

\xy <1cm,0cm>:
0 ; (4,0) **\dir{-} *+!CL{E_C} ,
(0,-1) ; (4,-1) **\dir{--} *+!CL{E_i}  ,
(0,-1.65) ; (4,-1.65) **\dir{-} *+!CL{E_F} ,
(0,-2) ; (4,-2) **\dir{-} *+!CL{E_V}
\endxy

\caption{p-type}

\end{table}

\subsection{Non Equilibrium Conditions}

Equilibrium conditions may be disturbed in several ways. Holes and
electrons can be introduced into a semiconductor by means of a metal of
semiconductor contact. They can be produced in pairs by illuminating the
semiconductor with light of an appropriate wavelength. The protons
transfer their energy to some of the valence electrons, there by
breaking bonds and producing electrons and holes. Even in non
equilibrium conditions, $n - p = N_{D} - N_{A}$ provided the excess
electrons $n'$ is equal to the excess holes $h'$.

Note: $\eta_{i}^{2} \neq np$ under non equilibrium conditions.
There is a tendancy for the excess carriers to recombine. Conduction
electrons fall back into vacancies, thereby removing both themselves and
an equal number of holes.

\subsection{Rate Of Recombination}

R(pairs/$cm^{3}$/sec) is given by:

\[ R = \frac{n'}{\psi} or R = \frac{p'}{\psi} \]

where $\psi$ is the lifetime of excess carriers. $\psi$ depends on the
chemical and metalorigal nature of the semiconductor.

\[ 1 ns < \psi < 500 \mu s \]

\subsection{Transport of Electrons and Holes}

If we consider a semiconductor which is not subject to any external
influence, then at temperatures in excess of $0^{o}K$, the carriers will
undergo thermal agitation and their motion in the crystal will be
random. However, if an electric field $\varepsilon$, is applied to the
semiconductor, an additonal velocity component will be superimposed upon
the thermal motion of the carriers. This is called the drift velocity
and will have (for electrons) a direction opposide to the electric
field.

\begin{table}[hbtp]

\xy <1cm,0cm>:
0 ; (4,0) **\dir{-} *\dir{>} *+!CL{\varepsilon} ,
(0,-0.5) ; (0.5,-0.5) *\dir{<} **\dir{--} ,
(0.5,-0.5) ; (1,-0.5) *\dir{<} **\dir{--} ,
(1,-0.5) ; (1.5,-0.5) *\dir{<} **\dir{--} ,
(1.5,-0.5) ; (2,-0.5) *\dir{<} **\dir{--} ,
(2,-0.5) ; (2.5,-0.5) *\dir{<} **\dir{--} ,
(2.5,-0.5) ; (3,-0.5) *\dir{<} **\dir{--} ,
(3,-0.5) ; (3.5,-0.5) *\dir{<} **\dir{--} ,
(3.5,-0.5) ; (4,-0.5) *\dir{<} **\dir{--}
\endxy

\end{table}

% 2003-10-09

\subsection{Carrier Mobility}

The net drift of carriers under the influence of an electric field
depends on the relative ease with which they move within the crystalline
material of the semiconductor. Carrier mobility is defined as the
velocity with which a carrier will drift, on average in a unit field
strength and is given as:

\[ \mu_{n} = \frac{q \psi_{n}}{m_{e}^{*}} \mbox{for electrons} \]

\[ \mu_{p} = \frac{q \psi_{p}}{m_{p}^{*}} \mbox{for holes} \]

\[ m_{e}^{*} : \mbox{Effective mass of an electron} \]

\[ m_{p}^{*} : \mbox{Effective mass of a hole} \]

\[ \psi_{n}, \psi_{p} : \mbox{Average time between collisions} \]

\[ q : \mbox{Magnitude of charge on a carrier} \]

Mobility of electons in the conduction band is much greater
that that of holes in the valence band.

\subsection{Summary}

Drift - Always associated with $\varepsilon$.

Diffusion - Always associated with concentration gradient
$\frac{dn}{dx}$.

\subsection{Total Drift Current}

When charge carriers drift in a uniform manner across a semiconductor,
there is a constant and uniform flux of charge across the material. The
charge flow per unit area is referred to as the charge flux density $J$.

\begin{table}[hbtp]

\xy <1cm,0cm>:
% q Directional Arrow
(1,1) ; (2,2) **\dir{-} *\dir{>} ?(-0.05) *+!UR{q} ,
%*+!CL{q} ,
% Horizontals and Verticals
(1,0) ; (3,0) **\dir{-} ,
(1,-2) ; (3,-2) **\dir{-} ,
(1,0) ; (1,-2) **\dir{-} ,
(3,0) ; (3,-2) **\dir{-} ,
% Diagonals
(1,0) ; (3,2) **\dir{-} ,
%(1,-2) ; (3,0) **\dir{-} ,
(3,0) ; (5,2) **\dir{-} ,
(3,-2) ; (5,0) **\dir{-} ,
% Charge Flux Density Arrows
(0.5,-1.5) ; (1.5,-0.5) **\dir{-} *\dir{>} ,
(1,-1.5) ; (2,-0.5) **\dir{-} *\dir{>} ,
(1.5,-1.5) ; (2.5,-0.5) **\dir{-} *\dir{>} ,
(0.6,-2) ; (1.6,-1) **\dir{-} *\dir{>} ,
(1.1,-2) ; (2.2,-1) **\dir{-} *\dir{>} ,
(1.6,-2) ; (2.6,-1) **\dir{-} *\dir{>}
\endxy

\caption{Charge Flux Density}

\end{table}

\[ J_{n drift} = n q \mu_{n} \varepsilon \mbox{electons} \]

\[ J_{p drift} = p q \mu_{p} \varepsilon \mbox{holes} \]

Electrons and holes will drift in opposide directions in a
given electric field. However, since the charge on the electron is
negative, both electrons and holes will make positive contributions to
conventional currents specified in the direction of the field.

When both fluxes are integrated over the area $A$, then the
total drift current is obtained as:

\[ I_{drift} = (J_{n drift} A) + (J_{p drift} A) = A \varepsilon q (n
\mu_{n} + p \mu_{p}) \cdots \mbox{Important} \]

Note:

\[ \sigma = q (n \mu_{n} + p \mu_{p}) \]

where $\sigma = \mbox{conductivity} = \frac{1}{\mbox{resistivity}} $.

\subsection{Diffusion}

Carriers in a semiconductor diffuse in a carrier gradient by random
thermal motion and by scattering from a lattice and impurities.

\begin{table}[hbtp]

\xy <1cm,0cm>:
0 ; (5.25,0) **\dir{-} *\dir{>} *+!CL{x} , 
(2,0) ; (3,0) **\crv{(2.5,7.5)} ?*@{}*^+!UL{t_{1}},
(1,0) ; (4,0) **\crv{(2.5,5.0)} ?*@{}*^+!UL{t_{2}},
0 ; (5,0) **\crv{(2.5,2.5)} ?*@{}*^+!UL{t_{3}},
0 ; (2.5,0) *+!UR{0} ;
(2.5,4.25) **\dir{-} *\dir{>} *+!CR{n(x)} ,
(4,4) ; (2.55,4) **\dir{-} *\dir{>} ?(-0.05) *+!CL{t = 0}
\endxy

\end{table}

Consider a pulse of electrons injected at time $t = 0$.
Initially, they are concentrated, but as time goes by, rhey diffuse to
different regions of lower concentrations. This process is known as
diffusion and gives rise to a diffusion current. Note that the movement
of carriers is always in the direction of decreasing concentration. The
diffusion of carriers will continue as long as the concentration
gradient is maintained. The resulting current will flow in a direction
which depends on the direction of the concentration gradient and the
charge on the carrier. Diffusion depends on these parameters:

\begin{enumerate}

\item

$\frac{dn}{dx}$, $\frac{dp}{dx}$: The Concentration Gradients

\item

$\pm q$ : The Charge of the Carrier

\item

$D_{n}$, $D_{p}$ : The Diffusion Coefficients ofor the Carriers

\end{enumerate}

$D$ is a measure of the ease with which the carriers can move
in the semiconductor. The charge flux density for diffusion is:

\[ J_{n diff} = - q D_{n} - \frac{dn}{dx} = q D_{n} \frac{dn}{dx}
\mbox{electrons} \]

\[ J_{p diff} = + q D_{p} - \frac{dp}{dx} = - q D_{p} \frac{dp}{dx}
\mbox{holes} \]

Note that the negative signs in front of the concentration
gradients denotes that the carriers move in the direction of decreasing
concentration. If the concentration gradients for both holes and
electrons are in the same direction, then both types of carrier will be
travelling in the same direction and the charge flows will tend to
cancel each other. For a homogeneous semiconductor of uniform cross
sectional area, the total diffusion current due to both electrons and
holes.

% 2003-10-13

\[ I_{diff} = J_{n diff} A + J_{p diff} A \]

\[ I_{diff} = A q \left(D_{n} \frac{dn}{dx} - D_{p} \frac{dp}{dx}
\right) \]

The polarity of the gradients will determine the direction of the flow
of individual carriers and hence the overall diffusion current.

\subsection{Einstein Relation}

The diffusion coefficient is related to the mobility of carriers:

\[ D_{n} = \mu_{n} \frac{k T}{q} \]

and

\[ D_{p} = \mu_{p} \frac{k T}{q} \]

\[ \frac{k T}{q} = 26 mV \mbox{at} 300^{o}K \]

Mobility is heavily concentration dependant. Typical values
for mobility for carrier concentrations less than $10^{16} cm^{-3}$ are:

\[ \mu_{n} = 1350 \frac{cm^{2}}{V sec} \]

\[ \mu_{p} = 480 \frac{cm^{2}}{V sec} \]

\begin{table}[hbtp]

\begin{pspicture}(0,0)(15,10)
\psset{xunit=1cm,yunit=1.5cm}
\psset{plotpoints=200}
\psplot{0}{2}{2.1}
%\psplot{2}{6}{2 y * + x x *}
\psplot{0}{2}{1.5}
%\psplot{2}{6}{}
\psaxes[Ox=14,arrows=->](7,4)
\end{pspicture}

\end{table}

\section{Diffusion And Recombination}

Recombination must be considered in the description of the conduction
process, since this can cause a variation in the carrier distribution.
Suppose an external source is used to inject excess minority charge
carriers into the semiconductor. Equilibrium conditions will be
disrupted as the excess minority carriers begin to recombine with the
majority carriers present within the semiconductor. If we consider an
elemental length of semiconductor $\Delta x$ with cross sectional area
$A$ into which excess carriers are injected.

\begin{table}[hbtp]

\xy <1cm,0cm>:
0 ; (2,0) **\dir{-} *\dir{>} ?(0.05) *+!UR{J_{n}(x)} ,
(1,0) ; (1,2) **\dir{-} ,
(1,2) ; (4,-1) **\dir{-} ,
(4,-1) ; (4,-3) **\dir{-} ,
(4,-3) ; (1,0) **\dir{-} ,
%
(4,-3) ; (7,-3) **\dir{-} ?(0.75) *+!UR{x} ,
(4,-1) ; (7,-1) **\dir{-} ,
(1,2) ; (4,2) **\dir{-} ,
%
(4,2) ; (7,-1) **\dir{-} ,
(7,-1) ; (7,-3) **\dir{-} ,
%
(7,-3) ; (8,-3) **\dir{-} ,
(8,-3) ; (8,-1) **\dir{-} ,
(8,-1) ; (7,-1) **\dir{-} ,
(4,2) ; (5,2) **\dir{-} ?(0.25) *+!DL{\Delta x} ,
(5,2) ; (8,-1) **\dir{-} ,
%
(5,2) ; (8,2) **\dir{-} ?(0.25) *+!UL{A} ,
(8,-1) ; (11,-1) **\dir{-} ,
(8,-3) ; (11,-3) **\dir{-} ?(0.25) *+!UR{x + \Delta x} ,
%
(8,2) ; (11,-1) **\dir{-} ,
(11,-1) ; (11,-3) **\dir{-} ,
%
(10,0) ; (12,0) **\dir{-} *\dir{>} ?(0.25) *+!UL{J_{n}(x + \Delta x)}
\endxy

\end{table}

\[\mbox{Rate of Change of Excess Carriers} = \mbox{Rate of Cariers
Entering Volume} - \mbox{Rate of Carriers Leaving Volume} - \mbox{Rate
of Recombination} \]

\[ \frac{\delta n'}{\delta t} = \frac{I_{n}(x)}{- q \Delta V} - \frac{I_{n}(x +
\Delta x)}{-q \Delta V} - \frac{n'}{\psi_{n}} \]

$\psi_{n}$ is the mean carrier lifetime (ie the average time
before a carrier entering the volume recombines).

Then:

\[ \frac{\delta n'}{\delta t} = \frac{J_{n} (x) A}{- q A \Delta x} -
\frac{J_{n}(x + \Delta x) A}{- q A \Delta x} - \frac{n'}{\psi_{n}} \]

\[ \Rightarrow \frac{\delta n'}{\delta t} = \frac{1}{q} \left[
\frac{J_{n}(x + \Delta x) - J_{n}(x)}{\Delta x}\right] -
\frac{n'}{\psi_{n}} \]

For small $\Delta x$:

\[J_{n}(x + \Delta x) - J_{n}(x) = \Delta J_{n}(x) \]

and

\[ \frac{\delta n'}{\delta t} = \frac{1}{q} \frac{\Delta
J_{n}(x)}{\Delta x} - \frac{n'}{\psi_{n}} \]

This is known as the Charge Continuity Equation for Electrons.

A similar expression may be derived for holes:

\[ \frac{\delta p'}{\delta t} = \frac{1}{q} \frac{\Delta
J_{p}(k)}{\Delta k} - \frac{p'}{\psi_{p}} \]

If the component of current due to drift is negligible (no
electric field), then injected carriers will move by diffusion and

\[ J_{n diff} = q D_{n} \frac{dn}{dx} \]

We can then substitute for $J_{n}$ in the charge continuity
equation. Taking the very general 3D case, but restricting to the
x-direction only within the volume $A \Delta x$ and considering the
diffusion of excess carriers only, we have

\[ \frac{\delta J_{n}(x)}{\delta x} = q \frac{D_{n}
\delta_{2}n'(x)}{\delta x^{2}} \]

which gives

\[ \frac{\delta n (x)}{\delta t} = D_{n} \frac{\delta_{2}n'(x)}{\delta
x^{2}} - \frac{n'(x)}{\psi_{n}} \cdots \mbox{Important} \]

When substituted into the charge continuity equation, the
above equation is known as the diffusion equation and under steady-state
conditions (no change in charge within $A \Delta k$), then 

\[ \frac{\delta n'(x)}{\delta t} = 0 \]

and

\[ \frac{\delta_{2} n'(x)}{\delta x^{2}} = \frac{n'(x)}{D_{n} \psi_{n}}
\]

\subsection{Diffusion Length}

This is a measure of the distance $L n$ (for electrons) or $L p$
(for holes) will travel on average before recombining.

\[ L n^{2} = D_{n} \psi_{n} \]

and so

\[ \frac{\delta_{2}n'(x)}{\delta x^{2}} = \frac{1}{L n^{2}} n'(x) \]

The general solution of this differential equation is of the
form:

\[ n'(x) = A e^{-\frac{X}{Ln}} + B e^{\frac{X}{L n}} \]

where $A$ and $B$ are constants which may be determined from
known boundry conditions.

% 2003-10-15

(1) When $x \to \infty \Rightarrow$ Total Recombination

and 

\[ n'(x) \to 0 \]

\[ \Rightarrow 0 = 0 + B e^{\frac{X}{L n}} \]

\[ \Rightarrow B = 0 \]


(2) When $x \to 0 \Rightarrow$ No Recombination

and

\[ n'(x) = n'(x = 0) = \mbox{constant} \]

and

\[ n'(x = 0) = A \]

\[ \Rightarrow n'(x) = n'(x = 0) e^{- \frac{X}{L n}} \]

So the excess minority carrier concentration decreases
exponentially with distance $L$.

\begin{table}[hbtp]

\begin{pspicture}(0,0)(15,10)
\psset{xunit=7.5cm,yunit=1.5cm}
\psset{plotpoints=200}
\psplot[linestyle=dashed]{0}{1.5}{1}
%\psplot[linestyle=solid]{0}{1.5}{}
\psaxes[arrows=->](1.5,4)
\end{pspicture}

\end{table}

Production of excess carriers in a semiconductor applies to
both intrinsic and extrinsic material. In the latter case, excess
minority carriers are of particular importance as the change represents
a much larger proportion of the equilibrium density. Consider a $Ge$
sample with:

\[ N_{D} = 6.3 x 10^{14} cm^{-3} \]

\[ P = 1 x 10^{12} cm^{-3} \]

\[ (\eta_{i} = 2.4 x 10^{13} cm^{-3}) \]

If $10^{12}$ carriers $cm^{-3}$ of each sign are introduced,
the minority carrier density will increase by $\approx 0.2 \% \left(
\frac{1 x 10^{12}}{6.3 x 10^{14}} \times 100 = 0.16\right) $.

\section{PN Junction}

\begin{table}[hbtp]

\xy <1cm,0cm>:
% To +ve
(-1.75,1.25) ; (-1.75,0.5) **\dir{-} ?(0.25) *+!CR{+ 5V},
(-1.75,0.5) ; (-1,0.5) **\dir{-} ,
% PN Junction
(0,0) *=(1,1)!DR\txt{P} *\frm{-} ,
(0,0) *=(1,1)!DL\txt{N} *\frm{-} ,
% To Earth
(1.75,-0.25) ; (1.75,0.5) **\dir{-} ,
(1.75,0.5) ; (1,0.5) **\dir{-} ,
(1.2,0.15) ; (2.3,0.15) **\dir{-} ?(1) *+!DL{GND} ,
(1.4,0) ; (2.1,0) **\dir{-} ,
(1.6,-0.15) ; (1.9,-0.15) **\dir{-} ,
\endxy

\end{table}

P junctions rectify, ie they permit the passage of current in
one direction only. When a positive potential is applied to the P region
(forward bias), a current begins to flow at a very small voltage. By
contrast, when a positive potential is applied to the N region (reverse
bias), very little current flows when the voltage is low. If the reverse
bias is made sufficiently large, then current can be made to flow and
this condition is referred to as junction breakdown.

\begin{table}[hbtp]

\begin{pspicture}(15,10)
\psaxes[arrows=->](-4,-4)(4,4)
\end{pspicture}

\vspace{25mm}

\end{table}

\begin{table}[hbtp]

\xy <1cm,0cm>:
0 ; *=(2,2)!DR\txt{P} *\frm{-} ,
(0,0) *=(0.5,2)!DR\txt{-} *\frm{-} ,
(0,0) *=(0.5,2)!DL\txt{+} *\frm{-} ,
(0,0) *=(2,2)!DL\txt{N} *\frm{-} ,
(-0.5,2.5) ; (0.5,2.5) **\dir{-} *\dir{>} ?(0.75) *+!DR{+} ,
(0.75,1.4) ; (-0.75,1.4) **\dir{-} *\dir{>} ?(1.05) *+!RC{+} ,
(-0.75,0.7) ; (0.75,0.7) **\dir{-} *\dir{>} ?(1.05) *+!LC{-} ,
(0.5,-0.5) ; (-0.5,-0.5) **\dir{-} *\dir{>} ?(0.75) *+!UL{\varepsilon} 
\endxy

\end{table}

When two regions of an semiconductor of differing signs are
brought into intimite contact, a flux of electrons and holes will flow
in such a direction as to even out the large concentration gradients
existing between the two regions. If the holes and electrons were not
charged, this flux would continue until the gradients were evened out.
Because they \emph{are} charged and because the semiconductor contains
ionised impurities atoms, the situation is different. When a net flow of
electrons $\to$ P region and holes $\to$ the N region has taken place, a
space charge due to the donor and acceptor ions is formed. Hence an
electric field forms in the vicinity of the junction and this repels
holes $\to$ P region and electrons $\to$ N region. Note that in
equilibrium, the net flux of both electrons and holes will be zero. The
diffusion flux of each carrier at the junction will exactly balance the
flux of that carrier due to the electric field $\varepsilon$.

\begin{table}[hbtp]

\xy <1cm,0cm>:
0 ; (4,0) **\dir{-} ?(1.1) *+!UC{E_{C}} ,
(0,-0.5) ; (4,-0.5) **\dir{-} ?(1.1) *+!UC{E_{F N}} ,
(1,-0.75) *+!UC{P} ,
(3,-0.75) *+!UC{N} ,
(0,-1.5) ; (4,-1.5) **\dir{-} ?(1.1) *+!UC{E_{F 0}} ,
(0,-2) ; (4,-2) **\dir{-} ?(1.1) *+!UC{E_{V}}
\endxy

\end{table}

\begin{table}[hbtp]

\xy <1cm,0cm>:
0 ; (4,0) **\dir{-} ,
(2.5,0) *=(0.5,2)!UR\txt{-} *\frm{-} ,
(2,0) *=(0.5,2)!UR\txt{+} *\frm{-} ,
(0.75,-0.75) *+!UC{P} ,
(3.25,-0.75) *+!UC{N} ,
(0,-2) ; (4,-2) **\dir{-}
\endxy

\[ E_{C}: \mbox{Conduction Band} \]
\[ E_{V}: \mbox{Valence Band} \]
\[ E_{i}: \mbox{Fermi Level} \]

\xy <1cm,0cm>:
0 ; (3,0) **\dir{-} ?(-0.35) *+!CL{E_{CP}} ,
(0,-1.12) ; (3,-1.12) **\dir{--} ?(-0.35) *+!CL{E_{i}} ,
(0,-2.12) ; (5,-2.12) **\dir{-} ?(-0.2) *+!DL{E_{FP}} ,
(0,-2.25) ; (3,-2.25) **\dir{-} ?(-0.35) *+!UL{E_{VP}},
%
(3,0) ; (5,-2) **\dir{-} ,
(3,-1.12) ; (5,-3.12) **\dir{--} ,
(3,-2.25) ; (5,-4.25) **\dir{-} ,
%
(5,-2) ; (8,-2) **\dir{-} ?(1) *+!UL{E_{CN}} ,
(5,-3.12) ; (8,-3.12) **\dir{--} ?(1) *+!CL{E_{i}} ,
(5,-2.12) ; (8,-2.12) **\dir{-} ?(1) *+!DL{E_{FN}} ,
(5,-4.25) ; (8,-4.25) **\dir{-} ?(1) *+!CL{E_{VN}}
\endxy

\caption{Energy Band Diagrams}

\end{table}

This representation is for the case where $N_{A} \approx
N_{D}$.

\subsection{Carrier Flow}

Hole $\longrightarrow$ Diffusion \\
Hole $\longleftarrow$ Drift \\
Electron $\longleftarrow$ Diffusion \\
ElectronHole $\longrightarrow$ Drift \\

\subsection{Barrier Potential}

Barrier Potential (also called Biult In Voltage), $V_{B}$ (or
$\varphi_{B}$) under equilibrium conditions, the flow of carriers (both
electrons and holes) by diffusion is exactly balanced by the flow due to
drift and so the net current flow is zero. It follows then, that for
electrons:

\[ J_{n drift} + J_{n diff} = 0 \]

\[ \Rightarrow q \mu_{n} n \varepsilon + q D_{n} \frac{dn}{dx} = 0 \]

\[ \Rightarrow - \frac{\mu_{n} \varepsilon}{D_{n}} = \frac{1}{n}
\frac{dn}{dx} \]

Recall the Einstein Relation:

\[ \frac{\mu_{n}}{D_{n}} = \frac{1}{n} \frac{dn}{dx} \]

By definition, potential is the quality, whose gradient is in
the negative of the electric field $\varepsilon$.

ie

\[ \varepsilon = - \frac{dV}{dx} \]

\[ \Rightarrow \frac{q}{kT} \frac{dV}{dx} = \frac{1}{n} \frac{dn}{dx} \]

The built in voltage $V_{B}$ will vary, along with the field
$\varepsilon$ across the junction so we can write:

\[ \frac{q}{k T} \frac{DV(x)}{dx} = \frac{1}{n(x)} \frac{dn(x)}{dx} \]

This can now be now be integrated across the junction and with
appropriate limits will give $V_{B}$ ie:

\[ \frac{q}{kT} \int^{V_{n}}_{V_{p}} dV = \int^{\eta_{n o}}_{\eta_{p o}}
\frac{1}{n} dn \]

\[ \Rightarrow \frac{q}{kT} V_{B} = \ln{\eta_{n o}} - \ln{\eta_{p o}} \]

On the N side, if we consider all of the donor atoms to be
ionised, then

\[ \eta_{p o} \approx \frac{\eta_{i}^{2}}{N_{A}} \]

Hence

\[ V_{B} = \frac{kT}{q} \ln{\left(\frac{N_{A}
N_{D}}{\eta_{i}^{2}}\right)} \cdots \mbox{Important} \]

\subsection{Example}

\begin{table}[hbtp]

\xy <1cm,0cm>:
0 ; *=(3,3)!UR\txt{$N_{A} = 10^{20}$} *\frm{-} ,
(3,3) ; *=(3,3)!UL\txt{$N_{D} = 10^{16}$} *\frm{-} ,
(-0.5,0) ; (-0.5,-3) **\dir{--} ,
(0.5,0) ; (0.5,-3) **\dir{--} ,
(-0.5, 0.5) ; (0.5,0.5) **\dir{-} ?>*\dir{>} ?<*\dir{<} ?(0.25)
*+!DC{V_{B}} ,
\endxy

\end{table}

\[ V_{B} = \frac{kT}{q} \ln{\left(\frac{N_{A}
N_{D}}{\eta_{i}^{2}}\right)} \]

\[ k = 1.38 \star 10^{-23} \]

\[ T = 300^{o}K \]

\[ q = 1.6 \star 10^{-19} \]

\[ \frac{kT}{q} = \frac{1.3 \star 10^{-23} \star 300}{1.6 \star
10^{-19}} = 0.026V \]

\[ V_{B} = 0.026 \ln{\left( \frac{10^{20} \star 10^{16}}{2.25 \star
10^{20} } \right)} \]

\[ V_{B} = 0.93V \]

Therefore, when $N_{A}$ is $10^{15} cm^{-3}$:

\[ V_{B} = 0.63 V \]

\subsection{Forward Bias}

When a PN junction is under forward bias, the applied filed
$\varepsilon$ tends to counteract the built-in field and there is a
lowering of the barrier potential, by an amount equal to the applied
voltage. This permits an increased number of electrons to cross the
junction into the P region and an increased number of holes to cross
into the N region.

\begin{table}[hbtp]

\xy <1cm,0cm>:
% To +ve
(-1.75,1.25) ; (-1.75,0.5) **\dir{-} ?(0.25) *+!CR{+ 5V},
(-1.75,0.5) ; (-1,0.5) **\dir{-} ,
% PN Junction
(0,0) *=(1,1)!DR\txt{P} *\frm{-} ,
(0,0) *=(1,1)!DL\txt{N} *\frm{-} ,
% To Earth
(1.75,-0.25) ; (1.75,0.5) **\dir{-} ,
(1.75,0.5) ; (1,0.5) **\dir{-} ,
(1.2,0.15) ; (2.3,0.15) **\dir{-} ?(1) *+!DL{GND} ,
(1.4,0) ; (2.1,0) **\dir{-} ,
(1.6,-0.15) ; (1.9,-0.15) **\dir{-} ,
\endxy

\end{table}

This gives rise to an increase in minority carrier
concentration in both regions (see minority carrier concentration
diagrams). The increased levels of carrier concentration are maintained
as long as the external bias is maintained. Carriers which recombine as
they diffuse into the neutral regions away from the junction are
replaced by the external supply. 
% 2003-10-22
There will therefore be a continious flow of current under forward bias
conditions, excess carriers injected at the boundaries of the neutral
regions, move away from the junction by diffusion and recombine with 
majority carriers as they do so. The excess minority carrier
concentration is seen to drop off exponentially with distance. Carriers
crossing the junction do so by a drift mechanism.

\begin{table}[hbtp]

\xy <1cm,0cm>:
(4.75,0) *=(1.5,2)!DR\txt{P} *\frm{-} ,
(4.75,0) *=(1,2)!DL\txt{} *\frm{-} ,
(5.75,0) *=(1.5,2)!DL\txt{N} *\frm{-} ,
(5.25,0) ; (5.25,2) **\dir{--}
\endxy

\end{table}

\begin{table}[hbtp]

\xy <1cm,0cm>:
(0,-3) ; (1.5,-3) **\dir{-} ,
(0,-2) ; (1.5,-2) **\dir{-} ?(-0.5) *+!CL{\eta_{p o}} ,
(1.5,0) ; (1.5,-3) **\dir{-} ,
%
(2.5,-3) ; (4,-3) **\dir{-} ,
(2.5,-2) ; (4,-2) **\dir{-} ?(1) *+!CL{P_{n o}} ,
(2.5,0) ; (2.5,-3) **\dir{-}
\endxy

\caption{Equilibrium}

\end{table}

\begin{table}[hbtp]

\xy <1cm,0cm>:
(0,0) *=(1.5,2)!DR\txt{P} *\frm{-} ,
(0,0) *=(1,2)!DL\txt{} *\frm{-} ,
(1,0) *=(1.5,2)!DL\txt{N} *\frm{-} ,
(0.5,0) ; (0.5,2) **\dir{--} ,
\endxy

\end{table}

\begin{table}[hbtp]

\xy <1cm,0cm>:
(0,-3) ; (1.5,-3) **\dir{-} ,
(0,-2) ; (1.5,0) **\crv{(1.475,-1.975)} , %?(-0.5) *+!CL{\eta_{p o}} ,
(1.5,0) ; (1.5,-3) **\dir{-} ,
%
(2.5,-3) ; (4,-3) **\dir{-} ,
(2.5,0) ; (4,-2) **\crv{(2.575,-1.975)} , %?(1.0) *+!CL{P_{n o}} ,
(2.5,0) ; (2.5,-3) **\dir{-} ,
%
(1.25,-1) ; (0,-1) **\dir{-} *\dir{>},
(2.75,-1) ; (4,-1) **\dir{-} *\dir{>}
\endxy

\caption{Forward Bias}

\end{table}

\begin{table}[hbtp]

Minority Carrier Concentrations

\end{table}

% 2003-10-20

\subsection{PN Junction Examples}

1. (a) Explain what is meant by the term "potential barrier" in a PN
junction and give an account of how it arises.

Answer:

\begin{table}[hbtp]

\xy <1cm,0cm>:
(0,0) *=<2cm,2cm>!DR\txt{P} *\frm{-} ,
(0,0) *=<2cm,2cm>!DL\txt{N} *\frm{-} 
\endxy

\end{table}

Concentration gradient driving force for:

Electrons Holes

\begin{table}[hbtp]

\xy <1cm,0cm>:
(3,0) ; (1,0) **\dir{-} *\dir{>} ,
(4,0) ; (6,0) **\dir{-} *\dir{>}
\endxy

\end{table}

\begin{table}[hbtp]

\xy <1cm,0cm>:
0 ; *=<2cm,2cm>!DR\txt{P} *\frm{-} ,
(0,0) *=<0.5cm,2cm>!DR\txt{-} *\frm{-} ,
(0,0) *=<0.5cm,2cm>!DL\txt{+} *\frm{-} ,
(0,0) *=<2cm,2cm>!DL\txt{N} *\frm{-} 
\endxy

\end{table}

Resultant Charge $\Rightarrow \varepsilon$.

\begin{table}[hbtp]

\xy <1cm,0cm>:
0 ; *=<2cm,2cm>!DR\txt{P} *\frm{-} ,
(0,0) *=<0.5cm,2cm>!DR\txt{-} *\frm{-} ,
(0,0) *=<0.5cm,2cm>!DL\txt{+} *\frm{-} ,
(0,0) *=<2cm,2cm>!DL\txt{N} *\frm{-} ,
(0.5,-0.5) ; (-0.5,-0.5) **\dir{-} *\dir{>} ?(0.5) *+!UC{\varepsilon}
\endxy

\end{table}

$\varepsilon$ gives rise to $V_{B}$.

Equilibrium situation where by flux due to diffusion is balanced by the
flux due to drift.

(b) Show that the barrier potential $V_{B}$ is given by:

\begin{eqnarray*}
V_{B} & = & \frac{kT}{q} \ln{\left(\frac{N_{A}
N_{D}}{\eta_{i}^{2}}\right)} \\
J_{n drift} + J_{n diff} & = & 0 \\
q \mu_{n} n \varepsilon + q D_{n} \frac{dn}{dx} & = & 0 \\
- \frac{\mu_{n} \varepsilon}{D_{n}} & = & \frac{1}{n} \frac{dn}{dx} \\
\frac{\mu_{n}}{D_{n}} & = & \frac{1}{n} \frac{dn}{dx} \\
\varepsilon & = & - \frac{dV}{dx} \\
\frac{q}{kT} \frac{dV}{dx} & = & \frac{1}{n} \frac{dn}{dx} \\
\frac{q}{k T} \frac{dV(x)}{dx} & = &\frac{1}{n(x)} \frac{dn(x)}{dx} \\
 & = & 0 \\
\frac{q}{kT} \int^{V_{n}}_{V_{p}} dV & = & \int^{\eta_{n o}}_{\eta_{p o}}
\frac{1}{n} dn \\
\frac{q}{kT} V_{B} & = & \ln{\eta_{n o}} - \ln{\eta_{p o}} \\
\eta_{p o} & \approx & \frac{\eta_{i}^{2}}{N_{A}} \\
V_{B} & = & \frac{kT}{q} \ln{\left(\frac{N_{A}
N_{D}}{\eta_{i}^{2}}\right)} \\
\end{eqnarray*}

(c) A $Si$ PN junction has an acceptor concentration $N_{A} = 10^{17}
cm^{-3}$ and a donor concentration $N_{D} = 10^{15} cm^{-3}$. \\

Determine the barrier potential at room temperature given:

\[ \eta_{i} = 1.5 x 10^{10} cm^{-3} \]

\[ k = 1.38 x 10^{-23} cm^{-3}  J/^{o}K \]

\[ q = 1.6 x 10^{-19} C \]

\[ \Rightarrow V_{B} = \frac{kT}{q} \ln{\left(\frac{N_{A}
N_{D}}{\eta_{i}^{2}}\right)}
\]

\[ \Rightarrow V_{B} = 0.7 V \]

% 2003-10-22

(d) Explain what is meant by:

(i) Drift
(ii) Diffusion

\[ \mbox{Drift:} q \mu_{n} n \varepsilon \]

\[ \mbox{Diffusion:} q D_{n} \frac{dn}{dx} \]

2. Calculate $V_{B}$ for a $Ge$ PN junction.

\[ N_{A} = 10^{17} cm^{-3} \]
\[ N_{D} = 10^{15} cm^{-3} \]

\[ \eta_{i} = 2.4 x 10^{13} cm^{-3} \]

\[ k = 1.38 x 10^{-23} cm^{-3}  J/^{o}K \]

\[ q = 1.6 x 10^{-19} C \]

\[ \Rightarrow  V_{B} = \frac{kT}{q} \ln{\left(\frac{N_{A}
N_{D}}{\eta_{i}^{2}}\right)}
\]

\[ \Rightarrow V_{B} \approx 0.3V \]

% 2003-10-22

\subsection{Ideal Diode Equation}

By considering the diffusion of electrons and holes away from the
boundaries at the junction and taking account of the effect on the
excess minority carrier concentrations, it can be shown that:

\[ I = I_{0} \left( exp^{\frac{q V}{kT}} - 1 \right) \]

Where $I_{0}$ is the reverse saturation current and $V$ is the bias
voltage.

\begin{table}[htbp]

TODO - GNUPlot

\end{table}

For when $V$ is far greater than $O$,

\[ I = I_{0} exp^{\frac{q V}{kT}} \]

\[ \ln{I} = \frac{q V}{kT} - \ln{I_{0}} \]

\subsection{Reverse Bias}

When a PN junction is under reverse bias, the applied electric field and
the potential barrier is increased further (by an amount equal to the
applied reverse voltage). This has the effect of repelling an increased
number of carriers away from the junction. It follows that the minority
carrier concentration in the region will be lowered below the
equilibrium value. It should be noted that thermal generation will give
rise to some carriers crossing the junction and this flow of charge is
known as the reverse saturation current.

Note: In the neutral regions away from the junction, current
flow is primarily by diffusion.

% 2003-10-23

\begin{table}[hbtp]

% Minority Carrier Concentration diagrams

\end{table}

\subsection{The Real Diode}

Under conditions of a very low or of a very high injection, there is a
departure from \emph{ideal diode} conditions:

First, we introduce the ideality factor $n$, by replacing
$\frac{q}{kT}$ by $\frac{q}{nkT}$. The ideality factor is a measure of
the quality of the conditions under which the diode was manufactured.


Typically:

\[ 1 < n < 2 \]

with a value of $1.05$ considered good.

Under any reasonable amount of forward bias, the exponential
term will be far greater than $-1$, so:

\[ I \approx I_{0} \left( exp^{\frac{q V}{nkT}} \right) \]

or

\begin{eqnarray*}
\ln{I} & = & \ln{\left(I_{0} + \frac{q V}{nkT}\right)} \\
y & = & c + mx
\end{eqnarray*}

A plot of $\ln{I}$ versus $V$ will yield a straight line. The
intercept will give $I_{0}$ and $n$ can be determined from the slope.
There are two principle regions of non ideal behaviour.

\begin{table}[hbtp]

\xy <1cm,0cm>:
% Dotted line
(0,1.3) *+!CR{I_{0}} ; (2.5,4) **\dir{--} *+!DL{m = \frac{q}{nkT}} ,
% Curved Line
(0,1.55) *+!DR{I_{0} (measured)} ; (3.5,3.75)
**\crv{(0.2,1.4)&(1.05,2.65)&(2.1625,3.45)} ,
% Axes
0 ; (0,4) **\dir{-} *\dir{>} *+!UR{\ln{I}} ,
0 ; (4,0) **\dir{-} *\dir{>} *+!UR{V} ,
% Arrow
; (1.5,1.5) *+!DL{\mbox{Ideal when } n = 1} ; (1,2.25) **\dir{-} *\dir{>} ,
\endxy

\end{table}

\subsection{Forward Bias Deviations From Ideal Conditions}

Under low level injection conditions, deviation is due to cahrge
generation in the space-charge region, which supplies an additional
component to the diffusion current. Surface leakage is also more
apperant at low voltages under high injection conditions, the voltage
drop in the N and P regions becomes significant. There is also the
voltage drop across the semiconductor contacts. In a typical modern
diode, the linear region would extend over roughly $8$ decades of
current (eg $10^{19} - 10^{1}$).

A diode can be characterised very basically by measuring it's
forward bias $I - V$ characteristics and determining the value of $n$
from the slope.


\subsection{Some Other Diodes}

\subsection{Schottky Barrier Diode (SBDs)}

SBDs are based on rectifying properties of the Schottky barrier, which
forms when certain metals are deposited on the semiconductor surface.
Application of a forward bias cause the barrier to be lowered and
current flow is decribed by the diode equation:

\[ I = I_{0} \left( exp^{\frac{q V}{nkT}} -1 \right) \]

There are several differences between the SBDs and the PN
junction. Most notably, the turn-on voltage is much lower, typically
about $0.3V$. SBDs are important for high speed applications where their
switching speed is superior to that of the PN junction. Platinium and
Palladium are two metals commonly used.

\begin{table}[hbtp]

\xy <1cm,0cm>:
% Graphed Data
(0.6,0) *+!UR{0.3} ; (1.15,4) **\crv{(0.95,1.25)} *^+!CR{SBD} ,
(1.2,0) *+!UL{0.6} ; (1.75,4) **\crv{(1.55,1.25)} *^+!CL{PN} ,
% Axes
0 ; (0,4) **\dir{-} *\dir{>} ,
0 ; (4,0) **\dir{-} *\dir{>} ,
\endxy

\end{table}

\vspace{10mm}

\begin{table}[hbtp]

\xy <1cm,0cm>:
% Horizontal
0 ; (7.5,0) **\dir{-} ,
% Verticals
(2.5,0) ; (2.5,2.5) **\dir{-} ,
(5,0) ; (5,2.5) **\dir{-} ,
% Arrows
(2,2.25) *+!CR{-} *\frm{o} ; (0,2.25) **\dir{-} *\dir{>} ,
(5.5,2.25) *+!CR{+} *\frm{o} ; (7.5,2.25) **\dir{-} *\dir{>} ,
% Curves
(0,0.8) ; (2.5,2.5) **\crv{(2.35,1.1)} ,
(5,2.5) ; (7.5,0.8) **\crv{(5,0.6)} ,
(5,0) ; (7.5,0.8) **\crv{(5,0.9)} ,
\endxy

\end{table}

\subsection{Zener Diode}

With proper selection of the doping levels, a PN junction can be
designed to conduct in the reverse direction at a particular voltage,
provided the manufacturers limits on power dissipation are not exceeded.
The device will have a voltage which is relatively independant of
current drawn and so can be used as a voltage reference.

\begin{table}[hbtp]

\xy <1cm,0cm>:
% Axes
0 *+!UL{-20V} ; (8,0) **\dir{-} *\dir{>} *+!CL{V} ,
(4,-2) ; (4,2) **\dir{-} *\dir{>} *+!UR{I} ,
% Curve
(0,-2) ; (4,0) **\crv{(0.15,0.15)}
\endxy

\end{table}

eg As a current control with a suspect voltage source.

\begin{table}[hbtp]

%\begin{circuit}{0}
%\- 1 u \\connection1 {Load} l
%\end{circuit}

\end{table}

eg If the voltage level rises above $-20$, the voltage gets
connected to ground (GND), thus maintaining a fairly constant voltage
across the load.

% 2003-10-29

\section{Bipolar Junction Transistor}

Two types: NPN and PNP. The NPN is more popular as speed is more than
$2.5$ faster than that of the PNP.

\begin{table}[hbtp]

\xy <1cm,0cm>:
% Horizontals
0 ; (6,0) **\dir{-} ,
(0,2) ; (6,2) **\dir{-} ,
(0,3) ; (6,3) **\dir{-} ,
(2,2.75) ; (4,2.75) **\dir{-} ,
(1,2.3) ; (5,2.3) **\dir{-} ,
% Curves
(1.5,3) ; (2,2.75) **\crv{(1.55,2.75)} ,
(4,2.75) ; (4.5,3) **\crv{(4.45,2.75)} ,
(0.5,3) ; (1,2.3) **\crv{(0.55,2.3)} ,
(5,2.3) ; (5.5,3) **\crv{(5.45,2.3)} , 
% Pins
(3,3) ; (3,3.5) **\dir{-} *+!DL{E} *--\frm{o} ,
(4.75,3) ; (4.75,3.5) **\dir{-} *+!DL{B} *--\frm{o} ,
(3,0) ; (3,-0.5) **\dir{-} *+!UL{C} *--\frm{o} ,
% Arrows
(1.5,4) *+!CR{n^{+}} ; (2.5,3) **\dir{-} *\dir{>} ,
(4,4) *+!CL{10^{19}} ; (3.5,3) **\dir{-} *\dir{>} ,
(1,0) *\dir{<} ; (1,2) **\dir{-} *\dir{>} *!/_3mm/{500 \mu m} ,
% Labels
(2,1) *+!CL{~10^{21}} ,
(5,1) *+!CL{n^{+}} ,
(1,2.5) *+!CL{5 \times 10^{15}} ,
(5.75,2.5) *+!CL{n} ,
(4.75,2.75) *+!CL{p} ,
\endxy

\caption{$E$: Emitter. $B$: Base. $C$: Collector}

\end{table}

A biploar junction transistor (BJT) is simply two PN junctions
back to back. The substrate (for a discrete device) is heavily doped to
minimise series resistance, as the collector current must flow through
the entire thickness of the $Si$ wafer. The active collector requires a
much lower doping, so this is achieved by growing a $Si$ layer on top of
the $n^{+}$ substrate. This can be doped to the required level during
growth. 

\[ 7 \mu m = 7 x 10^{-4} cm - \mbox{Typical Thickness of the p region} 
\]

A p-type diffused layer forms the base and the emitter is formed
by diffusing a heavily doped $n^{+}$ layer into the base. Typical values
for the dopant concentration are:

\begin{eqnarray*}
N_{E} & \approx & 10^{19} cm^{-3} \\
N_{B} & \approx & 10^{17} cm^{-3} \\
N_{C} & \approx & 5 \times 10^{15} cm^{-3}
\end{eqnarray*}

\begin{table}[hbtp]

\xy <1cm,0cm>:
% Boxes
0 ; *=(2,2)!DR\txt{N} *\frm{-} ,
(2,0) *=(2,2)!DL\txt{N} *\frm{-} ,
(0,0) *=(2,2)!DL\txt{P} *\frm{-} ,
% Pins
(1,0) ; (1,-0.5) **\dir{-} *+!UL{I_{B}} *--\frm{o} ,
(4,1) ; (4.5,1) **\dir{-} *+!UL{I_{C}} *--\frm{o} ,
(-2,1) ; (-2.5,1) **\dir{-} *+!UR{I_{E}} *--\frm{o} ,
% Depletion Layers
(-0.1,0) ; (-0.1,2) **\dir{--} ,
(0.4,0) ; (0.4,2) **\dir{--} ,
(1.5,0) ; (1.5,2) **\dir{--} ,
(2.5,0) ; (2.5,2) **\dir{--} ,
%(-0.1,2.2) ; (0.4,2.2) *\ar@{<->},
%(1.5,2.2) ; (2.5,2.2) *\ar@{<->} ,
% Labels
(-1,0.5) *+!UL{n^{+}} ,
\endxy

\caption{Depletion Layers in an NPN Junction}

\end{table}

There are different ways in which the $E-B$ and $C-B$ junctions
may be biased, but in the "normal" mode of operation, $E-B$ will be
forward biased and $C-B$ will be reverse biased. In this mode, it can be
used to amplify an input signal.

\begin{table}[hbtp]

TODO - Circ

\end{table}

\subsection{Emitter - Base Region}

This junction is forward biased and the potential barrier is reduced to
the point where electrons are injected by the emitter into the base
region. They then diffuse across the base region. Most of these
electrons will reach the $C-B$ junction, but a small minority (typically
$\approx 1\%$) will recombine with holes in the base. Lowering of the
potential barrier also permits holes to drift across the junction from
the base and diffuse into the emitter region. Recall that the emitter
doping $N_{E}$ is far greater than $N_{B}$. base doping. This means that
many more electrons will b einjected into the base, compared to holes
into the emitter. It follows that the current flow, due to electons
(across $E-B$) will be very much greater than the current due to holes.

% 203-12-01

The emitter is formed by diffusing a heavily doped $n^{+}$
layer into the base. Typical values for dopant concentrations are: \\

\begin{eqnarray*}
N_{E} & \approx & 10^{19} cm^{-3} \\
N_{B} & \approx & 10^{17} cm^{-3} \\
N_{C} & \approx & 5 \times 10^{15} cm^{-3}
\end{eqnarray*}

\begin{table}[hbtp]

Diagram - Circ

\end{table}

There are different ways in which the E-B and C-B junction may
be biased, but in the "normal" mode of operation, E-B will be forward
biased and C-B will be reverse biased. In this mode, it can be used to
amplify an input signal.

\begin{table}[hbtp]

\xy <1cm,0cm>:
% NPN Junctions
0 ; *=(2,2)!DR\txt{} *\frm{-} ,
(2,0) *=(2,2)!DL\txt{} *\frm{-} ,
(0,0) *=(2,2)!DL\txt{} *\frm{-} ,
% Depletion Layers
(-0.1,0) ; (-0.1,2) **\dir{--} ,
(0.4,0) ; (0.4,2) **\dir{--} ,
(1.5,0) ; (1.5,2) **\dir{--} ,
(2.5,0) ; (2.5,2) **\dir{--} ,
% Circuit
(-2,1) ; (-3,1) **\dir{-} ,
(-3,1) ; (-3,3) **\dir{-} ,
(-3,3) ; (-0.9,3) **\dir{-} ,
(-0.9,2.75) ; (-0.9,3.25) **\dir{-} ,
(-0.7,2.5) ; (-0.7,3.5) **\dir{-} ,
(-0.7,3) ; (2.9,3) **\dir{-} ,
(2.9,2.75) ; (2.9,3.25) **\dir{-} ,
(3.1,2.5) ; (3.1,3.5) **\dir{-} ,
(3.1,3) ; (5,3) **\dir{-} ,
(5,3) ; (5,1) **\dir{-} ,
(4,1) ; (5,1) **\dir{-} ,
(1,2) ; (1,3) **\dir{-} *\dir{*},
% Arrows
(0.6,1.75) ; (-1,1.75) **\dir{.} ,
(-1,1.75) ; (-1,1.85) **\dir{.} ,
(-1,1.85) ; (-1.25,1.6) **\dir{.} ,
(-1.25,1.6) ; (-1,1.35) **\dir{.} ,
(-1,1.35) ; (-1,1.45) **\dir{.} ,
(-1,1.45) ; (0.6,1.45) **\dir{.} ,
(0.6,1.75) ; (0.6,1.9) **\dir{.} ,
(0.6,1.9) ; (0.9,1.9) **\dir{.} ,
(0.9,1.9) ; (0.9,1.25) **\dir{.} ,
(0.9,1.25) ; (1,1.25) **\dir{.} ,
(1,1.25) ; (0.75,1) **\dir{.} ,
(0.75,1) ; (0.5,1.25) **\dir{.} ,
(0.5,1.25) ; (0.6,1.25) **\dir{.} ,
(0.6,1.25) ; (0.6,1.45) **\dir{.} ,
%
(1.3,1.75) ; (3.25,1.75) **\dir{.} ,
(3.25,1.75) ; (3.25,1.55) **\dir{.} ,
(3.25,1.55) ; (1.3,1.55) **\dir{.} ,
(1.3,1.55) ; (1.3,1.5) **\dir{.} ,
(1.3,1.5) ; (1.15,1.65) **\dir{.} ,
(1.15,1.65) ; (1.3,1.8) **\dir{.} ,
(1.3,1.8) ; (1.3,1.75) **\dir{.} ,
%
(1.15,1.05) ; (1.15,1.25) **\dir{.} ,
(1.15,1.25) ; (3.1,1.25) **\dir{.} ,
(3.1,1.25) ; (3.1,1.3) **\dir{.} ,
(3.1,1.3) ; (3.25,1.15) **\dir{.} ,
(3.25,1.15) ; (3.1,1) **\dir{.} ,
(3.1,1) ; (3.1,1.05) **\dir{.} ,
(3.1,1.05) ; (1.15,1.05) **\dir{.} ,
%
(-1,0.55) ; (0.85,0.55) **\dir{.} ,
(0.85,0.55) ; (0.85,0.75) **\dir{.} ,
(0.85,0.75) ; (0.7,0.75) **\dir{.} ,
(0.7,0.75) ; (1,1) **\dir{.} ,
(1,1) ; (1.3,0.75) **\dir{.} ,
(1.3,0.75) ; (1.15,0.75) **\dir{.} ,
(1.15,0.75) ; (1.15,0.55) **\dir{.} ,
(1.15,0.55) ; (3.4,0.55) **\dir{.} ,
(3.4,0.55) ; (3.4,0.7) **\dir{.} ,
(3.4,0.7) ; (3.7,0.4) **\dir{.} ,
(3.7,0.4) ; (3.4,0.1) ** \dir{.} ,
(3.4,0.1) ; (3.4,0.25) **\dir{.} ,
(3.4,0.25) ; (-1,0.25) **\dir{.} ,
(-1,0.25) ; (-1,0.55) **\dir{.} ,
% Labels
(-2.75,2.25) *+!CL{I_{E}} ; (-2.75,2.75) **\dir{-} *\dir{>} ,
(1.25,2.75) ; (1.25,2.25) **\dir{-} *\dir{>} *+!CL{I_{B}} ,
(4.75,2.75) ; (4.75,2.25) **\dir{-} *\dir{>} *+!CR{I_{C}} ,
(-0.6,1.6) *+!CR{h^{+}} ,
(2.25,1.6) *+!CR{h^{+}} ,
(2.75,1.2) *+!CR{e^{-}} ,
(1.45,0.5) *+!CR{e^{-}} ,
\endxy

\end{table}

\begin{table}[hbtp]

\vspace{10mm}

\xy <1cm,0cm>:
% Horizontals
0 ; (7.5,0) **\dir{-} ,
(0,0.5) *+!CR{P_{EO}} ; (2.5,0.5) **\dir{--} ,
(3,1.25) ; (4.5,1.25) **\dir{--} *+!DR{N_{BO}} ,
(5,2.25) ; (7.5,2.25) **\dir{--} *+!CL{P_{CO}} ,
% Verticals
(2.5,0) ; (2.5,2.5) **\dir{--} ,
(3,0) ; (3,2.5) **\dir{--} ,
(4.5,0) ; (4.5,2.5) **\dir{--} ,
(5,0) ; (5,2.5) **\dir{--} ,
% Curves
(0,0.525) ; (2.475,1.25) **\crv{(2.25,0.475)} *^+!DR{P_{E}} ,
(3,2.15) ; (4.5,0) **\crv{(3.5,0)} ,
(5.025,0) ; (7.5,2.225) **\crv{(5,2.25)} ,
\endxy

\caption{Minority Carrier Distribution}

\end{table}

\begin{table}[hbtp]

\vspace{10mm}

\xy <1cm,0cm>:
0 ; (0,2.5) **\dir{--} ,
(0.1,0) ; (0.1,2.5) **\dir{-} ,
(0.5,0) ; (0.5,2.5) **\dir{--} , 
(1.5,0) ; (1.5,2.5) **\dir{--} ,
(1.75,0) ; (1.75,2.5) **\dir{-} ,
(2.25,0) ; (2.25,2.5) **\dir{--} ,
\endxy

\end{table}

\subsection{Collector - Base Junction}

Recall that when a PN junction is reverse biased, the minority
carrier concentration falls below the equilibrium (unbiased) value and
tends to zero. (This resilts from charge carriers being froced away from 
the junction by the increased barrier).

Recall also that there will still be thermally generated
carriers (electrons in the base and holes in the collector). These
carriers give rise to a small electron current into the collector and a
small hole current into the base. These two components of current give
rise to the reverse saturation current, which we had when considering a
PN junction in isolation. However, there is an additional C-B current
whioch dominates the electrons which drifted across the base region
without recombining, reach the edge of the C-B junction, where they come
under the influence of the field assocciated with the reverse biased C-B
junction. The direction of this field is such as to sweep the electrons
across the space charge region, where they are "collected" by the
collector. This is the principle electric component of current. The base
width is intentionally made very narrow ($N$ is far smaller than
$L_{n}$) in order to minimise numbers of electrons recombining with
majority carrier holes in the base. The holes that do not recombine aree
replenished by the base current, $I_{B}$. The other component of $I_{B}$
is the hole current injected into the emitter. This hole current is very
small and as a result, the collector current $I_{C}$ is only a little
bit smaller than $I_{E}$, where as $I_{B}$ is about two orders of
magnatude lower.

\[ I_{C} = I_{E} - I_{B} \]

Note: If we neglect recombination in the base region, we can
assume a linear carrier concentration profile across the this region.
The dotted line on the diagram takes into account recombination. It
should also be noted that the applied biasa voltages are dropped almost
entirely across the space-charge regions assocciated with the E-B and
C-B junctions. This means that in the neutral regions, current flow
acorss the depletion regions will be by drift, under conditions of high
level injections, a small electric field exists acorss the neutral
regions.

By injecting a small amount of charge into the base (forward
bias conditions), the resultant lowering of the potential barrier
permits a large collector current to flow. This demonstrates the
amplifying property of the transistor.

% 20031203

\begin{table}[hbtp]

\begin{center}

\begin{circuit}{0}
\npn1 {?} B l														% NPN Transistor
\frompin npn1C														% From the Transistor's Collector
\- 1 u																% Wire up
\.1																	% Junction #1
\- 2 r																% Wire right
\cc\fconnection1 {$V_{CE}$} c r 								% Connection V_{CE}
\htopin .1															% To right side of Junction #1
\- 1 u																% Wire up
\R1 {$I_C$} u														% Resistor I_C
\frompin R1t														% From top of Resistor I_C
\- 2 u																% Wire up
\.2 																	% Junction #2
\- 8 r																% Wire right
\cc\fconnection2 {} c r			 								% Connection #2
\centertext {$V_{CC} = 5V$} from fconnection2 to .2r	% V_{CC} label
\htopin .2														 	% To right side of Junction #2
\- 16 l																% Wire left
\cc\fconnection3 {} c l 										% Connection V_{CE}
\htopin .2															% To right side of Junction #2
\frompin npn1E														% From the Transistor's Emitter
\- 8 d 																% Wire down
\.3																	% Junction #3
\gnd1																	% GND symbol
\vtopin gnd1														% Connect Junction #3 to GND
\- 1 d																% Wire down
\atpin npn1B														% From the Transistor's Base
\- 2 l																% Wire left
\R2 {$I_B$} l														% Resistor I_B
\- 1 l																% Wire left
\nl\varU1 {} l														% Temp symbol for AC supply
\htopin varU1r														% Connect to right of varU1
\frompin varU1l													% From left of varU1
\- 2 l																% Wire left
\- 1 d																% Wire down
\vtopin R3t															% Connect to top of DC Supply #1
\nl\R3 {} d 														% Temp DC Supply #1
\frompin R3b														% Connect bottom of DCSupply #1 to
\vtopin R4t															% At the top
\nl\R4 {} d															% Temp DC Supply #2
\frompin R4b														% From the bottom of DC Supply #2
\- 1 d																% Wire down
\htopin .3															% Connect to Connection Point #3
\frompin .3l														% From the left of Connection #3
\- 8 r																% Wire right
\end{circuit}

\end{center}

\end{table}

\begin{table}[hbtp]

\vspace{10mm}
**\dir{--} 
\xy <1cm,0cm>:
% Axes
0 ; (8,0) **\dir{-} *\dir{>} *+!UR{V_{CE} (V)} ,
0 ; (0,5) **\dir{-} *\dir{>} *+!UR{I_{C} (nA)} ,
0 ; (1,0) *+!CC{+} *+!UC{1} 
; (2,0) *+!CC{+} *+!UC{2}
; (3,0) *+!CC{+} *+!UC{3}
; (4,0) *+!CC{+} *+!UC{4}
; (5,0) *+!CC{+} *+!UC{5} ,
% Diagonal
(0,4.75) *+!DL{\frac{V_{CC}}{R_{C}}} ; (5,0) **\dir{-} ,
% Data
0 ; (0.1,0.1) **\crv{(0,0.1)} ; (7.5,0.1) **\dir{-} ,
(0.1,0.1) ; (0.2,0.75) **\crv{(0.1,0.75)} ; (7.5,0.75) **\dir{-} *+!CL{20 mA} ,
(0.2,0.75) ; (0.4,1.5) **\crv{(0.3,1.5)} ; (7.5,1.5) **\dir{-} *+!CL{40 mA} ,
(0.4,1.5) ; (0.6,2.25) **\crv{(0.5,2.25)} ; (7.5,2.25) **\dir{-} *+!CL{60 mA} ,
(0.6,2.25) ; (0.8,3) **\crv{(0.7,3)} ; (7.5,3) **\dir{-} *+!CL{80 mA} ,
(0.8,3) ; (1,3.75) **\crv{(0.9,3.75)} ; (7.5,3.75) **\dir{-} *+!CL{100 mA} ,
%
(-2.5,2.25) ; (2.65,2.25) **\dir{--} ; (2.65,-1.5) **\dir{--} ,
(-2.5,1.5) ; (3.425,1.5) **\dir{--} ; (3.425,-1.5) **\dir{--} ,
(-2.5,0.75) ; (4.2,0.75) **\dir{--} ; (4.2,-1.5) **\dir{--} ,
%
(-2.5,1.5) ; (-1,1.5) **\crv{(-2,4.125)&(-1.5,-1.125)} ,
(3.425,-1) ; (3.425,-1.5) **\crv{(0.75,-1.17)&(6.05,-1.34)} ,
% Arrows and Labels
(-1.5,-1.5) *+!DL\txt{Cut Off} ; (0.5,0.1) **\dir{-} *\dir{>} ,
(2.5,5) *+!DL\txt{Saturation} ; (0.775,3.95) **\dir{-} *\dir{>} ,
(4,3.375) *+!\txt{Active Region} ,
\endxy

\caption{Output Characteristics}

\end{table}

First draw a load line and establish an operating point.

\begin{eqnarray}
I_{C} & = & \frac{V_{CC} - V_{CE}}{R_{C}} \\
		& = & \frac{V_{CC}}{R_{C}} - \frac{V_{CE}}{R_{C}}
\end{eqnarray}

Supposing $I_{C} = 0$, $V_{CC} = V_{CE}$. \\

Supposing $V_{CE} = 0$, $I_{C} = \frac{V_{CC}}{R_{C}}$. \\

If the transistor has a current gain $\beta$, then:

\[ I_{C} = \beta I B \]

when operating in the linear active regions, the transistor
behaves like an amplifier and the output signal is an almost undistorted
version of the input.

\subsection{Operation As A Switch}

Unlike an amplifier circuit which aims to process signals in
an analogue manner without distorting shapes, a switching circuit is
essentially two state. This type of circuit is the basis of digital
logic and storage circuits and is the essence of the electronic
computer.

In this mode of operation, the transisitor no longer operates
in the active regions (except during switching).

\begin{table}[hbtp]

\begin{circuit}{0}

\end{circuit}

\end{table}

% Invertor

Suppose $I_{B} = 0$: \\

When the input is \emph{LOW}, $V_{CE} = V_{CC}$. \\

When the input is \emph{HIGH}, $V_{CE} \approx 0$. \\

\subsection{Off Region}

If $I_{B} = 0$ or $V_{BE} \leq 0$, then $I_{C} = 0$ and the
transistor is said to be \emph{OFF}.

\begin{eqnarray}
I_{C} & = & 0 \\
I_{C} \times R_{C} & = & 0 \\
V_{0} = V_{CC} & = & V_{CE} = \emph{HIGH}
\end{eqnarray}

With $V_{I} = 0$ (\emph{LOW}) and $V_{0} = V_{CC}$
(\emph{HIGH}).

ie the transistor is providinf the logic function of
inversion:

\emph{LOW} in, \emph{HIGH} out.

\subsection{Saturation Region}

If a high base current is delivered to the transistor, we
might expect the collector current to be $I_{C} = \beta I B$ (as was
ther case when operating in the linear active region). However, $R_{C}$
is used to limit $IC$ to $I_{C MAX} = \frac{V_{CC}}{R_{C}}$ and if
$I_{B} \gg \frac{I_{C MAX}}{B}$ then the transistor cannot amplify
$I_{B}$ and is forced out of the active region into the saturation
region.

Consider our basic circuit again, with $V_{I} = V_{CC}$.

\[ I_{B} = \frac{V_{CC} - V_{CE}}{R_{B}} \]

and

\[ I_{C} = I_{C MAX} = \frac{V_{CC} - V_{CE Sat}}{R_{C}} \approx
\frac{V_{CC}}{R_{C}} \]

\begin{table}[hbtp]

TODO - Circ

\end{table}

\end{document}
