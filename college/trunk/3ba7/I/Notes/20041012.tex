% $Id$

\documentclass[a4paper,12pt]{article}
\usepackage{amssymb}
\usepackage[all]{xypic}

\setlength{\parindent}{0mm}
\setlength{\parskip}{7.5mm}

\begin{document}

\title{Course 3BA7: Compiler Design I \\ Additional Lecture Notes \\ $12^{th}$ October 2004}

\maketitle

David Abrahamson

Semester 1 - Compiler Design

david@cs.tcd.ie

6081716

ORI F11

2 Exercises - $20\%$ of $\frac{1}{2}$ 3BA7

No requirement to pass.

1 or more exam questions on exam based upon coursework.


\section{Language Processors}

Piece of software which produces an output language from an input
language.


Interpreter - Source Language $\to$ Execution

Translater -  Source Language $\to$ Object Language

(i) Assemblers
	
	- Low Level Language
	  1 to 1 translations

(ii) Compiler
	
	- High Level Languages
	  1 to many translations

\begin{figure}[ht]

\vspace{10mm}

\xy <1cm,0cm>:
0 *=(3,2)\frm{-}  ="A" ,
(4,0) *=(3,2)\frm{-} ="B" ,
(8,0) *=(3,2)\frm{-} ="C" ,
(4,-3) *=(3,2)\frm{-} ="D" ,
\POS (0,2) *!DC={\txt{Source Code}} \ar ; "A" !UC ,
\POS "C" !UC \ar ; +(0,1) *!DC={\txt{Object Code}} ,
\POS "A" !CR \ar ; "B" !CL , 
\POS "B" !CR \ar ; "C" !CL , 
\POS "A" !DC *\dir{<} ; (0,-3) **\dir{-} ; "D" !CL **\dir{-} *\dir{>} , 
\POS "B" !DC \ar@{<->} ; "D" !UC , 
\POS "C" !DC *\dir{<} ; (8,-3) **\dir{-} ; "D" !CR **\dir{-} *\dir{>} , 
\POS "A" *--\txt{Lexical \\ Analyser} ,
\POS "B" *--\txt{Syntax \\ Analyser} ,
\POS "C" *--\txt{Code \\ Generator} ,
\POS "D" *--\txt{Symbol Table,\\ etc...} ,
\endxy

\caption{Simple model of compiler}

\end{figure}



\section{Lexical Analysis}

Splits input into a sequence of lexical tokens

(Class),(Class,Variable),(Class,Value)


eg

\begin{verbatim}

	if x > y then
		A := B + C * D;

\end{verbatim}

Lexical Element $\equiv$ Lexem

\verb!(IF) (IDENT,!$\uparrow$\verb!x) (RELOP,GT) (IDENT,!$\uparrow$\verb!y) (THEN) ...!

\section{Syntax Analysis}

Verifies that the sequence of lecical tokens is syntactically correct
and translates them into a sequence of atoms that more closely refect
the order of execution at runtime.


Input to syntax analyser:


\verb!(IDENT,!$\uparrow$\verb! A), (ASSIGN), (IDENT,!$\uparrow$\verb!B), (OP, PLUS),! \\
\verb!(IDENT,!$\uparrow$\verb!C), (OP, TIMES), (IDENT,!$\uparrow$\verb!D)!

\begin{verbatim}
A := B + C * D
\end{verbatim}

Output:

\verb!MULT(!$\uparrow$\verb!C,!$\uparrow$\verb!D,!$\uparrow TR_{1}$\verb!)! \\
\verb!ADD(!$\uparrow$\verb!B,!$\uparrow TR_{1}$, $\uparrow TR_{2}$\verb!)! \\
\verb!STORE(!$\uparrow$\verb!A,!$\uparrow$ $TR_{2}$\verb!)!


\end{document}
