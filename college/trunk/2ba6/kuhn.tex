\documentclass[a4paper,12pt]{article}

\begin{document}

\title{An Example of a Scientific Revoltion}

\author{Conall O'Brien \\ 01734351 \\ conall@conall.net}

\maketitle

\section*{Kuhn's Examples}

In his book "The Structure of Scientific Revolutions", Thomas Kuhn
discusses what he sees as a scientific revolution, he refers to as a
"paradigm". He coniders a paradigm to be a fundamentally important
theory for a particular aspect of science. Examples of paradigms he 
gives in his book include Aristotle's Physica, Ptolomy's Almagest, 
Isaac Newton's Principia and Opticks, Benjiman Franklin's Electricity, 
Antoine Lavoisier's Chemistry, Charles Lyell's Geology and Charles 
Darwin's Origin of Species.

\vspace{10mm}

\noindent All of these theories, with perhaps the exception of Newton's 
Principia, have proven themselves over time and are the cornerstones of
their respective fields in modern day science, combined with other 
theories proven that corroberate these paradigms. In the case of 
Newton's Principia however, despite hos theories being the basic for 
sending an to the moon, it is beleived Newton could not fully understand 
how gravity worked, or that it was the one and the same force that
controlled the position of the moon to the Earth, the tides as well as
his example of drawing small objects towards the Earth. However, started
by Albert Einstein and continued by people such as John Schwarz, Ed
Witten, Eva Silverstein, Juan Maldacena, Jim Gates, Sir Michael
Atiyah and Brian Greene, there is a new theory forming within the quantum
mechanics scietific desipline called "Super String Theory" or just
String Theory, which fundementally states that every atom is crated from
a "string" of energy vibrating at a particular frequecy, unique to the
atom. Perhaps this is an example of what Kuhn would call a paradigm
emerging in this modern age.

\end{document}
