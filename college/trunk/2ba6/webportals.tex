% $Id: webportals.tex,v 1.1 2004/03/03 02:40:02 conall Exp $

\documentclass[a4paper,10pt]{article}
\usepackage{relsize}
\usepackage{fancyhdr}
%\usepackage[left=2.5cm,top=2.5cm,bottom=2.5cm,right=2.5cm]{geometry}

\pagestyle{fancy}
\setlength{\parindent}{0mm}
\setlength{\parskip}{5.0mm}

\begin{document}

\title{2BA6 Computers And Society \\ Group - Monday \\ Web Portals}
%\\Personal Web Portals}

\author{Conall O'Brien \\ 01734351 \\ conall@conall.net}

\maketitle

\section*{Personal Web Portals}

For years, web portal content has been virtually uncustomisable, or
restricted to a set list of information sources. A good example of such
a web portal extensively used is 
MyNetscape\footnote{http://my.netscape.com/}. It was one of the first 
web portals, launched in January 1999, allowing users to customise 
it's content from a list of available sources. This is a personal 
web portal, with information specified and layed out by the user.


In recent years, emerging technologies such as the 
World Wide Web Consortium specified standard
RSS\footnote{http://www.w3.org/RDF/\#agg} (RDF Site Summary) and the
newly developing Atom\footnote{http://www.atomenabled.org/} standard
have changed this. These emerging technologies are
eXtensible Markup Language (XML) based text documents, easily
generated using a simple script and a database. Due to the simplicity of
these formats, combined with their sudden popularity on the internet
spawned from the blogging movement, anyone can create their own web
portal, armed with little more than some free web space, a parser to
syndicate \textit{the feed} and the desire to have a personal web
portal.


Numerous parsing tools in various scripting languages exist nowadays,
and composing one from scratch is a trivial task, to someone
familiar with script writing. There is also a large community online
with interests in the content syndication, such as
Syndic8\footnote{http://www.syndic8.com/} which includes a community
maintained database of news feeds from various news agencies around the
globe, catering for everyone's taste.

\end{document}
