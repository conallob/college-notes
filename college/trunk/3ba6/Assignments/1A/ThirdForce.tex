% $Id: Unix-Vs-Windows.tex 378 2004-11-27 18:11:22Z conall $

\documentclass[a4paper,12pt]{article}

\begin{document}

\title{Title}

\author{Paul Clancy \\ Vivian Hunt \\ Stephen Keely \\ Nina Mann \\ Conall O'Brien}

\maketitle


\pagebreak

\section*{Brief Overview of the E-learning Industry}

The UK is currently Europe's largest e-learning market, with an
estimated value approaching £200 million per annum and it is expected
to grow by approximately 40 per cent in each of the next two years
because of strong Government support for IT literacy training.	
By contrast the American market, the world's largest single
e-learning market, is estimated to be worth about \$$11.5$ billion, with
both Government and larger corporations spending heavily on IT training.		


Businesses need computer literate staff in many areas of operation and
e-learning and subsequent e-testing and certification enable this
education to be delivered in a cost effective and measurable way.
Target Market for the ECDL/ICDL Approved Single User CD Syllabus 4.0:
ThirdForce teaches and tests the individual it sells its products to
organisations, for example, governments who are increasing expenditure
into IT literacy. Although Dublin based the main market for ThirdForce
products is the UK. The UK is also Europe's largest e-learning
market. It is expected to grow by $40$ percent in each of the next two
years because of continuing Government support for IT literacy training.

ThirdForce has a strong track record with the UK Government's
University for Industry (UfI) and provides its software to customers 
through UfI Learndirect network of centres across the UK. They are also 
building upon the strategic relationships that Electric Paper has 
internationally in China with ICDL Asia Pacific, in the Middle East with 
UNESCO and in Chile with Fundacion Chile.


While ThirdForce teaches and tests the individual it sells its products
to organisations, for example, governments who are increasing
expenditure into IT literacy. Although Dublin based the main market for
ThirdForce products is the UK.

\section*{Main Team behind ThirdForce, their \\ Background and Skills and how they
relate to the Business Mission}

\subsection*{Patrick McDonagh: Non Executive Chairman}

Patrick is the founder of Riverdeep plc, a web-based educational
company, which was listed on NASDAQ in March 2000. In 1983, Patrick
founded the business now owned by SkillSoft plc (previously Smartforce
plc) which was listed on NASDAQ in 1995.

\subsection*{Brendan O'Sullivan: Chief Executive Officer}

Brendan was formerly a director of Education Europe at Apple Europe
(based in Paris). Prior to that he held positions in the UK as director
and general manager of Apple UK (based in London) and managing director
of Xemplar Education Limited (in Cambridge). Brendan is originally from
Dublin where he was managing director of Apple Computer Sales Limited
and has held senior management positions with Apple Ireland, Tomorrow's
World Group and Guinness.

\subsection*{Michael Newton: Non Executive Director}

Michael brings sales, services and operational management experience to
the Board. He worked for Apple UK Limited between 1989 and 1996. From
1991 to 1996 he held the positions of managing director and vice
president. Between 1996 and 1998, Michael worked with Dell Computer
Corporation Limited during which time he held the positions of managing
director and vice president.

\subsection*{Jonathan Parkes: Executive Director}

Jonathan has specific responsibility for Electric Paper, of which he is
currently managing director. Jonathan graduated from Trinity College
Dublin in 1991 with a degree in Micro Electronic Engineering. He has 12
years in the multimedia industry including internships with Apple
Computer in 1990 and Sony Corporation in 1991. Jonathan joined Electric
Paper in 1992 and was appointed managing director in 1999. He completed
an MBA with The Open University Business School in the UK in 2001.

\subsection*{Eimer McGovern: Chief Financial Officer}

Eimer is a Chartered Accountant who worked with PriceWaterhouseCoopers
from 1984 to 1989. She was appointed Finance Director of Electric Paper
in October 2001, having spent the previous ten years in various roles
with DCC plc and latterly as Finance Director of DCC's supply chain
management division servicing the IT industry.

\subsection*{Denis McMahon: Non-Executive Director}

Denis has over 30 years experience in the computer industry. In 1979 he
founded Online Computing and sold it 10 years later to Mentec
International. Denis is non-executive chairman of three other companies
in the hi-tech sector and focuses on business development. In 1996,
Denis was appointed non-executive chairman of Electric Paper.

\subsection*{Michael Costello: Non-Executive Director}

Michael is a Chartered Accountant who worked with Price Waterhouse from
1988 to 1993 before joining Costello Ryan, Chartered Accountants. He is
currently the Managing Partner of the firm, which now practices under
the name of Moore Stephens Costello McElroy Chartered Accountants.

\section*{ThirdForce Vision and Overall Business \\ Strategy}

ThirdForce aims to grow its business and become a leading international
e-learning company. The company intends to do this by both organic
growth and by acquisition. The beliefs that underpin this strategy are:

\begin{itemize}

\item While the ultimate client is the individual that sits down to
learn something, the most efficient and practical way to access large
numbers of individuals is through organisations

\item Organisations, whether government, educational or private will pay
more and for a longer period of time for the education of their people
or customers than private individuals will on their own

\item There needs to be a way of controlling the usage of the product so
that each individual user will generate revenue.	

\item Its products are designed for teaching large numbers of people but
it sells them to large organisations, often government sponsored, for
them to deploy rather than selling to the users itself.	

\item In most of the countries where ThirdForce operates there is
extensive government funding available to support IT training both in
schools and for the adult population.

\item It has a strong emphasis on developing products that are capable
of generating recurring revenues such as licence payments charged
against metered use, although one-off development contracts are
sometimes undertaken if they take the Group into worthwhile new areas or
can be executed very cost effectively by leveraging off the existing
expertise.

\item The Company works hard to retain the rights to new products, even
when they are developed to meet a specific customer's requirement.
This enables it to re-package the software it develops for use in
different markets. A library of e-learning modules is currently being
assembled, which should make it easier to use these modules in the
future.	
\item	Acquisitions are expected to supplement the organic growth
operations and the central overhead necessary to support this process is
already in place. We expect any new businesses to add complementary
products or bring access to new markets.
\item ThirdForce considers the American market particularly attractive
and ultimately a place where it must operate if it is to become a global
leader, although we believe that the Management will seek further
acquisitions in Europe to increase the Group's size before buying in
America. The American market is much larger than the UK although it is
still very fragmented and it is well known to the Group's Chairman.

\end{itemize}

ThirdForce's experience in and understanding of testing and
assessment, coupled with its testing products can provide organisations
with the measurement that's key to justifying the training
investment.

\section*{Operation and Organizational Challenges in Delivering the ECDL Product
to the Target Market}

Existing products need to be updated when the certifying bodies
introduce new training syllabi and when Microsoft changes the screen
appearance of its products, as recently happened on the introduction of
Windows XP and Office 2003.

\subsection*{Core Computing Product}

\subsubsection*{ECDL/ICDL Approved Single User CD Syllabus 4.0}

The European Computer Driving Licence (ECDL) is the world's leading
end-user computer skills certification programme. Outside of Europe, the
ECDL is known as the International Computer Driving Licence (ICDL).
It is internationally recognised as the global benchmark for end-user
computer skills and is the leading certification to be adopted by
governments, international organisations and corporations alike.

Our core product series is designed specifically for the global ECDL/ICDL 
certification programme. The product containing over 80 hours of
training materials with built-in exercises, consolidation tasks, a
simulation-based testing system for accredited test centres; a complete
solution for the ECDL / ICDL programme. Our training and testing
materials are approved by the ECDL Foundation for ECDL/ICDL Syllabus
4.0.

\subsubsection{Other IT Products that Support the ECDL Product}

http://isservices.tcd.ie/training/ecdl.php \\
Advanced Module 3 V1.0 \\
Advanced Module 4 V1.0 \\
Advanced Module 5 V1.0 \\
Advanced Module 6 V1.0 \\
CAD V1.0 \\

\section*{Advantage of this Product over the \\ Competition}

\begin{itemize}

\item Takes the strain out of teaching
\item Cost effective and efficient
\item Reduces the learning time
\item Motivates learners
\item Addresses different learning styles
\item Provides self-paced, flexible learning
\item Provides unrivalled levels of interaction
\item A simulated learning environment

\end{itemize}

Existing products need to be updated when the certifying bodies
introduce new training syllabi and when Microsoft changes the screen
appearance of its products, as recently happened on the introduction of
Windows XP and Office 2003.

Other development expenditure has focused on systems to make it easier
to reuse learning modules that have been developed for one application
as building blocks in the construction of other products, so that
offerings for different markets can be rapidly assembled. Work has also
been done to deliver training in different languages to cover the
world's major markets. ThirdForce has products available in Chinese,
Arabic and Spanish.
	
In addition, the Company has developed interfaces to allow some of
its products to be delivered on hand-held devices such as PDAs and
mobile phones and on interactive television.

\section*{How Will ThirdForce be able to maintain This Advantage}

Electric Paper was the first to have a product certified for use with
the new Syllabus 4 of the ECDL and many of its smaller competitors
have recently dropped out of the market having failed to meet the
cost of updating from Syllabus 3.

Our own teams of scriptwriters, subject matter experts, instructional
design professionals, software developers and quality assurance
people have developed all of our products. As a result we own the
Intellectual Property contained within our products.

ThirdForce will continue its mission of building a global e-learning
company through acquisition and organic growth. 

\section*{Potential Threats to ThirdForce}

ThridForce Competitors offer EDCL notes and classes in various
formats, such as LearnECDL's online courses, Blackrock Education
Centre's taught courses in a classroom and CIA Training's pre-made
paper printed notes.

\section*{Future Plans on How ThirdForce can adapt to Competition and Maintain
its Course Towards its Vision}

ThirdForce has successfully repositioned itself in FY 2003 and has
entered the e-learning market with excellent products, strong
management, talented people and a clear strategic goal. As technology
becomes more and more a part of people's daily lives around the
globe a growing demand from individuals who want to learn about
technology and through the medium of technology can be seen.

ThirdForce is expected to combine organic growth with acquisitions.
ThirdForce has stated that it will continue to search for suitable
acquisitions. With a well experienced management and board the
company is well positioned to take advantage of consolidation
opportunities that arise in the sector. The company suggests it has a
strong buy vs. build discipline and will most likely focus on
horizontal expansion through accredited complimentary learning
products and vertical expansion through industry specific learning
software that allows an entry point for ThirdForce's core
learning products across a variety of industries. The company will
also focus attentions toward the larger US market. 

As the offering moves away from products for which there is a formal
testing and recognition or a royalty fee per user (as in the case of
the UFI relationship), towards e-learning within corporate companies
and the public sector, the company needs to be able to offer tailored
products and become the primary e-learning provider to the
organisation with a product suite in a corporate specific wrapper.
This will allow it generate a premium on its offering, charge on a
basis which will approximate with the number of users and become
entrenched in the organisation, therefore staying ahead of the
competition.

\end{document}
