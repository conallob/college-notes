% $Id: 20041018.tex 338 2004-10-27 12:48:06Z conall $

\documentclass[a4paper,12pt]{article}
\usepackage{amssymb}
\usepackage[all]{xypic}

\setlength{\parindent}{0mm}
\setlength{\parskip}{7.5mm}

\begin{document}

\title{Course 3BA2: Artificial Intellegence \\ Additional Lecture Notes \\ $18^{th}$ October 2004}

\maketitle

\section*{Prolog Syntax}

Everything is a term. \\

Terms are simple or complex ("Structured").

\subsection*{Simple}

"atom" - stands for something. Eg: mike, chair \\

\subsubsection*{Syntax:} 

start with a lower case letter and followed by zero or more 
alphanumeric characters or underscores.


A special character followed by zero or more special characters: $ \&, \#. /. \*, \$ $


Any string of characters enclosed in single quotes. Eg: 'The w
cat', 'foo.txt', 'can''t do it'

\subsubsection*{Numbers:}

Integers \\

Floats (Implementation dependant)

\subsubsection{Variables:}

Anonymous Variable: $\_$ \\


\begin{verbatim}

member(x,[_|R]).

\end{verbatim}

\subsubsection*{Named Variable:}

Capitol Letter followed by zero or more aplhanumeric characters \\

or \\

An underscrore followed by one or more alphanumerics or underscores. \\


Scope of variable in a clause is the clause itself if named or no scope if anonymous.

\subsection*{Complex:}

has a functor, which has a name (some syntax rules as an atom) and has
an arity of $< 0$. \\

\begin{verbatim}

composs_points(north,south,east,west).

\end{verbatim}

north - Prolog term

\begin{verbatim}

tree(_,R).

temp(L1,R).

foo(X).

bar(X,R) : foo(X),foo(X).

: - Infix operator

\end{verbatim}

inter operator - binary - $x + 1$ \\

Postfix operator - $24$ \\

Prefix operator - ??? \\

Prefix operator - ??? \\

Prefix operator - ??? \\


\begin{verbatim}

:- (bar(X,Y),\left(foo(X),foo(Y)\right)

\end{verbatim}

\subsection{Lists in Prolog}

A list has 2 arguements, the 'head' and the 'tail'/'head' and 'body'/'car' and 'cdr' (LISP method) \\

\begin{verbatim}

.('head','body').

[a,b,c]

.(a,.(b,.(c,[]))).

[] % - Emplty list atom

\end{verbatim}

Prolog is a logic programming language.

Clauses are axioms.

When running a prolog program, you're asking for a proof that the query you make is true.

\begin{verbatim}

member(X,[X|_]).

member(X,[_|R]) :- member(X,R).

\end{verbatim}

eg: query:


\begin{verbatim}

member(a,[61,4,9,a,a(b),X]).

\end{verbatim}

Prolog makes Logical Interfaces to prove this goal.


In C:


\xy <1cm,0cm>:
(1,-10) *= (2,-6) !UL\txt{} *\frm{-} ,
0 ; (15,0) ; (15,-10) **\dir{-} ;
(0,-10) **\dir{-} ;
0 **\dir{-} ,
\endxy


In prolog:

\begin{verbatim}

member(a,[61,a]).
   
member(x,[X|_]).

\end{verbatim}

attempt to unify each arguement with the goal with tit's corresponding arguement in the head.

	- if the arguements are identical
	
	- if one is an unbound variable and the other is not, then the unbound variable is 
	- made equal to the other term (it is instantiated). The two terms are then identical
	  In fact, they are the same term from then on.

	- If both are unbound variables, they are unified such that they are identical there after,
	  the same unbound variable.

	- Otherwise, unification fails and resolution fails.

\end{document}
