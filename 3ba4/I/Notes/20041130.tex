% $Id: 20041108.tex 353 2004-11-12 00:21:25Z conall $

\documentclass[a4paper,12pt]{article}
\usepackage{amssymb}
\usepackage{ulem}

\setlength{\parindent}{0mm}
\setlength{\parskip}{7.5mm}

\begin{document}

\title{Course 3BA4: Computer Architecture II \\ Lecture Notes \\ $30^{th}$ November 2004}

\maketitle

\section*{Cache Organisation}

Organisation of a cahce can be described as:

$L$ bytes per cache line (or block size).

$K$ cache lines per set (degree of associativity of $K$ way)

$N$ number of sets

Cache size $= L \times K \times N$

\subsection*{$N = 1$}

Full associative Cache - address tag composed with \textbf{all} cache
tags.

Address can map to any of the $K$ cache lines.

\subsection*{$K = 1$}

Direct Mapped Cache - address tag compared with only one cache tag.

Address can be mapped \textbf{only} onto a single cache.

\subsection*{$N > 1$, $K > 1$}

Set Associativity


\section*{Write Through Vs Write Back}

\subsection*{Write Through}

\subsubsection*{Write Hit}

Write Hit - Update main memory only.

\subsubsection*{Write Miss}

Update main memory only 

or

Allocate cache line, read cache line from memory, then write to cache
line and main memory to cache line [ write allocate cache]

Must read because cache line is $64$ bit and the memory address can only
be $16$ bit.

\subsection*{Write Back}

\subsubsection*{Write Hit}

Update cache line \textbf{only}, update main meory when cache line is
flushed or replaced.

\subsubsection*{Write Miss}

Allocate cache line, read cache line from memory, then write to cache
line.

Must write back previous data if cache line dirty (modified)

\section*{Direct Mapped Vs Associative Caches}

Will an associative cache always outperform a direct mapped cache
of the same size?? [$L \times K \times N $ equal]

Consider 2 caches:

$K = 4$, $N = 1$, $L = 16$ (fully associative)

$K = 1$, $N = 4$, $L = 16$ (directly mmapped)

and a repeating sequence of 5 addresses:

$a, a + 16, a + 32, a + 48, a + 64, a, a + 16, a + 32, \cdots$

\subsection*{Fully Associative Tags}

\begin{tabular}{r|c|c|c|c|}
\hline
set0&\sout{$a$}, $a+64$, $a+48$&\sout{$a+16$}, $a$&\sout{$a+32$},
$a+16$&$a+48$, $a+32$\\
\hline
\end{tabular}

\subsection*{Direct Mapped Tags}

\begin{tabular}{r|c|l}
\hline
set0	&	\sout{$a$}, $a+64$, $a$, $\cdots$	&	\\	
\hline
set1	&	$a+16$									&	\\	
\hline
set2	&	$a+32$									&	\\	
\hline
set3	&	$a+48$									&	\\	
\hline
\end{tabular}

Fully Associative: only 4 address can \emph{fit} in the 4 way cache so
that due to the CRU replacement policy, every access will be a miss.

Direct Mapped: since \textbf{only} address $a$ and $a + 64$ map to the
same set, they will conflict with each other, resulting in 2 misses and
3 hits per cycle.

The Cs mean that the conflict misses can be negative.

\end{document}
