\documentclass[a4paper,12pt]{article}
\usepackage{amssymb}

\begin{document}

\title{Course 2BA1: Michaelmas Term 2003 \\ Assignment II}

\author{Conall O'Brien \\ \\ 01734351 \\ \\ conall@conall.net}

\maketitle

\section{}

\subsection*{Question}

Prove that

\[ A \setminus (B \setminus C) = (A \setminus B) \cup (A \cap C) \]

\subsection*{Proof}

Let $D = A \setminus (B \setminus C)$ and $E = (A \setminus B) \cup (A
\cap C)$

\vspace{10mm}

\noindent If $x \in D$, then $x \in A$ and $x \in B \setminus C$, since
$x \in D$, $x \not\in B$ and $x \not\in C$. $x \in A \setminus B$, since 
$x \in A$ and $x \not\in B$. $x \in A \cap C$ and since $x \in A$ and 
$x \not\in C$. $x \in (A \setminus B) \cup (A \cap C)$, ie $x \in E$.

\vspace{10mm}

\noindent If $x \in E$. Therefore $x \in A \setminus B$ or $x \in A \cap
C$. If $x \in A \setminus B$, $x \in A$ and $x \not\in B$. If $x \in A $
or $x \in C$, $x \not\in B$. Therefore $x \in A \setminus (B \setminus
C)$, since $x \in A$. Therefore $x \in D$.

\vspace{10mm}

\noindent $ D \subset E$ and $E \subset D$.

\noindent Therefore $D = E$ and so $A \setminus (B \setminus C) = (A
\setminus B) \cup (A \cap C)$.

\section{}

\subsection*{(i)}

\noindent $xPy$ for all $x$, $y \in \mathbb{R}$ when $y = xk^{2}$. $k \in \mathbb{Z}$.

\subsubsection*{Reflexive?}

\noindent Therefore $xPx$ is true if and only if $x = xk^{2}$.

\vspace{10mm}

 \[1 = k^{2} \]

\[ 1 = k \]

\[ 1 \in \mathbb{Z} \]

\vspace{10mm}

\noindent Therefore $P$ is reflexive.

\subsubsection*{Symmetric?}

\noindent $xPy = yPx$ if and only if $y = xk^{2}$ and $x = yl^{2}$.
$xRy = yPx$ when $k$, $l \in \mathbb{Z}$.

\[ y = (yl^{2})k^{2} \]

\[ y = y l^{2} k^{2} \]

\[ 1 = l^{2} k^{2} \]

\[ 1 = k l \]

\noindent if $l = 3$ and $k = \frac{1}{3}$, then $1 = 1 \times
\frac{1}{3}$. However when $k = \frac{1}{3}$, $k \not\in \mathbb{Z}$.

\noindent Therefore $P$ is not symmetric.

\subsubsection*{Transitive?}

$xPy$ and $yPz$ are true, so $y = xk^{2}$ and $z = yl^{2}$. Therefore
$xPz$ is true, if $z = xj^{2}$ is also true.

\[ (yl^{2}) = xj^{2} \]

\[ k^{2}l^{2} = j^{2} \]

\[ kl = j \]

\noindent Therefore $j$, $k$, $l \in \mathbb{Z}$. Thus $P$ is transitive.

\subsubsection*{Anti-Symmetric?}

$x = $ if $xPy$ and $yPx$ are true, if and only if $y = x k^{2}$ and $x
= y l^{2}$.

\[ y = (y l^{2}) k^{2} \]

\[ y = y l^{2} k^{2} \]

\[ 1 = l^{2} k^{2} \]

\[ 1 = l k \]

\noindent If $k = 2$ and $l = \frac{1}{2}$, then $1 = 1$. However, when
$l = \frac{1}{2}$, $k \not\in \mathbb{Z}$. Hence $P$ is not
anti-symmetric.

\subsubsection*{Conclusion}

\noindent $P$ is neither an equivalence relation, nor a partial order, since it is
neither symmetric nor anti-symmetric.

\subsection*{(ii)}

$xQy$ if $x$, $y \in \mathbb{R}$ if and only if $y^{3} = x^{3} - x + y$.

\subsubsection*{Reflexive?}

$xQx$ is true if and only if $x^{3} = x^{3} - x + x$.

\[ x^{3} = x^{3} \]

\noindent Therefore, $Q$ is reflexive.

\subsubsection*{Symmetric?}

$xQy = yQx$ if and only if $y^{3} = x^{3} - x + y$ and $x^{3} = y^{3} -
y + x$.

\[ y^{3} = (y^{3} - y + x) - x + y \]

\[ y^{3} = y^{3} - y + x - x + y \]

\[ y^{3} = y^{3} \]

\noindent Therefore $Q$ is symmetric.

\subsubsection*{Transitive?}

\noindent If $xQy$ and $yQz$ are true, then $y^{3} = x^{3} - x + y$ and $z^{3} =
y^{3} - y + z$ are also true. Therefore is $z^{3} = x^{3} - x + z$ true,
hence $xQz$ true as well?

\[ (y^{3} - y + z) = x^{3} - x + z \]

\[ (x^{3} - x + y) - y + z = x^{3} - x + z \]

\[ x^{3} - x + z = x^{3} - x + z \]

\noindent Therefore $Q$ is transitive.

\subsubsection*{Anti-Symmetric?}

\noindent If $xQy$ and $yQx$ are true, and hence $y^{3} = x^{3} - x + y$ and
$x^{3} = y^{3} - y + x$, is $x = y$ true?

\[ y^{3} = (y^{3} - y + x) - x + y \]

\[ y^{3} = y^{3} - y + x - x + y \]

\[ y^{3} = y^{3} \]

\noindent Therefore, if $y = y$, $y \neq x$, thus $Q$ is not anti-symmetric.

\subsubsection*{Conclusion}

\noindent $Q$ is a not a partial order since it is not anti-symmetric. It is not an 
equivalence relation however, since it is reflexive, symmetric and
transitive.

\section{}

\subsection*{(i)}

$ f : [-1, 1] \to [-2, 2] $ for $f(x) = x^{3} + x$, $x \in [-1, 1]$.

\[ f'(x) = 3x^{2} + 1 = 0 \]

\noindent Therefore $f(x)$ is a strictly increasing function.

\[ f(-1) = (-1)^{3} + (-1) \]

\[ f(-1) = -2 \]

\[ f(1) = (1)^{3} + (1) = 2 \]

\[ f(1) = 2 \]

\[ f(-1) \neq f(1) \]

\noindent Therefore $f$ is injective.

\[ f(-1) = -2 \]

\[ f(1) = 2 \]

\noindent Therefore $f$ is surjective and hence $f$ is bijective, since it
is injective and surjective. Hence it is invertible.

\subsection*{(ii)}

\noindent $g : (-1, 1) \to \mathbb{R}$ with $g(x) = \frac{1}{1 - x^{2}}$ for all
values of $x \in (-1, 1)$.

\[ g(0) = \frac{1}{1 - (0)^{2}} = 1 \]

\[ g(0.5) = \frac{1}{1 - 0.25} = \frac{1}{0.75} = \frac{4}{3} \]

\[ g(-0.5) = \frac{1}{1 - 0.25} = \frac{1}{0.75} = \frac{4}{3} \]

\noindent Therefore $g$ is not injective since $0.5 \neq -0.5$.

\vspace{10mm}

\noindent However, $g(0.5) = \frac{4}{3}$, $g(-0.5) = \frac{4}{3}$. $\frac{4}{3}
\in \mathbb{R}$. Therefore $g$ is surjective.

\noindent Therefore $g$ is not bijective, because it is not injective.
Hence it is not invertible.


\end{document}
