\documentclass[a4paper,12pt]{article}
\usepackage{amssymb}

\begin{document}

\title{Course 2BA1: Michaelmas Term 2003 \\ Assignment I}

\author{Conall O'Brien \\ \\ 01734351 \\ \\ conall@conall.net}

\maketitle

\section{}

Prove by induction that

\[ \sum^{n}_{i = 1} \frac{2i + 1}{i^{2}(i + 1)^{2}} = 
\frac{n^{2} + 2n}{(n + 1)^{2}} \]

\subsection{When $n = 1$}

\begin{eqnarray*}
\sum^{1}_{i = 1} \frac{2(1) + 1}{(1)^{2}(1 + 1)^{2}} 
& = &\frac{(1)^{2} + 2(1)}{((1) + 1)^{2}} \\
\frac{2 + 1}{(1)^{2}(2)^{2}} 
& = & \frac{(1)^{2} + 2(1)}{(1 + 1)^{2}} \\
\frac{3}{1(2)^{2}} 
& = & \frac{1^{2} + 2}{2^{2}} \\
\frac{3}{4} 
& = & \frac{3}{4}
\end{eqnarray*}

\noindent Therefore, holds true for $n = 1$.

\subsection{When $n = m$.}

Assume equation holds true for $n = m$, so:

\[ \sum^{m}_{i = 1} \frac{2i + 1}{i^{2}(i + 1)^{2}} = 
\frac{m^{2} + 2m}{(m + 1)^{2}} \]

\subsection{When $n = m + 1$.}

\begin{eqnarray*}
\sum^{m + 1}_{i = 1} \frac{2i + 1}{i^{2}(i + 1)^{2}} \\
\sum^{m}_{i = 1} \frac{2i + 1}{i^{2}(i + 1)^{2}} + 
\frac{2(m + 1) + 1}{(m + 1)^{2}((m + 1) + 1)^{2}} 
& = & \frac{(m + 1)^{2} + 2(m + 1)}{((m+ 1) + 1)^{2}} \\
\frac{m^{2} + 2m}{(m + 1)^{2}} + 
\frac{2(m + 1) + 1}{(m + 1)^{2}((m + 1) + 1)^{2}}
& = & \frac{(m + 1)^{2} + 2(m + 1)}{((m+ 1) + 1)^{2}} \\
\frac{m^{2} + 2m}{(m + 1)^{2}} +
\frac{2(m + 1) + 1}{(m + 1)^{2}(m + 2)^{2}}
& = & \frac{(m + 1)^{2} + 2(m + 1)}{(m + 2)^{2}} \\
\frac{m(m + 2)(m + 2)^{2} + 2(m + 1) + 1}{(m + 1)^{2}(m + 2)^{2}}
& = & \frac{(m + 1)^{2} + 2(m + 1)}{(m + 2)^{2}} \\
\frac{m(m + 2)^{3} + 2(m + 1) + 1}{(m + 1)^{2}}
& = & (m + 1)^{2} + 2(m + 1) \\
m(m + 2)^{3} + 2(m + 1) + 1
& = & (m + 1 + 2)(m + 1)^{3} \\
m(m + 2)^{3} + 2(m + 1) + 1
& = & (m + 3)(m + 1)^{3} \\
m(m^{3} + 6m^{2} + 12m + 8) + 2(m + 1) + 1
& = & (m + 3)(m^{3} + 3m^{2} + 3m + 1) \\
m^{4} + 6m^{3} + 12m^{2} + 10m + 3
& = & m^{4} + 3m^{3} + 3m^{2} + m + 3m^{3} + 9m^{2} + 9m + 3 \\
m^{4} + 6m^{3} + 12m^{2} + 10m + 3
& = & m^{4} + 6m^{3} + 12m^{2} + 10m + 3
\end{eqnarray*}

\noindent Therefore true for $n = m + 1$.

\subsection{Conclusion}

Since $n = 1$ and $n = m + 1$ both hold true, thus it is true $\forall
\in \mathbb{N}$.

\pagebreak

\section{}

Prove by induction on $n$ that $(3n)! \geq \frac{1}{20} \times 120^{n}$
for all natural numbers $\mathbb{N}$ (where $n!$ denotes the product of
all natural numbers from $1$ to $n$ inclusive).

\subsection{When $n = 1$}

\begin{eqnarray*}
(3(1))! & \geq & \frac{1}{20} \times 120^{1} \\
3! & \geq & \frac{1}{20} \times 120 \\
6 & \geq & \frac{120}{20} \\
6 & \geq & 6
\end{eqnarray*}

\noindent Therefore, holds true for $n = 1$.

\subsection{When $n = m$}

Assume equation holds true for $n = m$, so:

\begin{eqnarray*}
(3m)! & \geq & \frac{1}{20} \times 120^{m}
\end{eqnarray*}

\subsection{When $n = m + 1$}

\begin{eqnarray*}
(3(m + 1))! & \geq & \frac{1}{20} \times 120^{m + 1} \\
(3m + 3)! & \geq & \frac{1}{20} \times 120^{m + 1} \\
(3m + 3) \times (3m + 2) \times (3m + 1) \times 3m! 
& \geq & \frac{1}{20} \times 120^{m + 1} \\
(3m + 3) \times (3m + 2) \times (3m + 1) \times 3m!
& \geq & \frac{1}{20} \times 120 \times 120^{m} \\
(3m + 3) \times (3m + 2) \times (3m + 1) & \geq & 120
\end{eqnarray*}

\noindent Therfore it holds true for $n = m + 1$.

\subsection{Conclusion}

Since it holds true for $n = 1 $ and $n = m + 1$, therfore it holds true
$\forall \in \mathbb{N}$.

\end{document}
