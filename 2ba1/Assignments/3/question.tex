\documentclass[a4paper,12pt]{article}

\usepackage{amsfonts}
\usepackage{epsf}

\newcommand{\IN}{\mathbb{N}}
\newcommand{\IZ}{\mathbb{Z}}
\newcommand{\IQ}{\mathbb{Q}}
\newcommand{\IR}{\mathbb{R}}
\newcommand{\IC}{\mathbb{C}}
\newcommand{\IF}{\mathbb{F}}
\newcommand{\IK}{\mathbb{K}}
\newcommand{\qed}{\nobreak \ifvmode \relax \else
      \ifdim\lastskip<1.5em \hskip-\lastskip
            \hskip1.5em plus0em minus0.5em \fi \nobreak
            \vrule height0.75em width0.5em depth0.25em\fi}

\newtheorem{theorem}{Theorem}[section]
\newtheorem{lemma}[theorem]{Lemma}
\newtheorem{proposition}[theorem]{Proposition}
\newtheorem{corollary}[theorem]{Corollary}

\newenvironment{proof}[1][Proof]{\begin{trivlist}
\item[\hskip \labelsep {\bfseries #1}]}{\end{trivlist}}
\newenvironment{altproof}[1][Alternative Proof]{\begin{trivlist}
\item[\hskip \labelsep {\bfseries #1}]}{\end{trivlist}}
\newenvironment{definition}[1][Definition]{\begin{trivlist}
\item[\hskip \labelsep {\bfseries #1}]}{\end{trivlist}}
\newenvironment{example}[1][Example]{\begin{trivlist}
\item[\hskip \labelsep {\bfseries #1}]}{\end{trivlist}}
\newenvironment{remark}[1][Remark]{\begin{trivlist}
\item[\hskip \labelsep {\bfseries #1}]}{\end{trivlist}}

\newcommand{\romanlistlabel}[1]{\hfill\normalfont\mdseries (#1)}
\newenvironment{romanlist}
   {\begin{list}{}{\setlength{\labelsep}{0.5em}
   \setlength{\labelwidth}{\leftmargin}
   \addtolength{\labelwidth}{-\labelsep}
   \setlength{\itemindent}{0pt}
   \let\makelabel\romanlistlabel}}{\end{list}}

\newcounter{question}
\newenvironment{question}{\refstepcounter{question}%
   \begin{list}{\arabic{question}.}{\setlength{\itemindent}{0pt}}\item}%
   {\end{list}}
\newenvironment{qpart}[1]{\begin{list}{(#1)}{\setlength{\labelwidth}{0pt}
   \setlength{\labelsep}{0.5em}
   \setlength{\itemindent}{\labelsep}
   \setlength{\leftmargin}{0pt}}\item}{\end{list}}

\newcommand{\half}{{\textstyle\frac{1}{2}}}
\newcommand{\mapright}[1]{{\buildrel #1\over \longrightarrow}}
\newcommand{\mapdown}[1]{\Big\downarrow
  \rlap{$\vcenter{\hbox{$\scriptstyle#1$}}$}}
\newcommand{\mapsoutheast}[1]{\searrow
  \rlap{$\vcenter{\hbox{$\scriptstyle#1$}}$}}

\begin{document}

\begin{center}
\Large\bfseries
Course 2BA1: Hilary Term 2004.\\[6pt]
Assignment III.\\[12pt]
\normalsize To be handed in by Wednesday 7th April, 2004.\\
Please include both name and student number on any work
handed in.
\end{center}

\begin{question}
Let $X$ denote the set
\[ \{ \ldots, -5, -3, -1, 1, 3, 5,\ldots \} \]
of odd integers, and let $\#$ denote the binary operation
on $X$ defined by $x \mathbin{\#} y = \frac{1}{2} (xy - x - y + 3)$ for
all odd integers $x$ and $y$.

\begin{qpart}{a}
Is $(X,\#)$ a monoid?  If so, what is its identity element?
\end{qpart}

\begin{qpart}{a}
Is $(X,\#)$ a group?
\end{qpart}

[Briefly justify your answers.]
\end{question}

\begin{question}
Let $q$ and $r$ be the quaternions given by
$q = 1 - i - j$ and $r = 2i + j - k$.  Calculate the quaternion
products $q \times r$ and $r \times q$ (expressing $q \times r$
and $r \times q$ in the form $w + x i + y j + z k$ for appropriate
real numbers $w$, $x$, $y$ and $z$).
\end{question}

\begin{question}
Devise a context-free grammar to generate the language over the
alphabet $\{ 0, 1 \}$ consisting of the strings
\[ 01,\enspace 0011,\enspace 000111,\enspace 00001111,\ldots \]
(i.e., consisting of $m$ zeros, for some non-negative integer~$m$,
followed by $m$ ones).  You should specify the nonterminals of the
grammar, the start symbol and the productions of the grammar.
\end{question}

\begin{question}
\begin{qpart}{a}
Devise a regular grammar to generate the language over the
alphabet $\{ a, (, ), 0, 1 \}$ consisting of all strings
such as \texttt{a(001)} and \texttt{a(1001010)} in which the
initial substring \texttt{a(} is followed by a non-empty string
of binary digits, which is followed by the character \texttt{)}.
\end{qpart}

\begin{qpart}{b}
Devise a finite state acceptor that accepts (i.e., determines)
the language described in (\textit{a}).  You should specify the
states of the machine, the start state, the finishing state(s),
and the transition table that defines the machine.
\end{qpart}
\end{question}

\end{document}
