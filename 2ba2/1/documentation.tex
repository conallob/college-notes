\documentclass[a4paper,12pt]{article}
\usepackage{amssymb}

\begin{document}

\title{2BA2: \\ Programming Techniques \\ Assignment I \\ Mastermind}

\author{Conall O'Brien \\ \\ 01734351 \\ \\ conall@conall.net}

\maketitle

\pagebreak

\section{Basic Design}

The logic behind the game is rather simple. After initialising the game
and generating the code, a simple loop iterates ten times, each time
asking the user for input. Whil in the loop, the input is then checked
for validity first, then two functions are used to run the necessary
checks for exactly correct digits (bulls) and correct digits in wrong
positions (cows).

\section{The Loop}

Basic error checking is done within the loop in regards to the user
input. It is compared to the upper and lower bounds ($1111$ and $6666$
respectively), before being passed into the comparison functions.

\pagebreak

\section{The Functions}

\subsection{generate\_code : INTEGER}

This function simply makes an object of type \emph{STD\_RAND} which is
used to generate a random number, using a provided seed which is
generated by the \emph{get\_seconds\_from\_epoch}. In order to ensure
the generated number contains only valid digits, SmartEiffel API for the
\emph{STD\_RAND} class is used to generate each digit with an upper
limit of $6$. Then the generated digit is added to the previously
generated number, after the previous digits are shifted one digit left
by multiplying by $10$. Finally, the valid four digit number is returned
to the main program.

\subsection{compare\_guess\_for\_exact\_matches(guess : INTEGER; code :
INTEGER) : INTEGER}

This function compares the user's input to the code as integers, looking
for an exact match. If an exact match is not found, it converts both
numbers into variables of type \emph{STRING}. Then each character of the
code is compared to each character of the guess from the users. If a
match occurs, a counter is incremented. Upon completion, the counter of
matches is returned, after a basic post condition that the number
returned is between $0$ and $4$.

\subsection{compare\_guess\_for\_wrong\_places(guess : INTEGER; code :
INTEGER) : INTEGER}

This function compares the the number of occurrences of each digit
contained in the user's input to the generated code. The total number of
occurrences is then error checked to enure it is between $0$ and $4$ as
it is returned by the function.

\subsection{is\_valid\_input : BOOLEAN}

This function is used as a condition to ensure the user input is valid
by only containing the digits $1$, $2$, $3$, $4$, $5$ and $6$ and that
the input is between the lower and upper bounds ($1111$ and $6666$
respectively). If an invalid digit is found, the function returns false,
else the entered input is valid and may be then tested.

\subsection{get\_seconds\_from\_epoch : INTEGER}

This function calls SmartEiffel specific API that accesses the system
clock. Once syncronised with the system clock, it returns the years,
months, days, hours, minutes and seconds since UNIX Epoch (00:00:00 1st
January 1970). After converting it to seconds via arithmetic and
checking that the number is greater than $0$, it is returned.

\pagebreak

\section{Source Code}

\pagebreak

\section{Sample Output}

\end{document}
