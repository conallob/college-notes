% $Id: 20041109.tex 367 2004-11-23 10:22:38Z conall $

\documentclass[a4paper,12pt]{article}

\usepackage{graphicx}

\setlength{\parindent}{0mm}
\setlength{\parskip}{7.5mm}

\begin{document}

\title{Course 3BA3: Communications \\ \vspace{10mm} Assignment 2 \\
\vspace{10mm} Is Mobile Communication the Future? \\ \vspace{10mm}
{\small Do you think the proliferation of mobile voice and data communication
technologies (Cellular Mobile Phones, WiFi, WiMax, Bluetooth, GPRS, etc)
coupled with Voice over IP (VoIP) technologies (Skype, etc) will make
traditional fixed telephony a thing of the past?}}

\author{Conall O'Brien \\ conallob@maths.tcd.ie \\ 01734351}

\maketitle

\section{Introduction}

In this paper, I will briefly discuss issues relating to the use of
\footnote{Voice over IP}{VoIP}, in particular with wireless 
\footnote{Local Area Networks}{LANs} and 
\footnote{Wide Area Networks}{WANs}. I will look into advantages
and disadvantages of VoIP from the view of the telecommunications 
industry. Finally, I will conclude whether or not utilising the
potential of VoIP is a danger to the traditional fixed line telephony
business.

\section{Bandwidth}

The many members of the 
\cite[traditional telecommunications industry]{wikipedia-isp}
became
\footnote{Internet Service Providers}{ISPs}
during the 1980s and 1990s when the Internet experienced huge growth,
particularly in home and business users as well as it's existing 
academic base. \cite[In the 1970s]{wikipedia-pstn}, the 
telecommunications industry begun researching replacing analogue
technologies with high capacity digital alternative, giving birth to
technologies such as 
\cite[\footnote{Digital Line Subscriber}{DSL} and 
\footnote{Integrated Systems Digital Network}{ISDN}]{wikipedia-pstn},
both of which have proven useful in the area of data transmission since
such products have been made available to the end users.


At the same time, ISPs have invested large amounts of money into high
bandwidth networks for data transmission. One key feature of VoIP in the
eyes on the telecommunications industry is the ability to connect a
large number of calls due to the ready access to high bandwidth. It is
quite likely their \footnote{Internet Protocol} datagram networks could
route a higher capacity of voice communications using VoIP than their
dedicated hardware \footnote{Public Switched Telephone Network}{PSTN},
which could offer in the long term, lower running costs.


Since the wide availability of high speed broadband internet access to
many residential, business and mobile users, using various mediums such as
\footnote{Digital Subscription Line}{DSL}, cable and 
\footnote{General Packet Radio Service} and 
\footnote{3G}{Third Generation} mobile phone
communications, the primary use for such bandwidth currently is 
multimedia content. DSL and cable access to the internet is frequently
used to gain access to high quality audio and video content while GPRS
and 3G connections from mobile phones are used for accessing medium
quality audio and video content.

\section{Voice over IP}

VoIP protocols are relatively recent examples of application layer
protocols designed to provide rich content to broadband users. There are
6 main VoIP options, 
\cite[the \footnote{Internet Engineering Task Force}{IETF}'s 
\footnote{Session Initiation Protocol}{SIP},
the \footnote{International Telecommunications Union}{ITU}'s H.323,
Cisco's Skinny Client Control Protocol, The IETF and ITU's 
Megaco (aka H.348), MiNET, Skype and
\footnote{Inter-Asterisk eXchange}{IAX} ]{wikipedia-voip}.


Unfortunately, each protocol has it's own advantages and disadvantages,
thus there is no single popular choice. H.323, SIP and Skype are all
popular protocols, although they all are known to have issues 
traversing firewall and \footnote{Network Address Translation}{NAT} 
setups, particularly H.323. Cisco's Skinny Client Control Protocol is
popular with business, where dedicated hardware VoIP
telephones and are often used in large 
\footnote{Wide Area Networks}{WANs}. Other issues for all VoIP protocols inherent to IP datagram networking 
include \footnote{Quality of Service}{QoS}, latency and data integrity.

\section{Wireless Security}

Cognitively, one implicit, basic security feature of the 802.3 Ethernet
protocol is it's
reliance on copper cables, which  can be given restricted network access
or physically restricted. The same cannot be said for the 802.11 WiFi
protocol. By it's very design, a malicious user has access to the
network, so other security methods are required.


One the last few years, wireless \footnote{Local Area Network}{LAN}
security recommendations have changed more than once. The 
\cite[802.11 standards define specifications for built in encryption,
entitled
\footnote{Wired Equivalent Privacy}{WEP}]{wikipedia-wep}. However,
concerns over the integrity of WEP, particularly events in \cite[2001,
including the University of California at Berkeley, a paper entitled
\quote{Weaknesses in the Key Scheduling Algorithm of RC4} by Fluhrer, 
Mantin, and Shamir and AT\&T]{wikipedia-80211} each highlighted various
issues with the WEP security model and choice of cipher algorithm.


At the time of writing this paper, the current de facto methods of
securing a wireless lan involve the use of \footnote{WiFi Protected
Access}{WPA} (which was created the result of WPA's shortcomings),
802.1x user authentication and the use of a \footnote{Virtual Private
Network}{VPN} to encrypt all traffic using high security ciphers.

\section{Convergence}

Convergence between VoIP protocols and high speed broadband and
wireless networking has already begun. 
\footnote{Such as the ZyXEL Prestige 2000W -
http://www.zyxel.com/product/P2000W.php}{Products}
offering VoIP services over \footnote{Wireless Fidelity}{WiFi}
are becoming available.


Telecommunication providers have already begun to use VoIP to transport
calls between switched routing stations to lower costs, while improving
throughput.

\section{Conclusions}

VoIP protocols, combined with broadband internet access and wireless
networking has great potential, however there a still a few issues
preventing it's widespread adoption. First off, there are too many
competing protocols, each of which attempt to address some of the
issues VoIP face concerning firewalls and QoS.


Another key show-stopper is the conversion from VoIP calls to the PSTN.
There are small providers who offer conversion from one to the other,
but large telecommunications companies have the resources and hardware
to supply this service much cheaper than existing operators. However,
this is unlikely to happen soon, since it is in the interest of the
telecommunications industry that cheap VoIP based calls are not freely
available, in an effort to keep prices high.


To conclude, the proliferation of VoIP amongst end users will will
continue, but it will slow down in the coming future, until the 
telecommunications industry decide to offer the end user VoIP 
services. This trend will probably be comparable to the uptake of
internet access in the 1990s which started as being expensive and
lacking clear standards before telecommunication companies started to
offer digital services such as DSL and ISDN to end users. 

\section{Bibliography}

\bibliographystyle{ieeetr}

\bibliography{WiFi-VoIP-Essay}

\end{document}
