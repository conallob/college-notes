% $Id$

\documentclass[a4paper,12pt]{article}
\usepackage{amssymb}

\setlength{\parindent}{0mm}
\setlength{\parskip}{7.5mm}

\begin{document}

\title{Course 3BA1 - Part I -  Statistics \\ Assignment 1 \\ $24^{th}$ January 2005}

\author{Conall O'Brien \\ conallob@maths.tcd.ie \\ 01734351}

\maketitle

\section{}

\subsection*{(a)}

$P(R) = \frac{0.51}{1} = 0.51$ (Calc)

$P(E) = \frac{0.59}{1} = 0.59$ (Calc)

$P(R \cap E) = \frac{0.28}{1} = 0.28$ (Calc)

$\therefore R$ and $E$ are independant.

$P(S|E) = \frac{P(S \cap E)}{P(E)}$

$P(S \cap E) = \frac{0.23}{1}$ (Calc)

$P(S|E) = \frac{0.23}{0.59} = 0.39$ (Calc)

$P(S|R) = \frac{P(S \cap R)}{P(R)}$ (Calc)

$P(S \cap R) = \frac{0.23}{1} = 0.23$ (Calc)

$P(S|R) = \frac{0.23}{0.51} = 0.45$ (Calc)

$P(E|R) = \frac{P(E \cap R)}{P(R)}$

$P(E|R) = \frac{0.28}{0.51} = 0.55$ (Calc)

$P(E|\overline{R}) = \frac{P(E \cap \overline{R})}{P(\overline{R})}$

$P(E|\overline{R}) = \frac{0.31}{0.49} = 0.63$ (Calc)

\subsection*{(b)}

$P(\overline{S}) \times P(\overline{S}) \times P(\overline{S})$

$\Rightarrow P(\overline{S}) = \frac{0.56}{1} = 0.56$ (Calc)

$\Rightarrow 0.56 \times 0.56 \times 0.56 = 0.18$ (Calc)

$P(\overline{S}) \times P(\overline{S}) \times P(\overline{S} \cap R)$

$\Rightarrow P(\overline{S} \cap R) =  0.56$ (Calc)

$\Rightarrow 0.56 \times 0.56 \times 0.28 = 0.08$ (Calc)

\section{}

$P(\overline{V}|D) = 0.9$

$P(\overline{V}|\overline{D}) = 0.1$

$P(\overline{R}|D) = 0.9$

$P(\overline{R}|\overline{D}) = 0.03$

$P(D) = 0.04$

$\Rightarrow P(\overline{V}|D) = \frac{P(\overline{V} \cap D)}{P(D)} = 0.9$

$\Rightarrow P(\overline{V} \cap D) = 0.9 \times 0.04$

$\Rightarrow P(\overline{V}) - P(D) = 0.036$

$\Rightarrow P(\overline{V}) = 0.04$

$\Rightarrow P(D|\overline{V}) = \frac{P(D \cap
\overline{V})}{P(\overline{V})} = 0$

$P(\overline{R}|D) = \frac{P(\overline{R} \cap D)}{P(D)} = 0.9$

$\Rightarrow (0.9) \times (0.04) = P(\overline{R}) - P(D)$

$\Rightarrow 0.036 + 0.04 = P(\overline{R})$

$\Rightarrow 0.04 = P(\overline{R})$

$P(R) = 1 - P(\overline{R}) = 1 - 0.04 = 0.96$

\section{}

\subsection*{(a)}

The Poisson distribution is a distribution of counts in a random
scenario. It is a reasonable distribution to use in the circumstance due
to the irregular, random patterns branches grown on trees forming knots 
on timber.

Poisson expression for:

Grade A: 

$x < 2$

$=POISSON(1.9,\lambda,0)$ where $\lambda = 2$

$\Rightarrow 0.270670566$

Grade B: 

$x < 5$

$=POISSON(4.9,\lambda,0) \mbox{ where } \lambda = 2$

$\Rightarrow 0.090223522$

Grade C: 

$> 5$

$ =POISSON(5.1,\lambda,0) \mbox{ where } \lambda = 2$

$\Rightarrow 0.036089409$

\subsection*{(b)}

\section{}

\subsection*{(a)}

Mean Lifetime of a type A monitor is $3$ years.

$\exp{- \lambda x}$

Probability of a monitor lasting $3$ years is $\exp{-1} \therefore 
0.36787944117$ (Calc)

Probability of a $1$ year old monitor lasting another $1$ year $= 
\exp{- 0.3 x} = 0.71653131057$ (Calc)

Innsuficient data given to determine the operation requirement of one or
more monitors. Unable to calculate the probability of the maximum of $2$
monitors being required for $3$ years continious use.

\subsection*{(b)}

The mean lifespan of a type B monitor is $16$ months.

The probability of a type B monitor lasting $2$ years is $(0.6)^{2}(2)
\exp{-(2)(2)} = 0.01318725999$ (Calc)

The probability of a $1$ year old monitor lasting another year is $1 -
\exp{-(0.6)(0.6)(2)} - (0.6)(0.6)(2) \exp{-(0.6)(0.6)(2)} = 1 -
\exp{-0.72} - 0.72 \exp{0.72}$ 

\section{}

$\mu = 1000$

$\omega^{2} = 900$

\subsection*{(a)}

The probability of a resistor with a resistance less than $975 \Omega$
is $0.000443�1$ (Calc)

$P(\frac{965 - 1000}{30} \leq Z \leq \frac{975 - 1000}{30} = -
\frac{7}{6} \leq Z \leq \frac{-5}{6}) = 0.00000009837299826606$ (Calc)

\end{document}
