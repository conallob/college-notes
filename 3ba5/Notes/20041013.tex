% $Id: 20041012.tex 336 2004-10-27 12:47:38Z conall $

\documentclass[a4paper,12pt]{article}
\usepackage{amssymb}
\usepackage[all]{xy}
\usepackage{tabularx}


\setlength{\parindent}{0mm}
\setlength{\parskip}{7.5mm}

\begin{document}

\title{Course 3BA5: Computer Engineering \\ Lecture Notes \\ $13^{th}$ October 2004}

\maketitle

This is easy to acomplish within one chip, but chip to chip and pcb to
pcb,

eg: memory to processor it is difficult to present so present day high
speed systems, LP and Cray XD1 series have $f$ approaching $5GHz$, hence
$\tau = 0.2 ns$ and $d = 6cm$.

Thus fast systems must be very compact. But every transistor generates
heat which must be drawn off or the machine overheats.

To see how heat and speed interact, consider a processor $p$ on an area
$A = l \times l$, running at clock rate $f_{0}$ and producing $W$ watts
of heat.

To speed it up, we half it's size.

\begin{table}[hbtp]

% Polish diagram labels

\xy <1cm,0cm>:
0 ; (3,0) **\dir{-} ; 
(5,2) **\dir{-} *+!UC{l} ; 
(2,2) **\dir{-} *+!DL{l} ;
0 **\dir{-} ; ?(0.25) *+!UC{p} ,
\endxy

\end{table}

Heat intensity $= H_{0} = \frac{W}{A}$ \\
Clock Rate $f = f_{0}$

\begin{table}[hbtp]

% Polish diagram labels

\xy <1cm,0cm>:
0 ; (3,0) **\dir{-} ; 
(5,2) **\dir{-} *+!UC{\frac{l}{2}} ; 
(2,2) **\dir{-} *+!DL{\frac{l}{2}} ;
0 **\dir{-} ; ?(0.25) *+!UC{p} ,
\endxy

\end{table}

Heat intensity $= H = \frac{W}{\frac{A}{4}} = 4 H_{0}$ \\
Clock Rate $f = 2 \times f_{0}$

\section*{Conclusion}

there has been an engineering compromise between $f$ and $H$ and after
this has been reached the only way to improve performance is to ensure
that more of their transistors are doing useful work in each clock
period. ie: to use parallelism.

This has resulted in generations of increasing larger and larger and
more parallel architecture. Two versions of parallelism have developed,
temporal parallelism (ie pipelining examplified by the Cray vetor
processors) and spacial parallelism (ie multiple identical processing
elements examplified by the Intel series)

Ref. H3 Figure 1 amd H \& X - p155-9

\section*{Taxonomy of Computer Architecture}

It is helpful to have a general framework to view the broad range of
computer systems.

Flynn's taxonomy has proven to be the most useful, though it is not the
most sophisticated.

\begin{tabularx}{\linewidth}{|l|X|}
\hline
Acronym	&	Description											\\
\hline
MM		 	&	Main (Primary) Memory							\\
\hline
CU		 	&	Control Unit - fetches the raw instruction code, decodes
it and from this generates the control signals for the PU	\\
\hline
PU			&	Processing Unit - fetches operands, circulates them
through the datapath, stores the result							\\
\hline
\end{tabularx}

Flynn then focusses on the no of streams in instructions (IS) and data
(DS) which are in the same phase of execution at the most constrained
point of execution.

It is sufficient to distinguish single (S) from multiple (M) streams,
leading to four distinct catagories of architecture.

SISD - Single Instruction, Single Data - von Neumann's Design

\end{document}
