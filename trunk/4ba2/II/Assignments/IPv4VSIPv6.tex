% $Id: /college/trunk/4ict9/Assignments/MobileMap/report.tex 888 2005-12-08T16:27:16.887239Z conall  $

\documentclass[a4paper,12pt]{article}

\usepackage{graphicx}

\setlength{\parindent}{0mm}
\setlength{\parskip}{7.5mm}

\begin{document}

\title{4BA2 \\ Essay \\ Analysis of IP suitability for various application domains}

\author{Brian Brazil (02017610) brazilb@tcd.ie
David Collins (02704030) collinda@tcd.ie \\ 
Conall O'Brien (01734351) conallob@maths.tcd.ie}

\maketitle

\section{Introduction}

In order to compare and contrast the suitability of IPv4 against IPv6
for specific applications, we must first look at the reasons why IPv6
was created in the first place.

\subsection{The Limitations of IPv4}

Over the last 25 years, since it's standardisation in RFC 791, published
in 1981, IPv4 has proven to be very popular. Despite it's popularity, it 
has some limitations, which have arisen over the years as new uses for 
IP connectivity are invented. Some even believe one of the key reasons
for it's popularity, Network Address Translation (NAT), is also a 
limitation, and in certain cases, it is. Other limitations
affecting IPv4 include address allocation, security concerns, growth of 
routing tables and quality of service (QoS).


NAT is a method by which a network with private IP space, as defined in
RFC 1918, is masked behind a single or a few public, globally unique IP
address or addresses. NAT works by mapping certain TCP/IP ports between
addresses in the private address network and the public IP address(es).
While many find this useful, it does come at a cost, which is becoming
more and more expensive. Certain applications used in modern times to
not perform as expected when their IP datagrams are rewritten by a NAT
router. Additionally, certain applications, which shall be discussed
later, require a significant amount of TCP/IP ports, which are not
available in a NAT enviroment.


Initially, IPv4 address allocation was done in 3 catagories, Class A, B
and C. Hindsight showed that this method was overly gratituitious and
wasted IPv4 addresses. As a result, Classless Inter Domain Routing
(CIDR) addressing was created to slow down, but not stop IPv4 addresses
being wasted. CIDR also managed to condense routing table information,
slow down another problem afflicting IPv4, where routing tables on core
routers were growing out of control, with the addition of routes to
various class A, B and C address allocations.


Finally, in recent years, IPv4 has had many security concerns concerning
Address Resolution Protocol (ARP), a protocol it uses to communicate
with Layer 2 protocols. ARP, by it's nature is an unauthenticated, 
broadcast protocol which is suseptible to attacks. Modern applications
which require realtime QoS have also suffered from a certain
inflexability of IPv4 in adapting to suit their needs.

\section{Technologies}

\subsection{Routing Protocols}

% RIP, OSPF, IS-IS, BGP

\subsection{Voice over IP (VoIP)}

% Skype, SIP, IAX, H.323

\subsection{Remote Computing}

% X11, SSH, Telnet, VNC, Teminal Services

\section{Network Configuration and Administration}

% Ping, DHCP, autoconf, Traceroute, ARP

\section{Conclusion}


\section{References}

\end{document}
