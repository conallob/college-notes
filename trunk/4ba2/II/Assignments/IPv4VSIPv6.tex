% $Id: /college/trunk/4ict9/Assignments/MobileMap/report.tex 888 2005-12-08T16:27:16.887239Z conall  $

\documentclass[a4paper,12pt]{article}

\usepackage{url}

\setlength{\parindent}{0mm}
\setlength{\parskip}{7.5mm}

\begin{document}

\title{4BA2 \\ Analysis of IP suitability for various application domains}

\author{Brian Brazil (02017610) brazilb@tcd.ie \\
David Collins (02704030) collinda@tcd.ie \\ 
Conall O'Brien (01734351) conallob@maths.tcd.ie}

\maketitle

\section{Introduction}

In order to compare and contrast the suitability of IPv4 against IPv6
for specific applications, we must first look at the reasons why IPv6
was created in the first place.

\subsection{The Limitations of IPv4}

Over the last 25 years, since it's standardisation in RFC 791, published
in 1981, IPv4 has proven to be very popular. Despite it's popularity, it 
has some limitations, which have arisen over the years as new uses for 
IP connectivity are invented. Some even believe one of the key reasons
for it's popularity, Network Address Translation (NAT), is also a 
limitation, and in certain cases, it is. Other limitations
affecting IPv4 include address allocation, security concerns, growth of 
routing tables and quality of service (QoS).


NAT is a method by which a network with private IP space, as defined in
RFC 1918, is masked behind a single or a few public, globally unique IP
address or addresses. NAT works by mapping certain TCP/IP ports between
addresses in the private address network and the public IP address(es).
While many find this useful, it does come at a cost, which is becoming
more and more expensive. Certain applications used in modern times to
not perform as expected when their IP datagrams are rewritten by a NAT
router. Additionally, certain applications, which shall be discussed
later, require a significant amount of TCP/IP ports, which are not
available in a NAT enviroment.


Initially, IPv4 address allocation was done in 3 catagories, Class A, B
and C. Hindsight showed that this method was overly gratituitious and
wasted IPv4 addresses. As a result, Classless Inter Domain Routing
(CIDR) addressing was created to slow down, but not stop IPv4 addresses
being wasted. CIDR also managed to condense routing table information,
slow down another problem afflicting IPv4, where routing tables on core
routers were growing out of control, with the addition of routes to
various class A, B and C address allocations.


Finally, in recent years, IPv4 has had many security concerns concerning
Address Resolution Protocol (ARP), a protocol it uses to communicate
with Layer 2 protocols. ARP, by it's nature is an unauthenticated, 
broadcast protocol which is suseptible to attacks. Modern applications
which require realtime QoS have also suffered from a certain
inflexability of IPv4 in adapting to suit their needs.

\section{Technologies}

\subsection{Routing Protocols}

% RIP, OSPF, IS-IS, BGP

\subsection{Voice over IP (VoIP)}

There are a large amount of Voice over IP (VoIP) protocols around. They
include popular, widely deployed (in enterprises) standards such as 
H.323, Session Initiation Protocol (SIP) and Cisco SCCP (known as
Skinny). There are also newer protocols, such as the propriatary one used
by Skype, and the open source Inter Asterisk Exchange (IAX) protocol,
designed for call trunking and to address limitations NAT causes H.323
and SIP.


VoIP protocols are one of the application catagories most affected by
the limitations of IPv4. The primary issue is the widespread deployment
of NAT. Since a limited subset of TCP and UDP ports are mapped to each
privately addresses host, VoIP protcols which rely upon multiple UDP
based connections between nodes are adversely affected by NAT.
Additionally NAT IP header rewriting causes additional delays, which 
multimedia protocols (and their users) are particularly sensitive 
towards.


In addition, VoIP and other multimedia protocols are held to a higher
standard regarding delays, such as network congestion, network jitter 
and other interruptions. IP datagrams are designed to allow delivery 
in a non-sequential order, before the destination node reassembles them
sequentially. This behaviour causes noticible delay and distortions in 
multimedia applications. Attempts to address this issue are primarily
done using QoS, which is not widely supported over IPv4 networks.


For these reasons, VoIP protocols (and their multimedia cousins) are
better suited over IPv6, as they raise many of the limitations
encoutered with IPv4. 

\subsection{Remote Computing}

% X11, SSH, Telnet, VNC, Teminal Services

\section{Network Configuration and Administration}

% Ping, DHCP, autoconf, Traceroute, ARP

\section{Conclusion}


\section{References}

Title: RFC 791 - Internet Protocol \\
Author(s): Jon Postel \\
Publisher: IETF
Published Date: Semtember 1981 \\
URL: \url{ftp://ftp.rfc-editor.org/in-notes/rfc791.txt}	 

Title: RFC 1918 - Address Allocation for Private Internets \\
Author(s): Y. Rekhter, B. Moskowitz, D. Karrenberg, G. J. de Groot, \
\indent E. Lear \\
Publisher: IETF \\
Published Date: February 1996  \\
URL: \url{ftp://ftp.rfc-editor.org/in-notes/rfc791.txt}	 

Title: RFC 3261 - SIP: Session Initiation Protocol \\
Author(s): J. Rosenberg, H. Schulzrinne, G. Camarillo, A. Johnston, \\
\indent J. Peterson, R. Sparks, M. Handley, E. Schooler \\
Publisher: IETF \\
Published Date: June 2002  \\
URL: \url{ftp://ftp.rfc-editor.org/in-notes/rfc3261.txt}	 

Title: IPv6 Network Administration \\
Author(s): Niall Richard Murphy, David Malone \\
Publisher: O'Reilly \\
Published Date: March 2005  \\
ISBN: 0-596-00934-8	 

\end{document}
