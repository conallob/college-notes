% $Id$

\documentclass[a4paper,12pt]{article}

\setlength{\parindent}{0mm}
\setlength{\parskip}{7.5mm}

\begin{document}

\bibliographystyle{ieeetr}

\title{4BA2 \\ Assignment 2}

\author{Conall O'Brien (01734351) conallob@maths.tcd.ie}

\maketitle

\section{What is a Mainframe?}

\cite[Mainframes (often colloquially referred to as big iron) are large and 
"expensive" computers used mainly by government institutions and 
large companies for mission critical applications, typically bulk data 
processing such as censuses, industry/consumer statistics, ERP, and 
financial transaction processing]{m1}


Nowadays IBM, Unisys, Fujitsu and Bull all remain selling mainframes,
with \cite[IBM mainframes dominating the market at over 90\%]{m1}.
Between the 1950s and 1970s, mainframes were also manufactured by 
Sperry and Burroughs (now owned by Unisys), Siemens and Amdahl (now 
owned by Fujitsu), Hitachi, Control Data Corporation, General Electric, 
Honeywell, NCR, RCA, UNIVAC, Telefunken and ICL. \cite [ IBM's 
dominance grew out of their development of the 360 series mainframes; 
this basic architecture has continued to evolve into their current 
zSeries/z9 mainframe, which is arguably the only mainframe architecture 
still extant that dates from this early period.]{m1}

\section{Features and Uses of the Mainframe}

\section{Alternatives to the Mainframe}

Super computers, such as grids and lattices have proven to be a
compeditor to the mainframe. The fundemental difference a super computer
exhibits is it consists of many inexpensive computers connected
together via a network. Depending on the design, some super computers 
are able to be repaired while in operation, just like a mainframe. 
Alternatively, corporations such as Google and Amazon simply add and 
subtract computers to their clusters in order to remove nodes for 
repairs or replacements without affecting the system as a whole.

\section{The Fall and Resurrection of the Mainframe}

Traditionally, Mainframes were accessed using monochrome, text 
based "Dumb Terminals" such as the DEC VT100 over DB9 serial 
connections. In the 1980s and 1990s, dumb terminals were being replaced
by personal computers (PCs) located on office desks when networks of PCs
became popular.


\cite[In the late 1990s, corporations found new uses for their
mainframes, since they can offer web server performance similar to that 
of hundreds of smaller machines, but with much lower power and 
administration costs. The growth of e-business has also dramatically 
increased the number of backend transactions processed by 
tried-and-true mainframe software as well as the size and 
throughput of databases. In late 2004, IBM's mainframe revenues were
increasing even with price reductions, thanks to attractive TCOs.]{m1}

\section{Conslusion}

\section{Bibliography}

\bibliography{Mainframes}

\end{document}
