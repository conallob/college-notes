% $Id$

\documentclass[a4paper,12pt]{article}
\usepackage{amssymb}

\setlength{\parindent}{0mm}
\setlength{\parskip}{7.5mm}

\begin{document}

\title{Course 3BA1: Statistics \\ Additional Lecture Notes \\ $14^{th}$ October 2004}

\maketitle

$ 1, 2, 3, 4, 5, 6 \to $ Sample Space \\

"Even Numbers" \\

A, B \\

\\

\[ W W W   = 3 - 0 \] 
\[ W W L W =       \]
\[ W L W W = 3 - 1 \]
\[ L W W W =       \]

\\

2 dice - outcomes

$2 - 12$ sum on two dice.

\begin{table}[cc]
66 & 55 \\
65 & 54 \\
.  & .  \\
.  & .  \\
.  & .  \\
61 & 51 \\
\end{table}

\\

\begin{table}[cccccc]
66 & . & . & . & . & 16 \\
.  & . & . & . & . & .  \\
.  & . & . & . & . & .  \\
.  & . & . & . & . & .  \\
.  & . & . & . & . & .  \\
61 & . & . & . & . & 11 \\
\end{table}

In this setup, the probability of each outcome is the same. \\

\\

How do we calculate $P(E) = ?$ \\

\\

What is $ P(E) $? \\

\\

1930 - Kolmogorov \\

\\

% Diagram

\\

\[ P(H) + P(T) 	& = & P(H) + P(H) 	\]
\[ 		& = & 2 P(H)		\]
\[		& = & 1			\]

6 sided dice \\

6 outcomes each $P = \frac{1}{6} $. \\

\[ P(Event) & = & \frac{\mbox{no of likely outcomes where event happens}}{\mbox{total no of outcomes}} \]

\end{document}
