% $Id$

\documentclass[a4paper,12pt]{article}
\usepackage{amssymb}

\setlength{\parindent}{0mm}
\setlength{\parskip}{7.5mm}

\begin{document}

\title{Course 3BA4: Computer Architecture II \\ Lecture Notes \\ $26^{th}$ October 2004}

\maketitle

\section*{Register File Overflow}

PSW (status register) contains CWP (current window pointer) and SWP
(save window pointer) fields.


Diagram

\begin{itemize}

\item before a call/callr instruction is executed, the following 
test is made:

\subitem if (CWP $- 2 ==$ SWP)
\subsubitem TRAP(register file overflow)

\item the trap handler must push the registers pointed to by SWP 
 onto a stack in main memory (a global register, r1. r9 asking as a 
 stack pointer). How is this done?
  
\end{itemize}

\section*{Some problems with Multiple Register Sets}


\begin{itemize}

\item Must save \textbf{all} registers on an overflow (even if only a 
few may actually be in use).

\item Referencing variables held in registers \textbf{by address} as
registers normally do not have addresses.

\begin{verbatim}

		P(int i,int* j)
		{
			*j = ...

		}

		Q()
		{
			int i,j; // can j be allocated to a register?

			... // eg i in r16, j in r17

			P(i, &j);

			...

		}
\end{verbatim}

\item Referencing variables at intermediate levels in block structured
languages such as Modula-2 and Pascal

\item 3 Impact a context (process) switch times.

\subitem must save \textbf{all} used register windows

\end{itemize}

\section*{RISC-I Pipeline}

\begin{itemize}

\item 2 stage pipeline - fetch and execute units

\item normal instructions \\


\begin{tabular}{l|l|l|l}
fetch i1		&					&					&					\\
				&	execute i1	&					&					\\
				&	fetch i2		&	execute i2	&					\\
				&					&	fetch i3		&					\\
				&					&					&	execute i3	\\
\end{tabular}

\item load/store instructions

\item Diagram

\begin{tabular}{l|l|l|l}
fetch i1		&	compute addr	& mem access	&					\\
				&	fetch i2			&	--stall--	&	execute i2	\\
				&						&					&	fetch i3		\\
\end{tabular}


\item pipeline stall arises because it's not possible to access memory
twice in the same clock cycle - to fetch the next instruction and to
access the target of load/store

\subitem load/store - 2 cycles (latency 3 cycles)

\subitem others - 1 cycle (latency 2 cycles)

\end{itemize}

\section*{Delayed Jumps}

- jump/branch/call/ret take place after the immediately following
  instructions have been executed.

\begin{verbatim}
 1. 		sub r16,#1,r16 {C}
 2. 		jne L
 3. 		xor r0,r0,r16
 4. 		syb r17,#1,r17


10. L:	sll r16,2,r16
\end{verbatim}

- if conditional jump is taken, effective execution order is 1, 3, 2,
  10, ...

- if conditional jump is \textbf{not} taken, effective execution order
  is 1, 3, 2, 4.

\end{document}
