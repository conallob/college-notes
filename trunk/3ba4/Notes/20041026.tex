% $Id$

\documentclass[a4paper,12pt]{article}
\usepackage{amssymb}

\setlength{\parindent}{0mm}
\setlength{\parskip}{7.5mm}

\begin{document}

\title{Course 3BA4: Computer Architecture II \\ Lecture Notes \\ $26^{th}$ October 2004}

\maketitle

\section*{Register File Overflow}

PSW (status register) contains CWP (current window pointer) and SWP
(save window pointer) fields.


Diagram

\begin{itemize}

\item before a call/callr instruction is executed, the following 
test is made:

\subitem if (CWP $- 2 ==$ SWP)
\subsubitem TRAP(register file overflow)

\item the trap handler must push the registers pointed to by SWP 
 onto a stack in main memory (a global register, r1. r9 asking as a 
 stack pointer). How is this done?
  
\end{itemize}

\section*{Some problems with Multiple Register Sets}


\begin{itemize}

\item Must save \textbf{all} registers on an overflow (even if only a 
few may actually be in use).

\item Referencing variables held in registers \textbf{by address} as
registers normally do not have addresses.

\begin{verbatim}

		P(int i,int* j)
		{
			*j = ...

		}

		Q()
		{
			int i,j; // can j be allocated to a register?

			... // eg i in r16, j in r17

			P(i, &j);

			...

		}
\end{verbatim}

\item Referencing variables at intermediate levels in block structured
languages such as Modula-2 and Pascal

\item 3 Impact a context (process) switch times.

\subitem must save \textbf{all} used register windows

\end{itemize}

\section*{RISC-I Pipeline}

\begin{itemize}

\item 2 stage pipeline - fetch and execute units

\item normal instructions \\


\begin{tabular}{l|l|l|l}
fetch i1		&					&					&					\\
				&	execute i1	&					&					\\
				&	fetch i2		&	execute i2	&					\\
				&					&	fetch i3		&					\\
				&					&					&	execute i3	\\
\end{tabular}

\item load/store instructions

\item Diagram

\begin{tabular}{l|l|l|l}
fetch i1		&	compute addr	& mem access	&					\\
				&	fetch i2			&	--stall--	&	execute i2	\\
				&						&					&	fetch i3		\\
\end{tabular}


\item pipeline stall arises because it's not possible to access memory
twice in the same clock cycle - to fetch the next instruction and to
access the target of load/store

\subitem load/store - 2 cycles (latency 3 cycles)

\subitem others - 1 cycle (latency 2 cycles)

\end{itemize}

\section*{Delayed Jumps}

- jump/branch/call/ret take place after the immediately following
  instructions have been executed.

\begin{verbatim}
 1. 		sub r16,#1,r16 {C}
 2. 		jne L
 3. 		xor r0,r0,r16
 4. 		syb r17,#1,r17


10. L:	sll r16,2,r16
\end{verbatim}

- if conditional jump is taken, effective execution order is 1, 3, 2,
  10, ...

- if conditional jump is \textbf{not} taken, effective execution order
  is 1, 3, 2, 4.

\section*{Delayed Jump Examples}

Consider the RISC-I code for the following code segment:

\begin{verbatim}

(k = 10;)
i = 0; // assume i in r16
while (i < j) // assume j in r17
      i += f(i); // assume parameter & result in r0

\end{verbatim}

\subsection*{Unoptimised}

\begin{verbatim}

    add r0,r0,r16	// i = 0
L0: sub r16,r17,r0 {C} // i< j ?
    jge L1
	 nop
	 add r0,r16,r10 //move i into r10
	 callr f
	 nop
	 add r10,r16,r16 // i += f(i)
	 jmp L0
	 nop

L1: 

\end{verbatim}

\subsection*{Reorganised and Optimised}

\begin{verbatim}

    add r0,r0,r16
L0: sub r16,r17,r0 {C} // i < j ?
    jge L1
	 nop
	 call f
	 add r0,r16,10 // move i into r10
	 jmp L0
	 add r10,r16,r16 // i += f(i)

L1:

\end{verbatim}

\begin{tabular}{|l|l|l|l|}
fetch i1 		& execute i1 	&						&					\\
					& fetch jump 	& compute addr		&					\\
--delay slot--	&		 			& fetch i3			& execute i3	\\
					&					&						& fetch i4		\\
\end{tabular}

The destination of the jump instruction is i4 (if jump \textbf{not}
taken, this will be the instruction after the delay slot.)

i3 is executed in the delay slot.

Better to execute in the delay slot than leaving the execution unit
idle.

Since the instruction in the delay slot is fetched anyway, we might as
well execute it.

$60$\% if delay slots can be filled with useful instructions.

\subsection*{What about?}

\begin{verbatim}

i0 ...
jmp L1 // unconditional jump
jmp L2 // unconditiuonal jump
< = ?

L1: i10 ...
    i11 ...
    <?>

L2: i20 ...
    i21 ...

\end{verbatim}

\begin{tabular}{|l|l|l|l|l|l|}

i10	& EX	 	&			&			&			&			\\
		& jmp L1	& EX'		&			&			&			\\
		&		 	& jmp L2	& EX'		&			&			\\
		&		 	&			& i10		& EX		&			\\
		&		 	&			&			& i20		& EX		\\
		&		 	&			&			&			& i21		\\
\end{tabular}

EX : Execute \\
EX': Execute calculate jump, destination address.

\section*{Pipelining}

Key implementation technique for speeding up CPUs (see Hennessy \&
Patterson - Chapter 6)

Break each instruction into a series of small steps and execute them in
parallel (steps from different instructions)

\indent Think of a car assembly line

Clock rate set by hte time needed for the longest step - ideally time
for each step should be equal.

Consider a 5 stage insttuction pipeline for the hypothetical DLX
microporcessor.

IF : Instruction Fetch \\
ID	: Instruction Decode and Register Fetch \\
EX : Execution \& Effective Address Calculation \\
MA : Memory Access \\
WB : Write Back


\begin{tabular}{|l|l|l|l|l|l|}
i			&	IF		&	ID	 	&	EX		&	MA		&	WB	\\
i + 1		&			&	IF		&	ID		&	ID		&	MA	\\
i + 2		&			&			&	IF		&			&	EX	\\
i + 3		&			&			&			&	IF		&		\\
i + 4		&			&			&			&			&	IF	\\
\end{tabular}

\end{document}
