% $Id: 20041108.tex 353 2004-11-12 00:21:25Z conall $

\documentclass[a4paper,12pt]{article}
\usepackage{amssymb}

\setlength{\parindent}{0mm}
\setlength{\parskip}{7.5mm}

\begin{document}

\title{Course 3BA4: Computer Architecture II \\ Lecture Notes \\ $23^{th}$ November 2004}

\maketitle

Types when $V = 1$ (Valid):

\begin{itemize}

\item $\verb!MEM!$ maps virtual to physical.

\item $\verb!LOCK!$ is the same as $\verb!MEM!$ except page is locked
into physical memory.

\subitem $\verb!vlock(va)!$ - system call - [superuser only]

\item $\verb!SPY!$ maps virtual addresses [va] to a specific physical
address [pa]

\subitem Can be used to map device registers

\subitem $\verb!vspy(va, pa)!$ system call - [superuser only]

\end{itemize}


Types when $V = 0$ (Invalid):

\begin{itemize}

\item $\verb!NULL!$ page \textbf{not} mapped to physical memory.

\item $\verb!DISK!$ page not mapped to physical memory, but when mapped
the data for the page is stored on the disk - PTE contains the disk
block number.

\item $\verb!IOP!$ indicates that the disk I/O is in progress

\item $\verb!SPT!$ shared PTE

\subitem Allows code sharing between processes

\subitem Contains a pointer to a PTE in another table.

\end{itemize}

\begin{verbatim}
// Global Variables


int a[] = {0, 1, 2, 3, 4, 5, 6, 7, 8, 9};

char *s[] = {"zero", "one", "two", ...};


int c[100];

int d[1000];
\end{verbatim}

\section*{Initial Execution of Unix [Windows] Process:}

After the initial page table(s) have been created, the process can start
execution [start address in header] - will instantly generate a page
fault as the first instruction is still on disk.

Page faults will occur as the process executes and each PTE type fault
will be handled as follows:

\begin{tabular}{|l|l|}
\hline
Type				&	Action								\\
\hline
$\verb!DISK!$	&	Allocate a page of physical memory and read data \\
					&	from the disk (context switch while waiting) \\
					&		\\
					&	Code/initialised data paged \emph{"on demand"}.	\\
					&	$\verb!DISK! \Rightarrow \verb!IOP! \Rightarrow
					\verb!MEM!$ \\
\hline
$\verb!NULL!$	&	Physical memory has not yet been allocated. \\
					&		\\
					&	The virtual fault address is checked to see if it is in range. \\
					&	If outside the uninitialised data or heap or stack, \\
					&	then it's considered to be an illegal memory access and a \\
					&	memory access violation is signalled \textbf{otherwise} a \\
					&	page of [zeroed] physical memory is allocated. \\
					&		\\
					&	$\verb!NULL! \Rightarrow \verb!MEM!$ \\
\hline
$\verb!MEM!$	& Protection level fault [eg writing to text via a NULL pointer] \\
\hline
$\verb!IOP!$	&	Wait for I/O to complete (context switch)	\\
\hline
$\verb!SPY!$	&	$\verb!panic()!$	\\
\hline
$\verb!LOCK!$	&	$\verb!panic()!$	\\
\hline
\end{tabular}

\section*{Test Code Sharing}

\begin{itemize}

\item After the master page table has been created, the process page table for
the code [and trhe initialised data] regions is constructed from the
file map [so that code and possibly initialised data can be shared].

\subitem For those pages that are already in memory (ie, process has been
executed already), $\verb!MEM!$ type PTEs are copied from the file map -
hence sharing the same physical memory pages [code and initialised data
\textbf{only}]

\subitem The remaining PTEs for code and initialised data are allocated as SPT
type entries pointing to entries in the file map [point to $\verb!DISK!$
type PTEs in file map]

\subitem Remaining PTEs created as per non-standard case since each
instance needs its own copy of uninitialised data, heap and stack.

\item When OS handles an SPT type page fault, it follows the SPT PTE to
the corresponding entry in the field map and depending on the PTE type
performs the following actions

\subitem $\verb!DISK!$: allocate a memory page, reads data from the disk
and updates the file map and process to type $\verb!MEM!$.

\subitem $\verb!MEM!$: Copy PTE, but if the page is read/write [ie
initialised data], must make a copy of the page from the original
pointed to by the file map PTE [could be copy-on-write rather than
copy-on-access]

\end{itemize}

\end{document}
