% $Id$

\documentclass[a4paper,12pt]{article}
\usepackage{amssymb}

\setlength{\parindent}{0mm}
\setlength{\parskip}{7.5mm}

%\lstset{language=prolog}

\begin{document}

\title{Course 3BA3: Computer Architecture II \\ Tutorial Solution\\ $20^{th}$ October 2004}

\author{Conall O'Brien \\ conallob@maths.tcd.ie \\ 01734351}

\maketitle

\section{}

\begin{eqnarray*}
g & \equiv & r9 \\
i & \equiv & r26 \\
j & \equiv & r27 \\
k & \equiv & r16 \\
\end{eqnarray*}

\begin{verbatim}

	add 	r0,#256,r9		; g = 256

p:

	add 	r26,r27,r16		; k = i + j
	sll 	r16,#2,r1		; r1 = k << 2
	sub 	r1,#1,r1			; r1 = (k << 2) - 1
	ret 	r25,0				; r25 used to return trsult

q:

	add 	r0,r9,r10		; setup param 1
	sub 	r0,r26,r11		; setup param 2
	callr r25,p				; result in r1
	ret	r25,0				; return

f:
	sub r26,r0,r0 {c}		; test n
	jle f0					; jmp <=
	add r0,#1,r1			; delay slot, r1++
	call r25,f				; call f (recursive)
	add r26,#1,r10			; dely slot, 1st param
	add r26,r0,r10			; param
	call r25,times			; multiply
	add r1,r0,r11			; delay slot, retrieve value form previous times

f0:
	ret r25,0				; return
	nop						; delay slot

\end{verbatim}

\section{}

I am unsure as to how to proceed to calculate the number of proceedure
calls, stack underflows and stack overflows via experimentation.

\section{}

\subsection*{PowerPC}

System Used: 	$1.5GHz$	G4 PowerPC

Execution Time:

\begin{tabular}{rrr}
real 	&		&	0m0.032s		\\	
user 	&		&	0m0.010s		\\
sys	&		&	0m0.010s		\\
\end{tabular}

Assembly:

\begin{verbatim}

	.section __TEXT,__text,regular,pure_instructions
	.section __TEXT,__picsymbolstub1,symbol_stubs,pure_instructions,32
.section __TEXT,__text,regular,pure_instructions
	.align 2
	.align 2
	.globl __Z8ackermanii
.section __TEXT,__text,regular,pure_instructions
	.align 2
__Z8ackermanii:
LFB3:
	mflr r0
	stmw r30,-8(r1)
LCFI0:
	stw r0,8(r1)
LCFI1:
	stwu r1,-96(r1)
LCFI2:
	mr r30,r1
LCFI3:
	stw r3,120(r30)
	stw r4,124(r30)
	lwz r0,120(r30)
	cmpwi cr7,r0,0
	bne cr7,L2
	lwz r2,124(r30)
	addi r0,r2,1
	stw r0,64(r30)
	b L1
L2:
	lwz r0,124(r30)
	cmpwi cr7,r0,0
	bne cr7,L4
	lwz r2,120(r30)
	addi r0,r2,-1
	mr r3,r0
	li r4,1
	bl __Z8ackermanii
	mr r0,r3
	stw r0,64(r30)
	b L1
L4:
	lwz r2,124(r30)
	addi r0,r2,-1
	lwz r3,120(r30)
	mr r4,r0
	bl __Z8ackermanii
	mr r9,r3
	lwz r2,120(r30)
	addi r0,r2,-1
	mr r3,r0
	mr r4,r9
	bl __Z8ackermanii
	mr r0,r3
	stw r0,64(r30)
L1:
	lwz r3,64(r30)
	lwz r1,0(r1)
	lwz r0,8(r1)
	mtlr r0
	lmw r30,-8(r1)
	blr
LFE3:
	.align 2
	.globl _main
.section __TEXT,__text,regular,pure_instructions
	.align 2
_main:
LFB5:
	mflr r0
	stmw r30,-8(r1)
LCFI4:
	stw r0,8(r1)
LCFI5:
	stwu r1,-96(r1)
LCFI6:
	mr r30,r1
LCFI7:
	li r0,6
	stw r0,64(r30)
	li r3,3
	lwz r4,64(r30)
	bl __Z8ackermanii
	mr r0,r3
	stw r0,68(r30)
	lwz r0,68(r30)
	mr r3,r0
	lwz r1,0(r1)
	lwz r0,8(r1)
	mtlr r0
	lmw r30,-8(r1)
	blr
LFE5:
.data
.section __TEXT,__eh_frame,coalesced,no_toc+strip_static_syms
EH_frame1:
	.set L$set$0,LECIE1-LSCIE1
	.long L$set$0
LSCIE1:
	.long	0x0
	.byte	0x1
	.ascii "zPR\0"
	.byte	0x1
	.byte	0x7c
	.byte	0x41
	.byte	0x6
	.byte	0x90
	.long	L___gxx_personality_v0$non_lazy_ptr-.
	.byte	0x10
	.byte	0xc
	.byte	0x1
	.byte	0x0
	.align 2
LECIE1:
.globl _Z8ackermanii.eh
_Z8ackermanii.eh:
LSFDE1:
	.set L$set$1,LEFDE1-LASFDE1
	.long L$set$1
LASFDE1:
	.long	LASFDE1-EH_frame1
	.long	LFB3-.
	.set L$set$2,LFE3-LFB3
	.long L$set$2
	.byte	0x0
	.byte	0x4
	.set L$set$3,LCFI2-LFB3
	.long L$set$3
	.byte	0xe
	.byte	0x60
	.byte	0x11
	.byte	0x41
	.byte	0x7e
	.byte	0x9f
	.byte	0x1
	.byte	0x9e
	.byte	0x2
	.byte	0x4
	.set L$set$4,LCFI3-LCFI2
	.long L$set$4
	.byte	0xd
	.byte	0x1e
	.align 2
LEFDE1:
.globl main.eh
main.eh:
LSFDE3:
	.set L$set$5,LEFDE3-LASFDE3
	.long L$set$5
LASFDE3:
	.long	LASFDE3-EH_frame1
	.long	LFB5-.
	.set L$set$6,LFE5-LFB5
	.long L$set$6
	.byte	0x0
	.byte	0x4
	.set L$set$7,LCFI6-LFB5
	.long L$set$7
	.byte	0xe
	.byte	0x60
	.byte	0x11
	.byte	0x41
	.byte	0x7e
	.byte	0x9f
	.byte	0x1
	.byte	0x9e
	.byte	0x2
	.byte	0x4
	.set L$set$8,LCFI7-LCFI6
	.long L$set$8
	.byte	0xd
	.byte	0x1e
	.align 2
LEFDE3:
.data
.non_lazy_symbol_pointer
L___gxx_personality_v0$non_lazy_ptr:
	.indirect_symbol ___gxx_personality_v0
	.long	0
.data
.constructor
.data
.destructor
	.align 1

\end{verbatim}

\subsection*{x86}

System Used: 	Dual $3.0 GHz$ Intel Pentium 4 Xeon

Execution Time:

\begin{tabular}{rrr}
real	&		&	0m0.005s		\\
user	&		&	0m0.01ms 	\\
system&		&	0m0.00ms 	\\
\end{tabular}

Assembly:

\begin{verbatim}
	.file	"20041020.C"
	.version	"01.01"
gcc2_compiled.:
.text
	.p2align 2,0x90
.globl ackerman__Fii
		.type		 ackerman__Fii,@function
ackerman__Fii:
	pushl %ebp
	movl %esp,%ebp
	subl $8,%esp
	cmpl $0,8(%ebp)
	jne .L3
	movl 12(%ebp),%edx
	incl %edx
	movl %edx,%eax
	jmp .L2
	jmp .L4
	.p2align 2,0x90
.L3:
	cmpl $0,12(%ebp)
	jne .L5
	addl $-8,%esp
	pushl $1
	movl 8(%ebp),%eax
	decl %eax
	pushl %eax
	call ackerman__Fii
	addl $16,%esp
	movl %eax,%edx
	movl %edx,%eax
	jmp .L2
	jmp .L4
	.p2align 2,0x90
.L5:
	addl $-8,%esp
	addl $-8,%esp
	movl 12(%ebp),%eax
	decl %eax
	pushl %eax
	movl 8(%ebp),%eax
	pushl %eax
	call ackerman__Fii
	addl $16,%esp
	movl %eax,%eax
	pushl %eax
	movl 8(%ebp),%eax
	decl %eax
	pushl %eax
	call ackerman__Fii
	addl $16,%esp
	movl %eax,%edx
	movl %edx,%eax
	jmp .L2
	.p2align 2,0x90
.L6:
.L4:
	jmp .L7
	jmp .L2
	.p2align 2,0x90
.L7:
.L2:
	leave
	ret
.Lfe1:
		.size		 ackerman__Fii,.Lfe1-ackerman__Fii
	.p2align 2,0x90
.globl main
		.type		 main,@function
main:
	pushl %ebp
	movl %esp,%ebp
	subl $24,%esp
	movl $6,-4(%ebp)
	addl $-8,%esp
	movl -4(%ebp),%eax
	pushl %eax
	pushl $3
	call ackerman__Fii
	addl $16,%esp
	movl %eax,%eax
	movl %eax,-8(%ebp)
	movl -8(%ebp),%edx
	movl %edx,%eax
	jmp .L8
	xorl %eax,%eax
	jmp .L8
	.p2align 2,0x90
.L8:
	leave
	ret
.Lfe2:
		.size		 main,.Lfe2-main
	.ident	"GCC: (GNU) cplusplus 2.95.4 20020320 [FreeBSD]"

\end{verbatim}


\subsection*{Sparc}

System Used: 	Dual $450 GHz$	UltraSparc

Execution Time:

\begin{tabular}{rrr}
real 	&		&	0m0.070s		\\	
user 	&		&	0m0.060s		\\
sys	&		&	0m0.010s		\\
\end{tabular}

Assembly:

\begin{verbatim}

	.file	"20041020.C"
	.section	".text"
	.align 4
	.global _Z8ackermanii
	.type	_Z8ackermanii, #function
	.proc	04
_Z8ackermanii:
.LLFB2:
	!#PROLOGUE# 0
	save	%sp, -120, %sp
.LLCFI0:
	!#PROLOGUE# 1
	st	%i0, [%fp+68]
	st	%i1, [%fp+72]
	ld	[%fp+68], %g1
	cmp	%g1, 0
	bne	.LL2
	nop
	ld	[%fp+72], %g1
	add	%g1, 1, %g1
	st	%g1, [%fp-20]
	b	.LL1
	 nop
.LL2:
	ld	[%fp+72], %g1
	cmp	%g1, 0
	bne	.LL4
	nop
	ld	[%fp+68], %g1
	add	%g1, -1, %g1
	mov	%g1, %o0
	mov	1, %o1
	call	_Z8ackermanii, 0
	 nop
	mov	%o0, %g1
	st	%g1, [%fp-20]
	b	.LL1
	 nop
.LL4:
	ld	[%fp+72], %g1
	add	%g1, -1, %g1
	ld	[%fp+68], %o0
	mov	%g1, %o1
	call	_Z8ackermanii, 0
	 nop
	mov	%o0, %o5
	ld	[%fp+68], %g1
	add	%g1, -1, %g1
	mov	%g1, %o0
	mov	%o5, %o1
	call	_Z8ackermanii, 0
	 nop
	mov	%o0, %g1
	st	%g1, [%fp-20]
.LL1:
	ld	[%fp-20], %i0
	ret
	restore
.LLFE2:
	.size	_Z8ackermanii, .-_Z8ackermanii
	.align 4
	.global main
	.type	main, #function
	.proc	04
main:
.LLFB3:
	!#PROLOGUE# 0
	save	%sp, -120, %sp
.LLCFI1:
	!#PROLOGUE# 1
	mov	6, %g1
	st	%g1, [%fp-20]
	mov	3, %o0
	ld	[%fp-20], %o1
	call	_Z8ackermanii, 0
	 nop
	mov	%o0, %g1
	st	%g1, [%fp-24]
	ld	[%fp-24], %g1
	mov	%g1, %i0
	ret
	restore
.LLFE3:
	.size	main, .-main
	.section	".eh_frame",#alloc,#write
.LLframe1:
	.uaword	.LLECIE1-.LLSCIE1
.LLSCIE1:
	.uaword	0x0
	.byte	0x1
	.asciz	"zP"
	.byte	0x1
	.byte	0x7c
	.byte	0xf
	.byte	0x8
	.byte	0x50
	.align 4
	.long	__gxx_personality_v0
	.byte	0xc
	.byte	0xe
	.byte	0x0
	.align 4
.LLECIE1:
.LLSFDE1:
	.uaword	.LLEFDE1-.LLASFDE1
.LLASFDE1:
	.uaword	.LLASFDE1-.LLframe1
	.uaword	.LLFB2
	.uaword	.LLFE2-.LLFB2
	.byte	0x0
	.byte	0x4
	.uaword	.LLCFI0-.LLFB2
	.byte	0xd
	.byte	0x1e
	.byte	0x2d
	.byte	0x9
	.byte	0xf
	.byte	0x1f
	.align 4
.LLEFDE1:
.LLSFDE3:
	.uaword	.LLEFDE3-.LLASFDE3
.LLASFDE3:
	.uaword	.LLASFDE3-.LLframe1
	.uaword	.LLFB3
	.uaword	.LLFE3-.LLFB3
	.byte	0x0
	.byte	0x4
	.uaword	.LLCFI1-.LLFB3
	.byte	0xd
	.byte	0x1e
	.byte	0x2d
	.byte	0x9
	.byte	0xf
	.byte	0x1f
	.align 4
.LLEFDE3:
	.ident	"GCC: (GNU) 3.4.2"

\end{verbatim}


\end{document}
