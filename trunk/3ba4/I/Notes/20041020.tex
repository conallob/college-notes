% $Id$

\documentclass[a4paper,12pt]{article}
\usepackage{amssymb}

\setlength{\parindent}{0mm}
\setlength{\parskip}{7.5mm}

\begin{document}

\title{Course 3BA4: Computer Architecture II \\ Lecture Notes \\ $20^{th}$ October 2004}

\maketitle

\section*{MISD - Pipelined Vector Processor or Systolic Array}

Diagram


Since the $PU$; are simple pipeline stages, then the instruction 
($IS$); are at most just a few bits and change slowly.

Eg:

$\vec{x}$, $\vec{y}$ vectors of length $1000$.

\begin{eqnarray*}
\vec{z} & \leftarrow & \vec{x} + 7.3 \\
\vec{w} & \leftarrow & \vec{y} + \vec{z} \\
\end{eqnarray*}

\section*{MIMD - Multiprocessor, Multcomputer, Dataflow}

Diagram


Eg: $y_{k} = f(x_{k})$

$a \geq x_{k} \geq b$, $k = 1 \cdots 10^{2}$


Find roots.


- Each $CU_{i}$/$PU_{i}$ applies Newton-Raphson to locate all the roots,
  ie $x_{k}$, st $f(x_{k}) = 0$ over $\frac{1}{n}^{th}$ of the range
  $[a,b]$.

\section*{Taxonomy Retrospective}

\begin{itemize}

\item SISD has flourised since c 1945

\item SIMD flourised c 1980 - 1990, but their poor multi-tasking and
difficult I/O integration eventually halted their development.

\item MISD has flourised as pipeline subsystems, ever since their
introduction in the CDC 6600, circa 1983.

\item MIMD has flourised ever since the introduction of the 32 bit
single chip microprocessor, circa 1987.
\end{itemize}


\section*{Implementation}

Note that the $PU$ itself may be implemented by any member of this
taxonomy, leading to a hierarchical (and sometimes recursive)
implementations in practice.


For example, typical PC subsystems incorporate specialised  SISDs, for
the applications, I/O and graphics activities, so it's $PU$ consists of
three SISDs. (See handout)

\end{document}
