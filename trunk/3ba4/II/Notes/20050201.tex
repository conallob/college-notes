% $Id: 20041108.tex 353 2004-11-12 00:21:25Z conall $

\documentclass[a4paper,12pt]{article}
\usepackage{amssymb}
\usepackage{ulem}
\usepackage{pstricks}

\setlength{\parindent}{0mm}
\setlength{\parskip}{7.5mm}

\begin{document}

\title{Course 3BA4: Computer Architecture II - VLSI Design \\ Lecture Notes \\ $1^{st}$ February 2005}

\maketitle

\section*{N-Type - P-Type MOSFETs}

\begin{figure}[hbtp]

% Transistor Diagram

\caption{n-type}

\end{figure}


\begin{figure}[hbtp]

% Transistor Diagram

\caption{p-type}

\end{figure}

\subsection*{Build Logic Gate}

\begin{itemize}

\item Good restoring properties. Weak signal in $\to$ strong signal out.

\item Good noise margins. Weak signal - voltage near middle. Strong
signal - voltage near extremes.

\end{itemize}

\subsection*{Key Idea}

Inputs go to gates only - MOSFETs

Outputs are connected via switches to $PWR$ and $GND$

\subsection*{Pragmatic Fit}

Keep n-type together, distinct from p-type.


\subsection*{Two Networks}

\begin{figure}[hbtp]

% Transistor Network Diagram

\caption{Pull Up and Pull Down Networks}

\end{figure}

\begin{figure}[hbtp]

% Diagram

\caption{P-type - P-type: Does not work very well}

\end{figure}

\subsection*{$AND$}

\begin{figure}[hbt] 
\centering 

\input 20050201-ANDGate

\caption{$AND$ Gate}

\label{AND Gate} 

\end{figure} 


\begin{tabular}{cc|c}
$a$	&	$b$	&	$F$	\\
\hline
$0$	&	$0$	&	$0$	\\
$0$	&	$1$	&	$0$	\\
$1$	&	$0$	&	$0$	\\
$1$	&	$1$	&	$1$	
\end{tabular}

\begin{figure}[hbtp]

% Transistor Diagram

\end{figure}

\subsubsection*{Problem}

$a$ and $b$ lead and lead switches are open. Will not work.

Pull ups need to be p-type.

Can't build logic $AND$ as simple standard gate.

Implement $AND$ as $NAND$, then invert.

\begin{figure}[hbtp]

\centering 

\input 20050201-InvertNANDGate

\label{Inverted NAND Gate} 

\end{figure}

\subsubsection*{What Do We Build And How?}

Observation (for \emph{Std Gate} - 

All inputs $0$. $\Rightarrow$ output $= 1$. \\
All inputs $1$. $\Rightarrow$ output $= 0$.

\begin{tabular}{ccccc}
						&	\hspace{10mm}	&	Functions	&	\hspace{10mm}	&	Do-able?	\\
All $0$s bar 1		&					&	$\overline{A}$	&		&	Yes	\\
				&	& $A \cdot ((B \cdot C) \cdot D)$		&		&	? Yes	\\
					&						&	$A \cdot B$		&		&	No	\\
					&						&	$A \oplus B$		&		&	No	\\
Something		&						&	$A \cdot ((B \cdot C) + D)$	&	&	No	\\
Something		&				&	$\overline{A} \cdot \overline{D}$	& 		& ? Yes
\end{tabular}

\subsubsection*{Any Function of the Form}

$\overline{AND}$, $\overline{OR}$ and inputs (not negations) is
buildable as a \textbf{single} Std gate.

$\overline{A} \cdot \overline{B} \to$ DeMorgan's Law $\to
\overline{A+B}$ is buildable. Partial answer to ? wha % missing part

How do we build this?

Pull Up:

Logical $AND$ - Parallel Connection \\
Logical $OR$  - Series Connection 

Pull Down:

Logical $AND$ - Series Connection \\
Logical $OR$  - Parallel Connection

\begin{figure}[tbtp]

% Transistor Network Diagram

\caption{$F = \overline{(A \cdot ((B \cdot C) + D)}$}

\end{figure}

\subsubsection*{Design Goals - Implement $f$}

$f(a, b, c) \to Y$

When inputs are such that $f(input) = 1$. The \emph{path} exists in a
pull up network and \emph{no path exists} in a pull down network.

\emph{Path exists} if there is some chain of closed FETs from top to
botton of the network.

Cannot have two paths.

\begin{figure}

% Transistor network Diagram

\end{figure}

Therefore complementary situation required if inputs are such that
$f(input) = 0$.

\begin{figure}

% NOT Gate Diagram

\end{figure}

\begin{figure}

% Transistor network Diagram


\caption{Path Up}

\end{figure}

What can we implement as a \emph{"standard CMOS gate"}?

Who do we implement the \emph{single}.

Example: $F(a, b) = \overline{a \cdot b}$.


\begin{figure}[hbt] 
\centering 

\input 20050201-NANDGate

\label{NAND Gate} 

\end{figure} 

\begin{figure}

% Transistor network Diagram

\caption{Parallel. Either $0$ or closed leave path.}

\end{figure}

If and only If:

\begin{tabular}{|c|c|c|}
\hline
$a$	&	$b$	&	$out$	\\
\hline
$0$	&	$0$	&	$1$	\\
\hline
$0$	&	$1$	&	$1$	\\
\hline
$0$	&	$0$	&	$1$	\\
\hline
$1$	&	$1$	&	$0$	\\
\hline
\end{tabular}

When either $a$ or $b$ are $0$, path going up.
\end{document}
