% $Id: 20041108.tex 353 2004-11-12 00:21:25Z conall $

\documentclass[a4paper,12pt]{article}
\usepackage{amssymb}
\usepackage{ulem}

\setlength{\parindent}{0mm}
\setlength{\parskip}{7.5mm}

\begin{document}

\title{Course 3BA4: Computer Architecture II - VLSI Design \\ Lecture Notes \\ $8^{th}$ February 2005}

\maketitle

\begin{table}[hbtp]

% 3D Diagram

\caption{N-Device}

\end{table}

$tox$ - Oxide Thickness
$\omega$ - Channel Width
$L$ - Channel Length \emph{along} direction of current flow.

\[ \beta_{N} = \mu_{N} \frac{\epsilon_{i}}{tox} \times \frac{\omega}{L}
= \mu_{N} C_{i} \frac{W}{L} \]

$\epsilon_{i}$ - Permitivity of the oxide insulator \\
$\mu_{N}$ - mobility of the electrons in $Si$.


Parallel plate capactiro of area $A$.

Plates are of distance $t$ apart.

$\frac{\epsilon}{t} =$ capacitance per unit area.

$C = \frac{\epsilon A}{t}$

\[ \beta_{P} = \mu_{P} \times \frac{\epsilon_{i}}{tox} \times
\frac{W}{L} = \mu_{P} \times C_{i} \times \frac{W}{L} \]


$\mu_{N} \times \frac{\epsilon_{i}}{tox}$ - Determined by manufacturer.

$\frac{W}{L}$ - Determined by design.


\[ \beta = \mu_{N} \times \frac{\epsilon_{i}}{tox} \times \frac{W}{L} \]


We can tune $\beta$s by adjusting $W$, $L$. We need to compensate for
differences between $\mu_{N}$ and $\mu_{P}$.


$\mu_{P}$ - Mobility of holes

Typically $\mu_{N} \approx 3 \mu_{P}$ (according to text books).
For modern \emph{"sub micron"} devices $\mu_{N} \approx 2 \mu_{P}$

\begin{table}[hbtp]

% transistor network diagram

\end{table}

$\beta_{P} = \beta_{N} \rightarrow$ we need to set $\frac{W_{P}}{L_{P}}
= 2 \frac{W_{N}}{L_{N}}$


Want devices as small as possible.

How small is that?

Depends on manufacturer - they specify \emph{"Minimum Failure Size"} -
call this $\lambda$.

Want to minimise channel length - Why? Device speed is related to
channel length.

Keep $L = \lambda$ for all devices. Assume $\lambda = 1$.

With all $L = \lambda$:

\[ W_{P} = 2 W_{N} \]


\begin{table}[hbtp]

% Transistor network diagram

\end{table}

$i_{DS} = \beta ( \cdots )$

$\beta \propto W$

$V = 1 R$ \\
$R = Y$ \\
$R \propto \frac{1}{\omega}$

\begin{table}[hbtp]

$\beta \propto \omega$

% transistor diagram

(very) $\approx$

$R \propto \frac{1}{\omega}$

% resistor diagram

\end{table}

Series connection, add resistance.

Parallel connection $R_{!} \parallel R{2} = \frac{R_{1} R_{2}}{R_{1} + R_{2}}$

Transistors in series

$n$ transistors in series, increase all widths by factor of $n$.

\begin{table}[hbtp]

% transistor network diagram

\caption{$n$ transistors in series}

\end{table}


\begin{table}[hbtp]

% transistor network diagram

\caption{$n$ transistors in parallel}

\end{table}

Not both always on at the same time. Which case do we use? Worst case -
one branch only is active. Therefore, we do not adjust for parallel
combinations.


\subsection*{Proceedure}

\begin{enumerate}

\item \subitem Set all n-devices $\omega = 1$
		
		\subitem Set all p-devices $\omega = 2$

\item For every series chain of length $n$, \textbf{multiply widths} in
chain by $n$. (Do this \textbf{once only} per transistor)

\item Scale down to minimum size.

\end{enumerate}


\begin{table}[hbtp]

% transistor network diagram

\caption{Ex: $NOR 2$ - in $NOR$ gates, p-width $\propto$ fan-in (no of
inputs)}

\end{table}


\begin{table}[hbtp]

% transistor network diagram

\caption{$NAND$ - Cheaper than $NOR$ (in terms of gate width)}

\end{table}


\section*{Tutorial Solutions}

\subsection*{1}

\[ F_{1} = \overline{A \cdot \left( B + C \right) } \]

\[ F_{2} = \overline{A \cdot \left( B + C \cdot D \right) } \]

Design CMOS circuit:

\indent - Network

\indent - Size Transistors

\begin{table}[hbtp]

% transistor network diagram

\caption{$F_{1} = \overline{A \cdot \left( B + C \right) }$}

\end{table}


\begin{table}[hbtp]

% transistor network diagram

\caption{$F_{2} = \overline{A \cdot \left( B + C \cdot D \right) }$}

\end{table}

\subsubsection*{Marking Scheme}

Begins at $10$

\begin{tabular}{ccc}
$-1$	&	\hspace{10mm}	&	Not scaled down											\\
		&						&																	\\
$-2$	&	\hspace{10mm}	&	Failed to do series multiplier						\\
		&						&																	\\
$-2$	&	\hspace{10mm}	&	Did it twice or more to a transistor				\\
		&						&																	\\
$-3$	&	\hspace{10mm}	&	Wrong Network												\\
		&						&																	\\
$-4$	&	\hspace{10mm}	&	p-devices in pull down or n-devices in pull up	\\
\end{tabular}

\subsection*{2}

$8$ Input $OR$ Gate

Minimise total sum of widths (areas). Value $< 70$ is possible


\begin{table}[hbtp]

% transistor network diagram

\end{table}


\begin{table}[hbtp]

% transistor network diagram

\end{table}

\end{document}
