% $Id: 20041108.tex 353 2004-11-12 00:21:25Z conall $

\documentclass[a4paper,12pt]{article}
\usepackage{amssymb}
\usepackage{ulem}

\setlength{\parindent}{0mm}
\setlength{\parskip}{7.5mm}

\begin{document}

\title{Course 3BA4: Computer Architecture II - VLSI Design \\ Lecture Notes \\ $7^{th}$ February 2005}

\maketitle

% Transistor Diagram

How does $i_{PS}$ relate to $V_{GS}$, $V_{DS}$?


(i) $V_{GS} \leq V_{T}$, $I_{DS} = 0$ Cut Off

(ii) $V_{GS} > V_{T}$ $V_{GS} - V_{T}$ ammount of excess \emph{"drive"}
over threshold.

\indent (a) $V_{DS} \leq V_{GS} - V_{T}$ \\ 
$i_{DS} = \beta \left( \left(V_{GS} - V_{T} \right) V_{DS} - \frac{V_{DS}^{2}}{2} 
\right)$

\indent (b) $V_{DS} \geq V_{GS} - V_{T}$ \\
$i_{DS} = \beta \left( \frac{\left( V_{GS} - V_{T} \right)^{2}}{2}
\right)$ Saturated

$-V_{GS} > V_{T} (1 unit) \Rightarrow V_{GS} - V_{T} = 1$ \\

$-V_{GS} > V_{T} (2 units) \Rightarrow V_{GS} - V_{T} = w$ \\

% Graph Diagram

\begin{table}[hbtp]

% Transistor Diagram

\caption{Compute $V_{OUT}$ as a function of $V_{IN}$}

\end{table}

$V_{GS1N} = V_{IN}$ \\
$V_{DS1N} = V_{OUT}$ \\
$V_{GS1P} = V_{IN} - V_{PWR}$ \\
$V_{DS1P} = V_{OUT} - V_{PWR}$ \\
$i_{DS1P} = - i_{DS1N}$ \\

$V_{T1N} \approx \frac{V_{PWR}}{5}$ \\

$V_{T1P} \approx - \frac{V_{PWR}}{5}$ \\

$V_{IN} = 0$

$\Rightarrow$

$V_{GS1N} = 0 \Rightarrow V_{DS1N} = 0 = i_{DS1P}$ N-Device CUT OFF 

$V_{GS1P} = 0 -V_{PWR} = - V_{PWR}$ P-Device ON

$V_{GS1P} \gg V_{T1P}$ (ON) but $i_{DS1P} = 0$

$\Rightarrow V_{DS1P} = 0$ - Equation (a)

$\Rightarrow V_{OUT} - V_{PWE}= 0$

$\Rightarrow V_{OUT} = V_{PWR}$ 

$V_{IN} = V_{PWR} \Rightarrow V_{OUT} = 0V (GND)$

$i_{DS1N} = i_{DS1P}$

Extremes look good.

Analysis works for $V_{IN} \leq V_{T1N}$ - Logic $0$ \\
$V_{IN} \geq V_{PWR} + V_{TP}$ - Logic $1$ \\

No current flows when input and output are good logic values (close to
power rails)

No power consumption while outputting fixed logic value

\emph{"Electrical Design of CMOS Logic Gates"} is simply a matter of
fixing the ratio of $\beta_{N}$ and $\beta_{P}$ (ideally
$\frac{B_{N}}{B_{P}} = 1$).

\begin{table}[hbtp]

% Graph Diagram

\end{table}

\end{document}
