% $Id: 20041109.tex 367 2004-11-23 10:22:38Z conall $

\documentclass[a4paper,12pt]{article}

\usepackage{graphicx}

\setlength{\parindent}{0mm}
\setlength{\parskip}{7.5mm}

\begin{document}

\title{Course 3BA3: Comunications \\ \vspace{10mm} Assignment 2 \\
\vspace{10mm} Is Mobile Communication the Future? \\ \vspace{10mm}
{\small Do you think the proliferation of mobile voice and data communication
technologies (Cellular Mobile Phones, WiFi, WiMax, Bluetooth, GPRS, etc)
coupled with Voice over IP (VoIP) technologies (Skype, etc) will make
traditional fixed telephony a thing of the past?}}

\author{Conall O'Brien \\ conallob@maths.tcd.ie \\ 01734351}

\maketitle

\section{Introduction}

In this paper, I will briefly outline the details of existing
technologies used by the telecommunications industry, for both
telephonic communications and in their most recent roles as internet
service providers.


I will also discuss the various wireless technologies and
\footnote{Voice over \footnote{Internet Protocol}{IP}}{VoIP} which are
gaining increased deployment and interest, particularly from the
telecommunication industry.

\part{Traditional Technologies}

\section{The Telecommunications Industry}

Over the years, telephone networks around the world converged together
to form the \footnote{Public Switched Telephone Network}{PSTN} as it
effectivly exists today. \cite[In the 1970s]{wikipedia-pstn}, the 
telecommunications industry begun researching replacing analogue
technologies with high capacity digital alternative, giving birth to
technologies such as 
\cite[\footnote{Digital Line Subscriber}{DSL} and 
\footnote{Integrated Systems Digital Network}{ISDN}]{wikipedia-pstn}.


The telecoms industry are always interesting in more efficient, highly
scalable systems to cater for the magnitude of the PSTN, while keeping
or improving \footnote{Quality of Service}{QoS} and throughput.

\subsection{Internet Service Providers (ISPs)}

Over the last 20 years, since the emergance of the Internet, the
telecoms industry have become \footnote{Internet Service
Providers}{ISPs}, providing \footnote{Internet Protocol}{IP}
connectivity over various mediums, such as copper transmission lines
(ordinary phone lines), fibre optic transmission lines, ISDN, etc, as
well as 
\footnote{Groupe Sp{\'e}cial Mobile/Global System for Mobile Communications}{GSM}
and 
\footnote{General Packet Radio Service}{GPRS} with mobile or
\quote{cell} phones

\part{Emerging Technologies}

\section{Wireless Technologies}

\subsection{Protocols}

The 802.11 \footnote{Wireless Fidelity}{WiFi} 
\footnote{Institute of Electrical and Electronics Engineers}{IEEE}
family of standards is a member of the 
\cite[\footnote{Media Access Control - The lower sublayer of the OSI Data Link
Layer}{MAC} sublayer]{tanenbaum}. The various standards (eg 802.11a, 802.11b, 
802.11g, 802.11i, etc...) are all closely related, mainly using the
\cite[2.4GHz or 5GHz]{wikipedia-80211} (in the case of 802.11a), thus
allowing the use of the TCP/IP networking model, replacing the Ethernet
MAC sublayer with radio waves. 

\subsection{Security Concerns}

Cognitively, one implicit, basic security feature of the 802.3 Ethernet protocol is it's
reliance on copper cables, which  can be given restricted network access
or physically restricted. The same cannot be said for the 802.11 WiFi
protocol. By it's very design, a malicious user has access to the
network, so other security methods are required.


One the last few years, wireless \footnote{Local Area Network}{LAN{
security reccommendations have changed more than once. The 
\cite[802.11 standards define specifications for built in encryption, entitled
\footnote{Wired Equivalent Privacy}{WEP}]{wikipedia-wep}. However,
concerns over the integrity of WEP, particulary events in \cite[2001,
including the University of California at Berkeley, a paper entitled
\quote{Weaknesses in the Key Scheduling Algorithm of RC4} by Fluhrer, 
Mantin, and Shamir and AT\&T]{wikipedia-80211} each highlighted various
issues with the WEP security model and choice of cypher algorithm.


At the time of writing this paper, the current de facto methods of
securing a wireless lan involve the use of \footnote{WiFi Protected
Access}{WPA} (which was created the result of WPA's shortcomings),
802.1x user authenticaion

\section{VoIP Technologies}

\subsection{Competing Protocols}

\subsection{Proliferation}

\section{Convergencing the Technologies}

\subsection{Mobile Telecommunications}

\subsection{Enhancing the Private Branck eXchange (PBX) Model}

\part{Conclusions}

\bibliographystyle{ieeetr}

\bibliography{WiFi-VoIP-Essay}

\end{document}
