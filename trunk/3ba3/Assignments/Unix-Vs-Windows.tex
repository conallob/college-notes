% $Id$

\documentclass[a4paper,12pt]{article}
\usepackage{amssymb}

\setlength{\parindent}{0mm}
\setlength{\parskip}{7.5mm}

\begin{document}

\title{3BA3 - Systems Software \\ Unix Vs Windows}

\author{Conall O'Brien \\ conallob@maths.tcd.ie \\ 01734351 }

\maketitle

\section*{Introduction}

In this essay, I will be comparing and contrasting the Unix operating
system to the Windows operating system. When I make reference to Unix, I
will be referring to all Unix style operating systems, including the BSD
(Berkeley Software Distribution) family (including FreeBSD, OpenBSD,
NetBSD, DrangonFly BSD, Darwin, Mac OS X and other minor forks of
these), commercial Unix derivatives (such as Sun's Solaris, SGI's IRIX,
IBM's AIX, Hewlett Packard's HP-UX, Compaq's TRU64, etc) and Unix
clones including Linux, GNU HURD and Minix.

My references to Windows will refer to all members of the Microsoft
Windows Fammily (including Windows 3.1/3.11, Windows 95/98/ME, Windows NT 
3/3.5/4, Windows 2000 and Windows XP). IBM's OS/2 Warp is not included
in this reference, despite it's close ties to the Windows 95/98/ME
dynasty, mainly due to it's small usage.

I will compare and contrast the Unix family against the Windows family
in a number of areas, including: Traditional Uses; Usability;
Reliability; Security; Compatibity; Programs Available and Total Cost of
Ownership. After discussing these aspects of both operating system 
families, using facts and references to backup my arguement, I will
conlude which operating system is the winner of my arguement.

\section*{Traditional Use}

\section*{Usability}

\section*{Reliability}

\section*{Security}

\section*{Compatiblity}

\section*{Programs}

\section*{Total Cost of Ownership}

\section*{Conclusion}

\end{document}
