% $Id$

\documentclass[a4paper,12pt]{article}
\usepackage{amssymb}

\usepackage[left=2cm,right=2cm,top=2cm,bottom=2cm]{geometry}

\setlength{\parindent}{0mm}
\setlength{\parskip}{5.0mm}

\begin{document}

\title{3BA3 - Systems Software \\ Unix Vs Windows \\ \vspace{20mm}}

\author{Conall O'Brien \\ conallob@maths.tcd.ie \\ 01734351 \\
\vspace{20mm}}

\maketitle

\newpage

\section*{Introduction}

In this essay, I will be comparing and contrasting the Unix operating
system to the Windows operating system. When I make reference to Unix, I
will be referring to all Unix style operating systems, including the BSD
(Berkeley Software Distribution) family (including FreeBSD, OpenBSD,
NetBSD, DragonflyBSD, Darwin, Mac OS X and other minor forks of
these), commercial Unix derivatives (such as Sun's Solaris, SGI's IRIX,
IBM's AIX, Hewlett Packard's HP-UX, Compaq's TRU64, etc) and Unix
clones including Linux, GNU HURD and Minix.

My references to Windows will refer to all members of the Microsoft
Windows Family (including Windows 3.1/3.11, Windows 95/98/ME, Windows NT 
3/3.5/4, Windows 2000 and Windows XP). IBM's OS/2 Warp is not included
in this reference, despite it's close ties to the Windows 95/98/ME
dynasty, mainly due to it's small usage.

I will compare and contrast the Unix family against the Windows family
in a number of areas, including: Traditional Uses; Usability;
Reliability; Security; Compatibility; Programs Available and Total Cost of
Ownership. After discussing these aspects of both operating system 
families, using facts and references to backup my argument, I will
conclude which operating system is the winner of my argument.

\section*{Traditional Use}

Traditionally, since it's widespread take-up in 1992 with Windows 3.1,
Windows has been used as primarily as a desktop operating system. Users 
would required to be physically in front of the system and the system 
would only be usable by the currently logged in user.

After 1995, with the release of the Windows NT strain of Windows, a
version specifically aimed at servers was introduced. Features added to
the server releases include Terminal Services, which allowed a user
to use the system without being physically in front of the system. This
feature has been enhanced in recent years to allow multiple users use the 
system remotely using Terminal Services.

Unix systems on the other hand have usually been first choices to power
servers since the 1970s when AT\&T first introduced Unix to the world.
It's flexible, modular design gave it the ability to scale to meet the
task given to it. During the 1980s, the original Unix releases suffered
due to the rising popularity of the BSD Unix derivative. A similar
uptake is currently visible with the migration from modern Unix 
derivatives to Linux, a free Unix clone which is the primary example of
Open Source Programming projects.

Unix derivative systems have constantly been multi user systems. Users
can use the system locally or remotely and are able to run tasks pseudo
simultaneously, independent of each other. As a result, until recent
years, Unix derivatives were almost exclusively for servers only. In
recent years, with the widespread use of X Windows implementations, Unix
has been used on high end workstations, from manufacturers like Sun and
SGI and more recently, Linux and modern BSD systems are becoming more
and more popular as desktop systems.

\section*{Usability}

Over the last 11 years, since Windows 3.1, the command line interface
existing from MS-DOS has been gradually phased out due to the belief
that a command line interface to an operating system is complicated and
has a sleep learning curve for new or inexperienced users. As a result,
Windows initially ran on top of MS-DOS, but by Windows 2000, the MS-DOs
prerequisite was removed and a MS-DOS emulator was introduced instead.
As a result, Windows is almost exclusively used via GUI interfaces,
primarily for their user friendliness, except for doing more complex
tasks or problem solving, when experienced users and administrators find
the command line interface to Windows much easier and more informative
than existing GUI windows.

Due to the common perception that a command line interface is
complicated and required learning curve, Unix derivatives are usually
considered hard to use. Due to this perception, various projects
(such as SuSE Linux's YaST\footnote{SuSE YaST -
http://www.suse.com/us/private/products/suse\_linux/prof/yast.html - 
$27^{th}$ October $2004$} - Yet Another Setup Tool) aim to be
comprehensive GUI wrappers to the Unix command line in order to provide
friendlier methods of using a system effectively via the command line.

\section*{Reliability}

For a number of decades, Unix derivate have kept their reputation in
terms of reliability. With minimal or even no special preparation, a
Unix derivative can typically have an uptime (time since last reboot) 
over 30 days. According to Netcraft\footnote{Netcraft - Top 50 Uptimes 
-  http://uptime.netcraft.com/up/today/top.avg.html - $26^{th}$ October
$2004$}, the highest uptime of all the servers it monitors all run a BSD
Unix derivative. However this ability in itself does not prove Unix's
reliability, but does demonstrate that it's clean, modular approach to 
it's kernel components and even how the system surrounding is laid out 
is the key to it's flexibility and reliability for numerous tasks.

On the other hand, Windows systems are not known for it's clean, modular
design. While for certain tasks can be performed regularly and maintain
a high uptime for a Windows system, uptime for a Windows machine is
always far lower than a Unix uptime. Discounting software errors and
bugs, official updates from Microsoft can often require a system reboot,
even if they don't directly affect the core services and the kernel.
However, the equivalent update for a Unix system will rarely affect the
system's kernel, unless it is a direct kernel update.

\section*{Security}

Windows has a bad reputation when it comes to security. This is the
result of a combination of non technical issues such as it's dominant 
market share and social engineering, combined with repetitive vulnerabilities 
(such as the Windows RPC component or the known 
list\footnote{Windows XP SP2 Compatibility List -
http://support.microsoft.com/default.aspx?kbid=842242\&product=windowsxpsp2
- $30^{th}$ October $2004$} of compatibility issues with Windows XP's
recently released Service Pack 2 aka SP2 which added long overdue security issues
some programs assumed were not an issue). The Windows mono user approach
also contributes to this, where too much simplification removes
functionality.

Unix systems on the other hand have a more trusted security model. The
multi user approach in Unix is also used to secure specific programs by
using a \emph{"chroot jail"}, which in effect means a daemon process on a
Unix server is run as a specific, pseudo user only used to run that
program. While security vulnerabilities do also exist for Unix systems,
it's modularity and interest groups ensure vulnerabilities are
thoroughly resolved and are applicable with little or no disruption to
most of the system.

\section*{Compatibility}

Due to the popularity of Intel's x86 (i386) and IA-64 (Itanium)
architectures (and AMD equivalents of x86 and AMD64) with computer 
manufacturers, Windows primarily is only available for x86 and soon
IA-64/AMD64 architectures. Windows previously was also available for
the DEC Alpha architecture as an incentive to migrate DEC Alpha Unix
systems to Windows. As a result of the popularity of these three main 
hardware architectures, there is little or no incentive for Microsoft to
make Windows compatible with other architectures.

The principle goal of Dennis Ritchie and Ken Thompson in AT\&T in the
1970s was to write an operating system which was compatible with
numerous legacy hardware architectures. To aid this goal, the
programming language C was born as a foundation for Unix. In modern times,
Unix derivatives, but especially Linux made it their aim to try and be
compatible with as many architecture types as possible. Currently, Linux
is the most architecture compatible Unix derivative, due to it's close
relationship with the GNU C Compiler (gcc). As a result, Linux is
compatible with these architectures\footnote{GCC Architecture
Compatibility List - http://gcc.gnu.org/install/specific.html - $30^{th}$
October $2004$} as well as being used in mobile phones, TiVo (a digital
television recording device) and numerous other embedded devices.

\section*{Programs}

Thanks mainly to the large market share Windows currently holds, Windows
has support for all the commonly used programs and formats. Program
suites such as Microsoft Office and Adobe's Photoshop are used for
general office documents to photo montage and graphics design.

Unix has a large amount of software available to be used, especially
since Unix users can take control of numerous choices available to them
to customise a system. Although Unix has a number of such programs and
various methods to support comprehensive, open formats, tools such as 
Samba (to share files over a network) and StarOffice/OpenOffice (a Unix
targeted Office Suite with support for Microsoft Office formats) exist
to bridge the gap between Windows and Unix.

\section*{Total Cost of Ownership}

Total cost of ownership is not an accurate or even comparable
measurement to compare operating systems by. The reason being that
running an operating system involves many variables which can influence
the running costs.

In order to run a Windows system properly, you must buy a license from
Microsoft, have modern hardware less than $5$ years old so the machine
can meet the minimum system requirements comfortably, an anti virus
package with a paid subscription for crucial updates and support, a
trusted firewall for protection from security issues such as internet
works, as well as 3rd party support for Windows, since Microsoft don't
offer various support contracts for Windows.

A Unix derivative, especially a Linux distribution for example, is
cheaper to buy, will be able to run on any hardware, no matter how old
thanks to Unix' compatibility, does not require an anti virus
(considering the small number of known Linux virii\footnote{Linux Vs
Windows Viruses -
http://www.theregister.co.uk/2003/10/06/linux\_vs\_windows\_viruses/ -
$30^{th}$ October $2004$}, already comes with an integrated firewall as
standard, has support from the distribution themselves from the start
and requires less human support hours for maintenance because it can be
accessed remotely and usually has one core, powerful  packaging system
which can install, remove or update almost any component of the system.

\section*{Conclusion}

To conclude, there is no \emph{"better"} operating system, in a
comparison between Unix derivatives and Windows. Both have good and bad
features, but the key factor should be \emph{"which operating system is 
better for task X?"}. 

Currently, Windows is a better choice for a desktop system, despite it's
security issues and it's TCO. Reliability is not really a deciding
factor for a desktop system, however usability and available programs
are.

Unix derivatives such as Mac OS X and other projects and Linux
distributions aim to improve Unix so it is also a viable desktop system 
alternative. Unix is a better choice for servers and other core system. 
It is easily scalable, reliable, compatible with any hardware and requires 
less attention from support staff. Currently desktop Unix systems are
still in heavy development in order to simplify their use as a
mainstream desktop system.

Windows systems are currently better for general use by a user not
asking too much of the system and for compatability with de facto
standards used today. Unix desktops are better for technical 
orientated work, high performance operations and for power users 
interested in how the system works and how to achieve better system 
performance through customisation.

\end{document}
