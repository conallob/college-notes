% $Id$

\documentclass[a4paper,10pt]{article}
\usepackage{amssymb}

\usepackage[left=2cm,right=2cm,top=2cm,bottom=2cm]{geometry}

\setlength{\parindent}{0mm}
\setlength{\parskip}{5.0mm}

\begin{document}

\title{3BA3 - Systems Software \\ Unix Vs Windows}

\author{Conall O'Brien \\ conallob@maths.tcd.ie \\ 01734351 }

\maketitle

\section*{Introduction}

In this essay, I will be comparing and contrasting the Unix operating
system to the Windows operating system. When I make reference to Unix, I
will be referring to all Unix style operating systems, including the BSD
(Berkeley Software Distribution) family (including FreeBSD, OpenBSD,
NetBSD, DrangonFly BSD, Darwin, Mac OS X and other minor forks of
these), commercial Unix derivatives (such as Sun's Solaris, SGI's IRIX,
IBM's AIX, Hewlett Packard's HP-UX, Compaq's TRU64, etc) and Unix
clones including Linux, GNU HURD and Minix.

My references to Windows will refer to all members of the Microsoft
Windows Fammily (including Windows 3.1/3.11, Windows 95/98/ME, Windows NT 
3/3.5/4, Windows 2000 and Windows XP). IBM's OS/2 Warp is not included
in this reference, despite it's close ties to the Windows 95/98/ME
dynasty, mainly due to it's small usage.

I will compare and contrast the Unix family against the Windows family
in a number of areas, including: Traditional Uses; Usability;
Reliability; Security; Compatibity; Programs Available and Total Cost of
Ownership. After discussing these aspects of both operating system 
families, using facts and references to backup my arguement, I will
conlude which operating system is the winner of my arguement.

\section*{Traditional Use}

Traditionally, since it's widespead takeup in 1992 with Windows 3.1,
Windows has been used as primarily as a desktop operating system. Users 
would reuired to be physically in front of the system and the system 
would only be usable by the currently logged in user.

After 1995, with the release of the Windows NT strain of Windows, a
version specifically aimed at servers was introduced. Features added to
the server releases include Terminal Services, which allowed a user
to use the system without being physically in front of the system. This
feature has been enhanced in recent years to allow multiple users use the 
system remotely using Terminal Services.

Unix systems on the other hand have usually been first choices to power
servers since the 1970s when AT\&T first introduced Unix to the world.
It's flexible, modular design gave it the ability to scale to meet the
task given to it. During the 1980s, the original Unix releases suffered
due to the rising popularity of the BSD Unix derivative. A similar
uptake is currently visible with the migration from modern Unix 
derivatives to Linux, a free Unix clone which is the primary example of
Open Source Programming projects.

Unix derivative systems have constantly been multi user systems. Users
can use the system locally or remotely and are able to run tasks pseudo
simulatneously, independant of each other. As a result, until recent
years, Unix derivatives were almost exclusively for servers only. In
recent years, with the widepsread use of X Windows implementations, Unix
has been used on high end workstations, from manufacturers like Sun and
SGI and more recently, Linux and modern BSD systems are becoming more
and more popular as desktop systems.

\section*{Usability}

Over the last 11 years, since Windows 3.1, the command line interface
existing from MS-DOS has been gradually phased out due to the belief
that a command line interface to an operating system is complicated and
has a sleep learning curve for new or inexperienced users. As a result,
Windows initially ran on top of MS-DOS, but by Windows 2000, the MS-DOs
prereqresuite was removed and a MS-DOS emuator was introduced instead.
As a result, Windows is almost exclusively used via GUI interfaces,
primarily for their user friendlyness, except for doing more complex
tasks or problem solving, when expreienced users and administrators find
the command line interface to Windows much easier and more informative
than existing GUI windows.

Due to the comman perseption that a command line interface is
complicated and required learning curve, Unix derivatives are usually
considered hard to use. Due to this perception, various projects
(such as SuSE Linux's YaST\footnote{SuSE YaST -
http://www.suse.com/us/private/products/suse\_linux/prof/yast.html - 
$27^{th}$ October $2004$} - Yet Another Setup Tool) aim to be
comprehensive GUI wrappers to the Unix command line in order to provide
friendlier methods of using a system effectively via the command line.

\section*{Reliability}

For a number of decades, Unix derivates have kept their reputation in
terms of reliability. With minimal or even no special preparation, a
Unix derivative can typically have an uptime (time since last reboot) 
over 30 days. According to Netcraft\footnote{Netcraft - Top 50 Uptimes 
-  http://uptime.netcraft.com/up/today/top.avg.html - $26^{th}$ October
$2004$}, the highest uptime of all the servers it monitors all run a BSD
Unix derivative. However this ability in itself does not prove Unix's
reliability, but does demonstrate that it's clean, modular approach to 
it's kernel components and even how the system surrounding is layed out 
is the key to it's flexability and reliability for numerous tasks.

On the other hand, Windows systems are not known for it's clean, modular
design. While for certain tasks can be performed regularly and maintain
a high uptime for a Windows system, uptime for a Windows machine is
always far lower than a Unix uptime. Discounting software errors and
bugs, official updates from Microsoft can often require a system reboot,
even if they don't directly affect the core services and the kernel.
However, the equivalent update for a Unix system will rarely affect the
system's kernel, unless it is a direct kernel update.

\section*{Security}

Virii, worms, trojans. Security patches for patches for patches.

Multi user system - Single User Mode - chroot jails.

\section*{Compatiblity}

Windows - x86, IA64, Alpha. 3rd party driver support

Unix - Everything, mobile phones, TiVo, Embedded devices. Open drivers

\section*{Programs}

Windows - Popular - Widely used - Mainstream software

Unix - Core user base - Creating tools to work with the rest of the
world - Samba, OpenOffice/StarOffice

\section*{Total Cost of Ownership}

B\$.

Windows Licence + supported hardware + av + firewall + human support
hours. No generic support from M\$

Unix - No Licence - Any hadware + no av + included firewall + minimal
supprt time. - Support contract available from distros

\section*{Conclusion}

\end{document}
