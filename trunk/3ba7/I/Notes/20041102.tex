% $Id: 20041026.tex 336 2004-10-27 12:47:38Z conall $

\documentclass[a4paper,12pt]{article}
\usepackage{amssymb}

\setlength{\parindent}{0mm}
\setlength{\parskip}{7.5mm}

\begin{document}

\title{Course 3BA7: Compiler Design I \\ Additional Lecture Notes \\ $2^{nd}$ November 2004}

\maketitle

\section*{A Grammar for Arithmetic Expressions}

\subsection*{Productions}

\begin{eqnarray}
\langle E \rangle		&	\to	&	\langle T \rangle		\\
\langle E \rangle		&	\to	&	\langle T \rangle		\\
\langle T \rangle		&	\to	&	\langle T \rangle \ast \langle P \rangle \\
\langle T \rangle		&	\to	&	\langle P \rangle \\
\langle P \rangle		&	\to	&	\left( \langle E \rangle \right) \\
\langle P \rangle		&	\to	&	CONST
\end{eqnarray}

This is the $\langle E \rangle + \langle T \rangle$ grammar.

$\langle E \rangle$ is an Expression. \\
$\langle T \rangle$ is a Term. \\
$\langle F \rangle$ is a Factor (not shown). \\
$\langle P \rangle$ is a Primary.


Given this grammar, we should be able to derive the expression $1 + 2
\ast 3 + 4$.

\begin{eqnarray*}
\langle E \rangle	&	\stackrel{1}{\Rightarrow}	& \langle E \rangle + \langle T \rangle \\
						&	\stackrel{1}{\Rightarrow}	& \langle E \rangle + \langle T \rangle + \langle T \rangle \\
						&	\stackrel{2}{\Rightarrow}	& \langle T \rangle + \langle T \rangle + \langle T \rangle \\
						&	\stackrel{4}{\rightarrow}	& \langle P \rangle + \langle T \rangle + \langle T \rangle \\
						&	\stackrel{6}{\rightarrow}	& CONST_{1} + \langle T \rangle + \langle T \rangle \\
						&	\stackrel{3}{\rightarrow}	& CONST_{1} + \langle T \rangle \ast \langle P \rangle + \langle T \rangle \\
						&	\stackrel{6}{\rightarrow}	& CONST_{1} + CONST_{2} \ast \langle P \rangle + \langle T \rangle \\
						&	.	& \\
						&	.	&	\\
						&	.	&	\\
						&	\stackrel{*}{\rightarrow}	& CONST_{1} + CONST_{2} \ast CONST_{3} + CONST_{4} \\
\end{eqnarray*}

\subsection*{Derivation Tree (Parse Tree)}

% Diagram

LR Parsing: Scan input left to right and produce a reverse of the
rightmost % something - Ask irokie

\section*{Another Grammar for Arithmetic Expressions}

Consider

\begin{eqnarray}
\setcounter{equation}{1}
\langle E \rangle		& \to	& \langle E \rangle + \langle T \rangle \\
\langle E \rangle		& \to	& \langle T \rangle \\
\end{eqnarray}

\begin{eqnarray*}
\langle E \rangle	&	\stackrel{}{\Rightarrow}	&	\langle E \rangle + \langle T \rangle \\
						&	\stackrel{1}{\Rightarrow}	&	\langle E \rangle + \langle T \rangle + \langle T \rangle \\
						&	\stackrel{2}{\Rightarrow}	&	\langle T \rangle + \langle T \rangle + \langle T \rangle 
\end{eqnarray*}

\subsection*{Productions}

\begin{eqnarray}
\setcounter{equation}{1}
\langle E \rangle			& 	\to	& \langle T \rangle \langle E-List \rangle \\
\langle E-List \rangle	&	\to	& + \langle T \rangle \langle E-List \rangle \\
\langle E-List \rangle	&	\to	&	\epsilon 
\end{eqnarray}

\begin{eqnarray*}
\langle E \rangle	&	\Rightarrow	&	\langle T \rangle + \langle E-List \rangle \\
						&	\Rightarrow	&	\langle T \rangle + \langle T \rangle + \langle E-List \rangle \\
						&	\Rightarrow	&	\langle T \rangle + \langle T \rangle + \langle T \rangle \langle E-List \rangle \\ 
						&	\Rightarrow	&	\langle T \rangle + \langle T \rangle + \langle T \rangle 
\end{eqnarray*}

Consider

\begin{eqnarray}
\langle T \rangle	& 	\to	& \langle T \rangle \langle P \rangle \\
\langle T \rangle	&	\to	& \langle P \rangle \\
\end{eqnarray}

\begin{eqnarray*}
\langle T \rangle	&	\Rightarrow	&	\langle P \rangle \ast \langle P
\rangle \ast \langle P \rangle 
\end{eqnarray*}

\subsection*{The $E-Link$ Grammar}

\begin{eqnarray}
\setcounter{equation}{1}
\langle E \rangle			& 	\to	& \langle T \rangle \langle E-List \rangle \\
\langle E-List \rangle	&	\to	& + \langle T \rangle \langle E-List \rangle \\
\langle E-List \rangle	&	\to	& \epsilon \\
\langle T \rangle 		&	\to	& \langle P \rangle \langle T-List \rangle \\
\langle T-List \rangle	&	\to	& \ast \langle P \rangle \langle T-List \rangle \\
\langle P \rangle			&	\to	& \left( \langle E \rangle \right) \\
\langle P \rangle			&	\to	& COUNT
\end{eqnarray}

eg:

\begin{eqnarray*}
\langle T \rangle	&	\Rightarrow	&	\langle P \rangle \\
						&	\Rightarrow	&	\langle P \rangle \ast \langle P \rangle \langle T-List \rangle \\
						& 	\Rightarrow	& 	\langle P \rangle \ast \langle P \rangle \ast \langle P \rangle \ast \langle T-List \rangle \\
						&	\Rightarrow	&	\langle P \rangle \ast \langle P \rangle \ast \langle P \rangle \\ 
\end{eqnarray*}

\section*{Exercise}

\begin{eqnarray*}
\langle E \rangle	&	\stackrel{1}{\Rightarrow}	&	\langle T \rangle \langle E-List \rangle \\
						&	\stackrel{4}{\Rightarrow}	&	\langle P \rangle \langle T-List \rangle \langle E-List \rangle \\
						&	\stackrel{2}{\Rightarrow}	&	COUNT_{1} \langle T-List \rangle \langle E-List \rangle \\
						&	\stackrel{6}{\Rightarrow}	&	COUNT_{1} \langle E-List \rangle \\
						&	\stackrel{2}{\Rightarrow}	&	COUNT_{1} + \langle T \rangle \langle E-List \rangle \\
						&	\stackrel{6}{\Rightarrow}	&	COUNT_{1} + \langle P \rangle \langle T-List \rangle \langle E-List \rangle \\
						&	\stackrel{8}{\Rightarrow}	&	COUNT_{1} + COUNT_{2} \langle T-List \rangle \langle E-List \rangle \\
						&	\stackrel{ }{\Rightarrow}	&	COUNT_{1} + COUNT_{2} \ast \langle T-List \rangle \langle E-List \rangle \\
						&	.	&	\\
						&	.	&	\\
						&	.	&	\\
						&	\stackrel{ }{\Rightarrow}	&	COUNT_{1} + COUNT_{2} \ast COUNT_{3} \ast COUNT_{4}
\end{eqnarray*}

\subsection*{Derivation Tree}

% Diagram

\end{document}
