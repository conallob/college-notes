% $Id: 20041026.tex 336 2004-10-27 12:47:38Z conall $

\documentclass[a4paper,12pt]{article}
\usepackage{amssymb}
\usepackage[usenames]{color}
\usepackage{pst-all}
\usepackage{lscape}

\setlength{\parindent}{0mm}
\setlength{\parskip}{7.5mm}


\newcommand{\actionsym}[1]{{\color{Emerald}{\{$#1$\}}}}
\newcommand{\nonterminal}[1]{$\langle #1 \rangle$}

\begin{document}

\title{Course 3BA7: Compiler Design I \\ Additional Lecture Notes \\ $4^{th}$ November 2004}

\maketitle

A language may be described by many different context free grammars.

A language is defined by a context free grammar and a set of semantic
restrictions.


\begin{verbatim}
FOR I := M TO N
      DO -...
\end{verbatim}

What is the value of \verb!I!?


\begin{eqnarray*}
\langle E \rangle		&	\to			& \langle E \rangle \langle OP \rangle \langle T \rangle \\
\langle E \rangle		&	\to			& \langle T \rangle \\
\langle OP \rangle	&	\to			& + \\
\langle OP \rangle	&	\to			& OR \\
\langle T \rangle		&	\to			& IDENT \\
							&					&		\\
\langle E \rangle 	&	\Rightarrow	& A + B  \hspace{2mm} OR \hspace{2mm} C
\end{eqnarray*}

\section*{Translation Grammars}

A translation grammar is a context free grammar where the set of
terminal symbols is partitioned into a set of input symbols and a set of
action symbols.

Strings in the language described by a translation grammar are called
action sequences.

Eg:

Design a grammar to convert arithmetic expressions from infix to postfix
form.

$A + B \ast C + D \Rightarrow ABC \ast + D +$

$A$ \actionsym{A} $+ B$ \actionsym{B} $ \ast \hspace{2mm} C $
\actionsym{C} \actionsym{\ast} \actionsym{+} $+ D$
\actionsym{D} \actionsym{+}

This is an action sequence.

$A$ - Terminal Symbol \\
\actionsym{A} - Action Symbol (in this case, to output the symbol
$A$)

A context free grammar may be converted into a translation grammar by
inserting action symbols at appropriate locations in the right hand side
of reductions from the context free grammar.

\begin{eqnarray*}
\langle E \rangle		&	\to	&	\langle E \rangle + \langle T \rangle \mbox{\actionsym{+}} \\
\langle E \rangle		&	\to	&	\langle T \rangle \\
\langle T \rangle		&	\to	&	\langle T \rangle \ast \langle P \rangle \mbox{\actionsym{\ast}} \\
\langle T \rangle		&	\to	&	\langle P \rangle \\ 
\langle P \rangle		&	\to	&	\left( \langle P \rangle \right) \\
\langle P \rangle		&	\to	&	IDENT \mbox{\actionsym{IDENT}} \\
\langle P \rangle		&	\to	&	A \mbox{\actionsym{A}} \\
\langle P \rangle		&	\to	&	B \mbox{\actionsym{B}} \\
\langle P \rangle		&	\to	&	C \mbox{\actionsym{C}} \\
\langle P \rangle		&	\to	&	D \mbox{\actionsym{D}}
\end{eqnarray*}

\begin{eqnarray*}
\langle E \rangle		&	\Rightarrow	&	\langle E \rangle + \langle T \rangle \mbox{\actionsym{+}} \\
\langle E \rangle		&	\Rightarrow	&	\langle E \rangle + \langle T \rangle \mbox{\actionsym{+}} + \langle T \rangle \mbox{\actionsym{+}} \\
\langle T \rangle		&	\Rightarrow	&	\langle T \rangle + \langle T \rangle \mbox{\actionsym{+}} + \langle T \rangle \mbox{\actionsym{1}} \\
							&	\Rightarrow	&	\langle P \rangle + \langle T \rangle \mbox{\actionsym{+}} + \langle T \rangle \mbox{\actionsym{+}} + \langle T \rangle \mbox{\actionsym{+}} \\
							&	\Rightarrow	&	IDENT \mbox{\actionsym{IDENT}} + \langle T \rangle \mbox{\actionsym{+}} + \langle T \rangle \mbox{\actionsym{+}} \\
							&	\Rightarrow	&	IDENT \mbox{\actionsym{IDENT}} + \langle T \rangle \ast \langle P \rangle \mbox{\actionsym{\ast}} \mbox{\actionsym{+}} + \langle T \rangle \mbox{\actionsym{+}}
\end{eqnarray*}

\begin{figure}[ht]

\psset{levelsep=15mm,nodesep=1mm}

\pstree{\TR{\nonterminal{E}}}{
	\pstree{\TR{\nonterminal{E}}}{
		\pstree{\TR{\nonterminal{E}}}{
			\pstree{\TR{\nonterminal{T}}}{
				\pstree{\TR{\nonterminal{P}}}{
					\TR{$A$}
					\TR{\actionsym{A}}
				}
				\Tn
			}
			\Tn
		}
		\TR{$+$}
		\pstree{\TR{\nonterminal{T}}}{
			\pstree{\TR{\nonterminal{T}}}{
				\pstree{\TR{\nonterminal{P}}}{
					\TR{$B$}
					\TR{\actionsym{B}}
				}
			}
		\TR{$\ast$}
		\pstree{\TR{\nonterminal{P}}}{
			\TR{$C$}
			\TR{\actionsym{C}}
		}
		\TR{\actionsym{\ast}}
	}
	\TR{\actionsym{+}}
	}
	\TR{$+$}
	\pstree{\TR{\nonterminal{T}}}{
		\pstree{\TR{\nonterminal{P}}}{
			\TR{$D$}
			\TR{\actionsym{D}}
		}
	}
	\TR{\actionsym{$+$}}
}

\caption{This is an example of a syntax directed translation.}

\end{figure}


A translation grammar whole all the action symbols specifically output
actions is termed a string translation grammar.

\begin{eqnarray*}
\langle E \rangle			& \to	&	\langle T \rangle \langle E-List \rangle \\
\langle E-List \rangle	& \to	&	+ \langle T \rangle \langle E-List \rangle \\
\langle E-List \rangle	& \to &	\epsilon \\
\langle T \rangle			& \to	&	\langle P \rangle \langle T-List \rangle \\
\langle T-List \rangle	& \to	&	\ast \langle P \rangle \langle T-List \rangle \\
\langle T-List \rangle	& \to	&	\epsilon \\
\langle P \rangle			& \to	&	\left( \langle E \rangle \right) \\
\langle P \rangle			& \to &	IDENT
\end{eqnarray*}

\begin{eqnarray*}
\langle E \rangle			& \to	&	\langle T \rangle \langle E-List \rangle \\
\langle E-List \rangle	& \to	&	+ \langle T \rangle \mbox{\actionsym{+}} \langle E-List \rangle \\
\langle E-List \rangle	& \to &	\epsilon \\
\langle T \rangle			& \to	&	\langle P \rangle \mbox{\actionsym{\ast}} \langle T-List \rangle \\
\langle T-List \rangle	& \to	&	\ast \langle P \rangle \langle T-List \rangle \\
\langle T-List \rangle	& \to	&	\epsilon \\
\langle P \rangle			& \to	&	\left( \langle E \rangle \right) \\
\langle P \rangle			& \to &	IDENT
\end{eqnarray*}

$A + B \ast C + D$

\begin{eqnarray*}
\langle E \rangle		&	\Rightarrow	&	\langle T \rangle	\langle E-List \rangle \\
							&	\Rightarrow	&	\langle P \rangle	\langle T-List \rangle \langle E-List \rangle \\
							&	\Rightarrow	&	\langle P \rangle \langle T-List \rangle + \langle T \rangle \langle E-List \rangle \\
							&	\Rightarrow	&	\langle P \rangle \langle T-List \rangle +
\end{eqnarray*}


$A + B$

\begin{eqnarray*}
\langle E \rangle	&	\Rightarrow	&	\langle T \rangle \langle E-List \rangle \\
						&	\Rightarrow	&	\langle P \rangle \langle T-List \rangle \langle E-List \rangle \\
						&	\Rightarrow &	IDENT \mbox{\actionsym{IDENT}} \langle T-List \rangle \langle E-List \rangle \\
						&	\Rightarrow	&	IDENT \mbox{\actionsym{IDENT}} \langle E-List \rangle \\
						& \rightarrow	&	IDENT \mbox{\actionsym{IDENT}} + \langle T \rangle \mbox{\actionsym{+}} \langle E-List \rangle
\end{eqnarray*}

\begin{figure}[ht]

\psset{levelsep=15mm,nodesep=1mm}

\pstree{\TR{\nonterminal{E}}}{
	\pstree{\TR{\nonterminal{T}}}{
		\pstree{\TR{\nonterminal{P}}}{
			\pstree{\TR{$IDENT$ \actionsym{IDENT}}}{
				\TR{$A$ \actionsym{A}}
				\Tn
			}
			\Tn
		}
		\pstree{\TR{\nonterminal{T-List}}}{
			\TR{$\epsilon$}
		}
	}
	\pstree{\TR{\nonterminal{E-List}}}{
		\pstree{\TR{\nonterminal{T}}}{
			\pstree{\TR{\nonterminal{P}}}{
				\pstree{\TR{$IDENT$ \actionsym{IDENT}}}{
					\TR{$B$ \actionsym{B}}
					\Tn
				}
				\Tn
			}
			\pstree{\TR{\nonterminal{T-List}}}{
				\TR{$\epsilon$}
			}
		}
		\TR{\actionsym{+}}
		\pstree{\TR{\nonterminal{E-List}}}{
			\TR{$\epsilon$}
		}
	}
}

\caption{$A + B \Rightarrow A B +$}

\end{figure}

\begin{landscape}

\begin{figure}[ht]

\psset{levelsep=15mm,nodesep=1mm}

\pstree{\TR{\nonterminal{E}}}{
	\pstree{\TR{\nonterminal{T}}}{
		\pstree{\TR{\nonterminal{P}}}{
   		\TR{$ID$}
			\TR{\actionsym{ID}}
		}
		\pstree{\TR{\nonterminal{T-List}}}{
   		\Tn
			\TR{$\epsilon$}
		}
	}
	\pstree{\TR{\nonterminal{E-List}}}{
		\psset{levelsep=20mm,nodesep=1mm}
		\TR{$+$}
		\pstree{\TR{\nonterminal{T}}}{
			\psset{levelsep=15mm,nodesep=1mm}
			\pstree{\TR{\nonterminal{P}}}{
   			\TR{$ID$}
				\TR{\actionsym{ID}}
			}	
			\pstree{\TR{\nonterminal{T-List}}}{
				\psset{levelsep=15mm,nodesep=1mm}
   			\TR{$\ast$}
				\pstree{\TR{\nonterminal{P}}}{
					\TR{$ID$}
					\TR{\actionsym{ID}}
				}
				\TR{\actionsym{\ast}}
				\pstree{\TR{\nonterminal{T-List}}}{
					\Tn
					\TR{$\epsilon$}
				}
			}	
		}
		\TR{\actionsym{+}}
		\pstree{\TR{\nonterminal{E-List}}}{
			\psset{levelsep=15mm,nodesep=1mm}
   		\TR{$+$}
			\pstree{\TR{\nonterminal{T}}}{
				\pstree{\TR{\nonterminal{P}}}{
					\TR{$ID$}
					\TR{\actionsym{ID}}
				}
				\pstree{\TR{\nonterminal{T-List}}}{
					\Tn
					\TR{$\epsilon$}
			}
			\TR{\actionsym{+}}
			\pstree{\TR{\nonterminal{E-List}}}{
				\Tn
				\TR{$\epsilon$}
			}
		}
	}
}
}

\caption{$A + B \ast C + D \Rightarrow A B C \ast + D +$}

\end{figure}

\end{landscape}

\vspace{10mm}

\begin{figure}[ht]

\psset{levelsep=20mm,nodesep=1mm}

\pstree{\TR{$+$}}{
	\pstree{\TR{$+$}}{
		\TR{$A$}
		\pstree{\TR{$\ast$}}{
			\TR{$B$}
			\TR{$C$}
		}
	}
	\TR{$D$}
}


\caption{Translation Tree}

\end{figure}

\end{document}
