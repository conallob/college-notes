% $Id$

\documentclass[a4paper,12pt]{article}
\usepackage{amssymb}

\setlength{\parindent}{0mm}
\setlength{\parskip}{7.5mm}

\begin{document}

\title{Course 3BA7: Compiler Design I \\ Additional Lecture Notes \\ $12^{th}$ October 2004}

\maketitle

David Abrahamson

Semester 1 - Compiler Design

david@cs.tcd.ie

6081716

ORI F11

2 Exercises - $20\%$ of $\frac{1}{2}$ 3BA7

No requirement to pass.

1 or more exam questions on exam based upon coursework.


\section{Language Processors}

Piece of software which produces an output language from an input
language.


Interpreter - Source Language $\to$ Execution

Translater -  Source Language $\to$ Object Language

(i) Assemblers
	
	- Low Level Language
	  1 to 1 translations

(ii) Compiler
	
	- High Level Languages
	  1 to many translations


% Diagram

Simple model of compiler

\section{Lexical Analysis}

Splits input into a sequence of lexical tokens

(Class),(Class,Variable),(Class,Value)


eg

\begin{verbatim}

	if x > y then
		A := B + C * D;

\end{verbatim}

Lexical Element $\equiv$ Lexem

%\begin{verbatim}

(IF) (IDENT,$\uparrow$ x) (RELOP,GT) (IDENT,$\uparrow$ y) (THEN) ...

%\end{verbatim}

\section{Syntax Analysis}

Verifies that the sequence of lecical tokens is syntactically correct
and translates them into a sequence of atoms that more closely refect
the order of execution at runtime.


Input to syntax analyser:

%\begin{verbatim}

(IDENT,$\uparrow$ A), (ASSIGN), (IDENT,$\uparrow$ B), (OP, PLUS),
(IDENT,$\uparrow$ C), (OP, TIMES), (IDENT,$\uparrow$ D)

A := B + C * D

%\end{verbatim}

Output:

%\begin{verbatim}

MULT($\uparrow$ C, $\uparrow$ D, $\uparrow TR_{1}$)
ADD($\uparrow$ B,$\uparrow TR_{1}$, $\uparrow TR_{2}$)
STORE($\uparrow$ A, $\uparrow$ $TR_{2}$)

%\end{verbatim}


\end{document}
