% $Id$

\documentclass[a4paper,12pt]{article}
\usepackage{amssymb}
\usepackage[usenames]{color}
\usepackage{pst-all}

%
% Crazy Tree Diagram Arrows Still TO DO
%

\setlength{\parindent}{0mm}
\setlength{\parskip}{7.5mm}

\newcommand{\nonterminal}[1]{$\langle #1 \rangle$}
\newcommand{\actionsym}[1]{{\mbox{\color{Emerald}{\{$#1$\}}}}}

\newcommand{\synth}[1]{\mbox{{\color{Maroon}{$#1$}}}}


\begin{document}

\title{Course 3BA7: Compiler Design I \\ Additional Lecture Notes \\ $9^{th}$ November 2004}

\maketitle

Synthesized Attributes:

$\left( 1 + 2 \right) \ast \left( 3 + 4 \right) $

\begin{eqnarray*}
\langle S \rangle	&	\to	&	\langle E \rangle \mbox{\actionsym{Answer}} \\
						&			&  \hspace{5mm} \synth{p} \hspace{10mm} \synth{q} \hspace{10mm} \synth{p \leftarrow q}	\\
\langle E \rangle	&	\to	&	\langle E \rangle + \langle T \rangle	\\
\hspace{5mm}\synth{p}	&			&	\hspace{5mm} \synth{q} \hspace{10mm} \synth{r} \hspace{10mm} \synth{p \leftarrow q + r} \\
\langle E \rangle	&	\to	&	\langle T \rangle \\
\hspace{5mm}\synth{p}	&			&	\hspace{5mm} \synth{q} \hspace{23mm} \synth{p \leftarrow q} \\
\langle T \rangle	&	\to	&	\langle T \rangle \ast \langle P \rangle \\
\hspace{5mm}\synth{p}	&			&	\hspace{5mm} \synth{q} \hspace{10mm} \synth{r} \hspace{10mm} \synth{p \leftarrow q \ast r} \\
\langle T \rangle	&	\to	&	\langle P \rangle \\
\hspace{5mm}\synth{p}	&			&	\hspace{5mm} \synth{q} \hspace{23mm} \synth{p \leftarrow q} \\
\langle P \rangle	&	\to	&	\left( \langle E \rangle \right) \\
\hspace{5mm}\synth{p}	&			&	\hspace{5mm} \synth{q} \hspace{23mm} \synth{p \leftarrow q} \\
\langle P \rangle	&	\to	&	CONST \\
\hspace{5mm}\synth{p}	&			&	\hspace{5mm} \synth{q} \hspace{23mm} \synth{p \leftarrow q} \\
\end{eqnarray*}

\begin{figure}[ht]

\psset{levelsep=15mm,nodesep=1mm}

\pstree{\TR{\nonterminal{S}}}{
	\pstree{\TR{\nonterminal{E}}}{
		\pstree{\TR{\nonterminal{T}}}{
			\pstree{\TR{\nonterminal{T}}}{
				\pstree{\TR{\nonterminal{P}}}{
					\TR{$($}
					\pstree{\TR{\nonterminal{E}}}{
						\pstree{\TR{\nonterminal{E}}}{
							\pstree{\TR{\nonterminal{T}}}{
								\pstree{\TR{\nonterminal{P}}}{
									\TR{$CONST$}
								}
							}
						}
						\TR{$+$}
						\pstree{\TR{\nonterminal{T}}}{
							\pstree{\TR{\nonterminal{T}}}{
								\pstree{\TR{\nonterminal{P}}}{
									\TR{$CONST$}
								}
							}
						}
					}
					\TR{$)$}
				}
			}
			\TR{$\ast$}
			\pstree{\TR{\nonterminal{P}}}{
				\TR{$($}
				\pstree{\TR{\nonterminal{E}}}{
					\pstree{\TR{\nonterminal{E}}}{
						\pstree{\TR{\nonterminal{T}}}{
							\pstree{\TR{\nonterminal{P}}}{
								\TR{$CONST$}
							}
						}
					}
					\TR{$+$}
					\pstree{\TR{\nonterminal{T}}}{
						\pstree{\TR{\nonterminal{P}}}{
							\TR{$CONST$}
						}
					}
				}
				\TR{$)$}
			}
		}
	}
	\TR{\actionsym{ANSWER}}
}

\caption{$\uparrow$: Synthesis}

\end{figure}


Each non-terminal symbol has a single attribute that synthesizes the
value of the sub-expression that it generates.

Where
\indent $\langle E \rangle_{\synth{p}}, \langle T \rangle_{\synth{p}}$
and $\langle P \rangle_{\synth{r}}$ synthesized - $\synth{p}$ and
$\actionsym{Answer}_{\synth{p}}$ inherited $\synth{p}$.


This is an attributed translation grammar. In this case, the attributes
of the non-terminal symbols are evaluated from the bottom up.


\subsection*{Inherited Attributes}

$REAL x,y$

\begin{eqnarray*}
\langle DECL \rangle				&	\to	&	TYPE_{\synth{p}} IDENT_{\synth{q}} \actionsym{SETTYPE}_{\synth{t,s}} \langle MORE IDENTS \rangle \\
\langle MORE IDENTS \rangle	&	\to	& , IDENT \actionsym{SETTYPE} \langle MORE IDENTS \rangle \\
\langle MORE IDENTS \rangle	&	\to	& \epsilon
\end{eqnarray*}


Where $TYPE$, $IDENT$ and $,$ are lexical tokens.

The value part of $TYPE$ us $REAL$, $INTEGER$ or $BOOLEAN$, and the
value part of the $IDENT$ is a pointer to a symbol table entry for this
identifier.

\begin{table}[h]

\begin{tabular}{|c|c|c|}
\hline
Name		&	Type		&			\\
\hline	&				&\hspace{15mm}	\\
$X$		&	$REAL$	&					\\
$Y$		&	$REAL$	&					\\
			&				&					\\
			&				&					\\
\hline
\end{tabular}

\caption{Symbol Table}

\end{table}

\begin{figure}[ht]

\psset{levelsep=15mm,nodesep=1mm}

\pstree{\TR{\nonterminal{DECL}}}{
	\TR{$TYPE$}
	\TR{$IDENT$}
	\pstree{\TR{\nonterminal{$MORE$ $IDENTS$}}}{
		\TR{$,$}
		\TR{$IDENT$}
		\pstree{\TR{\nonterminal{$MORE$ $IDENTS$}}}{
			\TR{$\epsilon$}
		}
	}
}


\caption{\actionsym{SETTYPE}}

\end{figure}


\begin{eqnarray*}
\langle DECL \rangle	&	\to	& TYPE_{\synth{p}} IDENT_{\synth{q}} \actionsym{SETTYPE}_{\synth{r,s}} \langle MORE INDENTS \rangle_{\synth{t}} \\
\langle MORE IDENTS \rangle	&	\to	& IDENT_{\synth{q}} \actionsym{SETTYPE}_{\synth{r,s}} \langle MORE IDENTS \rangle + \\
\langle MORE IDENTS \rangle	&	\to	& \epsilon 
\end{eqnarray*}

Where $\langle MORE INDENTS \rangle_{\synth{p}}$ and
$\actionsym{SETTYPE}_{\synth{p,q}}$ inherited $\synth{p,q}$. Inherited 
attributes are passed across and down the derivation tree.

$\synth{r \leftarrow q }$ \\
$\synth{s \leftarrow p }$ \\
$\synth{r \leftarrow v }$ \\
$\synth{(s,t) \leftarrow p }$ \\

\begin{figure}[ht]

\psset{levelsep=15mm,nodesep=1mm}

\pstree{\TR{\nonterminal{DECL}}}{
	\TR{$TYPE$}
	\TR{$IDENT$}
	\TR{\actionsym{SETTYPE}}
	\pstree{\TR{\nonterminal{MORE IDENTS}}}{
		\TR{$,$}
		\TR{$IDENT$}
		\TR{\actionsym{SETTYPE}}
		\pstree{\TR{\nonterminal{MORE IDENTS}}}{
			\TR{$\epsilon$}
		}
	}
}

\end{figure}

\end{document}
