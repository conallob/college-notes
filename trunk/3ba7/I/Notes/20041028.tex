% $Id: 20041026.tex 336 2004-10-27 12:47:38Z conall $

\documentclass[a4paper,12pt]{article}
\usepackage{amssymb}
\usepackage{pst-all}

\setlength{\parindent}{0mm}
\setlength{\parskip}{7.5mm}

\newcommand{\nonterminal}[1]{$\langle #1 \rangle$}

\begin{document}

\title{Course 3BA7: Compiler Design I \\ Additional Lecture Notes \\ $28^{th}$ October 2004}

\maketitle

\section{Pushdown Machines (Continued)}

2 States

\begin{tabular}{r|l|l|l|}
\hline
		&		$0$						&			$1$	&	$\dashv$	\\
\hline
$X$	&	Pop State($2$), Advance	& Push($X$), State($2$), Advance	& Reject \\
\hline
$\triangledown$	&	Reject	& Push($X$), State($1$), Advance	& Reject \\
\hline
\end{tabular}

State 1. No $1$s. $N = 0$.

\begin{tabular}{r|l|l|l|}
\hline
		&		$0$						&			$1$	&	$\dashv$	\\
\hline
$X$	&	Pop State($2$), Advance	& Reject	& Reject 		\\
\hline
$\triangledown$	&	Reject	& Reject & Accept			\\
\hline
\end{tabular}

State 2. % Missing Label


Starting State: $\triangledown$

Stack Moves:

\begin{tabular}{rcl}
Stack								&	State		&	Input				\\
$\triangledown$			&	$[s1]$	&	$1100\dashv$	\\	
$\triangledown X$			&	$[s1]$	&	$100\dashv$	\\	
$\triangledown X X$		&	$[s1]$	&	$00\dashv$	\\	
$\triangledown X$			&	$[s2]$	&	$0\dashv$	\\	
$\triangledown$			&	$[s2]$	&	$\dashv$	\\	
\end{tabular}

$\Rightarrow$ ACCEPT


Extended Stack Operation: \verb!REPLACE!

\verb!REPLACE(ABC)! $\equiv$ \verb!pop,push(A),push(B),push(C)!


A different machine to do the same job:

Initialise stack to

\begin{tabular}{|c|}
		\\
\hspace{20mm}	\\
		\\
\hline
$Y$	\\
\hline
$X$	\\
\hline
$\triangledown$	\\
\hline
\end{tabular}

Every time we see a $1$ at head of input, \verb!replace(X,Y)! and
advance input


\begin{tabular}{r|l|l|l|}
\hline
		&		$1$		&			$0$			&	$\dashv$				\\
\hline
$X$	&	Reject		&	Pop, Advance 		& Reject 				\\
\hline
$Y$	&	Replace($X$,$Y$),Advance	&	Pop,Retain		& Reject \\
\hline
$\triangledown$	&	Reject		& Reject 			& Accept	\\
\hline
\end{tabular}


Starting State: $\triangledown Y$.

Stack Movie:

\begin{tabular}{rcl}
Stack								&		&	Input				\\
$\triangledown Y$			&		&	$1100\dashv$	\\	
$\triangledown X Y$		&		&	$100\dashv$	\\	
$\triangledown X X Y$	&		&	$00\dashv$	\\	
$\triangledown X X$		&		&	$0\dashv$	\\	
$\triangledown X$			&		&	$\dashv$	\\	
$\triangledown$			&		&	$\dashv$	\\	
\end{tabular}

$\Rightarrow$ ACCEPT

Slower than the other machine, but smaller.


There is in general no reduced (minimal) push-down machine.

\verb!OUT(X)! $\to$	Output Symbol $X$

A push down translator is a pushdown machine that produces an output.

Eg:

Design a machine to translate an arbitary sequence of ones and zeros
into the sequence $0^{N}1^{1}$ where $M, N$ are the number of zeros and
ones in their original input sequence.

$010111 \Rightarrow 00111$

\begin{tabular}{|c|}
		\\
\hspace{20mm}	\\
		\\
\hline
$X$	\\
\hline
$Y$	\\
\hline
$X$	\\
\hline
$\triangledown$	\\
\hline
\end{tabular}


\begin{tabular}{r|l|l|l|}
\hline
		&		$0$		&			$1$			&	$\dashv$				\\
\hline
$X$	& Out($0$),Advance	&	Push($X$), Advance 		& Pop,Out($1$),Retain \\
\hline
$\triangledown$	&	Out($0$),Advance		& Push($X$),Advance	& Accept	\\
\hline
\end{tabular}

Starting State: $\triangledown$


Stack Movie:

\begin{tabular}{rcl}
Stack								&				&	Input				\\
$\triangledown$			& out($0$)	&	$01011\dashv$	\\	
$\triangledown$			&				&	$1011\dashv$	\\	
$\triangledown X$			& out($0$)	&	$01\dashv$		\\	
$\triangledown X$			&				&	$11\dashv$		\\	
$\triangledown X X$		&				&	$1\dashv$		\\	
$\triangledown X X X$	& out($1$)	&	$\dashv$			\\	
$\triangledown X X$		& out($1$)	&	$\dashv$			\\	
$\triangledown X$			& out($1$)	&	$\dashv$			\\	
$\triangledown$			&				&	$\dashv$			\\	
\end{tabular}

$\Rightarrow$ Accept $\Rightarrow 01011 \Rightarrow 00111$

\section*{Context Free Grammars}

\begin{enumerate}

\item A finite set of terminal symbols

\item A finite set of non-terminal symbols. (Disjoint from the set of
terminal symbols)

\item A finite set of \textbf{productions} with the form $\langle A
\rangle \to X$, where $A$ is a non-terminal symbol and $A$ is a sequence
(possibly the null sequence) of terminal and non-terminal symbols.

\item A starrting non-terminal symbol.

\end{enumerate}

\subsection*{Productions}

\begin{enumerate}

\item $\langle A \rangle \to a \langle A \rangle \langle A \rangle
\langle B \rangle C$ (starting non-terminal by convention). $\langle A
\rangle$ and $\langle B \rangle$ are non-terminal symbols, $a$ and $C$
are terminal symbols.

\item $\langle A \rangle \to \langle A \rangle b$

\item $\langle A \rangle \to \epsilon$ (Epsilon - the null sequence)

\item $\langle B \rangle \to b$

\item $\langle B \rangle \to \epsilon$

\end{enumerate}

A language is a set of all sequences of terminal symbols that can be
derived from the starting non-terminal of a given context free grammar.

\subsection*{Derivations}

\begin{tabular}{lcl|l}
$\langle A \rangle$	&	$\stackrel{1}{\Rightarrow}$	&	$a \langle A \rangle
\langle B \rangle c$	&	$\langle A \rangle \Rightarrow abc$ \\
							&	$\stackrel{2}{\Rightarrow}$	&	$a \langle A \rangle
b \langle B \rangle c$	&	 \\	
							&	$\stackrel{3}{\Rightarrow}$	&	$a b \langle B \rangle c$	&	 \\
							&	$\stackrel{5}{\Rightarrow}$	&	$a b c $	&	\\
\end{tabular}

This is called \textbf{left most derivation}

$\Rightarrow : $ Derives in one step \\
$\stackrel{*}{\Rightarrow} : $ Derives in $0$ or more steps \\
$\stackrel{+}{\Rightarrow} : $ Derives in $1$ or more steps \\
$\stackrel{n}{\Rightarrow} : $ Derives using production $n$ \\

\begin{eqnarray*}
\langle A \rangle &	\stackrel{1}{\Rightarrow}	&	a \langle A \rangle \langle B \rangle C \\
						&	\stackrel{2}{\Rightarrow}	&	a \langle B \rangle c \\
						&	\stackrel{4}{\Rightarrow}	&	abc	\\
\end{eqnarray*}

Because the string $abc$ can be derived in two different ways, the
grammar is said to be \textbf{ambiguous}.

Note: Ambiguity is a \emph{property of a grammar}, \textbf{not} a
language.

Corresponding to each derivation, it is possible to draw an assocciated
derivation tree.

\begin{figure}[ht]

\psset{levelsep=15mm,nodesep=1mm}

\pstree{\TR{\nonterminal{A}}}{
	\TR{$a$}
	\pstree{\TR{\nonterminal{A}}}{
		\pstree{\TR{\nonterminal{A}}}{
				\TR{$e$}
				\Tn
		}
	\TR{$b$}
	}
	\pstree{\TR{\nonterminal{B}}}{
		\Tn
		\TR{$\epsilon$}
	}
	\TR{$c$}
}

\end{figure}


\begin{figure}[ht]

\psset{levelsep=15mm,nodesep=1mm}

\pstree{\TR{\nonterminal{A}}}{
	\TR{$a$}
	\pstree{\TR{\nonterminal{A}}}{
		\TR{$e$}
		\Tn
	}
	\pstree{\TR{\nonterminal{B}}}{
		\Tn
		\TR{$b$}
	}
	\TR{$c$}
}

\end{figure}

A derivation tree hides the order of the deviation.

\begin{eqnarray*}
\langle A \rangle &	\Rightarrow	&	a \langle A \rangle \langle B \rangle C \\
						&	\Rightarrow	&	a \langle A \rangle b c \\
						&	\Rightarrow	&	abc	\\
\end{eqnarray*}

Right most derivation.


Corresponding to every string, there is one or more something
derivations.

Corresponding to every derivation tree, there is a unique left most and
a unique right most derivation.

When each string derivable from the starting non-terminal of a given
context free grammar has only one derivation tree, the grammar is said
to be \textbf{unambiguous}.

\end{document}
