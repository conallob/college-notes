 % $Id: 20041026.tex 336 2004-10-27 12:47:38Z conall $

\documentclass[a4paper,12pt]{article}
\usepackage{amssymb}
\usepackage[usenames]{color}
\usepackage[arc,poly,all]{xy}
\usepackage{ulem}

\setlength{\parindent}{0mm}
\setlength{\parskip}{7.5mm}


\newcommand{\actionsym}[1]{{\mbox{\color{Emerald}{\{$#1$\}}}}}

\newcommand{\inherit}[1]{\mbox{{\color{NavyBlue}{$#1$}}}}

\newcommand{\synth}[1]{\mbox{{\color{Maroon}{$#1$}}}}

\newcommand{\yellowify}[1]{\mbox{{\color{Yellow}{$#1$}}}}

\newcommand{\nonterminal}[1]{\langle #1 \rangle}

\begin{document}

\title{Course 3BA7: Compiler Design I \\ Additional Lecture Notes \\	$13^{th}$ January 2005}

\maketitle

\begin{eqnarray}
\nonterminal{S}	&	\to	&	\verb!REPEAT! \nonterminal{SS} \verb!UNTIL! \nonterminal{C}	\\
\nonterminal{S}	&	\to	&	\verb!REPEAT! \actionsym{LABEL}_{\synth{p}} \nonterminal{SS} \verb!UNTIL! \nonterminal{C}_{\inherit{q}} \actionsym{JUMPF}_{\synth{r, s}}
\end{eqnarray}

Where $\nonterminal{C}_{\synth{p}}$ synthesized $\synth{p}$ and all
action symbol attributes are inherited.

\begin{eqnarray*}
\synth{r}		&	\gets	&	\inherit{q}		\\
(\synth{p, s})	&	\gets	&	NEWL				\\
\end{eqnarray*}

Rewritten to be

\begin{eqnarray*}
\nonterminal{S}	&	\to	&	\verb!REPEAT!
\actionsym{LABEL}_{\synth{p}} \nonterminal{SS} \verb!UNTIL! \nonterminal{C}_{\inherit{q}} \actionsym{FALSE}_{\inherit{q}, \synth{p}}
\end{eqnarray*}

\begin{eqnarray*}
\synth{p}	&	\gets	&	NEWL				\\
\end{eqnarray*}

\begin{verbatim}
PROC S(OTHER SYMBOLSET);
     LOCAL P, Q: ATTR;
     BEGIN {PARSE AN S }
     SKIPTO([REPEAT | WHILE | IF | ... | OTHERSYMBOLS])
     IF SYMBOLS = REPEAT THEN
           BEGIN { PARSE REPEAT STATEMENT }
                 NEXTSYMBOL P := NEWL;
                 OUT(LABEL P);
                 SS([UNTIL] + OTHERSYMBOLS);
     IF SYMBOLS = UNTIL THEN
                 NEXTSYMBOL;
                 C(Q, OTHERSYMBOLS);
                 OUT(JUMPF, Q, P);
           END
     END { PARSE AN S };
\end{verbatim}

\section*{An Arguement Push Down Machine (for processing L-Attributed Translation Grammers)}

A rule is a copy rule if it of the form $a \gets b$ or $(a, b, c)	\gets d$.

$a$ - sink \\
$b$ - source \\

A set of copy rules is independant if the souce of each rule only
appears once in the set of rules.

$(A, B) \gets C \& D \gets C \equiv (A, B, D) \gets C$

$(A, B, D) \gets Y \& (C, D) \gets B \equiv (A, B, C, D) \gets Y$

An L-attributed translation grammar is in simple assignment form  if and
only if:

\begin{enumerate}

\item The only attribute evaluation rules that are not copy rules are
rules to compute synthesised values of action symbols.

\item The set of copy rules associated with productions for the same
non-terminal, are independant.

\end{enumerate}

$\nonterminal{A}	\to	b_{\inherit{p}} \nonterminal{C}_{\synth{q}} \nonterminal{D}_{r}$

\begin{eqnarray*}
\yellowify{r}	&	\to	&	\inherit{p} + \synth{q}		\\
					&	\to	&	f (\inherit{p}, \synth{q})
\end{eqnarray*}

Where $\nonterminal{C}_{\inherit{p}}$ synthesised $\inherit{p}$ and
$\nonterminal{D}_{\inherit{p}}$ - inherited $\inherit{p}$

$\nonterminal{A} \to b_{\inherit{p}} \nonterminal{C}_{\synth{q}} \actionsym{F}_{\yellowify{v, s} \synth{t}} \nonterminal{D}_{\yellowify{u}}$

$\synth{r}		\gets	\inherit{p}$	\\
$\yellowify{s}	\gets	\inherit{q}$	\\
$\yellowify{u}	\gets	\inherit{t}$	\\

$\actionsym{F}_{\inherit{p, q}, \synth{r}}$

$\synth{r}		\gets	\inherit{p + q}$	\\

\begin{eqnarray}
\setcounter{equation}{1}
\nonterminal{S}	&	\to	&	\nonterminal{E}_{\inherit{p}} \actionsym{ANSWER}_{\yellowify{q}}
\end{eqnarray}

\begin{eqnarray*}
\yellowify{q}	&	\gets	&	\inherit{p}
\end{eqnarray*}

\begin{eqnarray}
\setcounter{equation}{2}
\nonterminal{E}_{\synth{p}}	&	\to	&	+ \nonterminal{E}_{\yellowify{q}} \nonterminal{E}_{\inherit{r}}
\end{eqnarray}

\begin{eqnarray*}
\synth{p}	&	\gets	&	\yellowify{q} + \inherit{r}
\end{eqnarray*}

\begin{eqnarray}
\setcounter{equation}{3}
\nonterminal{E}_{\synth{p}}	&	\to	&	\ast
\nonterminal{E}_{\inherit{q}} \nonterminal{E}_{\inherit{r}}
\end{eqnarray}

\begin{eqnarray*}
\yellowify{q}	&	\gets	&	\yellowify{q} \times \inherit{r}
\end{eqnarray*}

\begin{eqnarray}
\setcounter{equation}{4}
\nonterminal{E}_{\synth{p}}	&	\to	&	CONST_{\inherit{q}}
\end{eqnarray}

\begin{eqnarray*}
\synth{q}	&	\gets	&	\inherit{q}
\end{eqnarray*}

Where $\nonterminal{E}$ synthesis is in $\actionsym{ANSWER}_{\synth{p}}$


\begin{eqnarray}
\setcounter{equation}{2}
\nonterminal{E}_{\synth{p}}	&	\to	&	+ \nonterminal{E}_{\yellowify{q}} \nonterminal{E}_{\inherit{r}} \actionsym{ADD}_{\inherit{s, t} \synth{u}}	\\
\nonterminal{E}_{\synth{p}}	&	\to	&	\ast \nonterminal{E}_{\inherit{q}} \nonterminal{E}_{\inherit{r}} \actionsym{MULT}_{\inherit{s, t} \synth{u}}
\end{eqnarray}

\begin{eqnarray*}
\setcounter{equation}{1}
\synth{s}	&	\gets	&	\inherit{q}	\\
\synth{t}	&	\gets	&	\inherit{r}	\\
\synth{p}	&	\gets	&	\inherit{u}
\end{eqnarray*}

$\actionsym{ADD}_{\synth{p} \inherit{q}}$ for $f(MULT)_{\synth{p} \inherit{q} \inherit{r}}$

$\synth{r}	\gets	\synth{p} + \inherit{q}$

$\actionsym{MULT}_{\synth{p} \inherit{q} \inherit{r}} T \gets \synth{p} \ast \inherit{q}$

\begin{tabular}{|c|c|c|c|c|}
\hline
							&	\hspace{8mm}	$CONST$	\hspace{8mm}	&	\hspace{8mm}	$+$ \hspace{8mm}	&	\hspace{8mm}	$\ast$	\hspace{8mm}	&	$\dashv$		\\
\hline
$\nonterminal{S}$		&	$\#1$							&	$\#1$					&	$\#1$							&					\\
\hline
$\nonterminal{E}$		&	$\#4$							&	$\#2$					&	$\#3$							&					\\
\hline
$\triangledown$		&									&							&									&	$ACCEPT$		\\
\hline
$\actionsym{ANSWER}$	&	\multicolumn{4}{c|}{$OUTPUT(ANSWER P), RETAIN, POP$}	\\
\hline
$\actionsym{ADD}$		&	\multicolumn{4}{c|}{$ADD OPERANDS, STORE VALUE, POP, RETAIN$} \\
\hline
$\actionsym{MULT}$	&	\multicolumn{4}{c|}{$MULT OPERANDS, STORE VALUE, POP, RETAIN$}	\\
\hline
\end{tabular}

Starting Stack: $\triangledown \nonterminal{S}$

All blanks represent reject.


$\#1$ : $REPLACE(\actionsym{ANSWER} \nonterminal{E}), RETAIN$	\\
$\#2$ : $REPLACE(\actionsym{ADD} \nonterminal{E} \nonterminal{E}), ADVANCE$	\\
$\#3$ : $REPLACE(\actionsym{MULT} \nonterminal{E} \nonterminal{E}), ADVANCE$	\\
$\#4$ : $POP, ADVANCE$	\\

\subsection*{Stack Symbols}

\begin{table}[hbtp]

% xy-pic diagram

\end{table}

\begin{table}[hbtp]

% xy-pic diagram

\end{table}

\begin{table}[hbtp]

% xy-pic diagram

\end{table}

\begin{table}[hbtp]

% xy-pic diagram

\end{table}

\begin{table}[hbtp]

% xy-pic diagram

\end{table}

\begin{table}[hbtp]

% xy-pic diagram

\end{table}

\begin{table}[hbtp]

% xy-pic diagram

\end{table}

\end{document}
