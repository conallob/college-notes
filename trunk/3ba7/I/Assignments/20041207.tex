% $Id$

\documentclass[a4paper,12pt]{article}
\usepackage{amssymb}
\usepackage[usenames]{color}
\usepackage[arc,poly,all]{xy}
\usepackage{ulem}

\setlength{\parindent}{0mm}
\setlength{\parskip}{7.5mm}


\newcommand{\actionsym}[1]{{\mbox{\color{Emerald}{\{$#1$\}}}}}

\newcommand{\synth}[1]{\mbox{{\color{Maroon}{$#1$}}}}

\newcommand{\nonterminal}[1]{\langle #1 \rangle}

\begin{document}

\title{Course 3BA7: Compiler Design I \\ Assignment \\ $7^{nd}$ December 2004}

\author{Conall O'Brien \\ 01734351 \\ conallob@maths.tcd.ie}

\maketitle

\section*{Questions}

\subsection*{1}

\begin{eqnarray}
\nonterminal{S}	&	\to	&	a \nonterminal{A} \nonterminal{B}	\\	
\nonterminal{S}	&	\to	&	b \nonterminal{B} \nonterminal{S}	\\	
\nonterminal{S}	&	\to	&	\epsilon										\\	
\nonterminal{A}	&	\to	&	c \nonterminal{B} \nonterminal{S}	\\	
\nonterminal{A}	&	\to	&	\epsilon										\\
\nonterminal{B}	&	\to	&	d \nonterminal{B}							\\	
\nonterminal{B}	&	\to	&	e 													
\end{eqnarray}

Show that this is a Q-Grammar and build a push down control for it.

\subsection*{2}

\begin{eqnarray}
\nonterminal{B} \to c \nonterminal{S} \nonterminal{A} e
\end{eqnarray}

How can this be added into the existing grammer from $1$?

\section*{Solution}

\subsection*{1}

\begin{tabular}{lcl}
$SELECT(1)$	&	\hspace{10mm}	&	$\{a\}$	\\
$SELECT(2)$	&						&	$\{b\}$	\\
$SELECT(3)$	&						&	$FOLLOW(\nonterminal{S})$	\\
$SELECT(4)$	&						&	$\{c\}$	\\
$SELECT(5)$	&						&	$FOLLOW(\nonterminal{A})$	\\
$SELECT(6)$	&						&	$\{d\}$	\\
$SELECT(7)$	&						&	$\{e\}$	\\
\end{tabular}

$FOLLOW(\nonterminal{A}) = \{d,e\} \\
FOLLOW(\nonterminal{S}) = FOLLOW(\nonterminal{A}) + \{\dashv\} =
\{d,e,\dashv\}$

\begin{eqnarray*}
SELECT(1) \cap	SELECT(2)	&	=	&	\{a\} \cap \{b\}	= \emptyset	\\ 
SELECT(2) \cap	SELECT(3)	&	=	&	\{b\} \cap \{d,e,\dashv\}	= \emptyset	\\ 
SELECT(1) \cap	SELECT(3)	&	=	&	\{a\} \cap \{c\}	= \emptyset	\\ 
SELECT(4) \cap	SELECT(5)	&	=	&	\{c\} \cap \{d,e\}	= \emptyset	\\ 
SELECT(6) \cap	SELECT(7)	&	=	&	\{d\} \cap \{e\}	= \emptyset	\\ 
\end{eqnarray*}

Therefore all selection sets for productions for the same non-temrinal
symbol are disjoint, proving this grammar is a Q-Grammar.

\begin{tabular}{|c|c|c|c|c|c|c|}
\hline
						&	$a$	&	$b$	&	$c$	&	$d$	&	$e$	&	$\dashv$ \\
\hline
$\nonterminal{S}$	&	$\#1$	&	$\#2$	&			&			&			&	$\#5$		\\
\hline
$\nonterminal{A}$	&			&			&	$\#2$	&			&			&	$\#5$		\\
\hline
$\nonterminal{B}$	&			&			&			&	$\#3$	&	$\#4$	&				\\
\hline
$\triangledown$	&			&			&			&			&			& $ACCEPT$	\\
\hline
\end{tabular}

Starting Stack: $\triangledown \nonterminal{S}$

$\#1$: $REPLACE(\nonterminal{A}\nonterminal{B})$ \\
$\#2$: $REPLACE(\nonterminal{B}\nonterminal{S})$ \\
$\#3$: $REPLACE(\nonterminal{B})$ \\
$\#4$: $POP,ADVANCE$ \\
$\#5$: $POP,RETAIN$ \\

\subsection*{2}

\begin{tabular}{lcl}
$SELECT(8)$	&	\hspace{10mm}	&	$\{c,e\}$	\\
\end{tabular}

\begin{eqnarray*}
SELECT(8) \cap SELECT(6)	&	=	&	\{c,e\} \cap \{d\} = \emptyset \\
SELECT(8) \cap SELECT(7)	&	=	&	\{c,e\} \cap \{e\} = \{e\} \\
\end{eqnarray*}

Therefore, adding in this production into the Q-Grammar from question $1$
will not satisfy the properties of a Q-Grammar, due to an intersection
of the selection sets between production $7$ and the additional
production $8$.

\end{document}
