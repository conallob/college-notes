\documentclass[a4paper,12pt]{article}
\usepackage{amssymb}

\begin{document}

\title{Course 3BA7: Compiler Design I \\ Additional Lecture Notes \\ $19^{th}$ October 2004}

\maketitle

Design a FSM to recognise any valid sequence that many follow the
keyword integer in Fortran.

Input Set: $\{v, c, ',', (, ) \}$

Where $v$ is a variable identifier specifying an adjustable dimension
and $c$ is an integer constant.

$	INTEGER 	x ( 2 , I , 5 ) , y 		$
$				v ( c , v , c ) , v		$ - Lexical Tokens
$			 1|2|3|4|5|6|5|4|7|8|2		$	

\begin{table}[c|c]
1 & Starting State 						\\
2 & Variable ID to be made integer	\\
3 & Left Paren								\\
4 & Constant								\\
5 & Comma									\\
6 & Variable								\\
7 & Right Arrow							\\
8 & Comma									\\

\end{table}

\begin{table}[c|c|c|c|c|c|c]

\hline
	&	$v$	&	$c$	&	$,$	&	$($	&	$)$	&	Accepting states \\
\hline
1	&	$2$	&			&			&			&			&	$0$					\\
\hline
2	&			&			&	$8$	&	$3$	&			&	$1$					\\
\hline
3	&	$6$	&	$4$	&			&			&			&	$0$					\\
\hline
4	&			&			&	$5$	&			&	$7$	&	$0$					\\
\hline
5	&	$6$	&	$4$	&			&			&			&	$0$					\\
\hline
6	&			&			&	$5$	&			&	$7$	&	$0$					\\
\hline
7	&			&			&	$8$	&			&			&	$1$					\\
\hline
8	&	$2$	&			&			&			&			&	$0$					\\
\hline
E	&			&			&			&			&			&							\\
\hline

\end{table}

Al blank entries represent transitions to the error state.

\section{Extraneous State}

Make list of non extroneous states:

$\{1,2,8,3,6,4,5,7,E \}$

\section{State Equivalence}

Machine: N, M

States:	S, T

\section{Destinguishing Sequences}

State equivalence is:
	
	reflexive
	symmetric
	transitive

Partition states into 2 sets: accepting and rejecting.

\[ P_{0} & = & [\{2,7\},\{1,3,4,5,6,8,E\}] \]

Partition with $v$:

\[ P_{1} & = & [\{2,7\},\{1,8\}, {3,4,5,6,E\}] \]

Partition with $)$:

\[ P_{2} & = & [\{2,7\},\{1,8\},\{3,5,E\},\{4,6\}] \]

Partition with $v$:

\[ P_{3} & = & [\{2,7\},\{1,8\},\{3,5\},\{4,6\},\{E\}] \]

Partition with $($:

\[ P_{3} & = & [\{2\},\{7\},\{1,8\},\{3,5\},\{4,6\},\{E\}] \]


\[ \{1,8\} & - & A	\]
\[ \{3,5\} & - & B	\]
\[ \{4,6\} & - & C	\]

\end{document}
