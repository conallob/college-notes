\documentclass[a4paper,12pt]{article}
\usepackage{amssymb}

\begin{document}

\title{2BA2: \\ Programming Techniques \\ Assignment II \\ Word Search}

\author{Conall O'Brien \\ \\ 01734351 \\ \\ conall@conall.net}

\maketitle

\pagebreak

\section{Basic Design}

\noindent Instead of approaching this program in a brute force manner to
check every posible permutation of adjacent characters to find the
required words, I decided to approach the problem in the way a human
would. Humans don't read characters, they read words. Humans can
identify words correctly, even if misspelt, so long as the first and
last characters are in the correct places. Read
http://science.slashdot.org/science/03/09/15/2227256.shtml?tid=133 for
more information 


\pagebreak

\section{The Text Parser}

\subsection{parse\_grid\_input(file: STRING; i, j : INTEGER) :
ARRAY2[CHARACTER]}

\noindent This function parses a text file containing 15 lines, each
with 15 characters and returns them in a simple, yet useful data
sctructure consisting of an \emph{ARRAY2} of \emph{CHARACTER} variables.
This system was chosen for the functions available in the SmartEiffel
libraries for these classes.

\subsection{parse\_list\_input(file: STRING; length : INTEGER) :
ARRAY[STRING]}

\noindent This function parses a list of words layed out in a file one
word per line. The list is returned, in the same order as in the file as
a one dimensional \emph{ARRAY} of \emph{STRING} variables, allowing
access to many useful functions available to the SmartEiffel
\emph{STRING} class.

\section{The Grid Searcher}

\subsection{search(word: STRING; matrix : ARRAY2[CHARACTER]) : INTEGER}

\noindent This is the wrapper function called by the main loop which
iterates through the list of words. If the selected word is found, it
returns the value $1$, otherwise it defaults to $0$. Inside it, it calls
numerous other functions during the two main stages of searching,
scanning and close inspection.

\subsection{scan(word : STRING; grid : ARRAY2[CHARACTER]; f, l :
CHARACTER; len, x, y : INTEGER) : INTEGER}


\noindent This function does the basic scanning for the fitst character
and the last character, allowing for the offset between them equaling
the length of the word being searchied for. This function uses
SmartEiffel functions to check for valid bounds to ensure the scanning
is as efficient as possible, eliminating as many possibliities as
possible using basic intellegence (ie the known length of a word and the
end points or poles).

\subsection{further\_investigation(word : STRING; grid :
ARRAY2[CHARACTER]; x1, y1, x2, y2 : INTEGER) :INTEGER}

\noindent This is the function used to iterate through a finite no of
elements in the grid of characters, once a positive mathc for the end
characters with the appropriate length as been detected. If every
character matches the word being searched for, the search function has
completed.

\subsection{forward\_or\_reverse(a, b : INTEGER) : INTEGER}

\noindent This is a simple function used to return the smaller of two
numers, in order to clean up code and save reusing the same code
repeaditly, decreasing the program's efficiency. It is required by the
\emph{further\_investigation(word : STRING; grid : ARRAY2[CHARACTER];
x1, y1, x2, y2 : INTEGER) :INTEGER} function to establish the vertical
and horizonal directions (right $\to$ left, right $\to$ left, bottom
$\to$ top, top $\to$ bottom.)

\pagebreak

\section{Source Code}

\pagebreak

\end{document}
