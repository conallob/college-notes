% $Id: doc.tex,v 1.1 2004/03/05 13:11:18 conall Exp $

\documentclass[a4paper,12pt]{article}

\setlength{\parindent}{0mm}
\setlength{\parskip}{7.5mm}

\begin{document}

\title{2BA2: \\ Programming Techniques \\ Assignment III \\ Knight's Journey}

\author{Conall O'Brien \\ \\ 01734351 \\ \\ conall@conall.net}

\maketitle

\pagebreak

\section{Knight's Journey}

My implementaion relys heavily upon the code provided in the lecture
notes for a Knight's Tour. Apart from simple syntactic conversions into
SmartEiffel, I will explain and changes I made to each of the following
functions.

\subsection{init\_moves}

No changes made to this function. It simply generates two arrays
containing every permutation of the appropriate offsets to represent 
the movement of a Knight in all eight directions.

\subsection{try\_all(k, x,y : INTEGER)}

I took the the supplied version of this function and adapted it slightly
when it decides if a journey has been found. My version includes a check
to see that move no $m \times n$, the final move the knight can make is
in the specified finishing position. If it is, a Knight's journey has
been found, else not.

\subsection{acceptable(s,t : INTEGER):BOOLEAN}

No changes from the supplied source code.

\subsection{display\_board is}

No changes from the supplied source code.

\subsection{Sample Output}

\emph{nice ./specific-knight 6 6 1 1 3 2} \\

Knights Journey is given by \\

1 8 5 20 3 10 \\
6 19 2 9 30 21 \\
15 36 7 4 11 32 \\
18 25 16 31 22 29 \\
35 14 27 24 33 12 \\
26 17 34 13 28 23 \\
 
\section{Paraberry Knight's Tour}

For this proglem, I've adapted my implmentation of a Knight's Journey
from a start point to a destination point. Instead of asking the user
for starting and finishing coordinates, it instead just asks for grid
dimensions which must be even.


It then calculates each of the four smaller tours, transferring them
into the master array for the final tour.

\subsection{init\_moves}

No changes from the supplied source code.

\subsection{try\_all(k, x,y : INTEGER)}

I adapted my implementation of this function to only generate a tour
using the generic start and end coordinates required for a Paraberry
Tour. Once each tour has been generated, it's transferred from the 3D
array used to represent the the minature board into the corresponding 
quadrant of the large 2D array representing the full board, 

\subsection{acceptable(s,t : INTEGER):BOOLEAN}

No changes from the supplied source code.

\subsection{display\_board is}

No changes from the supplied source code.

\subsection{Sample Output}

Due to the estimated running time of this program for an $12 x 12$ grid,
I was unable to supply sample output in time for submission.

\end{document}
