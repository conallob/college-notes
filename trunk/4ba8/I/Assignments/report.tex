% $Id$

\documentclass[a4paper,10pt]{article}

\setlength{\parindent}{0mm}
\setlength{\parskip}{7.5mm}

\usepackage{amssymb}
\usepackage[usenames]{color}
\usepackage[arc,poly,all]{xy}
\usepackage{ulem}
\usepackage{listings}
\usepackage{verbatim}
\usepackage{fancyvrb}
\usepackage[left=2cm,right=2cm,top=4cm,bottom=4cm]{geometry}
\usepackage{graphicx}
\usepackage{url}

\begin{document}

\title{4BA8 \\ Assignment}

\author{Conall O'Brien \\ 01734351 \\ conallob@maths.tcd.ie}

\maketitle

\section{Introduction}

Given the broad guidelines relating to the implementation of this 
assignment,  programming language and Web Services, I have decided to 
use three technologies, Perl, MySQL and XML-RPC.


I have chosen to use Perl for multiple reasons, primarily it's
popular use developing server-side applications and dynamic web content.
In addition, the extensive library of modules available to Perl ensures
a complete implementation of numerous web service technologies.


I have chosen to use XML-RPC for the exchanging of RPC messages due to
it's lightweight and uncomplicated definition. It is my belief that the 
use of XML-RPC over compeditors such as SOAP is appropariate for an
application of this size and complexity.


For backend database storage, I have opted to use MySQL, the open source
SQL complient database over other Web Services backend solutions, such 
as UDDI standard, who's definition is incomplete.


In addition, I have chosen to run my Assignment on web hosting servers
running the Apache 2.0 web server running on the Debian GNU/Linux
platform. To satisfy the assignment requirement of "system must be 
deployable across the Internet. It should, for example, be possible for 
services to interoperate in an environment controlled by firewalls and 
other Internet infrastructure", I have used MacOS X, FreeBSD and Ubuntu 
GNU/Linux, systems connected to the TCD network, located behind the
main TCD firewall configuration (consisting of a Cisco PIX 515 and a 
Cisco PIX 525).

\section{Architectural Design}

My application design is simple, based upon the client - server
architecture. A Perl script on the primary server acts as a translator
between XML-RPC calls to the MySQL backend and the client script. An 
additional feature of the server script not implemented by the 
assignment of the deadline allows two server scripts to communicate
between each other, allowing for backend replication between multiple,
resielient hosts using XML-RPC.


A second Perl script, the client script is a Command Line Interface 
(CLI) to the server. It allows a user to do multiple actions, such as
tourism service searchs (queries on the database), comparison of tourism 
services (comparison of query results) and creation of new service
instances (insert queries on the backend).

\subsection{Server API}

% $Id$

\subsection{AddServiceType}

\subsubsection{Description}

Adds a new service catagory to the backend database.

\subsubsection{Parameters}

\begin{tabular}{lllll}
Variable Name	&		&	Data Type(s)		&	&	Description	\\
				&	&	&	&	\\
\verb!name! & \hspace{15mm} & $\langle string\rangle $ & \hspace{15mm} & Catagory Name \\
\end{tabular}

\subsubsection{Returns}

\subsubsection{On Success}

\begin{tabular}{lllll}
Variable Name	&		&	Data Type(s)		&	&	Description	\\
				&	&	&	&	\\
\verb!result! & \hspace{15mm} & $\langle boolean\rangle $ & \hspace{15mm} & True if Sucessful \\
\end{tabular}

\subsubsection{On Failure}

\begin{tabular}{lllll}
Variable Name	&		&	Data Type(s)		&	&	Description	\\
				&	&	&	&	\\
\verb!faultString! & \hspace{15mm} & $\langle string\rangle $ & \hspace{15mm} & SQL Error Experienced \\
\end{tabular}

\subsection{EditServiceType}

\subsubsection{Description}

Renames a Service Type. Affects all Operators in this catagory
automatically. 

\subsubsection{Parameters}

\begin{tabular}{lllll}
Variable Name	&		&	Data Type(s)		&	&	Description	\\
				&	&	&	&	\\
\verb!oldname! & \hspace{15mm} & $\langle string\rangle $ & \hspace{15mm} & Old Catagory Name \\
\verb!newname! & \hspace{15mm} & $\langle string\rangle $ & \hspace{15mm} & New Catagory Name \\
\end{tabular}

\subsubsection{Returns}

\subsubsection{On Success}

\begin{tabular}{lllll}
Variable Name	&		&	Data Type(s)		&	&	Description	\\
				&	&	&	&	\\
\verb!result! & \hspace{15mm} & $\langle boolean\rangle $ & \hspace{15mm} & True if Sucessful \\
\end{tabular}

\subsubsection{On Failure}

\begin{tabular}{lllll}
Variable Name	&		&	Data Type(s)		&	&	Description	\\
				&	&	&	&	\\
\verb!faultString! & \hspace{15mm} & $\langle string\rangle $ & \hspace{15mm} & SQL Error Experienced \\
\end{tabular}


\subsection{DeleteServiceType}

\subsubsection{Description}

Removed a Service Type. Automatically deletes all Operators,
Services and Bookings dependant on removed catagory.

\subsubsection{Parameters}

\begin{tabular}{lllll}
Variable Name	&		&	Data Type(s)		&	&	Description	\\
				&	&	&	&	\\
\verb!name! & \hspace{15mm} & $\langle string\rangle $ & \hspace{15mm} & Catagory Name \\
\end{tabular}

\subsubsection{Returns}

\subsubsection{On Success}

\begin{tabular}{lllll}
Variable Name	&		&	Data Type(s)		&	&	Description	\\
				&	&	&	&	\\
\verb!result! & \hspace{15mm} & $\langle boolean\rangle $ & \hspace{15mm} & True if Sucessful \\
\end{tabular}

\subsubsection{On Failure}

\begin{tabular}{lllll}
Variable Name	&		&	Data Type(s)		&	&	Description	\\
				&	&	&	&	\\
\verb!faultString! & \hspace{15mm} & $\langle string\rangle $ & \hspace{15mm} & SQL Error Experienced \\
\end{tabular}


\subsection{GetServiceType}

\subsubsection{Description}

Confirm if a string is a service catagory.

\subsubsection{Parameters}

\begin{tabular}{lllll}
Variable Name	&		&	Data Type(s)		&	&	Description	\\
				&	&	&	&	\\
\verb!name! & \hspace{15mm} & $\langle string\rangle $ & \hspace{15mm} & Catagory Name \\
\end{tabular}

\subsubsection{Returns}

\subsubsection{On Success}

\begin{tabular}{lllll}
Variable Name	&		&	Data Type(s)		&	&	Description	\\
				&	&	&	&	\\
\verb!TypeName! & \hspace{15mm} & $\langle string\rangle $ & \hspace{15mm} & Catagory Name \\
\end{tabular}
\subsubsection{On Failure}

\begin{tabular}{lllll}
Variable Name	&		&	Data Type(s)		&	&	Description	\\
				&	&	&	&	\\
\verb!faultString! & \hspace{15mm} & $\langle string\rangle $ & \hspace{15mm} & SQL Error Experienced \\
\end{tabular}


\subsection{AddOperator}

\subsubsection{Description}

Adds a service provider to the system. Details can optionally include
opening and closing dates and times, stating when operator is active.  

\subsubsection{Parameters}

\begin{tabular}{lllll}
Variable Name	&		&	Data Type(s)		&	&	Description	\\
				&	&	&	&	\\
\verb!name! & \hspace{15mm} & $\langle string\rangle $ & \hspace{15mm} & Operator Name \\
\verb!type! & \hspace{15mm} & $\langle string\rangle $ & \hspace{15mm} & Catagory Name \\
\verb!opening! & \hspace{15mm} & $\langle null\rangle  \mid \langle datetime\rangle $ & \hspace{15mm} & Opening Date/Time \\
\verb!closing! & \hspace{15mm} & $\langle null\rangle  \mid \langle datetime\rangle $ & \hspace{15mm} & Closing Date/Time \\
\end{tabular}

\subsubsection{Returns}

\subsubsection{On Success}

\begin{tabular}{lllll}
Variable Name	&		&	Data Type(s)		&	&	Description	\\
				&	&	&	&	\\
\verb!result! & \hspace{15mm} & $\langle boolean\rangle $ & \hspace{15mm} & True if Sucessful \\
\end{tabular}

\subsubsection{On Failure}

\begin{tabular}{lllll}
Variable Name	&		&	Data Type(s)		&	&	Description	\\
				&	&	&	&	\\
\verb!faultString! & \hspace{15mm} & $\langle string\rangle $ & \hspace{15mm} & SQL Error Experienced \\
\end{tabular}


\subsection{EditOperator}

\subsubsection{Description}

Updates the recorded details of a Service Provider in the backend.

\subsubsection{Parameters}

\begin{tabular}{lllll}
Variable Name	&		&	Data Type(s)		&	&	Description	\\
				&	&	&	&	\\
\verb!oldname! & \hspace{15mm} & $\langle string\rangle $ & \hspace{15mm} & Current Operator Name \\
\verb!prop! & \hspace{15mm} & $\langle string\rangle $ 	& \hspace{15mm} & Attribute. Expects  $"ServiceName" \mid$ \\
	&	&	&	&	$"ServiceType"  \mid "Opening" \mid "Closing"$\\
\verb!value! & \hspace{15mm} & $\langle string\rangle  \mid \langle datetime\rangle  \mid \langle null\rangle $ & \hspace{15mm} & Desired Value \\
\end{tabular}

\subsubsection{Returns}

\subsubsection{On Success}

\begin{tabular}{lllll}
Variable Name	&		&	Data Type(s)		&	&	Description	\\
				&	&	&	&	\\
\verb!result! & \hspace{15mm} & $\langle boolean\rangle $ & \hspace{15mm} & True if Sucessful \\
\end{tabular}

\subsubsection{On Failure}

\begin{tabular}{lllll}
Variable Name	&		&	Data Type(s)		&	&	Description	\\
				&	&	&	&	\\
\verb!faultString! & \hspace{15mm} & $\langle string\rangle $ & \hspace{15mm} & SQL Error Experienced \\
\end{tabular}


\subsection{DeleteOperator}

\subsubsection{Description}

Removes a Service Provider, including any dependant services and
bookings.  

\subsubsection{Parameters}

\begin{tabular}{lllll}
Variable Name	&		&	Data Type(s)		&	&	Description	\\
				&	&	&	&	\\
\verb!oldname! & \hspace{15mm} & $\langle string\rangle $ & \hspace{15mm} & Current Operator Name \\
\end{tabular}

\subsubsection{Returns}

\subsubsection{On Success}

\begin{tabular}{lllll}
Variable Name	&		&	Data Type(s)		&	&	Description	\\
				&	&	&	&	\\
\verb!result! & \hspace{15mm} & $\langle boolean\rangle $ & \hspace{15mm} & True if Sucessful \\
\end{tabular}

\subsubsection{On Failure}

\begin{tabular}{lllll}
Variable Name	&		&	Data Type(s)		&	&	Description	\\
				&	&	&	&	\\
\verb!faultString! & \hspace{15mm} & $\langle string\rangle $ & \hspace{15mm} & SQL Error Experienced \\
\end{tabular}


\subsection{GetOperator}

\subsubsection{Description}

Returns details on service provider.

\subsubsection{Parameters}

\begin{tabular}{lllll}
Variable Name	&		&	Data Type(s)		&	&	Description	\\
				&	&	&	&	\\
\verb!name! & \hspace{15mm} & $\langle string\rangle $ & \hspace{15mm} & Operator Name \\
\end{tabular}

\subsubsection{Returns}

\subsubsection{On Success}

\begin{tabular}{lllll}
Variable Name	&		&	Data Type(s)		&	&	Description	\\
				&	&	&	&	\\
\verb!ServiceName! & \hspace{15mm} & $\langle string\rangle $ & \hspace{15mm} & Operator Name \\
\verb!ServiceType! & \hspace{15mm} & $\langle string\rangle $ & \hspace{15mm} & Operator Catagory \\
\verb!Opening! & \hspace{15mm} & $\langle null\rangle  \mid \langle datetime\rangle $ & \hspace{15mm} & Operator Opeing Date/Time \\
\verb!Closing! & \hspace{15mm} & $\langle null\rangle  \mid \langle datetime\rangle $ & \hspace{15mm} & Operator Closing Date/Time \\
\end{tabular}

\subsubsection{On Failure}

\begin{tabular}{lllll}
Variable Name	&		&	Data Type(s)		&	&	Description	\\
				&	&	&	&	\\
\verb!faultString! & \hspace{15mm} & $\langle string\rangle $ & \hspace{15mm} & SQL Error Experienced \\
\end{tabular}


\subsection{AddInstance}

\subsubsection{Description}

Adds details of a specific service from a service provider to the
system.  

\subsubsection{Parameters}

\begin{tabular}{lllll}
Variable Name	&		&	Data Type(s)		&	&	Description	\\
				&	&	&	&	\\
\verb!operator! & \hspace{15mm} & $\langle string\rangle $ 	& \hspace{15mm} & Operator Name \\
\verb!date! 	 & \hspace{15mm} & $\langle datetime\rangle $ & \hspace{15mm} & Date/Time of Service\\
\verb!capacity! & \hspace{15mm} & $\langle int\rangle $ 		& \hspace{15mm} & Capacity available \\
\verb!cost! 	 & \hspace{15mm} & $\langle int\rangle $ 		& \hspace{15mm} & Cost per Person/Service \\
\verb!src! 		 & \hspace{15mm} & $\langle null\rangle  \mid \langle string\rangle $ & \hspace{15mm} & Source of Service (Optional) \\
\verb!dest! 	 & \hspace{15mm} & $\langle null\rangle  \mid \langle string\rangle $ & \hspace{15mm} & Destination of Service (Optional) \\
\verb!details!  & \hspace{15mm} & $\langle null\rangle  \mid \langle string\rangle $ & \hspace{15mm} & Additonal Service Details (Optional) \\
\end{tabular}

\subsubsection{Returns}

\subsubsection{On Success}

\begin{tabular}{lllll}
				&	&	&	&	\\
Variable Name	&		&	Data Type(s)		&	&	Description	\\
\verb!result! & \hspace{15mm} & $\langle boolean\rangle $ & \hspace{15mm} & True if Sucessful \\
\end{tabular}

\subsubsection{On Failure}

\begin{tabular}{lllll}
				&	&	&	&	\\
Variable Name	&		&	Data Type(s)		&	&	Description	\\
\verb!faultString! & \hspace{15mm} & $\langle string\rangle $ & \hspace{15mm} & SQL Error Experienced \\
\end{tabular}


\subsection{EditInstance}

\subsubsection{Description}

Edits details of an existing service.

\subsubsection{Parameters}

\begin{tabular}{lllll}
				&	&	&	&	\\
Variable Name	&		&	Data Type(s)		&	&	Description	\\
\verb!id! & \hspace{15mm} & $\langle int\rangle $ & \hspace{15mm} & Service ID \\
\verb!prop! & \hspace{15mm} & $\langle string\rangle $ Property. Expects
& \hspace{15mm} & Attribute  Expects \\
	&	&	&	&	$"ServiceDate" \mid "Capacity" \mid$\\ 
	&	&	&	&	$"Cost" \mid "Source" \mid $ \\
	&	&	&	&	$"Destination" \mid"Details$ \\
\verb!value! & \hspace{15mm} & $\langle string\rangle  \mid \langle datetime\rangle  \mid \langle int\rangle  \mid
\langle null\rangle $  & \hspace{15mm} & Desired Value \\
\end{tabular}

\subsubsection{Returns}

\subsubsection{On Success}

\begin{tabular}{lllll}
Variable Name	&		&	Data Type(s)		&	&	Description	\\
\verb!result! & \hspace{15mm} & $\langle boolean\rangle $ & \hspace{15mm} & True if Sucessful \\
\end{tabular}

\subsubsection{On Failure}

\begin{tabular}{lllll}
Variable Name	&		&	Data Type(s)		&	&	Description	\\
				&	&	&	&	\\
\verb!faultString! & \hspace{15mm} & $\langle string\rangle $ & \hspace{15mm} & SQL Error Experienced \\
\end{tabular}


\subsection{DeleteInstance}

\subsubsection{Description}

Deletes a specified service and any bookings connected to the specified
service.

\subsubsection{Parameters}

\begin{tabular}{lllll}
Variable Name	&		&	Data Type(s)		&	&	Description	\\
				&	&	&	&	\\
\verb!id! & \hspace{15mm} & $\langle int\rangle $ & \hspace{15mm} & Service ID \\
\end{tabular}

\subsubsection{Returns}

\subsubsection{On Success}

\begin{tabular}{lllll}
Variable Name	&		&	Data Type(s)		&	&	Description	\\
				&	&	&	&	\\
\verb!result! & \hspace{15mm} & $\langle boolean\rangle $ & \hspace{15mm} & True if Sucessful \\
\end{tabular}

\subsubsection{On Failure}

\begin{tabular}{lllll}
Variable Name	&		&	Data Type(s)		&	&	Description	\\
				&	&	&	&	\\
\verb!faultString! & \hspace{15mm} & $\langle string\rangle $ & \hspace{15mm} & SQL Error Experienced \\
\end{tabular}

\subsection{GetInstance}

\subsubsection{Description}

Returns the complete details of a service.

\subsubsection{Parameters}

\begin{tabular}{lllll}
Variable Name	&		&	Data Type(s)		&	&	Description	\\
				&	&	&	&	\\
\verb!id! & \hspace{15mm} & $\langle int\rangle $ & \hspace{15mm} & Service ID \\
\end{tabular}

\subsubsection{Returns}

\subsubsection{On Success}

\begin{tabular}{lllll}
Variable Name	&		&	Data Type(s)		&	&	Description	\\
				&	&	&	&	\\
\verb!ServiceID! & \hspace{15mm} & $\langle int\rangle $ & \hspace{15mm} & Service ID \\
\verb!ServiceDate! & \hspace{15mm} & $\langle datetime\rangle $ & \hspace{15mm} & Service Date \\
\verb!Capacity! & \hspace{15mm} & $\langle int\rangle $ & \hspace{15mm} & Capacity Available \\
\verb!Cost! & \hspace{15mm} & $\langle int\rangle $ & \hspace{15mm} & Unit Cost \\
\verb!Source! & \hspace{15mm} & $\langle string\rangle  \mid \langle null\rangle $ & \hspace{15mm} & Source of Service \\
\verb!Destination! & \hspace{15mm} & $\langle string\rangle  \mid  \langle null\rangle $ & \hspace{15mm} & Destination of Service \\
\verb!Details! & \hspace{15mm} & $\langle string\rangle  \mid \langle null\rangle $ & \hspace{15mm} & Additional Details \\
\end{tabular}
\subsubsection{On Failure}

\begin{tabular}{lllll}
				&	&	&	&	\\
Variable Name	&		&	Data Type(s)		&	&	Description	\\
\verb!faultString! & \hspace{15mm} & $\langle string\rangle $ & \hspace{15mm} & SQL Error Experienced \\
\end{tabular}


\subsection{AddBooking}

\subsubsection{Description}

Makes a booking for a specified service.

\subsubsection{Parameters}

\begin{tabular}{lllll}
Variable Name	&		&	Data Type(s)		&	&	Description	\\
				&	&	&	&	\\
\verb!id! & \hspace{15mm} & $\langle int\rangle $ & \hspace{15mm} & Service ID \\
\end{tabular}

\subsubsection{Returns}

\subsubsection{On Success}

\begin{tabular}{lllll}
Variable Name	&		&	Data Type(s)		&	&	Description	\\
				&	&	&	&	\\
\verb!result! & \hspace{15mm} & $\langle boolean\rangle $ & \hspace{15mm} & True if Sucessful \\
\end{tabular}

\subsubsection{On Failure}

\begin{tabular}{lllll}
Variable Name	&		&	Data Type(s)		&	&	Description	\\
				&	&	&	&	\\
\verb!faultString! & \hspace{15mm} & $\langle string\rangle $ & \hspace{15mm} & SQL Error Experienced \\
\end{tabular}


\subsection{EditBooking}

\subsubsection{Description}

Edit the status flags of an existing booking.

\subsubsection{Parameters}

\begin{tabular}{lllll}
				&	&	&	&	\\
Variable Name	&		&	Data Type(s)		&	&	Description	\\
\verb!id! & \hspace{15mm} & $\langle int\rangle $ & \hspace{15mm} & Booking ID \\
\verb!flag! & \hspace{15mm} & $\langle string\rangle $ Property. Expects
& \hspace{15mm} & Attribute  Expects \\
	&	&	&	&	$"Booked" \mid "Flexible" \mid$\\ 
	&	&	&	&	$"Cancelled" \mid "Confirmed"$ \\
\verb!newvalue! & \hspace{15mm} & $\langle boolean \rangle $  & \hspace{15mm} & Desired Value \\
\end{tabular}

\subsubsection{Returns}

\subsubsection{On Success}

\begin{tabular}{lllll}
Variable Name	&		&	Data Type(s)		&	&	Description	\\
				&	&	&	&	\\
\verb!result! & \hspace{15mm} & $\langle boolean\rangle $ & \hspace{15mm} & True if Sucessful \\
\end{tabular}

\subsubsection{On Failure}

\begin{tabular}{lllll}
Variable Name	&		&	Data Type(s)		&	&	Description	\\
				&	&	&	&	\\
\verb!faultString! & \hspace{15mm} & $\langle string\rangle $ & \hspace{15mm} & SQL Error Experienced \\
\end{tabular}


\subsection{DeleteBooking}

\subsubsection{Description}

Delete an existing booking.

\subsubsection{Parameters}

\begin{tabular}{lllll}
				&	&	&	&	\\
Variable Name	&		&	Data Type(s)		&	&	Description	\\
\verb!id! & \hspace{15mm} & $\langle int\rangle $ & \hspace{15mm} & Booking ID \\
\end{tabular}

\subsubsection{Returns}

\subsubsection{On Success}

\begin{tabular}{lllll}
Variable Name	&		&	Data Type(s)		&	&	Description	\\
				&	&	&	&	\\
\verb!result! & \hspace{15mm} & $\langle boolean\rangle $ & \hspace{15mm} & True if Sucessful \\
\end{tabular}

\subsubsection{On Failure}

\begin{tabular}{lllll}
Variable Name	&		&	Data Type(s)		&	&	Description	\\
				&	&	&	&	\\
\verb!faultString! & \hspace{15mm} & $\langle string\rangle $ & \hspace{15mm} & SQL Error Experienced \\
\end{tabular}


\subsection{GetBooking}

\subsubsection{Description}

Returns details on an existing booking.

\subsubsection{Parameters}

\begin{tabular}{lllll}
				&	&	&	&	\\
Variable Name	&		&	Data Type(s)		&	&	Description	\\
\verb!id! & \hspace{15mm} & $\langle int\rangle $ & \hspace{15mm} & Booking ID \\
\end{tabular}

\subsubsection{Returns}

\subsubsection{On Success}

\begin{tabular}{lllll}
Variable Name	&		&	Data Type(s)		&	&	Description	\\
				&	&	&	&	\\
\verb!Booked! & \hspace{15mm} & $\langle boolean\rangle $ & \hspace{15mm} & Booking Details \\
\verb!Flexible! & \hspace{15mm} & $\langle boolean\rangle $ & \hspace{15mm} & Booking Details\\
\verb!Cancelled! & \hspace{15mm} & $\langle boolean\rangle $ & \hspace{15mm} &  Booking Details\\
\verb!Confirmed! & \hspace{15mm} & $\langle boolean\rangle $ & \hspace{15mm} &  Booking Details\\
\end{tabular}
\subsubsection{On Failure}

\begin{tabular}{lllll}
Variable Name	&		&	Data Type(s)		&	&	Description	\\
				&	&	&	&	\\
\verb!faultString! & \hspace{15mm} & $\langle string\rangle $ & \hspace{15mm} & SQL Error Experienced \\
\end{tabular}


\section{Code}

\subsection{SQL Database}

%\lsttset{language=sql,basicstyle=\footnotesize,fancyvrb=true}

%\lstinputlisting{backend.sql}

\subsection{Server}

\subsection{Client}

\section{Working Prototype}

A working demonstration of my server and client are available online.


My server configuration is setup on a server situated in Tampa, Florida.
The URL of the CGI script is
\url{http://master-4ba8.conall.ie/cgi-bin/xmlrpc.cgi}


My client script is hosted on a server situated in Dublin, Ireland. The
URL for the client PHP script is
\url{http://client-4ba8.conall.ie}


Both servers are powered by Debian GNU/Linux, secured by restrictive, 
stateful iptables firewall, making only the necessary ports available.

\section{Technical References}

\begin{tabular}{lll}
Title	&	Author(s)	&	Publisher	\\
Edition	&	ISBN	&	Publish Date 	\\
			&			&						\\
			&			&						\\
Programming Web Services with Perl	&	Pavel Kulchenko,  Randy J. Ray	&
O'Reilly	\\
1st&	0-596-00206-8 & December 2002	\\
			&			&						\\
Programming Web Services with XML-RPC & Edd Dumbill, Joe Johnston, Simon
St. Laurent &	O'Reilly	\\
3rd & 0-596-00119-3	&	June 2001	\\
			&			&						\\
Real World Web Services	&	Will Iverson	&	O'Reilly	\\
1st & 0-596-00642-X	&	October 2004	\\
			&			&						\\
Learning Perl	&	Randal L. Schwartz, Tom Phoenix	&	O'Reilly	\\
3rd  		&	0-596-00132-0	&	July 2001	\\
			&			&						\\
Programming Perl	&	Larry Wall, Tom Christiansen, Jon Orwant	&
O'Reilly	\\
3rd		&	0-596-00027-8	&	July 2000	\\
\end{tabular}

\end{document}
