% $Id$

\documentclass[a4paper,12pt]{article}

\setlength{\parindent}{0mm}
\setlength{\parskip}{7.5mm}

\usepackage{amssymb}
\usepackage[usenames]{color}
\usepackage[arc,poly,all]{xy}
\usepackage{ulem}
\usepackage{listing}
\usepackage{verbatim}
\usepackage{fancyvrb}
\usepackage[left=2cm,right=2cm,top=4cm,bottom=4cm]{geometry}
\usepackage{graphicx}

\begin{document}

\title{4BA8 \\ Assignment}

\author{Conall O'Brien \\ 01734351 \\ conallob@maths.tcd.ie}

\maketitle

\section{Introduction}

Given the broad guidelines relating to the implementation of this 
assignment,  programming language and Web Services, I have decided to 
use three technologies, Perl, MySQL and XML-RPC.


I have chosen to use Perl for multiple reasons, primarily it's
popular use developing server-side applications and dynamic web content.
In addition, the extensive library of modules available to Perl ensures
a complete implementation of numerous web service technologies.


I have chosen to use XML-RPC for the exchanging of RPC messages due to
it's lightweight and uncomplicated definition. It is my belief that the 
use of XML-RPC over compeditors such as SOAP is appropariate for an
application of this size and complexity.


For backend database storage, I have opted to use MySQL, the open source
SQL complient database over other Web Services backend solutions, such 
as UDDI standard, who's definition is incomplete.


In addition, I have chosen to run my Assignment on web hosting servers
running the Apache 2.0 web server running on the Debian GNU/Linux
platform. To satisfy the assignment requirement of "system must be 
deployable across the Internet. It should, for example, be possible for 
services to interoperate in an environment controlled by firewalls and 
other Internet infrastructure", I have used MacOS X, FreeBSD and Ubuntu 
GNU/Linux, systems connected to the TCD network, located behind the
main TCD firewall configuration (consisting of a Cisco PIX 515 and a 
Cisco PIX 525).

\section{Architectural Design}

My application design is simple, based upon the client - server
architecture. A Perl script on the primary server acts as a translator
between XML-RPC calls to the MySQL backend and the client script. An 
additional feature of the server script not implemented by the 
assignment of the deadline allows two server scripts to communicate
between each other, allowing for backend replication between multiple,
resielient hosts using XML-RPC.


A second Perl script, the client script is a Command Line Interface 
(CLI) to the server. It allows a user to do multiple actions, such as
tourism service searchs (queries on the database), comparison of tourism 
services (comparison of query results) and creation of new service
instances (insert queries on the backend).

\subsection{Description of Functions}

% function: AddService

\subsubsection{AddService}

This function is used to add a new type of service into the system. 

To accomodate dynamic creation of new service types, this function
performs one, optionally two operations. When an administrator invokes 
AddService, they are given the choice to select an existing service 
type, or add a new one. Once they have selected their service type, they
then add service details, such as name and create a unique identifer
used throughout the backend system.

% function: RmService

\subsection{RmService}

This function compliments AddService. It simply removes a service
operator from the list of known operators. Additionally, in order to
prevent hanging dependancies, RmService also check for service instances
accreddited to the deleted service and remove them, in order to ensure
data consistincy in the booking system.

% function: EditService

\subsection{EditService}

This function allows a service partner to edit their service details,
such as their name.

% function: SearchService

\subsection{SearchService}

This function is the core of the intented user experience. Based upon
supplied criteria, it searches the collection of service instances 
currently available and returns summary details of the results to enable
comparisons of the search results.

% function: GetService

\subsection{GetService}

GetService retrieves detailed information of a selected service,
including Date, Time, Cost, Capacity, and additional details.

% function: MkBooking

\subsection{MkBooking}

MkBooking allows a system user to create a booking of a selected service
instance. At booking time, a user can make their booking flexible,
meaning a different user may be able to reserve the booking.

% function: GetBooking

\subsection{GetBooking}

GetBooking returns the status of an existing booking, confirming if a
booking exists andf whether it is confirmed, flexible or cancelled.

% function: EditBooking

\subsection{EditBooking}

EditBooking allows the booking system to edit the status flags of an
existing booking. It can be used to mark a booking flexible or
inflexible, to confirm or unconfirm a booking and to cancel a booking.

\section{Code}

\subsection{SQL Database}

%\lsttset{language=Python,basicstyle=\footnotesize,fancyvrb=true}

%\lstinputlisting{backend.sql}

\subsection{Server}

\subsection{Client}

\section{Sample Output}

\section{Bibliography}



\end{document}
