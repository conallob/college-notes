% $Id: 20041012.tex 336 2004-10-27 12:47:38Z conall $

\documentclass[a4paper,12pt]{article}
\usepackage{amssymb}
\usepackage[all]{xy}


\setlength{\parindent}{0mm}
\setlength{\parskip}{7.5mm}

\begin{document}

\title{Course 3BA7: Computer Engineering \\ Lecture Notes \\ $12^{th}$ October 2004}

\maketitle

Dan McCarthy

Room: ORI G26

mccarthy@cs.tcd.ie

\section*{Objective}

To acquire an understanding of computer architecture, both behaviour and
design, from von Neuman's single processor up to the highly parallel
pipelined and multiprocessor systems in use today.

\section*{References}

See Handout

\section*{Course Organisation}

Lectures will be self contained, drawing on material from Zarghan, Hwang
and Xu, and Quinn.

Tutorials will be based on problem material drawn from these texts,
supplemented with my own questions.

Projects will require you to implement both pipelined and multiprocessor
functions. 3 projects, 1 per term.

Assessment:

\begin{tabular}{lccr}
Exam based on lectures, projects and tutorials	&	&	&	$80\%$	\\
Projects													 	&	&	&	$20\%$	
\end{tabular}

A pass is required for the project material.

\section*{Introduction: Fundamental Constraints}

We can transport information as electro-magnetic waves at $C = 300,000
km/s$ in free space; in copper they travel at about $0.99 C$, slowing to
about $0.001C$ passing through trnasistors (gates).

% Circuit diagram

For a given clock rate $f$, this puts a limit on the physical seperation
of registers and hence system size.

Consider for example, a process $P$ with a clock rate $f = 1GHz =
1000 MHz = 109 cycles/s$

Then the period $\tau = \frac{1}{f} = \frac{1}{109} = 10^{-9} = 1 ns$

Graphically:

% Graph Diagram

Some registers load values.

Some registers load results.


Clearly the maximum distance between the source and destination
registers is:

\[ d = c \times \tau = 3 \times 10^{10} \times 10^{-9} = 30 cm\]

\end{document}
