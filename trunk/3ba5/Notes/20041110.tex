% $Id$

\documentclass[a4paper,12pt]{article}
\usepackage{amssymb}
\usepackage{amsfonts}
\usepackage{amsmath}
\usepackage[all]{xy}
\usepackage{tabularx}


\setlength{\parindent}{0mm}
\setlength{\parskip}{7.5mm}

\begin{document}

\title{Course 3BA5: Computer Engineering \\ Lecture Notes \\ $10^{th}$ November 2004}

\maketitle

\section*{27$^{\mbox{th}}$ October Tutorial Solutions}

\subsection*{4}

(a)

\begin{verbatim}
     XN(i) = x;
     do i = 2 to 8
            XN(i) = x * XN(i - 1);
     end
\end{verbatim}

(b)

On a dataflow machine:

\begin{table}[hbtp]

% xy-pic diagram

\end{table}

\begin{table}[hbtp]

% xy-pic schematic diagram

\end{table}

\begin{table}[hbtp]

% xy-pic schematic diagram

\end{table}

\subsection*{5}

\begin{eqnarray*}
f(x)	&	=	&	3 + 5x - 2x^{2} + 4x^{3} - 6x^{4} = \sum^{4}_{1} a_{i} x	\\
		&	=	&	3 + x \left(5 + x \left(-2 + x \left(4 + x \left(6\right) \right) \right) \right)	\\ 
\end{eqnarray*}

(a)

On a SISD - Horner's Method

\begin{verbatim}
     f = a(4);
     do i = 3 to 0
            f = a(i) + x * f;
     end
\end{verbatim}

(b)

On a dataflow

\begin{table}[hbtp]

% xy-pic diagram

\end{table}

\end{document}
