% $Id$

\documentclass[a4paper,12pt]{article}
\usepackage{amssymb}
\usepackage{amsfonts}
\usepackage{amsmath}
\usepackage[all]{xy}
\usepackage{tabularx}


\setlength{\parindent}{0mm}
\setlength{\parskip}{7.5mm}

\begin{document}

\title{Course 3BA5: Computer Engineering \\ Lecture Notes \\ $10^{th}$ November 2004}

\maketitle

\section*{27$^{\mbox{th}}$ October Tutorial Solutions}

\subsection*{4}

(a)

\begin{verbatim}
     XN(i) = x;
     do i = 2 to 8
            XN(i) = x * XN(i - 1);
     end
\end{verbatim}

(b)

On a dataflow machine:

\begin{table}[hbtp]

% xy-pic diagram

\end{table}

\begin{table}[hbtp]

% xy-pic schematic diagram

\end{table}

\begin{table}[hbtp]

% xy-pic schematic diagram

\end{table}

\subsection*{5}

\begin{eqnarray*}
f(x)	&	=	&	3 + 5x - 2x^{2} + 4x^{3} - 6x^{4} = \sum^{4}_{1} a_{i} x	\\
		&	=	&	3 + x \left(5 + x \left(-2 + x \left(4 + x \left(6\right) \right) \right) \right)	\\ 
\end{eqnarray*}

(a)

On a SISD - Horner's Method

\begin{verbatim}
     f = a(4);
     do i = 3 to 0
            f = a(i) + x * f;
     end
\end{verbatim}

(b)

On a dataflow

\begin{table}[hbtp]

% xy-pic diagram

\end{table}

\section*{6}

(a)

SISD will require $4 \times 2 = 8$ cycles

(b)

Dataflow will require $5$ instruction cycles and $3$ multiply circuits
and $1$ add circuit.

\section*{7}

$f = \sum^{n}_{i - 0} a_{i} x$

(a)

\begin{verbatim}
     f = a(n);
     do i = n - 1 to 0
            f = a(i) + x * f;
     end
\end{verbatim}

The body of this \verb!do\verb! loop is executed in times and each
execution requires $1$ multiplication and $1$ addition, thus
\verb!2n\verb! instruction cycles and required, ie it is $O(n)$.

(b)

A non-optimal but straightforward evaluation on a dataflow machine is:

\begin{tabular}{llll}
Operation					&	\# Ins Cycles	&	\# Multiply Circuits	&	\# Add Circuils	\\
								&						&								&							\\
Compute $x', i = 2, n$	&	$\log_{2}{n}$	&	$\frac{n}{2}$			&	$0$					\\
Compute $a_{i}$, $x^{\lambda}$', $i = 1, n$	&	$1$	&	$n$		&	$0$					\\
Compute $\sum a_{i} x'$	&	$log_{2}{n}$	&	$0$						&	$\frac{n}{2}$		\\
\end{tabular}

Total number of instruction cycles $= -\log{n} + 1 + \log{n} \Rightarrow
2 \log{n} + 1 \Rightarrow 2 \log{n}$

Thus dataflow requires time $O\left( \log{n} \right)$ versus SISD time
$O(n)$.

Total multiply$^{n}$ and add$^{n}$ circuits required - $n \times$
mult$^{n}$ circuits and $\frac{n}{2}$ add circuits - total
$\frac{3n}{2}$ circuits of SISD requires $2$.

\end{document}
