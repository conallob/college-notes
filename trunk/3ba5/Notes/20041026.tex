% $Id: 20041012.tex 336 2004-10-27 12:47:38Z conall $

\documentclass[a4paper,12pt]{article}
\usepackage{amssymb}
\usepackage[all]{xy}
\usepackage{tabularx}


\setlength{\parindent}{0mm}
\setlength{\parskip}{7.5mm}

\begin{document}

\title{Course 3BA5: Computer Engineering \\ Lecture Notes \\ $20^{th}$ October 2004}

\maketitle

\section*{Dataflow Machines}

Dataflow attempts to solve the \emph{"von Neumann"} bottleneck problem
by linking data to instructions and issuing the instructions fir
executuon as such as all the necessary data binding is complete.

Eg: $f \leftarrow 3 x^{2} + 6 x + 7$. $\therefore x \leftarrow 6$.

\begin{table}[hbtp]

% xy-pic Diagram

\end{table}

% Unresolved problem with \underset - 20050105

$\underset{\^}{x}, \underset{\^}{y} \left\{ 0 - 1000 \right\}$

$t$ - Scalar.

\begin{eqnarray*}
t						&	\leftarrow	&	3 \times 3				\\
\underset{^}{y}	&	\leftarrow	&	\underset{^}{X} + t	\\
\end{eqnarray*}

The pseudo random character of program assignments makes the efficient
design of the arbitration, distribution networks a formidable task.
Either, dataflow architecture has no inherent ability ot avail of the
efficiencies possible in executing vector assignments, such as
$\underset{\^}{n} \leftarrow + 3.7$ which is frequent, and execute very
efficiently on a vector processor.

So far, a number of experimental dataflow machines have been
constructed, but none have been marketed.

\section*{Systolic Arrays}

There are $2$ or $3$ dimensional pipelines, suitable only for
applications with very reegular data movement. For example - matrix
multiplication.

\begin{table}[hbtp]

% xp-pic diagram

\end{table}

\end{document}
