% $Id: 20041012.tex 336 2004-10-27 12:47:38Z conall $

\documentclass[a4paper,12pt]{article}
\usepackage{amssymb}
\usepackage[all]{xy}
\usepackage{tabularx}


\setlength{\parindent}{0mm}
\setlength{\parskip}{7.5mm}

\begin{document}

\title{Course 3BA5: Computer Engineering \\ Lecture Notes \\ $19^{th}$ October 2004}

\maketitle

\section*{Taxonomy}

\begin{table}[hbtp]

% xy-pic Diagram

\end{table}

Where

\begin{tabular}{lll}
$IS$	&	$=$	&	Instruction Set			\\
$IS'$	&	$=$	&	Decoded Instruction Set	\\
$DS$	&	$=$	&	Data Stream					\\
$IR$	&	$=$	&	Instruction Register		\\
\end{tabular}

\subsection*{SISD - Von Neumann}

This schematic view emphasises the seperate $IS$ and $DS$, giving them
seperate paths to the $MM$. In practise, only a single pathway is built,
leading to the so called Von Neumann bottleneck.

% xy-pic Schematic Diagram

(more accurate version of above)

Examples: IBM 360, DEC, PCs, Majority of processor systems.

\subsection*{SIMD - Array Processor}

% xy-pic Schematic diagram

Example:

$C$ is a scalar in each $Mi$.				\\
$\vec{X}$ vector $X_{i}$ in each $Mi$	\\
$\vec{Y}$ vector $Y_{i}$ in each $Mi$	\\

\begin{eqnarray*}
C			&	\leftarrow	&	7.3 \mbox{\hspace{5mm}(ie load direct)}	\\
\vec{Y}	&	\leftarrow	&	\vec{X} + C						
\end{eqnarray*}

Eg: The Connections Machine ($Cn1$, $Cn2$)

\end{document}
