% $Id$

\documentclass[a4paper,12pt]{article}
\usepackage{amssymb}
\usepackage{amsfonts}
\usepackage{amsmath}
\usepackage[all]{xy}
\usepackage{tabularx}


\setlength{\parindent}{0mm}
\setlength{\parskip}{7.5mm}

\begin{document}

\title{Course 3BA5: Computer Engineering \\ Lecture Notes \\ $9^{th}$ November 2004}

\maketitle

Another approach - Not in Zarghon

Split Level Control Memory (Nanocode)

When a typical instruction set is microcoded relatively few different
control words ($CW$) are found, which results in inefficient life for
the CROM.

Consider, for example a $64k$ CROM delivering a $250$ bit $CW$, then a
typical straightforward implementation will require:

\begin{table}[hbtp]

% xy-pic Diagram

\caption{Single Level Control State}

\end{table}

If however, in the CROM, we find only $512 = 2^{a}$ different $CW$s, we
may restructure it to reduce the total ROM.


\begin{table}[hbtp]

% xy-pic Diagram

\caption{Split Level Control State}

\end{table}

\subsection*{Compare Total ROM}

\begin{eqnarray*}
\mbox{Single Level}	&	=	&	2^{16} \times 270 \approx 2^{24} 		\\
\mbox{Two Level}		&	=	&	2^{16} \times 29 + 29 \times 250			\\
							&	=	&	2^{16} \times 25 + 2^{9} \times 2^{8}	\\
							&	=	&	2^{21} + 2^{17}								\\
							&	=	&	2^{21}											\\
\end{eqnarray*}

Single Level : Two Level - $2^{24} : 2^{21} \approx 8 : 1$

It is nearly one order or magnitude in reduction.

\section*{Hardwaired Vs Microcoded Implementations - A Changing
Perspective}

For a microcasm of implementation issues, consider two ways to implement
the following function:

\begin{table}[hbtp]

% xy-pic diagram

\end{table}


\begin{table}[hbtp]

% xy-pic schematic diagram

\caption{Implementation A}

\end{table}

\begin{table}[hbtp]

% xy-pic schematic diagram

\caption{Implementation B}

\end{table}

$F = ABC + AB\overline{C} + A\overline{B}C + \overline{A}B\overline{C} +
\overline{A}BC$

$2^{3} \times 3 - 1 AND$ Array $1 \times 8 - 1 OR$


\begin{table}[hbtp]

% xy-pic diagram

\caption{Microcoded Solution}

\end{table}

\subsection*{Traditional (1970 - 1995) Views}

\begin{tabular}{|l|l|l|}
\hline
Aspect					&	A - Hardwired	&	B - Microcoded		\\
\hline
Size						&	Minimum			&	Larger				\\
\hline
Propogation Delay		&	Minimum			&	Larger				\\
\hline
Power						&	Minimum			&	More					\\
\hline
Flexability				&	None				&	Complete				\\
\hline
Change					&	Redesign			&	Reconnect			\\
\hline
\end{tabular}

\subsection*{Traditional Conclusion}

Unless you are desperate for speed, use microcode.
But from the 90's, VLSI layout and CAD tools were able to convert from
microcode into near optimal PLA layout, so in a sophisticated development
enviroment, the traditional arguement doesn't stand.

\end{document}
