% $Id: bill-coffey.tex,v 1.27 2004/05/14 12:15:46 conall Exp $

\documentclass[a4paper,12pt]{article}

\usepackage{amssymb}

\usepackage[arc,poly,all]{xy}
\usepackage{pstcol,pst-plot}

\setlength{\parindent}{0mm}
\setlength{\parskip}{7.5mm}

\begin{document}

%\chapter{Bill Coffey}

% 20040105

\section{Introduction}

\begin{eqnarray*}
M & \to & kg \\
L & \to & m \\
T & \to & s \\
Q & \to & Coulombs
\end{eqnarray*}

Origins of Electricity - 600 BC - Thales of Miletus

Electricity - Greek work meaning "amber"

1687 - Newton - Principia - Gravitational Force

\begin{table}[hbtp]

\xy <1cm, 0cm>:
0 *+!CR{m_{1}} *+!UC{\vec{F_{12}}} ; (2.5,0) **\dir{-} *\dir{>} ?(1.15)
*+!CL{m_{2}} *+!DC{\vec{r_{12}}}
\endxy

\end{table}

\[ \vec{F_{12}} = \frac{\gamma m_{1} m_{2} \hat{\vec{r_{12}}}
}{r_{12}^{2} } \]

\[ \hat{\vec{r_{12}}} = \frac{\vec{r_{12}}}{\left| \vec{r_{12}} \right|}
\]

1770 - A Coulomb

\[\vec{F_{12}}^{el} = \frac{q_{1} q_{2} \hat{\vec{r_{12}}} }{ 4 \pi
\varepsilon_{0} r_{12}^{2} } \]

\begin{table}[hbtp]

\xy <1cm, 0cm>:
0 *+!CR{q_{1}} ; (2.5,0) **\dir{-} *\dir{>} ?(1.0)
*+!CL{q_{2}} *+!DC{\vec{r_{12}}}
\endxy

\end{table}

If the charges $q_{1}$ and $q_{2}$ are of the same sign, they
repel one another. And if they are opposite sign, they attract one
another. Moreover, the electrostatic force is billions of times stronger
than the gravitational force.

\[ \varepsilon_{0} = \frac{1}{36 \pi} \times 10^{-9} \mbox{  - Permitivity
of Free Space (Units: } Farads/m) \]

\[ i(t) = \frac{d Q(t)}{dt} = \frac{\mbox{Coulombs}}{\mbox{second}} = 
\mbox{Amperes (aka Amps) (Symbol: $A$)} \]

\begin{table}[hbtp]

\xy<1cm,0cm>:
0       ; (8,0) **\dir{-} ?(1.05) *+!UR{t} ,
(0,2)   ; (8,2) **\dir{-} ,
(0, -1) ; (0,3) **\dir{-} *+!UL{i(t)} ,
(4,0) *\dir{<} ; (4,2) **\dir{-} *\dir{>} ?(0.65) *+!UL{I}
\endxy

\end{table}

\begin{table}[hbtp]

\begin{pspicture}(-4,-1)(4,1)
\tiny
\psset{xunit=0.011,yunit=1}
\psaxes[Dx=60,Dy=1](0,0)(-359,-1)(360,1)
\psplot{-360}{360}{x sin}
\psplot{-360}{360}{x x mul sin}
\end{pspicture}

\end{table}

\[ i(t) = I_{m} sin \omega t \mbox{ - Amplitude of I} \]

$ \omega \mbox{  - Angular Frequency.} $ \\
$ \omega = 2 \pi f \mbox{ (radians/sec)} $ \\
$ \omega t = \Phi $ \\

We would like to have a measure which will make the AC behave
like DC.

\begin{table}[hbtp]

\xy <1cm,0cm>:
0 		; (8,0) **\dir{-} ,
(1,-1) ; (1,4) **\dir{-} ,
(3,-1) *+!DC{a} ; (3,4) **\dir{-} ,
(5,-1) *+!DC{b} ; (5,4) **\dir{-} ,
(7,-1) ; (7,4) **\dir{-} ,
(1.5,0.5)  ; (8,4) **\crv{(3,4)&(3,-3)} *+!DL{f(x)}
\endxy

\end{table}

$\overline{f(x)} = \frac{1}{b - a} \int^{b}_{a} f(x) dx$ \\
$\sin \omega t - \mbox{repeats every} T $ \\
$ T = \frac{2 \pi}{\omega} $

\begin{eqnarray*}
	& = & \sin \omega t \\
	& = & \sin \omega (t + \frac{2 \pi}{\omega}) \\
	& = & \sin \omega (\omega t + 2 \pi) \\
	& = & \sin \omega t \cos 2 \pi + \cos \pi t \sin 2 \pi \\
	& = & \sin \omega t
\end{eqnarray*}

\[ \frac{2 \pi}{\omega} = T \rightarrow \mbox{Periodic Time} \]

\[ \sin \omega (t + T) = \sin \omega t = \sin \omega (t - T) \]

\[ f(t \pm nT) = f(t) \]

\[ n = 1, 2, 3, ... \]

The period of every $\frac{2 \pi}{\omega}$ will show the
function for all time.

\begin{eqnarray*}
i(t) 				 & = & I_{m} \sin \omega t \\
\overline{i(t)} & = & \frac{1}{T} \int^{T}_{0} I_{m} \sin \omega t dt \\
					 & = & \frac{\omega}{2 \pi} \int^{\frac{2 \pi}{\omega}}_{0} I_{m} \sin \omega t dt \\
\Rightarrow \omega t & = & \Phi \\
		\omega dt & = & d \Phi \\
				  t & = & 0 \\
			  \Phi & = & 0 \\
				  t & = & \frac{2 \pi}{\omega} \\
			  \Phi & = & 2 \pi \\
\Rightarrow \overline{i(t)} & = & \frac{\omega}{2 \pi} \int^{2 \pi}_{0} \frac{I_{m} \sin \Phi d \Phi}{\omega} \\
					 & = & 2 \pi \int^{2 \pi}_{0} I_{m} \sin \Phi d \Phi \\
\frac{d}{dx} (\cos x)& = & - \sin x \\
\frac{-I_{m}}{2 \pi} \cos \Phi |^{2 \pi}_{0} & = & -
\frac{I_{m}}{2 \pi} \left[ 1 - 1\right] = 0
\end{eqnarray*}

No use talking about the average value of AC, since it will
always be zero.

\[ \overline{i^{2}(t)} = \frac{1}{T} \int^{T}_{0} i^{2}(t) dt =
\sqrt{i^{2}(t)} \]

\begin{table}[hbtp]

Diagram - To Do

\end{table}

When squared, the average value of $i(t)$ is no longer less
than or equal to zero.

% 20040107

$ \overline{f(t)} = \frac{1}{T} \int^{T}_{0} f(t) dt $ \\
$ f(t \pm nT) = f(t) $ \\

Periodic fn in st.

$ i(t) = I_{m} \sin \omega t $ \\
$ T = \frac{ 2 \pi}{\omega} $ \\

$fn$ repeats itself after a time period of $T = \frac{2 \pi}{\omega}$.

\begin{table}[hbtp]

\begin{pspicture}(-4,-1)(4,1)
\tiny
\psset{xunit=0.011,yunit=1}
\psaxes[Dx=60,Dy=1](0,0)(0,-2)(720,2)
\psplot{0}{720}{x sin}
\end{pspicture}

\end{table}

\[ T= \frac{2 \pi}{\omega} \]

Half Wave rectified Sine Wave

\begin{table}[hbtp]

Diagram - To Do

\end{table}

\[ T = \frac{\pi}{\omega} \]

Full Wave Rectifier

\[ i(t) = I_{m} \sin \omega t \]

\[ \left| \sin \omega t \right| \]

\[ \overline{f^{2}(t)} = \frac{1}{T} \int^{T}_{0} f^{2} (t) dt \]

\[ i^{2}(t) = I_{m}^{2} \sin^{2} \omega t \]

\[ \cos 2 A = 1 - 2 \sin^{2} A \]

\[ \sin^{2} A = \frac{1}{2} (1 - \cos 2A) \]

\[ i^{2}(t) = \frac{I_{m}^{2}}{2} (1 - \cos 2 \omega t) \]

\[ -1 \leq \cos^{2} 2 \omega t \leq 1 \]

The square of the current goes to zero, but never goes
negative. Furthermore, the square of the current oscillates with the
angular frequency $2 \omega$.

\[ \overline{i^{2}(t)} = \frac{1}{T} i^{2}(t) dt = \frac{1}{T}
\int^{T}_{0} \frac{I_{m}^{2}}{2} (1 - \cos 2 \omega t) dt \]

$T$ is of course the periodic time of $i(t)$. $T = \frac{2
\pi}{\omega}$.

\begin{eqnarray*}
\overline{i^{2}(t)} & = & \frac{\omega}{2 \pi} \frac{I_{m}^{2}}{2} \int^{\frac{2 \pi}{\omega}}_{0} (1 - \cos 2 \omega t) dt \\
& = & \frac{\omega}{2 \pi} \frac{I_{m}^{2}}{2} \int^{\frac{2 \pi}{\omega}}_{0} dt - \int^{\frac{2 \pi}{\omega}}_{0} \cos 2 \omega t dt \\
& = & \frac{\omega}{2 \pi} \frac{I_{m}^{2}}{2} \frac{2 \pi}{\omega} - \int^{\frac{2 \pi}{\omega}}_{0} \cos 2 \omega t dt \\
\int^{\frac{2 \pi}{\omega}}_{0} \cos 2 \omega t dt \\
\omega t & = &\Theta \\ 
\omega dt & = & d\Theta \\
t & = & 0 \\ 
\Theta & = & \frac{2 \pi}{\omega} \\ 
\Theta & = & 2 \pi \\
\int^{2 \pi}_{0} \frac{\cos 2 \Theta d \Theta}{\omega} & = & \frac{1}{\omega} \left[ \frac{\sin 2 \Theta}{2} \right]^{2 \pi}_{0} \\
\frac{1}{\omega} (0 - 0) \\
\overline{i^{2}(t)} & = & \frac{I_{m}^{2}}{2} \\
\sqrt{i^{2}(t)} & = & \frac{I_{m}}{\sqrt{2}} I
\end{eqnarray*}

$I = RMS$ value of $i(t) = I_{m} \sin \omega t$.

$V = VMS$ value of $V(t) < V_{m} \sin \omega t$.

\[ V = \frac{V_{m}}{\sqrt{2}} = \frac{0.707 V_{m}}{220V} \]


Tutorial Assignment: Find $\overline{V(t)}$ and $\overline{V^{2}(t)}$.

\subsection{$\overline{V(t)}$}

\begin{eqnarray*}
\overline{V(t)} & = & \frac{1}{T} \int^{T}_{0} V(t)  dt \\
					 & = & \frac{1}{T} \int^{T}_{0} V_{m} \times \cos\omega t dt \\
					 & = & \frac{V_{m}}{T} \left| \sin\omega t \right|^{T}_{0} \\
					 & = & \frac{V_{m}}{T} \times \left( \sin\omega T - \sin\omega 0 \right) \\
					 & = & \frac{V_{m}}{T} \times \sin\omega T \\
					 & = & \frac{V_{m} \omega}{2 \pi} \times \sin\omega \frac{2 \pi}{\omega} \\
					 & = & \frac{V_{m} \omega}{2 \pi} \times \sin2\pi \\
					 & = & 0
\end{eqnarray*}

\subsection{$\overline{V^{2}(t)}$}

\begin{eqnarray*}
\overline{V^{2}(t)} & = & \frac{1}{T} \int^{T}_{0} V^{2}(t)  dt \\
						  & = & \frac{1}{T} \int^{T}_{0} (V_{m}^{2} \cos^{2}{\omega t}) dt \\
						  & = & \frac{1}{T} \int^{T}_{0} V_{m}^{2} (1 + \cos{2 \omega t}) dt \\
						  & = & \frac{V_{m}^{2}}{T} \int^{T}_{0} dt + \cos{2 \omega t} dt \\
						  & = & \frac{V_{m}^{2}}{T} \left| t + \sin{2 \omega t} \right|^{T}_{0} \\
						  & = & \frac{V_{m}^{2}}{T} \times \left(T + \sin{2 \omega T} - 0 - \sin{2 \omega 0} \right) \\
						  & = & \frac{V_{m}^{2}}{T} \times \left(T + \sin{2 \omega T} \right) \\
						  & = & \frac{V_{m}^{2} \times \omega}{4 \pi} \times \left(\frac{2 \pi}{\omega} + \sin{4 \pi} \right) \\
						  & = & \frac{V_{m}^{2} \times \omega}{4 \pi} \times \frac{2 \pi}{\omega} \\
						  & = & \frac{V_{m}^{2}}{2}
\end{eqnarray*}

% 2004-01-08

\begin{table}[hbtp]

Diagram - To Do

\end{table}

\[ i(t) = I_{m} \sin \omega t \]

In general, in an AC circuit the instantaneous current $i(t)$
and instantaneous voltage $v(t)$ will not be in phase with each other.
They will lag or lead each other by an order greater than $\Phi$.

$\Phi$ is called $i(t) = I_{m} \sin \omega t$ in phase.

\begin{table}[hbtp]

Diagram - To Do

\end{table}

\begin{table}[hbtp]

Digram - To Do

\end{table}

\begin{table}[hbtp]

Diagram - To Do

\end{table}

If you have a purely resistive circuit, the voltage and
current are always applied in phase because Ohm's Law applies. ie $v(t)
= i(t) R$.

For capacitive circuit:

\[ v(t) = L \frac{q(t)}{C} \]

\[ i(t) = \frac{d q(t)}{dt} \]

\[ q(t) = \int^{t}_{0} i(t') dt' \]

\[ v(t) = \int^{t}_{0} \frac{d(t')}{C} \]

The applied voltage is proportional to the integral of the
current. Ohm's Law \emph{does not} apply. In such a circuit the current
leads the voltage by $90^{o}$ (the phase angle).

For the capacity inductive circuit:

\[ v(t) = L \frac{d i(t)}{dt} \]

The applied voltage is proportional to the derivative of the
current - Faraday and Lenz's Law. And the constant of proportionality
$L$ is called the self inductance of the circuit. Again Ohm's Law does
not apply, the voltage will lead the current by $90^{o}$.

\subsection{Tutorial Questions}

\[ i(t) = I_{m} \sin(\omega t + \Phi) \]

Find $i(t) and i^{2}(t)$.

\begin{eqnarray*}
\overline{i(t)} & = & \frac{1}{T} \int^{T}_{0} i(t) dt \\
					 & = & \frac{\omega}{\2 \pi} I_{m} \int^{\frac{2\pi}{\omega}}_{0} 
					 		 \sin(\omega t + \Phi) dt \\
\omega t + \Phi & = & \theta \\
\omega dt		 & = & d\theta \\
t					 & = & 0 \\
\theta			 & = & \Phi \\
t 					 & = & \frac{2 \pi}{\omega} \\
\theta			 & = & 2 \pi + \Phi \\
\overline{i(t)} & = & \frac{\omega}{2 \pi} I_{m} \int^{2 \pi + \Phi}_{\Phi} 
							 \sin\theta \frac{d\theta}{\omega} \\
\int^{b + a}_{b}& = & \int^{a}_{0} \\
\overline{i(t)} & = & \frac{I_{m}}{2 \pi} \int^{2 \pi}_{0} \sin\theta d\theta \\
					 & = & \frac{I_{m}}{2 \pi} \left[ - \cos\theta\right]^{2\pi}_{0} \\
					 & = & 0
\end{eqnarray*}

The phase angle $\Phi$ makes no difference when calculating
the average value.

Mean Square Value:

\begin{eqnarray*}
\overline{i^{2}(t)} & = & \frac{1}{T} \int^{T}_{0} i^{2}(t) dt \\
						  & = & \frac{\omega}{2 \pi} I_{m}^{2} \int^{\frac{2
						  \pi}{\omega}}_{0} \sin^{2}(\omega t + \Phi) dt \\
\omega t + \Phi	  & = & \theta \\
\omega dt			  & = & d \theta \\
dt 					  & = & \frac{d \theta}{\omega} \\
t						  & = & 0 \\
\theta				  & = & 0 \\
t						  & = & \frac{2 \pi}{\omega} \\
\theta				  & = & 2 \pi + \Phi \\
\overline{i^{2}(t)} & = & \frac{\omega}{2 \pi} I_{m}^{2} \int^{2 \pi +
\Phi}_{\Phi} \sin^{2} \theta \frac{d \theta}{\omega} \\
						  & = & \frac{I_{m}^{2}}{2 \pi} \int^{2 \pi}_{0}
						  \sin^{2} \theta \frac{d \theta}{\omega} \\
						  & = & \frac{I_{m}^{2}}{2 \pi} \int^{2 \pi}_{0}
						  \frac{1}{2} \left( 1 - \cos 2 \theta \right) d
						  \theta \\
						  & = & \frac{I_{m}^{2}}{2 \pi} \frac{1}{2} \left[
						  \theta - \frac{\sin 2 \theta}{2} \right]^{2 \pi}_{0} \\
						  & = & \frac{I_{m}^{2}}{4 \pi} \left[ 2 \pi - \sin{\frac{4 \pi}{2}} + \sin{\frac{\theta}{2}} \right] \\
						  & = & \frac{I_{m}^{2}}{2} \\
I						  & = & \frac{I_{m}}{\sqrt{2}}
\end{eqnarray*}

The phase angle makes no difference to the root mean square
value of the average value.

When you talk about $220V 50Hz$ supply, you talk about the RMS
value.

Speech and music are composed of complicated superpositions of
sine waves and the same relation between $I$ and $V$.

\[ i(t) = I_{m} \sin{\omega t} + I_{m2} \sin{2 \omega t} + I_{m3} \sin{3
\omega t} ... = \sum^{\infty}_{n = 1} \sin{n \omega t} \]

The term frequency, $\omega$, is the fundamental frequency
that with $2 \omega$ is called the second harmonic. $3 \omega$ is the
third harmonic and so on. How do you work out the mean square value of
such complex wave forms. Eg 

\[ i(t) = I_{m1} \sin{\omega t} + I_{m2} \sin{2 \omega t} \]

\begin{table}[hbtp]

Diagram - To Do

\end{table}

The fundamental that one executes one complete cycle in the
time $\frac{2 \pi}{\omega}$. It's amplitude is $I_{m1}$. The second
harmonic, because of it's frequency, is twice that of the fundamental,
executing $2$ oscillations in the fundamental time period $\frac{2
\pi}{\omega}$.

\[ T_{2} \mbox{(second harmonic)} = \frac{\pi}{\omega} \]

If you have a complex wave if it follows that averaging over
the period of the fundamental will include the averages over all the
regular harmonics.

\[ i(t) = \frac{1}{T} \int^{T}_{0} (I_{m1} \cdots + I_{m2}) dt \]

Prove for $\overline{i(t)} = 0$.

% 2004-01-12

\[ v(t) = V_{m1} sin{\omega t} + V_{m2} \sin{2 \omega t} \]

\begin{table}[hbtp]

Diagram - To Do

\end{table}

\[ T_{1} = \frac{2 \pi}{\omega} \]

\[ T_{2} = \frac{\pi}{\omega} \]

\[ \overline{v(t)} = \frac{1}{T} \int^{T}_{0} \left(V_{m1} \sin{\omega
t} + V_{m2} \sin{\omega t} \right) dt \]

\[ T = \frac{2 \pi}{\omega} \]

\begin{eqnarray*}
\overline{v(t)} & = & \frac{\omega}{2 \pi} \int^{\frac{2 \pi}{\omega}}_{0} \left( V_{m1} \sin{\omega t} + V_{m2} \sin{2 \omega t} \right) dt \\
\int^{\frac{2 \pi}{\omega}}_{0} \sin{\omega t}dt & ; & \int^{\frac{2 \pi}{\omega}}_{0} \sin{2 \omega t}dt \\
\omega t 		 & = & \theta \\
t					 & = & 0 \\
\theta			 & = & 0 \\
t					 & = & \frac{2 \pi}{\omega} \\
\theta 			 & = & 2 \pi \\
\int^{2 \pi}_{0} \sin{\theta}d\theta & ; & \int^{2 \pi}_{0} \sin{2 \theta}d\theta \\
\frac{d}{d\theta} \left(\cos{\theta} \right) & = & - \sin{\theta} \\
															& = & \left[ - \cos{\theta}
															\right]^{2 \pi}_{0} \\
															& = & - [1 - 1] \\
															& = & 0 \pi \\
\int^{2 \pi}{0} \sin{2 \theta}d\theta			& = & \left[ - \frac{\cos{2 \theta}}{2} \right]^{2 \pi}_{0} \\
															& = & - [1 - 1] \\
															& = & 0 
\end{eqnarray*}

\begin{table}[hbtp]

Diagram - To Do

\end{table}

\[ v(t) = V_{0} + V_{m1} \sin{\omega t} \]

\subsection{}

\begin{eqnarray*}
v(t)						& = & V_{m1} \sin{\omega t} + V_{m2} \sin{2 \omega t} \\
v^{2}(t)					& = & V^{2}_{m1} \sin^{2}{\omega t} + V^{2}_{m2} \sin^{2}{2 \omega t} + 2V_{m1}V_{m2} \sin{\omega t}\sin{2 \omega t} \\
\overline{v^{2}(t)}	& = & \frac{\omega}{2 \pi} \int^{\frac{2 \pi}{\omega}}_{0} V^{2}_{m1} \sin^{2}{\omega t} + V^{2}_{m2} \sin^{2}{2\omega t} + 2V_{m1} V_{m2} \sin{\omega t}\sin{2 \omega t} \\
\int^{\frac{2 \pi}{\omega}}_{0} \sin^{2}{\omega t} dt & & \\
\int^{\frac{2 \pi}{\omega}}_{0} \sin^{2}{2 \omega t} dt & & \\
\cos{2A}					& = & 1 - 2 \sin^{2}{A} \\
\sin^{2}{A}				& = & \frac{1}{2} \left( 1 - \cos{2A} \right) \\
\omega t					& = & \theta \\
\omega dt				& = & d \theta \\
t							& = & 0 \\
\theta					& = & 0 \\
t							& = & \frac{2 \pi}{\omega} \\
\theta					& = & 2 \pi \\
\int^{2 \pi}_{0} \sin^{2}{\frac{\theta d \theta}{\omega}} & = & \frac{1}{\omega} \int^{2 \pi}_{0} (1 - \cos{2 \theta}) dt \\ 
\int^{2 \pi}_{0} \sin^{2}{\frac{2 \theta d \theta}{\omega}} & = & \frac{1}{2\omega} \int^{2 \pi}_{0} (1 - \cos{4 \theta}) dt \\
\frac{1}{\omega} \int^{2 \pi}_{0} \sin^{2}{\theta} d\theta & = &
\frac{1}{2 \omega} \left[ 1 - \frac{\sin{2 \theta}}{2} \right]^{2 \pi}_{0} \\
											  & = & \frac{2 \pi}{2 \omega} \\
\frac{1}{\omega} \int^{2 \pi}_{0} \sin^{2}{2 \theta} d\theta & = &
\frac{1}{2 \omega} \left[ 1 - \frac{\sin{4 \theta}}{2} \right]^{2 \pi}_{0} \\
											  & = & \frac{2 \pi}{2 \omega} \\
\overline{v^{2}(t)}					  & = & \frac{\omega}{2 \pi} \left[V_{m1}^{2} \frac{\pi}{\omega} + V_{m2} \frac{\pi}{\omega} + 2 V_{m1}
V_{m2} \int^{frac{2 \pi}{\omega}}_{0} \sin{\omega t}\sin{2\omega t}dt \right] \\
\frac{\omega}{2 \pi} \int^{frac{2 \pi}{\omega}}_{0} \sin{\omega t}\sin{2\omega t}dt & & \\
t					& = & 0 \\
\omega t			& = & \theta \\
\omega dt		& = & d \theta \\
dt					& = & \frac{d \theta}{\omega} \\
t					& = & 0 \\
\theta			& = & 0 \\
t					& = & \frac{2 \pi}{\omega} \\
\theta			& = & 2 \pi \\
\frac{\omega}{2 \pi} \int^{2 \pi}_{0} \sin{\theta}\sin{\frac{2\theta d
\theta}{\omega}} & & \\
\sin{A}\sin{B}	& = & \frac{1}{2} \left[\cos{(A-B)} - \cos{(A+B)} \right] \\
\sin{\theta}\sin{2 \theta} & = & \frac{1}{2} \left[ \cos{\theta - 2
\theta)} - \cos{(\theta + 2 \theta)} \right] \\
\frac{1}{2 \pi} \int^{2 \pi}_{0} \sin{\theta}\sin{2 \theta} d \theta & = & 
				\frac{1}{4\pi} \int^{2 \pi}_{0} (\cos{(-\theta)} - \cos{(3\theta)}) d \theta \\
		& = & \frac{1}{4\pi} \int^{2 \pi}_{0} (\cos{\theta} -	\cos{3\theta}) d \theta \\
		& = & \frac{1}{4\pi} \left[ \sin{\theta} - \frac{\sin{3\theta}}{3}
		\right]^{2 \pi}_{0} \\
		& = & 0 \\
\overline{v^{2}(t)} & = & \frac{V^{2}_{m1}}{2} + \frac{V^{2}_{m2}}{2} \\
V_{1} & = & \frac{V_{m1}}{\sqrt{2}} \\
V_{2} & = & \frac{V_{m2}}{\sqrt{2}} \\
\end{eqnarray*}

The mean square value of a complex wave is the sum  of the
mean square values of the components.

\[ v(t) = V_{m1} \sin{\omega t} + V_{m3} \sin{3 \omega t} \]

Find $\overline{V}$ and $\overline{V^{2}}$.

% 2004-01-14

\begin{eqnarray*}
		v(t) & = & V_{m1} \cos{\omega t} + V_{m1} \sin{2\omega t} \\
\overline{v^{2}(t)}  & = & \frac{V_{m1}^{2}}{2} + \frac{V_{m2}^{2}}{2} \\
		     & = & V_{1}^{2} + V_{2}^{2} \\
V_{1}		     & = & \frac{V^{2}_{m}}{\sqrt{2}} \\
V_{2}                & = & \frac{V^{2}_{m}}{\sqrt{2}}
\end{eqnarray*}

In general, if one had:

\begin{eqnarray*}
	v(t) & = & \sum^{\infty}_{n = 0} V_{mn} \sin{\omega t} \\
	\mbox{or} & & \\
	v(t) & = & \sum^{\infty}_{n = 1} V_{mn} \cos{\omega t} \\
	v^{2}(t) & = & \sum^{\infty}_{n = 0} \frac{V_{mn}^{2}}{2} \\
				& = & \sum^{\infty}_{n = 0} V_{n}^{2} \\
	\mbox{or}& = & \\
				& = & \sum^{\infty}_{n = 1} \frac{V_{mn}^{2}}{2} \\
				& = & \sum^{\infty}_{n = 1} V_{n}^{2} \\
\end{eqnarray*}

The mean square value of a complex wave is equal to the sum of
the mean square values of it's frequency components.

\begin{tabular}{ll}
$v_{1}^{2}$	& mean square value of $V_{m1} \cos{\omega t}$ \\
$v_{2}^{2}$ & mean square value of $V_{m2} \cos{2 \omega t}$ \\
$v_{n}^{2}$ & mean square value of $V_{mn} \cos{n \omega t}$ \\
\end{tabular}

\begin{eqnarray*}
v(t) & = & V_{0} + V_{1} \cos{\omega t} + V_{2} \cos{2 \omega t} \\
	  &   &	\mbox{DC}	\hspace{18mm} \mbox{AC} \\
	  & = & \sum^{2}_{n = 0} V_{mn} \cos{n \omega t} \\
\overline{v^{2}(t)} & = & V_{0}^{2} + \frac{V_{1}^{2}}{2} + \frac{V_{2}^{2}}{2} \\
						  & = & V_{0}^{2} + \overline{V_{1}^{2}} + \overline{V_{2}^{2}} \\
 						  & = & V_{0}^{2} + V_{2}^{2} + V_{2}^{2}
\end{eqnarray*}

\emph{Note:} If we have a complex waveform, to get the mean
square value, we simply add up the individual mean square values.

\[ \frac{1}{T} \int^{T}_{0} f^{2} (t) \]

\[ V(t) = \sum^{zinfty}_{n = 0} V_{mn} \sin{n \omega t} \]

Squaring creates items like $\sin^{2}{n \omega t} - \sin{m
\omega t}\sin{n \omega t}$.

\[ V(t) = V_{m1} \sin{\omega t} + V_{m2} \sin{2 \omega t} \]

\[ \sin^{2}{\omega t} = \sin{\omega t}\sin{2\omega t} \]

\[ \overline{v^{2}(t)} = \sum^{\infty}_{n = 0} \frac{V_{mn}^{2}}{2} \]

All integrals of cross products where $n \neq m$, vanish. This
is due to the orthogonality property of the circle earth. The circular
functions are mutually perpendicular unit vectors in functional space.
Just an extension of linear algebra - The circular functions are
considered as unit vectors in function space (or as it is called, little
burke space).

\begin{eqnarray*}
\left\{e^{ i p \theta } \right\} & = & \\
p = - \infty \hspace{15mm} \cdots -1, 0, 1, \cdots \infty \\
\left\{e^{ i p \theta } \right\} & = & \frac{1}{2 \pi} \int^{2
\pi}_{0} e^{i p \theta} \times \exp{- i p \theta} d \theta \\
p & = & q \cdots \mbox{(simplest case)} \\
\left\{e^{ i p \theta } \right\} & = & \frac{1}{2 \pi} \int^{2
\pi}_{0} e^{i(p - q) \theta} d \theta \\
	& = & \frac{1}{2 \pi} \int^{2 \pi}_{0} d \theta \\
p & \neg & q \\
\left\{e^{ i p \theta } \right\} & = & \int^{2 \pi}_{0} \exp{i(p -
q)\theta} d \theta \\
	& = & \frac{1}{2 \pi} \left[ \frac{e^{i(p - q)\theta}}{i(p - q)}
	\right]^{2 \pi}_{0} \\
	& = & \frac{1}{2 \pi i (p - q)} \left[ e^{i(p - q) 2 \pi} - 1
	\right] \\
e^{2 \pi i} & = & \cos{2 \pi} + i \sin{2 \pi} \\
e^{i(p - q) 2 \pi} & = & \cos{(p - q) 2 \pi} + i \sin{(p - q) 2 \pi} \\
		& = & \frac{1}{2 \pi} \int^{2 \pi}_{0} e^{i(p - q)\theta} d
		\theta \\
		& = & 1 \\
p		& = & q \\
e^{i(p - q) 2 \pi} & = & \cos{(p - q) 2 \pi} + i \sin{(p - q) 2 \pi} \\
& = & \frac{1}{2 \pi} \int^{2 \pi}_{0} e^{i(p - q)\theta} d\theta \\
		& = & 0 \\
p		& \neq & q \\
\frac{1}{2 \pi} \int^{2 \pi}_{0} e^{i(p - q) \theta} d	\theta & = &
\delta_{pq}
\end{eqnarray*}

Kronechen's Delta

This may be considered a scalar product in function space,
the set $\left\{ e^{ip\theta}\right\}$ form an orthogonal basis set 
in function space. Orthogonal vectors are mutually perpendicular unit 
vectors.

\[ \frac{1}{2 \pi} \int^{2 \pi}_{0} \left[ \cos{(p - q)\theta} + i\sin{(p
- q)\theta} \right] d \theta \]

\[ \frac{1}{2 \pi} \int^{2 \pi}_{0} \left( \cos{p \theta}\cos{q \theta}
+ \sin{p \theta}\sin{q \theta} + i \sin{p \theta}\cos{q \theta} - i
\sin{q \theta}\cos{p \theta} \right) d \theta \]

\[ = 0 \]


\begin{eqnarray*}
\frac{1}{2 \pi} \int^{2 \pi}_{0} \left( \cos{p \theta}\cos{q \theta} +
\sin{p \theta}\sin{q \theta} \right)d \theta & = & 0 \\
\frac{1}{2 \pi} \int^{2 \pi}_{0} \left( \sin{p \theta}\cos{q \theta} -
\sin{p \theta}\cos{q \theta} \right)d \theta & = & 0 \\
\frac{1}{2 \pi} \int^{2 \pi}_{0} \cos{p \theta}\cos{q \theta}d \theta & = & 0 \\
p = q & = & 0 \\
\frac{1}{2 \pi} \int^{2 \pi}_{0} \cos{p \theta}\cos{q \theta} d \theta &
= & 0 \cdots \mbox{always}
\end{eqnarray*}

By inspection, all cross terms are equal to $0$.

\begin{eqnarray*}
\frac{1}{2 \pi} \int^{2 \pi}_{0} e^{i(p - q)\theta} d \theta & = &
\delta_{piq} \\
\frac{1}{\pi} \int^{2 \pi}_{0} \sin{p \theta}\cos{q \theta}d \theta &
= & 0 \cdots \mbox{always} \\
\frac{1}{\pi} \int^{2 \pi}_{0} \sin^{2}{p \theta}d \theta & = & 1 \\
\frac{1}{\pi} \int^{2 \pi}_{0} \cos^{2}{p \theta}d \theta & = & 1 \\
\frac{1}{\pi} \int^{2 \pi}_{0} \sin{p \theta}\cos{q \theta}d \theta & = & 0 \\
\frac{1}{\pi} \int^{2 \pi}_{0} \cos{p \theta}\cos{q \theta}d \theta & = & \delta_{pq} \\
\frac{1}{\pi} \int^{2 \pi}_{0} \sin{p \theta}\sin{q \theta} d \theta & = & \delta_{piq}
\end{eqnarray*}

We take the set of functions $\left\{ e^{i p \theta} \right\}$
and the set of functions $\left\{ e^{-i q \theta} \right\}$. We form
their scalar product (aka inner product) and if we get the result $1$,
$p = q$, otherwise $p \neq q$.

That property means $\left\{ e^{i p \theta} \right\}$ and
$\left\{ e^{-i q \theta} \right\}$ are mutually perpendicular unit
vectors. then we can use De Moivre's Theorem to write out the result in
trigonometric form. The integral of scalar $\sin{p \theta}\cos{q
\theta}$ is $0$. The integral of $\sin^{2}{p \theta}$ only has a value
when $p = q$. This means $\sin{p \theta}$ and $\sin{q \theta}$ are
mutually perpendicular unit vectors.

\[ \frac{1}{\pi} \int^{2 \pi}_{0} \cos{p \theta}\cos{q \theta} d \theta =
\delta_{pq} \]

$\cos{p \theta}$ and $\cos{p \theta}$ are also mutually
perpendicular unit vectors. If we take mean square values of such
complex waves, then the mean square value is the sum of the mean square
values of the sum of the individual harmonic components.

% 2004-01-15 -- Lecture Missing

% 2004-01-18

\subsection{Task 2 (ii)}

Find the value of a full rectified sine wave.

\begin{table}[hbtp]

Diagram To Do

\end{table}

\[ v(t) = V_{m} \sin{\omega t} \]

if

\[ 0 \leq t \leq \frac{\pi}{\omega} \]

\begin{eqnarray*}
\overline{v(t)} & = & \frac{1}{T} \int^{\pi}_{0} v(t) dt \\
					 & = & \frac{\omega}{T} \int^{\frac{\pi}{\omega}}_{0}
					 V_{m}\sin{\omega t}dt \\
					 & = & \frac{\omega}{T} V_{m} \left[ - \cos{\omega t}
					 \right]^{\frac{\pi}{\omega}}_{0} \\
					 & = & \frac{- \omega}{T} V_{m} \left[ -1 -1 \right] \\
					 & = & \frac{2 \omega V_{m}}{\pi}
\end{eqnarray*}

\subsection{Unknown}

\[ v(t) = V_{m} \sin{\omega t} \]

\[ 0 \leq t \leq \frac{2 \pi}{\omega} \]

\begin{table}[hbtp]

Diagram To Do

\end{table}

\[ v(t) = 0 \]

\[ \frac{\pi}{\omega} \leq t \leq \frac{2 \pi}{\omega} \]

\begin{eqnarray*}
\overline{v(t)} & = & \frac{1}{T} \int^{\frac{T}{2}}_{0} \sin{\omega t}dt 
							 + \frac{1}{T} \int^{T}_{\frac{T}{2}} 0 \\
					 & = & \frac{V_{m} \omega}{2 \pi} \int^{\frac{\pi}{\omega}}_{0} 
					 		 \sin{\omega t}dt \\
\mbox{Let} \omega t & = & \theta \\
d \omega t & = & d \theta \\
\mbox{change limits when} &  & \\
\omega t & = & 0 \\
\theta   & = & 0 \\
\omega t & = & \frac{\pi}{\omega} \\
\theta	& = & \pi \\
\overline{v(t)} & = & \frac{V_{m} \omega}{2 \pi} \int^{\pi}_{0}
\sin{\theta}d \theta \\
					 & = & \frac{V_{m} \omega}{2 \pi} \left[ - \cos{\pi} +
					 \cos{0} \right] \\
					 & = & \frac{V_{m} \omega}{2 \pi} \left(-(-1) + 1 \right) \\
					 & = & \frac{V_{m}}{\pi}
\end{eqnarray*}

\subsection{1}

\begin{eqnarray*}
	 v(t) & = & V_{0} + V_{m} \cos{\omega t} \\
v^{2}(t) & = & V^{2}_{0} + V_{m}^{2}\cos^{2}{\omega t} + 2V_{0}V_{m}	\cos{\omega t} \\
\overline{v^{2}(t)} & = & \frac{\omega}{2 \pi} \int^{\frac{2
\pi}{\omega}}_{0} \left(V_{0}^{2} + V_{m}^{2}\frac{1}{2}(1 + \cos{2
\omega t}) + 2 V_{0}V_{m} \cos{\omega t}\right)dt \\
			& = & \frac{\omega}{2 \pi} \left[ V_{0}^{2}t +
			\frac{V^{2}_{m}}{2}\left(t + \frac{\sin{2 \omega t}}{2}\right)
			+ 2V_{0}V_{m}\frac{\sin{\omega t}}{\omega} \right]^{\frac{2
			\pi}{\omega}}_{0} \\
			& = & \frac{\omega}{2 \pi} \left[ V_{0}^{2}\frac{2 \pi}{\omega}
			+ \frac{V_{m}^{2}}{2} \left(t + \frac{\sin{2 \omega t}}{2}\right) 
			+ 2V_{0}V_{m}\frac{-\sin{2 \pi}}{\omega} - 0 \right] \\
			& = & V^{2}_{0} + \frac{V_{m}^{2}}{2 \pi}(2 \pi) \\
			& = & V^{2}_{0} + V_{m}^{2} \\
			& = & \sqrt{V^{2}_{0} + V^{2}} \\
V			& = & \frac{V_{m}}{2}			
\end{eqnarray*}

\subsection{4}

\[ \int^{2 \pi}_{0} e^{i(m - n)x} dx \]

\[ m, n \in \mathbb{Z} \]

\begin{eqnarray*}
	\int^{2 \pi}_{0} e^{i(m - n)x} & = & \int^{2 \pi}_{0} (\cos{(m -n)}x
	+ i \sin{(m -n)}x)dx \\
	& = & \int^{2 \pi}_{0} \cos{(m - n)} xdx + i \int^{2 \pi}_{0} \sin{(m
	- n)}x dx \\
	& = & \frac{1}{m - n} [\sin{(m - n)}x]^{2 \pi}_{0} - \frac{i}{m - n}
	[\cos{(m - n)}x]^{2 \pi}_{0} \\
	i(t) & = & \frac{1}{m - n} [\sin{(m - n)}2 \pi - \sin{(m - n)}0 ] \\
		  & = & \frac{1}{m - n} [ \cos{(m - n)} 2 \pi - \cos{(m - n)}0 ] \\
		  & = & \frac{1}{m - n} (\sin{2 \pi} - \sin{2 \pi} + \cos{2 \pi} -
		  \cos{2 \pi}) \\
		  & = & 0
\end{eqnarray*}

\begin{eqnarray*}
	\frac{1}{\pi} \int^{2 \pi}_{0} \cos{mx}\cos{nx}dx & = & \delta_{mn} \\
	\frac{1}{\pi} \int^{2 \pi}_{0} \cos{nx}\sin{mx}dx & = & 0 \\
	\frac{1}{\pi} \int^{2 \pi}_{0} \sin{mx}\sin{nx}dx & = & \delta_{mn} \\
	\frac{1}{\pi} \int^{2 \pi}_{0} \cos{(m + n)x} + \cos{(m - n)x}dx & &
\end{eqnarray*}

% 2004-01-21

\subsection{Orthogonality Relationships of the Functions}

\[ \frac{1}{2 \pi} \int^{2 \pi}_{0} e^{i(p - q)\theta} d\theta =
\delta_{piq} \]

\[ piq = 1 \pm 2 \pm 3 \pm \cdots \]

Clearly

\[	\frac{1}{2 \pi} \int^{2 \pi}_{0} e^{i(p + q)\theta} d \theta =
\delta_{pi - q} \]

From these identities, one may deduce that

\[ \frac{1}{\pi} \int^{2 \pi}_{0} \sin{p \theta}\cos{q \theta}d\theta =
0 \forall piq \]

\[ \frac{1}{\pi} \int^{2 \pi}_{0} \sin{p \theta} \sin{q \theta}d \theta
= \delta_{piq} \]

$\sin{p \theta}$ and $\cos{q \theta}$ are mutually
perpendicular unit vectors.

\[ \frac{1}{\pi} \int^{2 \pi}_{0} \cos{p \theta} \cos{q \theta} d \theta
= \delta_{piq} \]

De Moivre:

\begin{eqnarray*}
	(e^{i \theta})^{n} & = & (\cos{\theta} + i\sin{\theta})^{n} \\
							 & = & e^{in \theta} \\
							 & = & \cos{n \theta} + i\sin{n \theta} \\
			\varepsilon  & = & (x + iy)^{n} \\
							 & = & r^{n}(\cos{n \theta} + i\sin{n \theta}) 
\end{eqnarray*}
							 

The orthogonality of the properties of the circular functions
allow one to easily evaluate integrals of the period of time (such as
rise) in calculating. In the same manner, the representation of
trigonometric functions in terms of complex exponentials greatly
simplifies the analysis of AC circuits.

Let us evaluate the mean power in an AC circuit. In general,
(we shall see why later), the instantaneous current and voltage
(instantaneous break ways) are not in phase.

\begin{eqnarray*}
		v(t) & = & V_{m} \cos{\omega t} \\
		i(t) & = & I_{m} \cos{(\omega t + n)} \mbox{ \hspace{5mm}... $n$ is a
		constant}\\		
	\Phi(t) & = & \omega t + n \\
  \psi (h) & = & \omega t \\
\mbox{Power} & = & \mbox{Rate of doing work} \\
				 & = & \frac{d \omega}{dt} \mbox{(Joules sec$^{-1}$ or
				 Watts)} \\
\mbox{Instantaneous Power} & = & v(t) i(t) \\
\mbox{log on case but in} & & \\
		  P(t) & = & I_{m}V_{m} \cos{\omega t} \cos{(\omega t + n)} 
\end{eqnarray*}

In this case, $P(t)$ varies sinusoidally with time $t$ at 
twice the frequency of the voltage and current. \\

Power is essentially going to behave like $\cos^{2}{\omega
t}$.

\begin{eqnarray*}
		 	 P_{m} & = & I_{m} V_{m} \\
\overline{P(t)} & = & \frac{1}{T} \int^{T}_{0} v(t) i(t) dt \\
				T	 & = & \frac{2 \pi}{\omega} \\
\overline{P(t)} & = & \frac{\omega}{2 \pi} \int^{\frac{2
\pi}{\omega}}_{0} (V_{m} \cos{\omega t}) I_{m}\cos{(\omega t + \phi)} dt \\
					 & = &\frac{\omega}{2 \pi} I_{m}V_{m} \int^{\frac{2
					 \pi}{\omega}}_{0} \cos{\omega t} \cos{(\omega t +
					 \phi)}dt \\
					 & = & \frac{\omega}{2 \pi} I_{m} V_{m} \int^{\frac{2
					 \pi}{\omega}}_{0} \cos{\omega t} [ \cos{\omega
					 t}\cos{\phi} - \sin{\omega t}\sin{\phi}]dt \\
\omega t			 & = & \theta \\
\omega dt		 & = & d \theta \\
t					 & = & 0 \\
\theta			 & = & 0 \\
t					 & = & \frac{2 \pi}{\omega} \\
\theta			 & = & 2 \pi \\
\overline{P(t)} & = & \frac{1}{2 \pi} I_{m}v_{m} \int^{2 \pi}_{0}
\cos{\theta}[\cos{theta}\cos{\phi} - \sin{\theta}\sin{\phi}] d \theta \\
\cos{\phi} \int^{2 \pi}_{0} \cos^{2}{\theta} d \theta 
					 & = & \pi \cos{\phi} \mbox{\hspace{5mm}... A} \\
\sin{\phi} \int^{2 \pi}_{0} \cos{\theta} \sin{\theta} d \theta
					 & = & 0 \mbox{\hspace{5mm}... B} \\
\end{eqnarray*}

The scalar or inner product of the sine and cosine functions
is always equal to $0$.

\[ \frac{1}{\pi} \int^{2 \pi}_{0} \cos{p \theta}\cos{q \theta} d\theta =
\delta_{piq} \]

\begin{eqnarray*}
\overline{P(t)} & = & \frac{1}{2 \pi} I_{m} V_{m} (\pi \cos{\phi}) \\
                & = & \frac{I_{m} V_{m}}{2} \cos{\phi} \\
I               & = & \frac{I_{m}}{\sqrt{2}} \\
V               & = & \frac{V_{m}}{\sqrt{2}} \\
\overline{P(t)} & = & IV \cos{\phi} Watts
\end{eqnarray*}

\subsection{Mean Power}

\[ I = RMS \mbox{at} i(t) \]

\[ V = RMS \mbox{at} v(t) \]

\begin{table}[hbtp]

Diagram To Do

\end{table}

If you have a purely resistive circuit:

\[ \phi = 0 \]

because $v(t)$ and $i(t0$ are in phase.

\[ \overline{P(t)} = \frac{I_{m} V_{m}}{2} \]

\[ \overline{P(t)_{peak}} = I_{m} V_{m} \]

In general, if the circuit contains conductors and capacitors,
the current $i(t)$ and the voltage $v(t)$ will not be in phase.

\[ \cos{\phi} \mbox{Power Factor} \]

\subsection{Exercise}

\[ i(t) = I_{m} \sin{(\omega t + \phi)} \]

\[ v(t) = V_{m} \sin{\omega t} \]

Prove that $\overline{P(t)} = IV\cos{\phi}$.

\subsection{Exercise}

A voltage $v(t) = 100 + 250 \sin{40 \pi t} + 120\sin{80 \pi
t}$ is applied to a resistance of $50 \Omega$. Calculate from first
principles, the mean power dissipated.

Hint:

\begin{eqnarray*}
v(t) & = & i(t) R \mbox{\hspace{5mm} Ohm's Law} \\
i(t) & = & \frac{v(t)}{R} \\
P(t) & = & i(t) v(t) \\
	  & = & \frac{v^{2}(t)}{R} \\
\overline{P(t)} & = & \frac{\overline{v^{2}(t)}}{R}
\end{eqnarray*}

The mean square value of voltage or current is the mean power
dissipated in a resistance of $1 \Omega$.

\begin{table}[hbtp]

Diagram - To Do

\end{table}

For a purely resistive circuit, we have Ohm's Law: Resistance
is portortional to the current.

If reactive elements such as inductors or capacitors are
present.

\begin{table}[hbtp]

Diagram - To Do

\end{table}

\begin{table}[hbtp]

Diagram - To Do

\end{table}

$\Phi$ (Weber - Volts/sec).

% Double check this paragraph... vvvvvv

Faraday discovered that the magnitude of fluxes in the coil
was proportional to the rate of change of the magnetic flux, linking the
coil.

\begin{eqnarray*}
e(t) & = & \mbox{Rate of change of flux per leakages} \\
	  & = & d \Phi (t) \\
	  & = & \frac{d \Phi}{dt}
\end{eqnarray*}

The sign of the induced EMFs such as to oppose the change
proposing.

Action and reaction are equal and opposite.

\[ e(t) = - \frac{d \Phi}{dt} \]

\[ \Phi(t) \propto i(t) \]

\[ e(t) = L \frac{di(t)}{dt} \]

The constant of proportionality $L$ is the self inductance
value of the circuit.

If a current changing at a rate of $1A$ induces an EMF of $1V$
in a circuit, then that circuit is said to have self inductance of $1H$.

\begin{table}[hbtp]

Diagram - To Do

\end{table}

\[ e(t) = L \frac{di(t)}{dt} \]

\subsection{AC Case}

\[ e(t) = \varepsilon_{m} \cos{\omega t} \]

\[ \varepsilon_{m} \cos{\omega t} = L \frac{di(t)}{dt} \]

This is a simple differential equation connecting the voltage
and the current.

Faraday:

\[ \varepsilon_{m} \cos{\omega t} = l \frac{di(t)}{dt} \]

Ohm:

\[ \varepsilon = \cos{\omega t} = i(t) R \]

\begin{eqnarray*}
\varepsilon_{m} \cos{\omega t} dt & = & Ldi \\
\varepsilon_{m} \int \cos{\omega t} dt & = & Li \\
\varepsilon_{m} \frac{\sin{\omega t}}{\omega} & = & Li \\
\varepsilon_{m} \sin{\omega t} & = & \omega Li \\
i(t) & = & \frac{\varepsilon_{m} \sin{\omega \omega t}}{\omega L}
\end{eqnarray*}

Current and voltage are out of phase.

How does phase of $i(t)$ relate to $e(t)$?

\begin{eqnarray*}
\cos{(A + B)} & = & \cos{A}\cos{B} + \sin{A}\sin{B} \\
\cos{(\omega t - \frac{\pi}{2})} & = & \cos{\omega t}\cos{\frac{\pi}{2}}
+ \sin{\omega t}\sin{\frac{\pi}{2}} \\
	& = & \sin{\omega t} \\
i(t) & = & \frac{\varepsilon_{m}}{\omega L} \cos{\left(\omega t -
\frac{\pi}{2}\right)} \\
\Phi & = & - \frac{\pi}{2}
\end{eqnarray*}

In a purely inductive circuit, the current lags the voltage by
$\frac{\pi}{2}$.

\begin{eqnarray*}
\overline{P(t)} & = & \frac{\varepsilon_{m}}{\sqrt{2}} \frac{I_{m}}{\omega
L \sqrt{2}} \cos{\left(-frac{\pi}{2}\right)} \\
	& = & 0
\end{eqnarray*}

An inductor is not a dissipater, it is just an electric mass.

\[ \omega L = X_{L} \mbox{\hspace{5mm} Inductance Reactance} \]

% 2004-01-26

\begin{table}[hbtp]

Diagram To Do

\end{table}

\[ e(t) = i(t) R \]

Current is proportional applied EMF.

\[ e(t) = \varepsilon_{m} \cos{\omega t} \]

\[ i(t) = \frac{\varepsilon_{m}}{R} \cos{\omega t} \]

\begin{table}[hbtp]

Diagram To Do

\end{table}

\begin{eqnarray*}
e(t) & = & L \frac{di(t)}{dt} \\
	  & = & \varepsilon_{m} \cos{\omega t} \\
\varepsilon_{m} \cos{\omega t} & = & L \frac{di(t)}{dt} \\
\frac{\varepsilon_{m} \sin{\omega t}}{\omega} & = & L i(t) \\
i(t) & = & \frac{\varepsilon_{m} \sin{\omega t}}{\omega L} \\
	  & = & \frac{\varepsilon_{m} \cos{(\omega t - \frac{\pi}{2})}}{\omega L}
\end{eqnarray*}

\begin{table}[hbtp]

Diagram To Do

\end{table}

\begin{eqnarray*}
\overline{i^{2}(t)} & = & \frac{I_{m}}{\sqrt{2}} \\
						  & = & I \\
						  & = & \frac{\varepsilon_{m}}{\sqrt{2} \omega L} \\
						  & = & \frac{E}{\omega L} < - 90^{o}
\end{eqnarray*}						  

\subsection{Inductive Reactance ($\omega L$/$X_{L}$)}

Newton's Law:

\begin{eqnarray*}
m \ddot{q} & = & F(t) \\
\ddot{q} & = & \mbox{displacement} \\
e(t) & = & L \frac{di}{dt} \\
d & = & \frac{dq}{dt} \\
L \ddot{q} & = & e(t)
\end{eqnarray*}

There is no dissipation in an inductive circuit, but there is
dissipation in a resistive circuit. It is impossible to make an
inductive circuit without resistance.

\section{Capacitance}

\begin{table}[hbtp]

Diagram To Do

\end{table}

\begin{table}[hbtp]

Diagram To Do

\end{table}

Lines of electric displacement flux density.

In a capacitor, you have lines of electric flux, which start as a
positive charge and ends as a negative charge.

\subsection{Electric Flux Density - $D$ (Coulombs/$m^{2}$)}

Maxwell wrote down Faraday + Lenz in mathematical form ($e(t) = -
\frac{d}{dt} (\mbox{Flux Leakages})$.

Lenz -- Arrow To Do % ---> -

Faraday - Arrow To Do % ---> \frac{d}{dt}

Maxwell connected an electrical quantity, EMF ($e(t)$) with a magnetic
quantity, $\Phi$ (Webers/sec). But he still had a problem why the
current appeared to flow in a capacitive circuit. He postulated that the
current that flows in a capacitive circuit is not conventional induction
current, but instead, due to the collapse and growth of the electric
flux lines between the plates, due to the altering sign, the charges on
the plates arising from the imposition of the time period $e(t)$.

It's a displacement current.

\[ e(t) = \frac{q(t)}{C} \]

Units of $C: Farads$.

If a body raised to a potential of $1V$ stores a charge of $1$ coulomb,
then that body is said to have the capacitance of $1 Farad$.

Earth's Capacitance: $700\mu F$.

\begin{table}[hbtp]

Diagram - To Do

\end{table}

\[ \dot{e}(t) = \frac{\dot{q}(t)}{C} = \frac{i(t)}{C} \]

Current, in a capacitive circuit is a time derivative of the supplied
voltage. So, if

\[ e(t) = \varepsilon_{m} \cos{\omega t} \]

then

\begin{eqnarray*}
\dot{e}(t) & = & - \omega \varepsilon_{m} \sin{\omega t} \\
			  & = & - \frac{i(t)}{C} \\
i(t)		  & = & - \omega C \varepsilon_{m} asin{\omega t}
\end{eqnarray*}

However, we would like to see what it's phase relation of the current to
the voltage.

\begin{eqnarray*}
\cos{(\omega t + \frac{\pi}{2})} & = & \cos{\omega t}
\cos{\frac{\pi}{2}} - \sin{\omega t}{\sin{\frac{\pi}{2}}} \\
& = & - \sin{\omega t} \\
X_{C} & = & \frac{1}{\omega C} \mbox{hspace{5mm}... Capacitance
Reactance} \\
i(t) & = & \omega C \varepsilon_{m} \cos{(\omega t + \frac{\pi}{2})} \\
	  & = & \frac{\varepsilon_{m}}{X_{C}} \cos{(\omega t +
	  \frac{\pi}{2})}
\end{eqnarray*}

the current leads the voltage by $90^{o}$ in a capacitive circuit.

\section{Capacitance Reactance}

It again has the unit Ohms. Here, there there appears to be no
dissipation, however capacitance is an electric spring.

\begin{table}[hbtp]

Diagram To Do

\end{table}

Exercise: Voltage $\varepsilon_{m} \sin{\omega t}$ is supplied to \\

(a) to a purely resistive circuit. \\ \\

(b) to a purely capacitive circuit. \\ \\

(c) to a purely inductive circuit. \\ \\

Find the relationship between the voltage and the current in each case
and draw the phaser diagram of each.

% 2004-01-28

\begin{table}[hbtp]

Diagram - To Do

\end{table}

\begin{table}[hbtp]

$i(t)$ leads $e(t)$ by $90^{o}$.

\end{table}

\begin{table}[hbtp]

Diagram - To Do

\end{table}

$i(t)$ in phase with $e(t)$

\begin{table}[hbtp]

Diagram - To Do

\end{table}

$i(t)$ lags $e(t)$ by $90^{o}$.

In electrical engineering however, it is not convenient to use sine and
cosine as the driving EMFs in AC theory. It is best to consider:

\[ e(t) = \varepsilon_{m} e^{j \omega t} \]

where $j = \sqrt{-1}$ . \\

\[ e(t) = \cos{\omega t} + j \sin{\omega t} \]

Resultant current is complex. \\

Real $i(t)$ gives response to $\cos{\omega t}$. \\

Imaginary $i(t)$ gives response to $\sin{\omega t}$. \\

Simple circuit:

\begin{eqnarray*}
e(t) & = & R i(t) \\
\varepsilon_{m} e^{j \omega t} & = & R i(t) \\
i(t) & = & \frac{\varepsilon_{m} e^{j \omega t}}{R} \mbox{hspace{5mm}
... Response} \\
\mbox{Real } i(t) & = & \frac{\varepsilon_{m} \cos{\omega t}}{R} \\
\mbox{Imaginary } i(t) & = & \frac{\varepsilon_{m} \sin{\omega t}}{R}
\end{eqnarray*}

% Missing

And it's easy to see the phase relationships.

\subsection{Inductive Circuit}

\begin{eqnarray*}
		L & = & \frac{di(t)}{dt} \\
  		  & = & \varepsilon_{m} e^{j \omega t} \\
Li		  & = & \frac{\varepsilon_{m} e^{j \omega t}}{j \omega} \\
i(t)	  & = & \frac{\varepsilon_{m} e^{j \omega t}}{j \omega L} \\
j		  & = & \cos{\frac{\pi}{2}} + j \sin{\frac{\pi}{2}} \\
		  & = & e^{j \frac{\pi}{2}} \\
j^{-1}  & = & e^{-j \frac{\pi}{2}} \\
i(t)	  & = & \frac{\varepsilon_{m} i \omega t e^{- j
\frac{\pi}{2}}}{\omega L} \\
		  & = & \frac{\varepsilon_{m} e^{i\left(\omega t -
		  \frac{\pi}{2}\right)}}{\omega L}
\end{eqnarray*}

\begin{table}[hbtp]

Diagram - To Do

\end{table}

\[ [j \hspace{2mm} \omega \hspace{2mm} L ] = [ R ] \]


$j \omega L = $ impedance of an impedance circuit, ie $X_{L}$, the
inductive reactance.

\subsection{Capacitive Circuit}

\begin{table}[hbtp]

Diagram - To Do

\end{table}

\begin{eqnarray*}
e(t)			& = & \frac{q(t)}{C} \\
\dot{e}(t)	& = & \frac{\dot{q}(t)}{C} \\
				& = & \frac{i(t)}{C} \\
i(t)			& = & C \dot{e}(t) \\
				& = & j \omega C \varepsilon_{m} e^{j \omega t} \\
j				& = & \cos{\frac{\pi}{2}} + i \sin{\frac{\pi}{2}} \\
				& = & e^{j \frac{\pi}{2}} \\
\dot{e}(t)	& = & \frac{\dot{q}(t)}{C} \\
i(t)			& = & \omega C \varepsilon_{m} e^{(\omega t +
\frac{\pi}{2})} \\
				& = & \frac{e(t)}{Z_{C}} \\
j \omega C	& = & - \frac{1}{\omega C}
\end{eqnarray*}

\begin{table}[hbtp]

Diagram - To Do

\end{table}

This is the impedance offered by a capacitor. But there is no
dissipation.

\[ X_{C} = \frac{1}{\omega C} \mbox{\hspace{5mm}Capacitance Reactance}
\]

\begin{eqnarray*}
Z_{R} & = & R_{1} \\
Z_{C} & = & \frac{1}{\omega C}
\end{eqnarray*}

It is impossible to build an electric circuit with resistance only, either
capacitance only or with inductance only.

In general, you will have the three circuit elements present.

\begin{table}[hbtp]

Diagram - To Do

\end{table}

\begin{eqnarray*}
Z_{T} & = & Z_{L} + Z_{R} + Z_{C} \\
		& = & j \omega L + R + \frac{1}{j \omega C} \\
		& = & j \omega L + R + \frac{j}{\omega C} \\
i(t)	& = & \frac{\varepsilon_{m} e^{j \omega t}}{r + j\left(\omega L -
\frac{1}{\omega C} \right)}
\end{eqnarray*}

We have got the current from a simple generalised form of Ohm's Law.
However, we do not know it's phase relative to the applied voltage
without computation.

\begin{eqnarray*}
i(t) & = & \frac{\varepsilon_{m} e^{j \omega t}}{R^{2} + \left| \omega L
- \frac{1}{\omega C} \right|} \times R - j \left(\omega L -
\frac{1}{\omega C} \right) \\
	  & = & X_{i} + i Y \\
\mbox{Arguement} X_{i} & = & \tan^{-1}{\frac{Y}{X}} \\
\left| X_{i} \right| & = & \sqrt{X^{2} + Y^{2}}
\end{eqnarray*}

\begin{table}[hbtp]

Diagram - To Do

\end{table}

Exercise: Show that

\[ \mbox{Arguement} X_{i} = \tan^{-1}{\left[ - \frac{\left( \omega L -
\frac{1}{\omega C} \right)}{R} \right]} \]

\[ \left| X_{i} \right| = \frac{1}{\sqrt{R^{2} + \left( \omega L -
\frac{1}{\omega C} \right)}} \]

% 2004-02-02

\begin{table}[hbtp]

Diagram - To Do

\end{table}

By Generalising Ohm's Law:

\[ i(t) = \frac{\varepsilon_{m} e^{j \omega t}}{Z(j \omega)} \]

We work out $Z$ in arguement form and modulus form.

\begin{eqnarray*}
Z 			& = & \left| Z \right| e^{j \varphi} \mbox{\hspace{5mm}... De Moivre's
Euler} \\
Z^{-1}	& = & \frac{1}{\left| Z \right| e^{j \varphi}} \\
\mbox{Thus Modulus} & & \\
\left| \frac{1}{Z} \right| & = & \frac{1}{\left| Z \right|} \\
\mbox{Arguement} \frac{1}{Z} & = & - \mbox{Arguement} Z \\
Z			& = & \left| Z \right| e^{j \varphi} \\
Z^{-1}	& = & \frac{1}{\left| Z \right|} e^{- j \varphi} \\
v(t)		& = & \frac{\varepsilon_{m} e^{j \omega t} e^{-j \varphi}}{\left|
Z \right|}
\end{eqnarray*}

\begin{eqnarray*}
\varphi & = & tan^{-1}{\frac{Y}{X}} \\
		  & = & tan^{-1}{\frac{\mbox{Imaginary} Z}{\mbox{Real} Z}} \\
Z		  & = & X + j Y
\end{eqnarray*}

The complex notation is just a way of manipulating AC quantities. The
actual currents and voltages are always either sine or cosine.

\begin{eqnarray*}
i(t) \mbox{ is complex.} & & \\
i'(t) + j i''(t) & & \\
i'(t), i''(t) \in \mathbb{R} & & \\
i' + i'' & = & \varepsilon_{m}{(\omega t - \varphi)} + 
\frac{j \varepsilon_{m} \sin{(\omega t - \varphi)}}{\left|Z \right|} \\
e^{j \alpha} & = & \cos{\alpha} + j \sin{\alpha} \\
i'(t) & = & \frac{\varepsilon_{m} \cos{(\omega t - \varphi)}}{\left| Z
\right|} \mbox{\hspace{5mm} Response to } \varepsilon_{m} \cos{\omega t} \\
i''(t) & = & \frac{\varepsilon_{m} \sin{(\omega t - \varphi)}}{\left| Z
\right|} \mbox{\hspace{5mm} Response to } \varepsilon_{m} \sin{\omega t} \\
\end{eqnarray*}

To solve any serious AC circuit, all you need to know is the complex
impedance.

\begin{eqnarray*}
I_{m}			& = & \frac{\varepsilon_{m}}{\left| Z \right|} \\
I				& = & \frac{I_{m}}{\sqrt{2}} \\
				& = & \frac{\varepsilon_{m}}{\sqrt{2} \left| Z \right|} \\
				& = & \frac{E}{\left| Z \right|}
\end{eqnarray*}

$R$, $L$ in series

\begin{table}[hbtp]

Diagram - To Do

\end{table}

\begin{eqnarray*}
i(t)			& = & \frac{\varepsilon_{m} e^{j (\omega t -
\varphi)}}{\left| Z \right|} \\
Z				& = & R + j \omega L \\
				& = & R + j X_{L} \\
\left| Z \right| & = & \sqrt{R^{2} + \omega^{2}L^{2}} \\
\varphi & = & \mbox{Arguement } Z \\
				& = & \tan^{-1}{\left(\frac{\omega L}{R}\right)} \\
				& = & \tan^{-1}{\left(\frac{\mbox{Imaginary } Z}{\mbox{Real
				} Z}\right)} \\
i(t)			& = & \frac{\varepsilon_{m} e^{j\left(\omega t -
\tan^{-1}{\left(\frac{\omega L}{R}\right)}\right)}}{\sqrt{R^{2} +
\omega^{2}L^{2}}} \mbox{\hspace{5mm} Instantaneous Current}
\end{eqnarray*}

Response to $\varepsilon_{m} \cos{\omega t}$.

\[ \mbox{Real } i(t) = \frac{\varepsilon_{m} \cos{\left( \omega t - 
\tan^{-1}{\frac{\omega L}{R}}\right)}}{\sqrt{R^{2} + \omega^{2} L^{2}}} \]

\begin{table}[hbtp]

Diagram - To Do

\end{table}

\begin{eqnarray*}
I_{m} & = & \frac{\varepsilon_{m}}{\sqrt{R^{2} + \omega^{2} L^{2}}} \\
I		& = & \frac{E < - \tan^{-1}{\left(\frac{\omega L}{R}\right)}}{\sqrt{R^{2} 
+ \omega^{2} L^{2}}}
\end{eqnarray*}

The RMS current is very strongly frequency dependent. At low
frequencies, the phase angle $\varphi$ is $0$. At very high frequencies,
the inductive reactance, $X_{L}$ dominates the circuit and the circuit
behaves like a pure inductor.

\[ \overline{P(t)} = I E \cos{\varphi} \]

\begin{table}[hbtp]

Diagram - To Do

\end{table}

\begin{eqnarray*}
\tan{\varphi} & = & \frac{- \omega L}{R} \\
\sqrt{R^{2} + \omega^{2} L^{2}} & = & \mbox{hyp} \\
\cos{\pi} & = & \frac{R}{\sqrt{R^{2} + \omega^{2} L^{2}}} \\
\overline{P(t)} & = & \frac{IE R}{\sqrt{R^{2} + \omega^{2} L^{2}}} \\
E & = & I \left| Z \right| \\
  & = & I \sqrt{R^{2} + \omega^{2} L^{2}} \\
\overline{P(t)} & = & I^{2} R \\
\cos{\varphi} & = & \frac{R}{\left| Z \right|} \\
\overline{P(t)} & = & I^{2} R
\end{eqnarray*}

\subsection{Summary}

Any AC circuit which is governed by linear differential equations, one
can always find the response that is the current, if one knows the
impedence. The complex $j$ notation allows one to easily understand the
phase relation.

\section{Power Factor Correction}

The problem however, is that many loads are highly inductive, hence
$\varphi$ is high, approaching $\frac{\pi}{2}$. This means we have hugh,
reactive currents flowing in the circuit, if it's inductive. We are then
far from the ideal conditions for peak power thanks to the supply
voltage. So we try yo make $\cos{\varphi}$ are near to $1$.

To solve this problem, we insert a capacitor into the circuit.

% 2004-02-02

Exercise: A voltage $e(t) = \frac{3}{5} \sin{\frac{3t}{4}}$ is applied
to a capacitor of $0.5 \mu F$.

Calculate from first principles, the instantaneous current, the maximum
current, the phase angle, the RMS value, the power consumed. Sketch the
variation of $\left| Z \right|$ with frequency. Draw a phasor diagram.

Diagram - To Do

\[ Z_{C} = - \frac{j}{\omega L} \]

% 2004-02-04

\section{Resonance}

\begin{table}[hbtp]

Diagram - To Do

\end{table}

\begin{eqnarray*}
Z & = & R + j \omega L - \frac{j}{\omega L} \\
i(t) & = & \frac{\varepsilon_{m} e^{j \omega t} }{Z} \\
	  & = & \frac{\varepsilon_{m} e^{j(\omega t - \varphi)}}{\left| Z
	  \right|} \\
\varphi & = & \mbox{Arg } Z \\
\left| Z \right| & = & \sqrt{R^{2} + \left(\omega L - \frac{1}{\omega
C}\right)^{2}} \\
Z & = & x + j y \\
\left| Z \right| & = &\sqrt{x^{2} + y^{2}} \\
Z & = & \left| Z \right| e^{j \varphi} \\
  & = & \left| Z \right| e^{j Arg  Z} \\
\mbox{Arg } Z & = & \tan^{-1}{\frac{y}{x}} \\
				  & = & \frac{\left( \omega L - \frac{1}{\omega C}
				  \right)}{R} \\
i(t)	& = & \frac{\varepsilon_{m} exp j \left\{ \omega t -
\tan^{-1}{\left[ \frac{\left(\omega L - \frac{1}{\omega C}\right)}{R}
\right]}\right\}}{\sqrt{R^{2} + \left(\omega L -  \frac{1}{\omega
C}\right)^{2}}} \\
i(t)	& = & i'(t) + j i''(t) \\
i'(t)  & = & \frac{\varepsilon_{m} \cos{\left\{ \omega t -
\tan^{-1}{\left[ \frac{\left(\omega L - \frac{1}{\omega C}\right)}{R}
\right]}\right\}}}{\sqrt{R^{2} + \left(\omega L -  \frac{1}{\omega
C}\right)^{2}}} \\
\exists \hspace{1mm} \omega & = & \omega_{0} \\
\mbox{ such that} & & \\
\omega_{0} L & = & \frac{1}{\omega_{0} C} \\
\tan^{-1}{\varphi} & = & 0
\end{eqnarray*}

There exists a special frequency $\omega_{0}$, at which the supply
current is in phase with the supply voltage. At $\omega = \omega_{0}$,
the circuit appears to behave as a pure resistance circuit.

At $\omega = \omega_{0}$:

\begin{eqnarray*}
\omega_{0} & = & \frac{1}{2 \pi\sqrt{LC}} \\
f_{0} & = & \frac{1}{2 \pi \sqrt{LC}} Hz
\end{eqnarray*}

This frequency is called the resonance frequency.

The frequency $f_{0} = \frac{1}{2 \pi \sqrt{LC}}$ is the resonance
frequency for the resonance circuit.

\begin{table}[hbtp]

Diagram - To Do

\end{table}

\begin{table}[hbtp]

Diagram - To Do

\end{table}

\begin{eqnarray*}
\tan^{-1}{x} & = & - \tan^{-1}{x} \\
\tan^{-1}{\infty} & = & \frac{\pi}{2} \\
\tan^{-1}{- \infty} & = & - \frac{\pi}{2}
\end{eqnarray*}

The function $\tan^{-1}{x}$, for very large positive values of it's
arguement, asimtotically approaches $\frac{\pi}{2}$. For very large
values and negative values of it's arguements asimtotically occur at $-
\frac{\pi}{2}$. The same behavious occurs for the positive and negative
at multiples of $n + \frac{\pi}{2}$.

When $\omega$ is far greater than $\omega_{0}$, $\omega L$ dominates the
phase. Than $\tan^{-1}$ approaches $\infty$, so the current lags the
voltage by $\frac{\pi}{2}$. $\left| Z \right|$ becomes extremely lerge.
The incoming signal is presented with a very high impedence, compared to
that when $\omega = \omega_{0}$.

At low frequencies, when $\omega$ is far smaller than $\omega_{0}$,
$\frac{1}{\omega C}$ dominates the phase. then the current leads the
voltage, so the phase angle asimtotically approaches $\frac{\pi}{2}$.
While the impedence is also very large.

% 2004-02-05

Exercise: An AC circuit of $50V$ at a frequency of $158 kHz$ is applied
to a series $RLC$ circuit consisting of an inductance of $1000 \mu H$, a
capacitance of $0.001 \mu F$ and a resistance of $10\Omega$. \\ \\

Find the: \\ 

(a) current flowing \\

(b) phase angle \\

(c) the voltage across the capacitor \\

(d) the voltage across the inductor \\

Fine the resonance frequency.

\begin{table}[hbtp]

Diagram - To Do

\end{table}

If an alternating voltage at $50V$ is applied at this frequency, find
the answers to a, b, c and d above.

% 2004-02-09

\begin{table}[hbtp]

Diagram - To Do

\end{table}

\[ I = \frac{E}{\left| Z \right|} \]

At $\omega = \omega_{0}$

\begin{eqnarray*}
\omega_{0} & = &\frac{1}{\sqrt{LC}} \\
I_{\omega} & = & \frac{E}{\sqrt{R^{2} + \left(\omega L - \frac{1}{\omega
C}\right)^{2}}}
\end{eqnarray*}

$I$ is a maximum at $\frac{E}{R}$.

\begin{table}[hbtp]

Diagram - To Do

\end{table}

If the resistance is very small, $I$ tends to $\infty$ at $\omega =
\omega_{0}$.

This is a high $Q$ factor circuit.

We can find the quality factor by calculating the $Q$ factor of the
resonance.

In this circuit:

\begin{eqnarray*}
Q factor & = & 2 \pi \times \mbox{peak energy stored at resonance
divided by energy dissipated per cycle} \\
v(t) & = & \frac{1}{2} L i^{2}(t) + \frac{1}{2}
Cv^{2}(t)\left(\frac{1}{2} mx^{2} + \frac{1}{2} kx^{2} \right)
\end{eqnarray*}

The electric potential energy stored in capacitance is $\frac{1}{2} k
x^{2}$.

We want the ratio of the energy stored to the energy dissipated.

\begin{eqnarray*}
\overline{V}(t) & = & \frac{1}{2} \overline{Li^{2}(t)} + \frac{1}{2}
\overline{C v^{2}(t)} \\
			& = & \frac{1}{2} LI^{2} + \frac{1}{2} CV^{2} \\
V_{C}		& = & \frac{I}{\omega C} \\
			& = & IX_{C} \\
\overline{V}(t) & = & \frac{1}{2} LI^{2} + \frac{1}{2}
\frac{CI}{\omega^{2} C^{2}} \\
			& = & \frac{1}{2} I^{2} \left( L + \frac{1}{\omega^{2} C}
			\right) \\
\omega & = & \omega_{0} \\
\overline{V}(t) & = & \frac{I^{2}}{2} \left(L + \frac{1}{\omega_{0} C}
\right)\\
		 & = & \frac{I^{2}}{2} (L + C) \\
		 & = & \frac{I^{2}}{2} 2L \\
		 & = & L I^{2} \\
\mbox{Since } I & = & \frac{I_{m}}{\sqrt{2}} \\
E & = & \frac{1}{2} LI_{m}^{2} \\
Q & = & \frac{2 \pi \left(\frac{1}{2} L I_{m}^{2} \right)}{I^{2} R
\frac{2 \pi}{\omega_{0}}} \\
I & = & \frac{I_{m}}{\sqrt{2}} \\
Q & = & \frac{\omega_{0} L}{R} \\
\omega_{0}^{2} & = & \frac{1}{LC} \\
Q & = & \frac{1}{\omega_{0} CR}
\end{eqnarray*}

\begin{table}[hbtp]

Diagram - To Do

\end{table}

\begin{table}[hbtp]

Diagram - To Do

\end{table}

In a car, you rotate the alternator machanically, so you are supplying
energy from some external, non electric agent.

\begin{eqnarray*}
\Phi(t) & = & \Phi_{m} \cos{\theta(t)} \\
e(t) & = & - \frac{d(\mbox{Flux Linkages})}{dt}
\end{eqnarray*}

Now, if the coil is made to rotate in the field, with angular velocity
$\omega$.

\begin{eqnarray*}
e(t) & = & - \frac{d}{dt} (\Phi (t)) \\
	  & = & - \frac{d}{dt} (\Phi_{max} \cos{\omega t}) \\
	  & = & \omega \Phi_{m} \sin{\omega t} \\
\Phi_{m} \theta & = & 0 \\
\overline{e^{2}}(t) & = & \frac{\omega^{2} \Phi_{m}^{2}}{2} \\
E & = & \frac{\omega \Phi_{m}}{\sqrt{2}} \mbox{\hspace{5mm} Elementary
Generator Equation} \\
  & = & \frac{2 \pi f \Phi_{m}}{\sqrt{2}} \\
  & = & 4.4 f \Phi_{m}
\end{eqnarray*}  

In practice, a DC current is often desirable. So, the AC has to be
rectified to make it unidirectional. This is accomplished in modern
alternators by usig diodes as rectifiers.

Before the invention of reliable, solid state diodes, DC current was
produced using a commumutator, which is an electromechanical device for
ensuring unidirectional current. This is a good method of generating AC 
at low frequencies. It is not suitable at high frequencies.

\begin{table}[hbtp]

Diagram - To Do

\end{table}

Hertz and Marconi made a large oscillator by connecting accululators or
batteries to a resistive circuit.

They switched the battery in and out, producing a continious sine wave.
This is an elementary high frequency oscillator called a \emph{Spark
Transmitter}.

Lee de Forest patented the vacuum and that allowed one to make a high
frequency oscillator.

% 2004-02-09

\begin{table}[hbtp]

Diagram - To Do

\end{table}

\begin{table}[hbtp]

Diagram - To Do

\end{table}

We saw that for impedences in series, we add:

\[ Z_{T} = Z_{C} + Z_{L} + Z_{R} \]

\begin{table}[hbtp]

Diagram - To Do

\end{table}

To deal with such parallel circuits, we introduce the idea of
admittence.

\[ Y(j \omega) = \frac{1}{Z} (\Omega^{-1}) \]

Just as impedences in series add up, admittences in parallel add up.

Let's consider $R$ and $L$ as constituting admittence and the 
capacitor as constituting an admittence.

\begin{eqnarray*}
Z_{C} & = & \frac{1}{j \omega C} \left( \frac{- j}{\omega C} \right) \\
Z_{LR} & = & R + j \omega L \\
Y_{C} & = & j \omega C \\
Y_{LR} & = & \frac{1}{R + j \omega L} \\
Y_{T} & = & j \omega C + \frac{1}{R + j \omega L} \\
Z_{T} & = & \frac{1}{Y_{T}} \\
E & = & I \left| Z \right| \\
I & = & \frac{E}{\left| Z \right|} \\
  & = & E \left| Y \right|
\end{eqnarray*}

This circuit has aproperty that the current is a mimimum when $Y(j)$
vanishes. This is called a parallel resonance or rejector circuit.

Current is a maximum in a series resonance circuit because the impedence
is at a minimum, so it's called an acceptor circuit.

In  parallel resonance circuit, when the resistance is at a maximum, the
current and the resonance are at a minimum.

Resonance is defined as the condition that the imaginary part of the
admittence (or impedence) vanishes. In practice, the supply current is
in phase with the supply voltage.

Exercise: Show that

\begin{eqnarray*}
\omega_{0}^{2} & = & \frac{1}{LC} - \frac{R^{2}}{L^{2}} \\
& & \\
& & \\
Y & = & j \omega L + \frac{R - j \omega L}{R^{2} + \omega^{2} L^{2}} \\
Y & = & \frac{R}{R^{2} + \omega^{2} L^{2}} + j \omega \left(C -
\frac{L}{R^{2} + \omega^{2} L^{2}} \right) \\
C - \frac{L}{R^{2} + \omega^{2} L^{2}} & = & 0 \\
C & = & \frac{L}{R^{2} + \omega_{0}^{2} L^{2}} \\
\frac{L}{C} & = & R^{2} + \omega_{0}^{2} L^{2} \\
\frac{L}{C} - R^{2} & = & \omega_{0}^{2} L^{2} \\
\frac{1}{LC} - \frac{R^{2}}{L^{2}} & = & \omega_{0}^{2} \\
Y & = & \frac{R}{R^{2} + \omega^{2} L^{2}}
\end{eqnarray*}

% 2004-02-16

\section{Ideal Transformer}

\begin{table}[hbtp]

Diagram - To Do

\end{table}

No power loss in

\[ E_{1} I_{1} = E_{2} I_{2} \]

\begin{table}[hbtp]

Diagram - To Do

\end{table}

Flux lines $L_{1}$ cut coil$_2$ and in general, not all the flux lines
in coil$_1$ will cut coil$_2$.

If a current changing at a rate of $1 A/sec$ in coil$_1$ induces in
coil$_2$ an EMF of $1V$, then the coils are said to have a mutual
inductance of $1H$.

\begin{eqnarray*}
e_{1}(t) & = & \frac{N_{1} d \Phi_{12}}{dt} \\
\Phi_{12} & = & \Phi_{21} \\
e_{2}(t) & = & \frac{N^{2} d \Phi}{dt} \\
\frac{E_{1}}{E_{2}} & = & \frac{N_{1}}{N_{2}} \\
& & \\
\mbox{All the flux from coil$_1$} \\
\mbox{links coil$_2$ and vice versa.} \\
& & \\
\frac{E_{1}}{E_{2}} & = & \frac{I_{2}}{I_{1}} \\
\frac{N_{1}}{N_{2}} & = & \frac{I_{2}}{I_{1}} \\
E_{2} & = & \frac{N_{2}}{N_{1}} E_{1} \\
\frac{N_{1}}{N_{2}} & = & n \mbox{\hspace{5mm} (Turns Ratio)} \\
E_{2} & = & \frac{E_{1}}{n} \\
I_{2} & = & n I_{1}
\end{eqnarray*}

The purpose of transformers is to alter the balance of % gap

\begin{table}[hbtp]

Daigram - To Do

\end{table}

Mobile Phone Charger

\begin{table}[hbtp]

Diagram - To Do

\end{table}

Ignition Coil

\vspace{10mm}

First of all, we get different Turns Ratios by tapping the primary
winding. The tapping responds to the values needed.

Widely used for impedence matching, eg old fashioned amplifiers have a
very high impedence.

You want to match the output, so you match the impedence using a
transformer.

\begin{table}[hbtp]

Diagram - To Do

\end{table}

In practice, $RL$ and $RLC$ circuits, etc are subject to simulate which
are switched on and off all the time. The most common example is in a
car engine.

\begin{table}[hbtp]

Diagram - To Do

\end{table}

What is the subsequent growth or decay?

\begin{table}[hbtp]

Diagram - To Do

\end{table}

\begin{eqnarray*}
U(t) & = & 1 \\
t &> & 0 \\
U(t) & = & \frac{1}{2} \\
t & = & 0 \\
U(t) & = & 0 \\
t & < & 0
\end{eqnarray*}

Represents switching on a $DC$ circuit.

What is $i(t)$ for $t > 0$

the home (Kivdnoff)

\begin{eqnarray*}
V_{R}(t) + V_{L}(t) & = & E U(t) \\
						  & = & E \\
t & > & 0 \\
i(t) R + L \frac{di(t)}{dt} & = & E
\end{eqnarray*}

First order linear ordinary differential equation.

\begin{eqnarray*}
R i(t) dt + L di & = & E dt \\
L di & = & (E - iR) dt \\
\frac{L di}{E - iR} & = & dt
\end{eqnarray*}

or

\begin{eqnarray*}
L \int \frac{di}{E - iR} & = & t + A \\
E - iR & = & \mu \\
- R di & = & d \mu \\
di & = & - \frac{d \mu}{R} \\
\int \frac{di}{E - IR} & = & - \frac{1}{R} \int \frac{d \mu}{\mu} \\
							  & = & \frac{1}{R} \log{\mu} \\
- \frac{L}{R} \log{(E - IR)} & = & t + A \\							  
\log{(E - iR)} & = & - \frac{R}{L} t - \frac{R}{L} A \\
E - iR & = & e^{-\frac{R t}{L}} e^{- \frac{R A}{L}} B \\
E - iR & = & B e^{- \frac{R t}{L}} \\
\mbox{At } t & = & 0 \\
i & = & 0 \\
B & = & E \\
E - iR & = & E e^{- \frac{R t}{L}} \\
i & = & \frac{E}{R} \left( 1 - e^{-\frac{R t}{L}} \right)
\end{eqnarray*}

\begin{table}[hbtp]

Diagram - To Do

\end{table}

If you differentiate the delta function ($1 - e^{- \frac{R t}{L}}$), the
result is a spike.

% 2004-02-18

\begin{table}[hbtp]

Diagram - To Do

\end{table}

Flowing at any instance t, following the imposition of a $DC$ current,
EMF $E$ is given by:

\begin{eqnarray*}
i(t) = \frac{E}{R} \left( 1 - \exp{-\frac{R t}{L}} \right) U(t) \\
U(t) & & \mbox{Response}
\end{eqnarray*}

The current is an inductive circuit, which behaves like:

\begin{table}[hbtp]

Diagram - To Do

\end{table}

Initially, there is no current because the switch has not been turned
on.

\begin{table}[hbtp]

Diagram - To Do

\end{table}

Because there is resistance and inductance in the circuit, the response
cannot instantaneously reach it's steady state value $\frac{E}{R}$, so
we say the current is just in a purely resistive circuit and the current
asymtotically approaches it's steady state value $\frac{E}{R}$.

No physical system can pobtain it's steady state value immediately.

\begin{eqnarray*}
\tau & = & \frac{E}{R} \\
i(t) & = & \frac{E}{R} \left( 1 - \exp^{- \frac{R}{L} \frac{L}{R}}
\right) \\
	  & = & \frac{E}{R} \left( 1 - e^{-1} \right) \\
\tau & = & 0.3333
\end{eqnarray*}

At this time, the current has reached $67\%$ of it's final value.

\[ t_{const} = \frac{L}{R} \mbox{(of inductive circuit)} \]

This is an important quantity because it determines how fast the
response is. If you have a very small time constant, then the steady
state response is obtained almost immediately. If you have a very large
time constant, it takes a long time to reach a steady state.

This is particularly relavent in digital systems, because you are
switching on and off the electrical circuit in sympathy with the pulses.

\begin{table}[hbtp]

Diagram - To Do

\end{table}

It will rise to the height of the pulse, the steady state position. The
height of the pulse is $\frac{E}{R}$. However, if you switch the EMF
off, the current would not decay immediately, but exponentially to $0$.

\begin{table}[hbtp]

Diagram - To Do

\end{table}

The decay transient is the mirror image of the rise transient.

We got all are results from

\begin{eqnarray*}
L \frac{di(t)}{dt} + R i(t) & = & E V(t) \\
V_{L} (e) + V_{R}(t) & = & E V(t) \\
i(0) & = & 0 \mbox{\hspace{5mm} (Initial Condition)} \\
\end{eqnarray*}


What are the voltages across $L$ and $R$ at any time $t > 0$?

\begin{eqnarray*}
V_{R}(t) & = & R i(t) \\
			& = & E \left( 1 - e^{- \frac{R}{L} t} \right) V t
\end{eqnarray*}			


At very small times for $t \ll \tau = \frac{L}{R}$.

\begin{eqnarray*}
V_{R}(t) & \simeq & 0 \\
i(t) & = & \frac{E}{R} \left(1 - e^{- \frac{R t}{L}} \right) \\
V_{L}(t) & = & L \frac{di}{dt} \\
			& = & \frac{LE}{R} \left(- \frac{R}{L} e^{- \frac{R t}{L}}
			\right) \\
			& = & - E e^{- \frac{R t}{L}} \\
V_{L}		& = & - E
\end{eqnarray*}

The full supply voltage appears across the inductor. Notice Lenz's
law: sine of the induced EMF is opposide to the sign of the EMF causing
the charge.

eg switching a starter motor powers approximitely $3kW$. Therefore you
do not need a started switch, which is a large inductive load. Instead
you remotely stithc it off a relay.

You also hyave transient effects if you have alternating currents being
switched on and off in stead like fashion.

\subsection{Decay transient}

Suppose that the battery has been switched into the circuit for a long
time. Then let us suppose that the circuit is broken at some initial
time, at $t = 0$.

$E$ applied at $t = - \infty$, then $E$ switched off at $T = 0$.

So for any time $t > 0$,

\begin{eqnarray*}
L \frac{di(t)}{dt} + R i(t) & = & 0 \\
i(0) & = & \frac{E}{R}
\end{eqnarray*}

Find $i(t)$.

\begin{eqnarray*}
L di + R i dt & = & 0 \\
dy + x y dx & = & 0 \\
d \frac{dy}{y} + x dx & = & 0 \\
\int \frac{di}{i} & = & - \frac{R}{L} \int dt + C \\
\log{i(t)} & = & - \frac{R}{L} (t + C) \\
i(0) & = & \frac{E}{R} \\
\mbox{At } t & = & 0 \\
\log{i(0)} & = & C \\
\log{\frac{E}{R}} & = & C \\
\log{i(t)} & = & - \frac{R t}{L} + \log{\frac{E}{R}} \\
\log{i(t)} - \log{\frac{E}{R}} & = & - \frac{R t}{L} \\
\log{\left( \frac{i(t) R}{E} \right)} & = & - \frac{R t}{L} \\
\frac{i R}{E} & = & e^{- \frac{R t}{L}} \\
i & = & \frac{E}{R} e^{- \frac{R t}{L}}
\end{eqnarray*}

Rise Transient: $\frac{E}{R} \left(1 - e^{-\frac{R t}{L}} \right)$

Decay Transient: $\frac{E}{R} e^{-\frac{R t}{L}}$

% 2004-02-19

Example: To Do

% 2004-02-23

\begin{table}[hbtp]

Diagram - To Do

\end{table}

\begin{table}[hbtp]

Diagram - To Do

\end{table}

\begin{eqnarray*}
E & = & V_{C}(t) + V_{R}(t) \\
  & = & \frac{q(t)}{C} + R \frac{d q(t)}{dt} \\
t & > & 0 \\
CR & & \\
q(t) &= &CE \left(1 - e^{-\frac{t}{CR}} \right)
\end{eqnarray*}

\begin{table}[hbtp]

Diagram - To Do

\end{table}

\begin{eqnarray*}
RL & & \\
i(t) & = & \frac{E}{R} \left(1 - e^{- \frac{Rt}{L}} \right)
\end{eqnarray*}

\begin{table}[hbtp]

Diagram - To Do

\end{table}

\begin{eqnarray*}
\mbox{at a time } \tau & = & CR \\
q(\tau) & = & 0.6666 \cdots CE \\
q(t) & = & CE \left( 1 - e^{- \frac{t}{CR}} \right) 1
\end{eqnarray*}

$R$ is called the time constant of the circuit. It's the time the charge
takes to reach $\frac{2}{3}$ of it's final value. In the $RL$ circuit 
at a time $t = \tau = \frac{L}{R}$, the current has reached
$\frac{2}{3}$ of it's final value $\frac{L}{R}$, thus is the time
constant for an $RL$ circuit.

\begin{table}[hbtp]

Diagram - To Do

\end{table}

After a very long time, $I$ short circuits the battery. Then the charge
will decay exponentially with a decay transient which is the mirror
image of the rise transient.

\begin{table}[hbtp]

Diagram - To Do

\end{table}

The response is rounded, because of the damping due to the resistance.

\begin{eqnarray*}
\dot{q}(t) & = & CE \left( - \frac{1}{CR} e^{- \frac{t}{CR}} \right) \\
\dot{i}(t) & = & \frac{E}{R} \left( e^{- \frac{t}{CR}} \right)
\end{eqnarray*}

When the capacitance is empty, no current flows.

The current rises exponentially to it's value, but also decays
exponentially back to $0$.

\[ R i(t) = E e^{-\frac{t}{CR}} \]

The voltage across the resistor decays exponentially with time. The
voltage across the capacitor grows exponentially to the supply value
$E$.

\begin{table}[hbtp]

Diagram - To Do

\end{table}

\begin{eqnarray*}
\mbox{Initial Charge } & = & CE \\
V_{C}(t) + V_{R}(t) & = & 0 \\
\frac{q(t)}{C} + \frac{R dq(t)}{dt} & = & 0 \\
R_{C}(t) & = & \frac{q}{C} \\
			& = & E \left(1 - e^{\frac{\tau}{\cdots}} \right) \\
\frac{q(t)}{CR} + \frac{dq}{dt} & = & 0 \\
\frac{q}{CR}dt + dq & = & 0 \\
\int \frac{dq}{q} & = & \int \frac{dt}{CR} \\
\log{q} & = & - \frac{t}{CR} + A \\
\mbox{At t = 0} & & \mbox{(taking $0$ as origin of time)} \\
q & = & CE \\
\log{CE} & = & A \\
\log{q} & = & - \frac{t}{CR} + \log{CE} \\
\log{q} - \log{CE} & = & - \frac{1}{CR} \\
\log{\frac{q}{CE}} & = & - \frac{t}{CR} \\
q(t) & = & CE e^{- \frac{t}{CR}}
\end{eqnarray*}

Now we said that a capacitor is an electrical spring and an inductor is
an electrical mass.

\begin{table}[hbtp]

Diagram - To Do

\end{table}

The capacitor is charged to a mid value $Q$ and let us close the switch
$S$ at time $t = 0$.

\begin{eqnarray*}
q(0) & = & Q \mbox{\hspace{5mm}($S$ is closed at $t = 0$)} \\
\mbox{So } t & > & 0 \\
V_{L}(t) + V_{C}(t) & = & 0 \\
L \frac{di}{dt} + \frac{q(t)}{C} & = & 0 \\
i(t) & = & \dot{q}(t) \\
L \ddot{q}(t) + \frac{q(t)}{C} & = & 0 \\
\ddot{q}(t) + \frac{q(t)}{LC} & = & 0 \\
\mbox{SHM Equation} & & \\
\omega_{n}^{2} & = & \frac{1}{LC} \\
fn & = & \frac{1}{2 \pi \sqrt{LC}} \mbox{\hspace{5mm} (Natural
Frequency)}
\end{eqnarray*}

If you disturb the system, it will oscillate with frequency $\frac{1}{2
\pi \sqrt{LC}}$. This circuit is an elementary high frequency
oscillator.

However, the inductor always has a resistance, therefore you must put
energy back into the system in order to compensate for the series
resistance of the inductor.

Modern technology uses solid state devices. Powerful ones still use
tubes.

% 2004-02-25

\begin{table}[hbtp]

Diagram - To Do

\end{table}

\begin{eqnarray*}
L \ddot{q} + \frac{q}{C} & = & E \\
t & > & 0 
\end{eqnarray*}

If you have a battery and you connect it to an $LC$ circuit, the circuit
will act as an electric harmonic oscillator.

\[ V_{C} = \frac{q(t)}{C} \]

The voltage across the capacitator will oscillate, as will the current
and the voltage across the inductor. In practive, it's impossible to make
an inductive circuit without resistance.

\begin{table}[hbtp]

Diagram - To Do

\end{table}

\begin{eqnarray*}
V_{L}(t) + V_{R}(t) + V_{C}(t) & = & E \\
L \frac{di}{dt} + iR + \frac{q}{C} & = & E \\
L \ddot{q}(t) + R \dot{q}(t) + \frac{q(t)}{C} & = & E \\
\mbox{Since } \dot{q}(t) & = & i(t)
\end{eqnarray*}

This is an oscillator with damping, because the mechanical equivalent of
capacitance is a dashpot. In this situation, the charge will
exponentially decay with time in damped oscillitory fashion.

\begin{table}[hbtp]

Diagram - To Do

\end{table}

Every so often, one must reactivate the oscillation by opening and
closing the switch, corresponding to the winding of a pendulum clock.
The onvolope of decay is exponential.

If you recall that an $RLC$ circuit, with small resistance, when
connected to a battery, can act as a high frequency oscillator.
Nowadays, active devices, such as transistors, coupled with a suitably
tuned circuit to replace the battery and switch, are used.


\section{Lorentz Force}

\xy <1cm, 0cm>:
0 ; (1.25,1.25) **\dir{-} *\dir{>} ?(1.25) *+!CR{\vec{V}} ;
0 ; (2.5,0) **\dir{-} *\dir{>} ?(1.25) *+!CR{\vec{B}} ;
0 ; (0,-2.5) **\dir{-} *\dir{>} ?(1.25) *+!CR{\vec{F}} ;
\endxy

\[ \vec{F} = q \vec{V} \times \vec{B} \]

The Lorentz Force is the force on a charge $Q$, when placed in a
magnetic field $\vec{B}$ (measured in teslas, whose unit is $\frac{V
sec}{m^{2}}$). If a charged particle moves with velocity $\vec{V}$, then
it experiences the Lorentz Force.

Lorentz discovered that the sense of the force is as follows: \textit{it
turns the direction of $\vec{B}$ in the direction $\vec{V}$ int he
direction a right handed screw would have from $\vec{V}$ to $\vec{B}$.}

Applications: 

\begin{itemize}
\item focussing electron beams is cathod ray tubes. The Lorentz Force 
enables one to have a television screen.

\item particle accellerators.

\item electric motors.
\end{itemize}

If we look at:

\begin{eqnarray*}
\vec{F} & = & q \vec{V} \times \vec{B} \\
		  & = & q (L T^{-1}) \times \vec{B} \\
		  & = & q T^{-1} \vec{L} \times \vec{B} \\
		  & = & i(\vec{L} \times \vec{B})
\end{eqnarray*}

A conductor of length $l$, when placed in a magnetic field $\vec{B}$,
experiences a force $i(\vec{L} \times \vec{B})$. The current through the
conductor is $i(t)$, and it behaves like a charge moving at velocity
$\vec{V}$.

We want an expression for the torgue on a conductor carrying $i Amps$
when it is placed in a magnetic field. Since torque is a mechanical
quantity, we would then have an electric motor.

\begin{table}[hbtp]

\xy <1cm, 0cm>:
(-2.0,0) *\txt{.} *+!RD{\vec{L}} ;
(0.75,0) ; (-1.75,0) **\dir{-} *\dir{>} ?(0.5) *+!RU{\vec{F}} ;
(0,-1.25) ; (0,-2.5) **\dir{-} *\dir{>} ;
(0.75,0) ; (0.75,-2.5) **\dir{-} *\dir{>} ;
(2,0) ; (2,-2.5) **\dir{-} *\dir{>} ?(0.5) *+!LC{\vec{B}} ;
(3.5,0) ; (3.5,-2.5) **\dir{-} *\dir{>} ;
(4.5,0) ; (4.5,-2.5) **\dir{-} *\dir{>} ;
\endxy

\end{table}

\[ \vec{F} = i \vec{L} \times \vec{B} \]

Suppose that the conductor has an area $A$, torque is force by
perdindicular distance. If the distance is $x$, then the torque for the
conductor is:

\[ \vec{I} = i \vec{A} \times \vec{B} \]

$\vec{A}$ is the vector area of the coil.

A conductor of area $A$, carrying a conductor $i Amps$, when placed in a
magnetic field $\vec{B}$, experienced a torque $i \vec{A} \times
\vec{B}$.

Furthermore, if the coil has $n$ turns, it % gap

Applications:

\begin{itemize}
\item elementary $AC$ motor

\item galvanometers

\item virtually all analogue ammeters and voltometers
\end{itemize}

The torque is enourmous, depending on the motor.


A coil, rotating in a magnetic field can act either as a motor or as a
generator. Eg alternators (as used in cars) always incorporate a
regulator, which is a voltage senstitive relay in order to ensure when
the battery is charged, the alternator does not start to act as a motor.

\section{Transmission Lines}

Electromagnetic waves, when received on an antenna, have to be fed into
a receiving device, such as a tv, by for example a coaxial cable. This
is called a transmission line.

\begin{table}[hbtp]

Diagram - To Do

\end{table}

Such cables have inductance per unit length and capacitance per unit
length.

Capacitance per unit meter:

\[ \frac{2 \pi \epsilon_{0}}{\log{\frac{b}{a}}} \]

\[ \epsilon_{0} = \frac{1}{36 \pi} \times 10^{-9} F / m \]

$\epsilon_{0}$ is the permeability of free space.

Inductance per unti length:

\[ L = \frac{\mu_{0}}{2 \pi} \log{\frac{b}{a}} \]

$\mu_{0}$ is the inductance per unit length of free space.

\begin{eqnarray*}
\mu_{0} & = & 4 \pi \times 10^{-7} H / m \\
\epsilon_{0} \mu_{0} & = & \frac{1}{C^{2}} \\
C & = & 3 \times 10^{8} m / s \\
\end{eqnarray*}

Light is propogated in free space at $3 \times 10^{8} m / s$ ($186$ $miles
/sec$). Eg The sun is $93$ $million$ $miles$ from the Earth, so light takes
$9$ $minutes$ to reach us.

% 2004-02-26 - Tutorial Missing

% 2004-03-01

\begin{eqnarray*}
\frac{\delta^{2} v}{\delta x^{2}} & = & L C \frac{\delta^{2} V (n,t)}{\delta t^{2}} \\
\frac{\delta^{2} i}{\delta x^{2}} & = & L C \frac{\delta^{2} i(x,t)}{\delta t^{2}} \\
C & = & \frac{1}{\sqrt{LC}} \\
\frac{e}{L} & = & \frac{2 \pi \epsilon_{0}}{\ln{\frac{b}{a}}} \\
\frac{L}{e} & = & \frac{\mu_{0}}{2 \pi} \ln{\frac{b}{a}} \\
V & = & \frac{1}{\sqrt{\epsilon_{0} \mu_{0}}} \\
  & = & 3 \times 10^{8} m s^{-1}
\end{eqnarray*}


The velocity of propogation of a magnetic wave on a transmission line is

\begin{table}[hbtp]

\xy <1cm, 0cm>:
{\ellipse(0.5){}}
{\ellipse(){}}
%0 ; (3.5,2.5) *+!UR{$\mu$} **\dir{-} ;
%0 ; (-1.5,-2.5) *+!DL{$\epsilon$} **\dir{-} ,
\endxy

\end{table}

If we fill the space between the conductors with relative permeability
$\mu$ and $\epsilon$ % gap

The product:

\begin{eqnarray*}
LC & \to & \frac{1}{\sqrt{\epsilon \mu \epsilon_{0} \mu_{0}}} \\
	& = & \frac{C_{LIGHT}}{\sqrt{\epsilon \mu}}
\end{eqnarray*}

So the electromagnetic waves get slowed down in a material medium.

Eg: Water

\begin{eqnarray*}
C_{L} & = & \frac{1}{N \sqrt{\epsilon_{0} \mu_{0}}} \\
		& = & \frac{3 \times 10^{8}}{\sqrt{80.1}}
\end{eqnarray*}

Application: can be a delay line

\subsection{Characteristic Impedence}

\[ Z_{0} = \frac{L}{C} \]

It's the voltage and the current.


Unlike the velocity of propogation, it depends on the geometry of the
line.

\begin{eqnarray*}
\sqrt{\frac{L}{C}} & = & 
\frac{1}{2 \pi}\sqrt{\frac{\mu_{0}}{\epsilon_{0}}} \ln{\frac{b}{a}} \\
\frac{\delta^{2} v}{\delta x^{2}} & = & 
\frac{1}{V^{2}} \frac{\delta^{2}v}{\delta t^{2}} \\
v(x,t) & = & f (x - Vt) + g(x + Vt)
\end{eqnarray*}

The solution consists of a wave travelling on the positive $x$ axis and
a wave travellign on the negative $x$ axis.

Eg: tides or incident waves on a seashore.

The wave equation, being a physical law, must contain this fact.

\begin{eqnarray*}
A \sin{\left(\omega t - \frac{x}{V}\right)} & & \\
\Phi & = & \omega (t - \frac{x}{V}) \\
V & = & \lambda f \\
A \sin{\left( \omega t - \frac{\omega x}{V} \right)} & = & 
A \sin{\left( \omega t - \frac{2 \pi f x}{V} \right)} \\
& = & A \sin{\left( \omega t - \frac{2 \pi f(x)}{x f} \right)} \\
& = & A \sin{(\omega t - k x)} \\
k & = & \frac{2 \pi}{\lambda}
\end{eqnarray*}

$k$ is the wave number (phase change). Units are meters.

Sinusiodal Wave:

\begin{eqnarray*}
A \sin{\left( \omega t \pm kx \right)} & & \\
y_{1} (n, t) & = & A \sin{\left( \omega t - k n \right)} \\
y_{2} (x, t) & = & A \sin{\left( \omega t + k x \right)} \\
& = & 2 A \sin{ \left( \frac{1}{2} \left[ \omega t - k x + \omega t + k x
\right] \right)} \cos{\left( \frac{1}{2} \left[ \omega t - k x - (\omega
t + k x) \right] \right)} \\
& = & 2 A \sin{(\omega t)} \cos{(- k x)} \\
& = & 2 A \sin{(\omega t)} \cos{(k x)}
\end{eqnarray*}

% Double check the 7 paragraphs below.

It's not like $ f(x \pm Vt)$. $A$ is not a travelling wave. It's called
a standing wave.

Discharges from a $V$ notch.

If you add two wave, with the same amplitude, a standing wave will be
setup.

A standing wave will occur on any place on a transmission line.

In acoustics, standing waves occur whenever a guitar or violin string
are plucked, because a wave will travel down to where it is connected
and be reflective.

In electrical enginering and communications, we are interested in the
velocity of propogation.

In amplitude propogation for example, an audio frequency is superimposed
onto a high frequewncy $AC$  signal.

\begin{eqnarray*}
y_{1} (n, t) & = & A \sin{\left( \omega t - k n \right)} \\
y_{2} (n, t) & = & A \sin{\left( \omega' t - k' x \right)} \\
2 \pi f_{1} & = & \omega \\
f_{1} & = &  200 kHz \\
f_{2} & = & 205kHz \\
y_{1} + y_{2} & = & 2 A \sin{\left( \frac{1}{2} \left[ \omega t - kx +
\omega' t - k' x\right] \right)} \cos{ \left( \frac{1}{2} \left[ \omega
t - \omega' t \right] x \right)} \\
& = & 2 A \sin{\left[ \frac{( \omega + \omega'}{2} - \frac{(k + k')}{2}
\right]} \cos{\left( \frac{x}{2} \left[ \omega - \omega' \right]
\right)} \\
(\omega - \omega') & \to & \delta \omega \\
(k' - k) & \to & \Delta K
\end{eqnarray*}

Sensible get

\begin{eqnarray*}
2 A \sin{\left( \omega t - k x \right)} \cos{\left( \frac{ \Delta \omega
t}{2} - \frac{\Delta k v}{2} \right)} \\
2 A \cos{ \left( \frac{\Delta \omega t}{2} - \frac{\Delta k x}{2}
\right)} \sin{\left( \omega t - k x \right)} \\
2 A \cos{\left( \frac{\Delta \omega t}{2} - \frac{\Delta kx}{2} \right)}
\sin{\left( \omega t - k x \right)} \\
& = & \Delta (\varphi) 5000 Hz
\end{eqnarray*}

You get back your original wave, however the amplitude of the original
character varies slowly.

\begin{eqnarray*}
U & = & \frac{\Delta \omega}{\Delta k}, \frac{ d \omega}{d k} \\
V & = & \frac{\omega}{k} \\
  & = & \frac{2 \pi f}{2 \pi} \\
  & = & \lambda f
\end{eqnarray*}

The information is the modular carrier.

The information is propogated by $\frac{d }{d }$. This is group
velocity.

You need a detector to remove the carrier wave.

% 2004-03-03

\begin{eqnarray*}
y & = & y_{1} + y_{2} \\
y(x, t) & = & 2 A \sin{\left( \omega t - k x \right)} \cos{\left(
\frac{\Delta k}{2} x - \frac{\Delta \omega t}{2} \right)} \\
\omega t - kx & & \\
\omega' t - k' x & & \\
\frac{\omega - \omega'}{2} & &
\end{eqnarray*}

\begin{table}[hbtp]

%\begin{pspicture}
%\psset{xunit=1cm,yunit=1cm}
%\psset{plotpoints=50}
%\psplot{0}{360}{x sin}
%\psaxes(0,0)(-1.5,1.5)
%\end{pspicture}

Graph - To Do

\end{table}

\begin{table}[hbtp]

Graph - To Do

\end{table}

This is what happens when you have a high frequency carrier wave,
superimposed on which is a wave of the same frequency ($205 kHz$),
arising from the modulation of the carrier wave by voice frequencies.

Suppose you add up the disturbances of $202 - 205$.

$205$ will be $\omega'4$ and have the wave number $k'$.

\begin{eqnarray*}
\omega - \omega' & \mbox{is} & \mbox{small} \\
k - k' & \mbox{is} & \mbox{small} \\
\Delta k & &
\end{eqnarray*}

Then the composite disturbance looks like that shown. The red curve is
original carrier wave of frequency $\omega$ and wave number $k$. The
black curve is the slowlyvarying amplitude which varies in cosine
fashion with angular frequency $\Delta \frac{\omega}{2}$ and wave number
$\Delta k$.

The frequency of the envelope is very low and it is the frequency of the
information which is superimposed on the carrier. In radio engineering,
the purpose of a detector is to rectify the carrier wave, turning it
into perfect $DC$ current, so as to receive the low frequency
information only. This is achieved using a diode.

\[ \mu = \frac{\Delta \omega}{\Delta k} = \frac{d \omega}{d k} \]

\subsection{Group Velocity}

Group velocity is the velocity of transmission of information. The 
removal of the carrier is the reason you need a detector in radio
engineering.

\section{Exam Topics}

\begin{itemize}

\item Impedence caclulations of series and parallel $RLC$ circuits.
Various properties of $RLC$ circuits (acceptor/rejector circuits),
calculate impedence at resonance, etc...

\item Ideal transformers, elementory generators, elementory electric
motors (using Lorentz Force).

\item $RMS$ values of various comp[lex $AC$ voltages and currents.
Derive the mean power expression in an $AC$ circuit.

\item Three phase circuits (see handout). Show interesting property of a
three phase circuit (no integrgals of negative instantaneous power,
unlike in a singe phase circuit).

\item Transience in circuits. What happens when a battery is suddenly
applied and suddenly removed. Derive expressions for rise and decay
transients for $LR$ and $CR$ circuits.

\item Loseless transmission line. Define the important quality known as
characteristic impedence and to appreciate the fact that characteristic
impedence depends on the geometry of the line. while, on the other hand,
the velocity of propogation does not depend on upon the geometry, but
only on the relative premeability and the relative premeability of the
material between the conductors.

\item Ability to derive the wave equation of the voltage and current at
any point $x$ on the line at time $t$. Ability to explain what a
standing wave is and also be able to explain group velocity and it's
importance in the transmission of information.

\end{itemize}

\end{document}
