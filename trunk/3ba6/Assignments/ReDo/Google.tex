% $Id$

\documentclass[a4paper,12pt]{article}

\usepackage{setspace}

\setlength{\parindent}{0mm}
\setlength{\parskip}{7.5mm}

\newcommand{\euro}{\kern.17em{}C\kern-.95em\raise.19ex\hbox{\it
=\/}\kern.17em{}}

\begin{document}

\title{Google Inc \\ Business Review}

\author{Conall O'Brien \\ 01734351 \\ conallob@maths.tcd.ie}

\maketitle

\newpage

\doublespacing

\section{Brief Overview}

Since it's founding on \cite[September 7, 1998]{timeline}, Google, Inc
has maintained it's \cite[mission is to organize the world's information
and make it universally accessible and useful]{overview}.


Google's core service is giving people access to the vast amount of
information available from around the internet. Since it's debut into
the internet search business, it's expanded it's services to also
include a portal page, access to news headlines, phone book information,
email, etc.

\section{The Team}

The team was formed initially by Larry Page and Sergey Brin, the two
company founders. Since it's humble beginnings, it's board has now
expanded to include seven other members.

\subsection{Background and Skills}

\subsubsection{Eric Schmidt} 

Since \cite[July 2001]{board}, he has served as the company's Chief Executive 
Officer. Before becoming CEO, he served as Chairman of the board of 
directors from \cite[March 2001 to April 2004]{board}. In 
\cite[April 2004]{board}, 
Eric was named Chairman of the Executive Committee of the board. 
Before coming to Google, Eric served in various positions on the board
of directors at \cite[Novell and sun Microsystems]{board} and still sits
on the board of \cite[Siebel System]{board}. \cite[Eric has a Bachelor of Science degree
in electrical engineering from Princeton University, and a Masters
degree and Ph.D. in computer science from the University of California
at Berkeley.]{board}

\subsubsection{Sergey Brin}

As one of Google's founders, Sergey has been on the board of directors 
since \cite[September 1998]{board} and since \cite[July 2001]{board},
has been the President of Technology after serving as President from 
\cite[September 1998 to July 2001]{board}, \cite[Sergey holds a Masters 
degree in computer science from Stanford University, a Bachelor of 
Science degree with high honours in mathematics and computer science 
from the University of Maryland at College Park and is currently on 
leave from the Ph.D. program in computer science at Stanford 
University.]{board}

\subsubsection{Larry Page}

As the second founder of Google, Larry has been a member of the board of
directors since \cite[September 1998]{board} and was made the President of
Products in \cite[July 2001]{board}. Between \cite[September 1998 to 
July 2001]{board}, he was the CEO and between \cite[September 1998 to 
July 2002]{board}, the Chief Financial Officer. \cite[Larry holds a 
Masters degree in computer science from Stanford University, a Bachelor 
of Science degree with high honours in engineering, with a concentration 
in computer engineering, from the University of Michigan and is 
currently on leave from the Ph.D. program in computer science at 
Stanford University.]{board}

\subsubsection{L. John Doerr} 

John has been on the Google board of directors since \cite[May
1999]{board}, after serving as a \cite[General Partner of Kleiner 
Perkins Caufield and Byers, since August 1980]{board}. He also 
sits on the board of directors at 
\cite[Amazon.com, drugstore.com, Homestore.com, Intuit, 
PalmOne, and Sun Microsystems]{board}. He holds a \cite[Masters of 
Business Administration degree from Harvard Business School and a 
Masters of Science degree in electrical engineering and computer 
science and a Bachelor of Science degree in electrical engineering 
from Rice University]{board}.

\subsubsection{John L. Hennessy}

John became a member of the Google board of directors in 
\cite[April 2004]{board}. Since \cite[September 2000]{board}, he has 
also served as the \cite[President of Stanford University]{board}. 
Between \cite[1994 and August 2000]{board}, he has held positions 
in Stanford, including \cite[Dean of the Stanford University School
of Engineering and Chair of the Stanford University Department of
Computer Science]{board}. John is also a member of the board of
directors at \cite[Cisco Systems and Atheros Communications]{board}. 
\cite[John holds a Master's degree and
Doctoral degree in computer science from the State University of New
York, Stony Brook and a Bachelor of Science degree in electrical
engineering from Villanova University.]{board}

\subsubsection{Arthur D. Levinson}

Arther became member of the board of directors in \cite[April
2004]{board}. He has also served on the boards of directors at 
\cite[Genentech]{board} and currently also serves currently on the 
\cite[Apple Computers]{board}, 
\cite[Arthur was a Postdoctoral Fellow in the Department of Microbiology at the
University of California, San Francisco. Art holds a Ph.D. in
biochemistry from Princeton University and a Bachelor of Science degree
in molecular biology from the University of Washington.]{board}

\subsubsection{Michael Moritz}

Michael became a member of the board of directors in
\cite[May 1999]{board}. Michael has been a 
\cite[General Partner of Sequoia Capital]{board} since
\cite[1986]{board}. Michael also serves as a member of the board at 
\cite[Saba Software and Flextronics International and RedEnvelope]{board}
\cite[Michael holds a Masters of Arts degree from Christ Church, 
University of Oxford.]{board}

\subsubsection{Paul S. Otellini}

Paul has served as a member of the board of directors 
\cite[April 2004]{board}. Paul has also held numerous positions at
\cite[Intel]{board}, including posts on it's board of directors. 
\cite[Paul holds a Master's degree from the University of California 
at Berkeley and a Bachelors degree in economics from the University 
of San Francisco.]{board}

\subsubsection{K. Ram Shriram}

Ram has served as a member of Google's board of directors since
\cite[September 1998]{board}. Since \cite[January 2000]{board}, 
Ram has served as managing partner of 
\cite[Sherpalo]{board}. \cite[From August 1998 to September 1999]{board},
Ram served as 
\cite[Vice President of Business Development at Amazon.com]{board},
after serving as \cite[President at Junglee]{board} and was 
\cite[an early member of the Netscape executive team]{board}. 
\cite[Ram holds a Bachelor of Science degree from the University of
Madras, India.]{board}

\section{Business Vision and Strategy}

Google's strategy is to provide internet users free services, including
their primary and secondary services. In order to fund such services,
they sell advertising space, cleverly tailored to appear to users of the
services.


Once Google's brand recognition was widely known, they also started to
offer searching and indexing services so that companies could have the
power of Google, but contained within their organisation. 

\section{Operational and Organisational Challenges}

One major obstacle Google has faced is maintaining all the data they
store in a system that is highly scalable and easily searched with as
short a delay as possible. They also had issues relating to building
cost effective super computers.


To address these obstacles, they adapted a recent design for high
performance super computers, commonly designed and built for number
crunching. Using relatively cheap, low cost computers they were able to
construct large, high performance systems allowing them to store
billions of web pages with numerous index and caching features to allow
search results to be available within milliseconds.

\section{Products and Services}

\subsection{Core Service}

Google's core service is it's free search tool of a large number of
websites around the internet, available for anyone to use free of 
charge.

\subsection{Secondary Services}

Google also have a large number of secondary services, such as:

\begin{tabular}{ll}
\cite[Gmail]{productdesc}	& a free email system 						\\ 
Google \cite[News]{productdesc}	& a highly customisable news system \\
\cite[Froogle]{productdesc}	&	 a comparison shopping database	\\
\cite[Blogger]{productdesc}	&	 a free Web Log (online journal) service \\
Google \cite[Maps]{productdesc}&	 a search-able address book and map directory \\
Google \cite[Language Tools]{productdesc}	&	 multilingual translators
and other tools	\\
\end{tabular}


They also have numerous other products, such as specialised searches in
specific areas to help return fewer, more precise results and various
other innovative tools often available for testing from the Google Lab. 

\section{Competing with Others}

\subsection{Competing Products}

Since Google's primary service is an internet search engine, they face
stiff competition from their competitors. The other large competitors in
the field include Yahoo!, MSN and Altavista.


Google face stiff competition in particular from MSN and Yahoo! when
it's secondary services are also considered, since all three offer
equivalent services of one form or another.

\subsection{Keeping the Edge}

Apart from their core searching service based upon a highly scalable
mathematical model originally implemented by the company Founders Page
and Brin, Google manages to keep it's market share for a number of
reasons. These reasons include their \cite[strict beliefs against non text
based advertising]{philosophy} and the steady stream of well thought out
products emerging from the Google \footnote{Research and Development}{R\&D}.

\subsection{Potential Threats}

The two main threats Google face are from Yahoo! and MSN, a division of
Microsoft. Both opponents offer similar search services as Google,
although neither use (or at least admit to use) the same algorithm as 
Google. Both MSN and Yahoo! have greater brand name recognition 
associated with their free and subscription email services. Yahoo! 
also has a well recognised portal page, which can be tailored to show
relevant news and information on a per user basis. 

\subsection{Future Plans}

Google has always been traditionally quite secretive about certain
details of it's business. To date, there has been speculation about the
total sum of nodes from their \footnote{Modern day super computers which
are the sum of many ordinary computers, arranged together to work as a
collective}{clusters}. Google themselves instruct their staff not to
disclose this information.


Google R\&D are also known to give relatively short notice about new and
emerging products. Instead of issuing a press release or having staff
hint after upcoming products, Google prefer to just add upcoming
products with a minimal or no announcement to their beta series
of products, which is their public testing area based upon the
traditional software alpha, beta testing system.

\section{References}

\bibliographystyle{ieeetr}

\bibliography{Google}

\end{document}
