% $Id$

\documentclass[a4paper,12pt]{article}

%\usepackage{eurosym}

\setlength{\parindent}{0mm}
\setlength{\parskip}{7.5mm}

\newcommand{\euro}{\kern.17em{}C\kern-.95em\raise.19ex\hbox{\it=\/}\kern.17em{}}

\begin{document}

\title{Business Plan Critique \\ Group 5 \\ TaraWAN}

\vspace{20mm}

\author{Conall O'Brien \\ 01734351 \\ conallob@maths.tcd.ie}

\maketitle

\newpage

\tableofcontents

\section{Introduction}

I have decided to critique the business plan of Group 5's TaraWAN
bunisness proposal to provide high speed, wireless broadband internet
access to the County Meath region of Ireland.


I will structure my crititique upon the existing headers and subheaders
of the original business plan, critiquing each section individually.
Finally, I will draw my conclusions and sum up my findings.

\section{The Plan}

\subsection{Management and Organisational Structure}

The business plan names \cite[3 employees]{busplan}, each one to manage
the three sections of the business \cite[Sales and Marketing,
Procurement and Technology Development]{busplan}. All three managers
come from a \cite[Computer Science]{busplan} background with experience
dealing with \cite[large scale wireless networks]{busplan}.

This personell structure has numerous oversights and obvious potential
flaws. 


First of all, 3 managers will not suffice to run a company who's
sole product is a service. Companies offering services need to also
offer support for their product, otherwise their userbase will either
diminise or fail to grow in the first place. Employing someone from a 
technical background with no business experience or management skills 
will almost certain be castastrophic to the company.

\subsection{Market}

The business plan discusses establishing a wide area wireless network of
a \cite[20 km radius]{busplan}, which is estimated to cover a potential
\cite[8000]{busplan} potential business customers.


This plan is well thought out and promising. Currently, there is a large
opening in the broadband internet market in this regional area, which
obviously is affectling local businesses established in the area. A
local business attempting to fill this niche would have the potential to
thrive and be very sucessful.

\subsection{Competition}

The business plan specifically names 
\cite[DigiWeb, Irish Broadband and Net1]{busplan} as key compeditors.


TaraWAN would face still competition from Digiweb specifically who have
been operating in this area for some time, thanks to their satellite
internet option, which has not suffered from the location difficulties
of securing DSL (Digital Line Subscription) based internet broadband
available from large and virtual telecom operators. Digiweb would have
name recognition in this region.


Irish Broadband have established a name for themselves for building a
high speed wireless network over large sections of the Dublin area.
Their wireless products and cost plans have proven sucessful as an
alternative to DSL based broadband options. While they would not have
good name recognition in the Meath area, their extensive experience in 
the area would prove to be a key advantage.


Net1 are a similar company to TaraWAN. They are a recent startup, based
within the area using similar technologies. They are new to the market,
but are gaining brand recognition and a good reputation, despite their
compeditive pricing plans.


The key compeditors are clearly identified, however the advantages of
exisiting market players such as Digiweb and Irish Broadband, who have
experience and name recognition appear to be underestimated in the
business plan. 

\subsection{Pricing}

The business plan states intially, hardware cost and installation will 
be subsidised to customers for \cite[\euro 150]{busplan}, desipte it's 
\cite[\euro 250]{busplan} retail cost. A reoccurring monthly charge of 
\cite[\euro 40]{busplan} would also be required for the internet
service.


The installation and hardware costs appear to be idealistic. TaraWAN's
technology plan states using the \cite[3.5GHz]{busplan} radio frequency
spectrum, which is licenced by ComReg, the Irish Communications
Regulauthority. This frequency is used by the 
\cite[802.16 WiMax]{wimax} draft standard. This standard is
currently being developed, so supporting hardware is scarce, if even
manufactured. This emerging standard is also currently competing with
the 802.11i standard which aims to also provide high speed wireless
broadband, but based upon the widely established and supported
\cite[802.11 (aka WiFi) standard]{wifi}


Monthly rental costs appear compeditive, but would need to be flexible,
in order for TaraWAN to remain compeditive in the event of a compeditor
dropping rental prices.

\subsection{Advertising and Promotion}

The business plan mentions plans to use \cite[local radio, local
newspapers, store signs and information pamphlets]{busplan} to raise the
profile of TaraWAN's website, which will contain information for
potential customers, an order form to begin the installation process and
methods to discuss issues with other TaraWAN customers.


The promotion methods planned are all solid, well based plans. However,
the informational pamphlets are aimed at \cite[residences of potential
customers]{busplan}. A second information pamphlet aimed at the business
market in the area should also be sent out ot each listed business
within the region to maximise awareness of TaraWAN's products and how
they could benefit lcoal businesses.

\subsection{Sales Management}

Sales management is planned on being primarily managed via the
\cite[TaraWAN website]{busplan} in an effort to \cite[keep sales costs
at a minumum]{busplan}.


While this plan will help to expediate the sales process and keep sales
costs at a minimum, there are a few weaknesses to such a plan. Internet
only sales means potential customers without ready internet access who
are interested in TaraWAN's services will be at a loss. As will
potential customers not content with internet shopping. Telephone, mail
and faximile order facilities should also be provided to cater for the
most common sales mediums.

\subsection{Sales Fortcasts}

The sales forecast anticipates \cite[5 customers in the first 2
months]{busplan}, followed by growth of \cite[10 additional customers
every two mmonths there-after]{busplan}.


This sales forecast is obviously very pessimistic in terms of scale.
However, given the projection of 
\cite[8000 potential customers]{busplan} in the region, it is
understandable and acceptable, considering obstacles such as
compeditors, lack of name recognition and general interest initially in
such services.

\subsection{SWOT Analysis}

The SWOT analysis in the plan is clear and consise, presented in bullet
points under each of the four sub headings. Although information is
clear, some of the points indicate a lack of research and thought.


Under strengths, \cite[knowledgeable staff and confirmed high site at
Tara]{busplan} are both listed. As previously discussed, at least one
named employee is unsuited to manage the Sales and Marketing department
due to lack of management experience or business qualifications.


The other point regarding a high site on Tara claims to be confirmed,
yet receiving planning permission from the Meath County Council for such
a site is an expensive and lengthy proceedure, requiring various surveys
and tests such as \cite[enviromental site surveys]{meathcoco}.
Considering the history of Tara and the tourism market in the local
vicinity, there are numerous potential objectors with good grounds, to 
such a planned structure. If planning permission were to be denied, it 
would strike a serious blow to TaraWAN's business and operational plans. 


Amongst the weaknesses listed in the SWOT analysis, the current status
of the WiMax draft standard is ackowledged, therefore demonstrating an
understanding of their business plan being based upon experimental
stardards which are still experimental and not widely supported.

\section{Conclusion}

To conclude, the business idea of supplying high speed broadband
internet to the county of Meath is a good idea with a rich market.
However, certain aspects of the TaraWAN proposal to tap into this market
need some rethinking. 


In particular, basing a business upon an unstandardised technology 
currently experiencing support ieeues amongst vendors is not
recommended. Another major issue which appears to have failed to be 
considered is understanding of the required planning permission upon 
which the entire business relies upon. The proposed management structure
and employee numbers need serious reworking to complement a service
providing business of this nature. Finally, a lack of sales options will
also likely hinder initial interest and long term uptake of customers.

\bibliographystyle{quote}

\bibliography{TaraWANCritique}

\end{document}
