% $Id$

\documentclass[a4paper,12pt]{article}
\usepackage{amssymb}

\setlength{\parindent}{0mm}
\setlength{\parskip}{7.5mm}

%\lstset{language=prolog}

\begin{document}

\title{Course 3BA2: Artificial Intellegence \\ Tutorial Solution\\ $20^{th}$ October 2004}

\maketitle

\section{Introduction}

Algorithms - Robert Sedgewick

% Diagram

\begin{verbatim}

in_trie(Element, TrieIn, TrieOut).

node(g, , ).

node(d,[],[]).

node(g,node(d(node(b,[],[]),node(f,[],[]),node(p,node(j,[],[]),node(3,[],[]))).

compose_trie(E,[],node(E,[],[])).

member(X,[x|_]).
member(X,[_|R]).
member(X,R).

compose_trie(E,[],node(E,[],[])).

compose_trie(E,node(V,L,R),node(V,Ln,R)) :- E@<V,compose_trie(E,L,Ln).

compose_trie(E,node(V,L,R),node(V,L,Rn)) :- E@>=V,compose_trie(E,R,Rn).

list_to_trie([],T,T).

list_to_trie([H|B],Ti,To) :- compose_trie(H,Ti,To),list_to_trie(B,T,To).

\end{verbatim}

Algorithms = Logic + Control - Robert Kowalski (1979)

\end{document}
