% $Id: 20041011.tex 338 2004-10-27 12:48:06Z conall $

\documentclass[a4paper,12pt]{article}
\usepackage{amssymb}

\setlength{\parindent}{0mm}
\setlength{\parskip}{7.5mm}

%\lstset{language=prolog}

\begin{document}

\title{Course 3BA2: Artificial Intellegence \\ Additional Lecture Notes \\ $11^{th}$ October 2004}

\maketitle

\section{Introduction}

Proceedural Languages - 

\begin{itemize}
\item Java
\item C/C++
\item Eiffel
\item $68K$ ASM
\item Perl
\item Python
\item PHP
\item QBASIC
\end{itemize}

Prolog: Non proceedural, declarative

\begin{eqnarray*}
A1 & : & 85 \\
B4 & : & 876 \\
C9 & : & A1 \times B4 \\
\end{eqnarray*}


Listener \\

Interpreter \\
 

\begin{verbatim}

parent(chris,liz). % a clause
parent(chris,mike).
parent(liz,aidan).
parent(liz,louise).
parent(make,clare).
parent(mike,kate).

?- parent(chris,liz).

yes


?-

:- % if

_ % anonymous variable

[1,2,3] % list

\end{verbatim}

\section{Predecate}


\begin{verbatim}

read_everything(File) :-  see(File),
.  
.  read(T),
.  process(T),
. 
. 
. 

seen.

process(end_of_file).
process(X) :- write(X),nl,read(Y),process(Y).

\end{verbatim}

\end{document}
