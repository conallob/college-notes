\documentclass[a4paper,12pt]{article}
\usepackage{amssymb}

\begin{document}

\title{Course 3BA2: Artificial Intellegence \\ Additional Lecture Notes \\ $18^{th}$ October 2004}

\maketitle

\section*{Prolog Syntax}

Everything is a term. \\

\\

Terms are simple or complex ("Structured").

\subsection*{Simple}

"atom" - stands for something. Eg: mike, chair \\

\\

\subsubsection*{Syntax:} 

start with a lower case letter and followed by zero or more 
alphanumeric characters or underscores.


A special character followed by zero or more special characters: $ \&, \#. /. \*, \$ $


Any string of characters enclosed in single quotes. Eg: 'The w cat', 'foo.txt', 'can''t do it'

\subsubsection*{Numbers:}

Integers \\

Floats (Implementation dependant)

\subsubsection{Variables:}

Anonymous Variable: $\_$ \\

member(x,[_|R]). \\

\subsubsection{Named Variable:}

Capitol Letter followed by zero or more aplhanumeric characters \\

or \\

An underscrore followed by one or more alphanumerics or underscores. \\


Scope of variable in a clause is the clause itself if named or no scope if anonymous.

\subsection{Complex:}

has a functor, which has a name (some syntax rules as an atom) and has an arity of $\gt 0$. \\

composs_points(north,south,east,west). \\

north - Prolog term \\

tree(_,R). \\

temp(L1,R). \\

foo(X). \\

bar(X,R) $ : $ foo(X),foo(X). \\

$ : $ - Infix operator \\


\end{document}
