The architectural model of the system is relatively straight forward,
consisting of a client and a server. However, there are multiple
operational modes for each component.

\begin{figure}[H]

\begin{center}
\ \xy<1cm,0cm>:
(3, 0) *=(2,2) !CR \txt{A} *\frm{-} ="nodeA" , 
(9, 0) *=(2,2) !CL \txt{B} *\frm{=} ="nodeB" , 
"nodeA" ; "nodeB" **\dir{-} *\dir{>} ?(0.5) *++!CC{1}*\frm{o} , 
\endxy
\end{center}

\label{fig:client}

\caption{Message 1 sent from Client A to Server B}

\end{figure}

\begin{figure}[H]

\begin{center}
\ \xy<1cm,0cm>:
(3, 0) *=(2,2) !CR \txt{A} *\frm{-} ="nodeA" , 
(9, 4) *=(2,2) !CL \txt{B} *\frm{=} ="nodeB" , 
(9,-4) *=(2,2) !CL \txt{C} *\frm{=} ="nodeC" , 
"nodeA" ; "nodeC" **\dir{-} *\dir{>} ?(0.5) *++!CC{2}*\frm{o} , 
"nodeC"!CC ; "nodeB"!DC **\dir{-} *\dir{>} ?(0.5) *++!CC{2}*\frm{o} , 
\endxy
\end{center}

\caption{Message 2 sent from Client A to Server B, routed through Server C}

\label{fig:relay}

\end{figure}

\subsection{Operating Modes of the Client}

There are two operational modes for the client, direct and relayed.

Direct mode is a pure distributed system, using the peer to peer model.
When transmitting a message, a client connects directly to the
recipient node, as illustrated in Figure \ref{fig:client}.


Unfortunately, due to modern security concerns, it is often common 
practice to restrict access to non local subnets by employing firewall
and/or routing table restrictions. Remote access is possible through
certain nodes, acting as proxies for allowed services. 


\subsection{Operating Modes of the Server}

The server can also operate in two modes, direct and proxy. In direct
mode, the server accepts connections from the client, processes the
message and attempts to deliver the message to a local user.


Alternatively, in proxy mode, the server simply accepts a message from a
client connection, buffers the message and relays it to the target 
destination, similar to the model currently used by SMTP, as depicted in
Figure \ref{fig:relay}.

\subsection{Operational Flow Chart}

\input include/flow-chart

