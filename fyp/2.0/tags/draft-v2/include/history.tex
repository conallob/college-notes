Since the release of Version 1 of AT\&T Unix upon the world 
in 1971\footnote{http://www.faqs.org/faqs/unix-faq/faq/part6/section-2.html}, 
there has been the \emph{write} tool, which allows multiple users on a Unix 
system to communicate with each other.


\emph{write} is quite simple and delivers a message one line at a time,
resulting in the recipient of a \emph{write} message waiting while the
sender composes the next line of his or her message.


Due to this behaviour, implementing a tool which buffers the message
prior to sending is a logical progression for improved communication
between users. Since this is an obvious enhancement, I believe this
enhancement may have been implemented by many people. However I am 
documenting the timeline of \emph{hey}, a buffer system for write,
which emerged originally from Dublin University, Trinity College in the 
early 1990s.


In 1991, \emph{hey} was created as a shell script by Adrian Colley. In 
1993/1994, Paul McGaley translated \emph{hey} to Perl and enhanced it's 
usability until 1996. Paul's Perl version of \emph{hey} resurfaced in 
Dublin City University's RedBrick Society in 1998, where it was enhanced
further by Colin Whittaker and others to interpret environmental variable 
settings and accept additional command line flags.


Due to it's popularity in RedBrick, Cian Synnott begun work in 1998/1999
on a version written in C to improve speed and portability, which is 
known as \emph{c-hey}. \emph{c-hey} has since received many 
contributions and additional features such as user definable formatting 
and at the time of writing, is currently maintained by Colm 
MacC\'{a}rthaigh.


Meanwhile in 2000/2001, back in Dublin University, Trinity College, Aidan 
Kehoe began work on his own rewrite of \emph{hey} in C. Over the 
following years, it inherited and shared many features with RedBrick's 
\emph{c-hey}, including ANSI colour formatting in 2002.


Finally, in 2000/2001, the idea of making \emph{hey} distributed across 
a network was formed by John Tobin from Dublin University, Trinity 
College.  He went on to christened this idea \emph{oi} and work on it 
as his final year project in 2002. However, John's implementation was
never completed.
