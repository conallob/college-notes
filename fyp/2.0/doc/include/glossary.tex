\textbf{6bone}

Testbed IPv6 network formed from multiple sites and tunnels. The 6bone
has been deprecated by RFC 3701 and will be non functional as of June
2006.

\textbf{ACL}

Access Control List, concept of restricting access to a resource by
forming a list of accpeted and/or rejected clients.

\textbf{anycast}

A networking delivery method which allows a any node in a group of 
machines to answer a request. Sometimes used in load balancing 
scenarios.

\textbf{API}

Application Programming Interface, a programming library with
standardised set of data structures, function calls and methods used to
simply programming.

\textbf{ARP}

Address Resolution Protocol, used in IPv4 TCP/IP stacks to translate 
between IP and Layer 2 address (eg Ethernet MAC addresses).

\textbf{bit}

The method of expressing data on most computers. It is a number in base
2, giving only two possible values, 0 and 1. Bits are combined togather
and operated on to perform almost all computer operations.

\textbf{BGP}

Border Gateway Protocol, a popular EGP protocol.  

\textbf{broadcast}

A network delivery method which delivers the same data to all nodes.

\textbf{CIDR}

Classless Inter Domain Routing, A method of allocating IPv4 IP addresses
without using the original Class A, B and C system.

\textbf{DAD}

Duplicate Address Detection, a feature of IPv6 to allow nodes to check
that an address isn't already used.

\textbf{DNS}

Domain Name System, a distributed database used to store information
such as FQDNs and IP addresses, mail servers, telephone numbers, etc

\textbf{DoS}

Denial of Service, a form of attack employed against a computer network
of service. It is a flood of additional traffic directed at a target,
which results in significant or complete impedence of the target. Also
available in a distributed flavour, when multiple notes are used to
generate the traffic.

\textbf{dual stack}

term used to describe a node with IPv4 and IPv6 TCP/IP networking stacks
enabled. Currently common during the rollout of IPv6 to allow
compatability between IPv4 and IPv6 networks.

\textbf{EGP}

Exterior Gateway Protocol, a routing protocol used to populate routing
tables between different, often adjacent networks. Compare to IGP.

\textbf{FQDN}

Fully Qualified Domain Name, the full name of a node, consisting of a
name of the node, followed by a domain name or sub domain name suffix.

\textbf{HTML}

HyperText Marckup Language, a markup language used to format documents.
Primary document format of the "World Wide Web", the Internet.

\textbf{ICMP}

Internet Control Message Protocol, the control protocol used to
communicate error and diagnostic messages. ICMP was revised and expanded 
during the design of ICMPv6, included within IPv6.

\textbf{IGP}

Interior Gateway Protocol, a protocol class used within a network to
learn and update routing tables within that network. Compare to EGP.

\textbf{IM}

Instant Messaging, an alternative to SMTP used to send often short
messages between users. Some believe it replaced IRC as the informal
messaging system on the Internet. 
% Examples of IM networks include 
% Microsoft Messenger, Yahoo! Messenger, AOL Instant Messenter (AIM) and 
% ICQ are all examples. Traditionally, these networks were not
% interoperable, resulting in users with accounts on multiple networks and
% propiatary clients for each network.

\textbf{IPsec}

technique used to secure network connections at Layer 3, ensuring
privacy, authentication and other security related services. 

\textbf{IPv4}

Internet Protocol, version 4. The IP protocol used in the internet
at the time of writing.

\textbf{IPv6}

A new version of the IP protocol, with significant design improvements
to address issues in IPv4. Significant differences to IPv4 include a
128 bit address space, extentible headers and greater emphasis on 
security and IP multicast.

\textbf{IRC}

Internet Relay Chat, as defined in RFC 1459 is an open standard real
time messaging system. Is able to support public "chat room" style
conversations in public, or user to user private messages.

\textbf{Node}

An individual computer, connected to a single network or subnet.

\textbf{Machine}

See node

\textbf{MLD}

Multicast Listener Discovery, a part of ICMPv6 which allows an IPv6
router to discover multicast addresses being listened to for a link.

\textbf{Multicast}

A way of sending packets to a certain group of machine. Duplication of
packet contents is down within the network.

\textbf{Multihoming}

a network with multiple upstream connections, commonly using different
upstream provides. Commonly found in ISP networks and significant
locations such as server farms and co-location data centres.

\textbf{NAT}

Network Address Translation, as described in RFC 1631 is a method
to rewrite IP source and destination between 2 networks. It is often
used to mask private IP ranges, as described in RFC 1918 behind a 
public, routable IPv4 address. 

\textbf{ND}

Neighbour Discovery, the new and improved equivalent of ARP for IPv6.

\textbf{OSI}

Open System Interconnection, a networking stack once the alternative to
TCP/IP. Known primarily now for the layered paradigm central to it's
design.

\textbf{password}

A word or code used to authenticate a user to a computer system.

\textbf{passphrase}

Used to perform the same authentication functions as a password, but was
originally designed to be longer than eight characters, the maximum
length for a password at the time.

\textbf{Public Private Keys}

A pair of asymmetric cryptography keys, one public, one private which 
are mathematically related to each other. 

\textbf{PIM}

Protocol Independent Multicast, a prptocol for routing multicast
traffic. There are two versions, Sparse Mode and Dense Mode, referring
to the density of nodes subnets in multicast enabled subnets.

\textbf{prefix}

A block or IPv4 or IPv6 addresses determined by fixing the first x bits
of an address. Eg 134.226.0.0/16 is the prefix for the network
containing IPv4 address 134.226.81.11. 2001:770:10:300::/64 is an IPv6
prefix containing the address 2001:770:10:300::86e2:510b. 

\textbf{QoS}

Quality of Service, the ability of gaurantee dedicated bandwidth to
specific traffic in an effort to reduce the risk delayed delivery a for
time sensitive payload.

\textbf{RFC}

Request For Comments, the nearest equivalent to complete standards that
govern the Internet and the protocols available.

\textbf{SMTP}

Simple Mail Transfer Protocol, the protocl used to transmit email
between nodes on a network, or the Internet.

\textbf{tunnel}

technique of sending packets of a particular protocol from one point on
a network to another without the intermediate network understanding the
tunneled payload. Commonly achieved by wrapping payload packets within
headers and tails understood by the intermediate network. This
technique is commonly used to provide upstream IPv6 connectivity to IPv6
islands without native IPv6 upstreams.

\textbf{tunnelbroker}

A dual stacked node used as a router to provide IPv6 connectivity to a
tunnel.

\textbf{Unicast}

Primary networking method used to create IP connections between nodes.
Compare with anycast, multicast and broadcast.

\textbf{Userland}

Tools and applications installed on an O/S outside of the kernel and
core subsystems. Typically refers to all tools available to a user of
the system.

\textbf{XML}

eXtensible Markup Language, a markup language for documents containing
structured information. Appears similar to HTML, but is much more
flexible and extentible. XML document structures, known as schema, are
often designed and standardised for various uses.

\textbf{XMPP}

eXensible Messaging and Presence Protocol, an open standard for
transmitting IM messages using XML schema.
