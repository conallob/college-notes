% $Id$

\subsection{Introduction}

Unicode, also referred to as UTF8, UTF-8 and i18n is a universal
character encoding standard designed to replace multiple existing
character sets such as ASCII, 
\footnote{The series consists of 16 parts, including 8859-1 "Latin-1" and the modern revision, 8859-15, both of which are used mainly in Western Europe}{ISO/IEC 8859 series} 
of character sets which cater for most languages using the Latin, 
Cyrillic, Arabic, Hewbrew and Thai alphabets.

\subsection{Characters Supported by Unicode}

Unicode has defined characters for all major languages, including the
Latin, Cyrillic, Arabic, Hewbrew, Thai, Kanji, Punjabi, Ethiopian, 
Braille, Runic, Ogham and numerous other alphabets. Additionally,
Unicode also includes punctuation marks, diacritics, mathematical
symbols, technical symbols, arrows, dingbats, etc. As of Unicode version
4.0, provides codes for 96447 characters and has 6400 standard and over
100000 supplemental code points for private use by vendors.

\subsection{Different Flavours}

There are three Unicode standards, UTF-8, UTF-16 and UTF-32 which allow 
the same data to be transmitted in a byte, word or longword format. All
three encoding forms encode the same information and conversion between
any two standards can be done without any data loss.


UTF-8 is a way of transforming all Unicode characters into a variable
length encoding of bytes. Symbols common to both Unicode and ASCII have
the same byte values, allowing existing software to use Unicode with
minimal effort. UTF-8 is popular in various documentation standards, 
such as HTML and XML.


UTF-16 is popular in enviroments which balance effecient access to
characters with economical use of storage. It is reasonably compact and
every heavily used symbol fit into a single UTF-16 code unit, while
less commonly used symbols are accessible via pairs of 16-bit code
units.


UTF-32 is popular where memory space is of no concern, but a fixed
width, single code unit access to characters using a single 32-bit code
unit.

\subsection{Glyphs}

Unicode does not define glyph images. Instead the Unicode standard
defines how characters are interpreted, rather than how to render the
sixe, shape or style of a glyph on screen. The software or hardware 
rendering engine of a computer interprets and renders the characters to 
the selected output device.

\subsection{Character Codes}

Unicode character codes are represented by a four hexadecimal digit with
a "U" prefix. For example, U+004F represents "O", the "Latin Capital 
Letter O" and U+1696 represents the symbol "Ogham Letter Or", "\u1696"
(which can be considered to be "Oi")

\tofix

\subsubsection{Combined Characters}

The Unicode standard specifices the order of characters in interpreting 
a combined character. The base character is first, followed by one or
more non-spacing marks. The order of the marks is unimportant, if the 
marks don't interact typographically, if hey do interact, order is
important.


Certain combined characters, eg \"{o} can also be represented as a
precomposed character. To generate the glyph \"{o} using Unicode, one
can use either the precomposed base character U+00F6, or combine the
characters o (U+006F) with the non-spacing character U+0308 (\"{ }).

\begin{table}

\begin{center}

\begin{tabular}{ccccc}
\'{a}	&	\`{e}	&	\"{u}	&	\^{o}	&	\~{n}	\\
\={o}	&	\.{o}	&	\u{o}	&	\v{c}	&	\H{o}	\\
\t{oo}&	\c{c}	&	\d{o}	&	\b{o}	&			\\
\end{tabular}

\end{center}

\caption{Examples of combined character codes}

\end{table}

\subsection{Principles}

Unicode was invented by a group os computer professionals, linguists and
scholars in the early 1990s. In an effort to become a worldwide
character set easily used for text encoding everywhere, they defined a
set of fundemental principles:

\begin{itemize}

\item Universal Repertoire

\item Logical Order

\item Efficiency

\item Unification

\item Characters, not Glyphs

\item Dynamic Composition

\item Semantics

\item Equivalent Sequence

\item Plain Text

\item Convertability

\end{itemize}

The first 256 characters within the Unicode standard are taken from the
widely used ASCII format. Duplicate entries across different languages
is avoided by unifying characters within scripts. CJK consolidation is
achieved by assigning a single code for each ideograph each time it
occurs in each language.

The Unicode standard also specifies an algorithm for the presentation of
text in a bidirectional manner, such as left to right languages (such as
English and other Latin alphabet based languages) and right to left
(such as Hebrew and Arabic) and special characters to specify changes in
direction. For all scripts, Unicode characters are arranged in a
logical order, corresponding to the order in which they can be typed on
a keyboard.

