% $Id$

During the course of this project, I learnt numerous lessons, primarily
relating to software engineering and the problems that can occur.

\subsection{Time Management}

For the first few weeks, research and productivity was slow, due to
inefficient time management with various personal, professional and
academic commitments. In an attempt to resolve this, I consulted
Stephen Barrett, who suggested a few methods of time keeping and task
management.


After incorporating some of his suggestions into my work method,
progress became more transparent and began accelerating. Over the next
few months, I incorporated additional suggestions from various printed
and online sources, in efforts to further improve my task agenda and 
email management. Results of adapting many of these suggestions into my
work method became apparent as I was more and more productive towards
the end of this project.

\subsection{Overly Ambitious}

In hindsight, the large number of features incorporating various, often
recently emerged technologies resulted in an ambitious specification.
Due to only administration and usage experiences with many of these
technologies, I did not possess much  programming experience with these
technologies. 

\subsection{Inexperienced Programmer}

Due to my relative inexperience doing this type of development, initial
programming was slow. Over time, this issue resolved itself as I became
more familiar with C programming styles and methods, as well as best
practices using incorporating libraries, system calls and functions. I
also became more skilled at diagnosing and repairing pointer related 
issues when working with character arrays as String objects.

\subsection{Coding is the Shortest Phase}

The majority of the time spent of this project was spent planning and
organising various steps and phases. Common activities such as research,
prototype testing, documentation and various meetings account for a 
substantial part of the time spent on the project. As a result, actual 
codebase additions and edits were effectively alloted the smallest 
amount of time.

