% $Id$

Given the requirements of dual stacked IPv4 and IPv6 networking,
portable code, cryptography to ensure security, unicode to support
multiple languages, and multicast support to avail of modern network
capabilities.


Since there are many existing messaging clients and protocols available,
what use is there for this client? Most IM clients available are GUI
based. While there are a select few CLI based clients for Unix systems,
but they are all interactive clients which work in the foreground of a
Unix console. These clients do not act as true Unix tools, which are
traditionally simple, consise and flexible, thus allowing a user to
harness their immense power, by combining multiple tools together.


Existing IM clients transmit unencrypted messages between users as
standard. Although there are 
recent enhancements\footnote{For example, Off The Record - http://www.cypherpunks.ca/otr/}
to encrypt messages, unencrypted messages are still send by default.
Additionally, these enhancements to existing IM protocols use different
cryptography cypers since there is no standard, de facto or otherwise.


IP Multicast networking has only become popular in recent times, so most
networks still don't support IP Multicast partially, or at all. One of
the reasons why Multicast is not widely deployed is because there is yet
to be a \emph{killer application} . Most multicast protocols distribute
multimedia formats to many users. Hence, these complex applications 
are not suited to perform basic multicast functionality tests. A simple,
lightweight messaging system, such as \emph{oi} is a much better  
application to test a multicast deployment, before attempting more
significant or complete tests.
