% $Id: /college/fyp/2.0/doc/objective.tex 2490 2006-03-19T22:58:59.629349Z conall  $

Client should be intuitive for a novice to use, but versatile for a
power user.


The client should transmit the message to the server on the destination
host. The server then will attempt delivery of the message to the
recipient and transmit status details to the sender via the client.


System should support IPv4 and IPv6 sockets to communicate between
client and server. 


System should be as portable as possible, compiling and working on 
various Unix and Unix related systems (eg commercial Unix
implementations like Solaris and MacOS X, Open Source Unix 
implementations such as the primary BSD flavours and open source Unix
related systems such as various Linux distributions) 


Messages between client and server should be cryptographically secure by
default. Due to the additional security concerns using SSL connections
and corresponding X.509 certificates to encrypt messages in transit
between servers on separate notes. Therefore public private keys should
be used and the encryption should be from end user to end user.
Therefore, OpenPGP should be the encryption model used.


Client should support unicode input from the user. Server should attempt
delivery in Unicode, if the recipient terminal is Unicode capable. 

Messages should be transmitted in unicode. Non unicode input should be
encoded into unicode prior to encryption and transmission. Messages
should be decoded from unicode after transmission and decryption, if
recipient is not able to receive unicode support. 


Server should be able to join IP multicast group, to allow multiple
servers to receive the same message transmitted by a client to an IP
multicast group. 
