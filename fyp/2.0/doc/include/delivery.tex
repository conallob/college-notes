% $Id$

Since Unix and Unix derivatives treat devices the same as file
descriptors, delivering a message to the TTY file descriptor 
representing the terminal session of a user on a system is the 
same as writing to a file.


However, for security there ownership and write access restrictions
applied. All TTY devices are owned by the user currently connected to
that tty and the "tty" system group. Only the owner has read and write
access to a specific terminal. If a user is willing to accept 
messages\footnote{Users are able to nominally enable or disable receiving messages by using the mesg y and mesg n commands},
their TTY is writable by members of the "tty" group.


In order for the daemon to deliver messages to any system user, it needs
to invoke the setgid() system call in order to execute as a member of 
the "tty" group. However, using setgid() is a security 
risk\footnote{A process needs to be run initially as the root super user in order to call setgid() properly. Additionally, if a setgid() application has a security vulnerability, a malicious user could gain privileges granted to the group}.


To avoid this security issue, the daemon shall use the 
\emph{write}\footnote{which is setgid to the "tty" group} application
to write a message to a user's terminal.
