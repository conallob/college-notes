% $Id$

\subsection{Concerns Raised}

During the initial brainstorming phase when possible requirements were
suggested to the authors of software such as the original \emph{hey},
the widely used \emph{c-hey} and \emph{Oi, version 1}, some concerns 
over the security model used in \emph{Oi, version 1} were raised by 
John Tobin and Adrian Colley. 


\emph{Oi, version 1} used X.509 managed SSL certificates to encrypt 
messages between nodes, so unencrypted messages were not transmitted 
across a network. Adrian expressed reservations over this security model 
for a multiuser, multiple-node architecture. In this model, if a
single node with multiple users is compromised, the messages to and from
any user on this system are also compromised. As a result, a malicious
user could potentially intercept messages within minimal effort.


John also raised concerns over using X.509 certificates as well,
primarily because his SSL socket functionality was created using the
OpenSSL open source library. However, due to lack of adequate
documentation during development, the pre-existing SSL support is
semi-functional.

\subsection{A Different Approach}

After considering the concerns highlighted regarding the pre-existing
security architecture, I decided to redesign the security model using a
different technology to X.509 security certificates. I decided to use the 
OpenPGP standard for encrypting transmissions between two individual 
users.


The principle difference using this approach to the pre-existing
codebase is that encryption is end-to-end, not node-to-node. As a
result, the model is more robust and privacy is achieved. Instead of a
single SSL certificate and private key on each node, every user has a
public and private key pair. Users also require the public key component
of their intended recipient to allow them to encrypt their message using
their recipient's public key.


Using a public-private key model, a compromised node is no longer as
critical as it was using X.509 SSL certificates. Additionally, messages
are still encrypted until delivered to the recipient, protecting
sensitive messages from being intercepted during the delivery process.


Finally, since OpenPGP keys are quite common, there is an existing
public key management system in widespread deployment, simplifying the
task of key management for the messaging system.
