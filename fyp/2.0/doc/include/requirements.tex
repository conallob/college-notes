% $Id: /college/fyp/2.0/doc/objective.tex 2490 2006-03-19T22:58:59.629349Z conall  $

\subsubsection{User Interface}

There is no need to provide a complete GUI interface for  a user. Most
Unix users are capable of, or restricted to, command line interfaces via
an interactive shell. Therefore, the client should comply with the 
Unix de facto CLI standard: input is received via \verb!stdin!, output
is sent to \verb!stdout! and errors are sent to \verb!stderr!. By
following these de facto standards, providing a user with much more 
power by availing of the I/O 
redirection\footnote{referred to as the pipe, represented by the symbol $\mid$}.

\subsubsection{Networking}

Building upon the existing software model, internet sockets should be 
primary IPC, used to communicate between nodes. In order to be modern,
IPv6 sockets should be supported in addition to IPv4 sockets.

\subsubsection{Portability}

In an effort to resolve one of the key issues experienced in 2002, 
ensure POSIX portability to allow compilation and execution on a wide 
range of Unix and Unix like platforms.

\subsubsection{Cryptography}

Due to the potential for sensitive messages being sent between nodes
across a network, some form of cryptography is required to ensure 
security of the message contents.

\subsubsection{Unicode}

Support for Unicode (UTF-8) formatted input, allowing a user to input 
text using any alphabet, such as Latin, Hebrew, Arabic, Cyrillic or 
many more\footnote{The full list of supported alphabets and character sets is available at http://www.unicode.org/charts/}.

\subsubsection{IP Multicast}

Multicast, a previously seldom used networking feature exists between 
unicast and broadcast delivery. For years, there were few uses for 
multicast connectivity. However, there is now a growing number of uses 
making it much more popular and cost effective. 

