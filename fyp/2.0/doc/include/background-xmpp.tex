% $Id$

\subsection{Introduction}

XMPP is the IETF name for a set of XML based streaming protocols for
instant messaging, as proven by the success of the Jabber IM protocol.
It consists of a transport layer, authentication and optional
encryption. Due to these abstractions, XMPP can be used for more than
simple text based messaging. Additional payload types include audio and
video data types.

\subsection{How It Works}



\subsection{Applications using XMPP}

The best known example of XMPP is the Jabber IM protocol. Jabber has
become increasingly popular as an IM protocol due to it's ability to
interface with propietary IM protocols, such as ICQ, AIM and MSN
Messenger. Additionally, a private Jabber network can be easily 
deployed within a network if an IM protocol is required, without 
requiring full internet access. 


A second example is the Google Talk network, while interoperable with
the existing Jabber network and propiatary networks, in addition it
supports Jingle\footnote{As specified by JEP-0167 - \url{http://www.jabber.org/jeps/jep-0167.html}},
a proposed extension to XMPP to allow audio conversations, similar to
other VoIP protocols such as SIP and IAX.


\subsection{Conclusion}

Further information about XMPP, XMPP compatible applications and
proposed XMPP extensions can be found at \url{http://www.jabber.org} and
\url{http://www.xmpp.org}.
