% $Id$

\subsection{Introduction}

XMPP is the IETF name for a set of XML based streaming protocols for
instant messaging, as proven by the success of the Jabber IM protocol.
It consists of a transport layer with an addressing scheme, 
authentication and optional encryption. Due to these abstractions, XMPP 
can be used for more than simple text based messaging. Additional 
payload types include audio and video data types.

\subsection{Identifiers (JIDs)}

XMPP/Jabber identifier addresses, known as JIDs, are valid URI 
addresses, using the Augmented Backus-Naur Form. JIDs are primarily used 
to identify a user, but can also be used to identify a multi-user 
chatroom. JIDs cannot be more than 3071 bytes in length. The format of a
JID is described in Table \ref{tab:jid}.

\begin{table}[p]

\label{tab:jid}

\begin{center}

\begin{tabular}{lll}
jid		&		&	[ node "@" ] domain [ "/" resource ]				\\
			&		&																	\\
domain	&		&	FQDN / DNS A or AAAA Record / Literal Address	\\
Literal Address	&		&	IPv4 / IPv6 IP Address						\\
resource &		&	Context / User Handle (Nickname)						\\
\end{tabular}

\end{center}

\caption{Construction of a JID Identifier}

\end{table}

\subsection{How XMPP Works}

There are 4 operational modes to XMPP: Client, Server, Gateway and
Network.

\subsubsection{Client}

XMPP clients connect directly to an XMPP server over a TCP connection,
on the recommended port 5222. Multiple clients can connect from 
different addresses can connect to a server simultaneously with the same 
authorisation credentials; resulting in differing resource identifiers
or JIDs.

\subsubsection{Server}

An XMPP Server is responsible for managing connections between XMPP
clients, servers and other entities and routing the appropriate XML stanzas
to the corresponding XMPP entity using XML streams. XMPP servers are
also commonly responsible for managing and storing user data, such as a
user's contact list.

\subsubsection{Gateway}

An XMPP gateway is a special case of an XMPP server. It's primary
function is to translate XMPP data into non XMPP data. Existing gateway 
examples include connections to propiatary IM networks such as AIM, ICQ,
MSN Messenger and Yahoo! Messenger; translating XMPP messages into email 
or SMS messages; or interfacing with IRC network. Other proposed gateway
translations include translating XMPP Jingle Messages into SIP or IAX
VoIP messages.

\subsubsection{Network}

Since XMPP servers are easily identified by network address, and server
to server connections are a simple extension of client to server
connections, an XMPP network is simply constructed by interconnecting
multiple servers together.

\subsection{Applications using XMPP}

The best known example of XMPP is the Jabber IM protocol. Jabber has
become increasingly popular as an IM protocol due to it's ability to
interface with propietary IM protocols, such as ICQ, AIM and MSN
Messenger. Additionally, a private Jabber network can be easily 
deployed within a network if an IM protocol is required, without 
requiring full internet access. 


A second example is the Google Talk network, while interoperable with
the existing Jabber network and propiatary networks, in addition it
supports Jingle\footnote{As specified by JEP-0167 - \url{http://www.jabber.org/jeps/jep-0167.html}},
a proposed extension to XMPP to allow audio conversations, similar to
other VoIP protocols such as SIP and IAX.

\subsection{Conclusion}

Further information about XMPP, XMPP compatible applications and
proposed XMPP extensions can be found at \url{http://www.jabber.org} and
\url{http://www.xmpp.org}
