% $Id$

Due to the events in the history of Unix over the last 30 years, there
are now 3 catagories of Unix based systems; commercial Unix
implementations, open source Unix implementations and Unix and Unix-like 
derivatives.


Commercial Unix implementations are primarily able to trace ties back to
the orginal AT\&T Unix dynasty and are thus referred to as System V (V5)
dreivatives. Examples include Sun Solaris, HP UX, Compaq
Tru64 and IBM AIX. 


By contrast, open source Unix implementations do not share ties to Unix 
System V, but to the 4.4BSD codebase, primarily due to the fallout from 
the lawsuit between USL and BSDI, which resulted in a requirement to
make the distinction between Unix and BSD. Leading up to the time the 
BSD project in the University of Berkeley concluded in 1995, The NetBSD
and FreeBSD projects, based upon the 4.3BSD (and ultimately 4.4BSD R2) 
codebase were formed in the community spirit of the CSRG who ran the BSD
project. Examples of BSD systems nowadsys include FreeBSD, NetBSD,
OpenBSD (a fork from the NetBSD project) and DragonFlyBSD (a fork from
the FreeBSD Project).


The last catagory consists of Unix and Unix-like derivatives, which
refer to systems which were never based upon the System V or BSD
codebases, yet mimic many of the features and functionality of a Unix or
BSD system. Examples include the popular Linux kernel, the Gnu/Herd,
and NeXTSTEP operating systems.


Finally, a special mention should be made for Apple's MacOS X, which is
a commercial BSD implementation consisting of a merger between the
FreeBSD codebase and the NeXTSTEP Mach kernel.


As a result of all of these options, there are a wide number of
differences in the userland enviroments of various these systems.
Therefore, portability becomes an issue, if a project is intended to run
on a wide range of Unix, BSD and Unix-like, since every system call, 
standard library or API interface cannot be anticiated on each system.
As a result, traditionally a number of conditions are placed in portable
programs, in order to edit the function calls to suit the system being
used.
