% $Id$

\subsection{Introduction}

Due to the events in the history of Unix over the last 35 years, there
are now 3 categories of Unix based systems; commercial Unix
implementations, open source Unix implementations and Unix and Unix-like 
derivatives.

\subsection{Different Flavours}

Commercial Unix implementations are primarily able to trace ties back to
the original AT\&T Unix dynasty and are thus referred to as System V (V5)
derivatives. Examples include Sun Solaris, HP UX, Compaq
Tru64 and IBM AIX. 


By contrast, open source Unix implementations do not share ties to Unix 
System V, but to the 4.4BSD code-base, primarily due to the fallout from 
the lawsuit between USL and BSDI, which resulted in a requirement to
make the distinction between Unix and BSD. Leading up to the time the 
BSD project in the University of Berkeley concluded in 1995, The NetBSD
and FreeBSD projects, based upon the 4.3BSD (and ultimately 4.4BSD R2) 
code-base were formed in the community spirit of the CSRG who ran the BSD
project. Examples of BSD systems nowadays include FreeBSD, NetBSD,
OpenBSD (a fork from the NetBSD project) and DragonFlyBSD (a fork from
the FreeBSD Project).


The last category consists of Unix and Unix-like derivatives, which
refer to systems which were never based upon the System V or BSD
code-bases, yet mimic many of the features and functionality of a Unix or
BSD system. Examples include the popular Linux kernel, the Gnu/Herd,
and NeXTStep operating systems.


Finally, a special mention should be made for Apple's MacOS X, which is
a commercial BSD implementation consisting of a merger between the
FreeBSD code-base and the NeXTStep Mach kernel.


As a result of all of these options, there are a wide number of
differences in the user-land environments of various these systems.
Therefore, portability becomes an issue, if a project is intended to run
on a wide range of Unix, BSD and Unix-like, since every system call, 
standard library or API interface cannot be anticipated on each system.
As a result, traditionally a number of conditions are placed in portable
programs, in order to edit the function calls to suit the system being
used.

\subsection{User-Land Variations}

Due to various reasons, such as the rapid popularity of Linux in recent
years, Linux distributions have modern, complete user-land
configurations and reliable package management systems (such as Debian
.deb packages and RedHat .rpm packages) used to maintain, upgrade and 
extend the user-land of a system. In order to be a viable alternative to 
Linux, the BSD family of Unix varients also has a high quality package 
management system known as the Ports System.


Most commercial Unix distributions do not have a common package, well
supported management tool. As a result, commercial Unix distributions
often do not have easily maintained user-land.

\subsection{Further Information}

Further information regarding the timeline of various Unix releases and
significant incidents can be found in the BSD Family 
Tree\footnote{\url{http://www.freebsd.org/cgi/cvsweb.cgi/src/share/misc/bsd-family-tree}}, 
maintained within the FreeBSD documentation. 


Information regarding the user-land capabilities for each distribution 
can be found in the corresponding distribution release notes Information
regarding the user-land capabilities for each distribution can be found 
in the corresponding distribution release notes.

