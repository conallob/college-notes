% $Id$

\subsection{Introduction}

Multicast is a data delivery method allowing data to be transmitted from
one-to-many and many-to-many delivery. It can be considered the middle
ground between the traditional delivery methods of Unicast and
Broadcast. The network itself is responsible for generating the 
duplicate traffic, thus using bandwidth more efficiently than using 
multiple Unicast connections between the source and multiple 
destinations. There are currently three categories of multicast, ASM, 
SSM and PIM-Bidir.

\subsection{What Multicast Is}

Multicast connections are based upon groups and group membership. In
practical terms, a multicast group is an IP number from the multicast IP
block, from the class D IP range 224.0.0.0/4. Group members are clients
who are interested in receiving the multicast transmissions.

\subsection{Flavours}

\subsubsection{Any Source Multicast (ASM)}
\label{sec:asm}

ASM is the current name referring to IP Multicast as defined in RFC
1112, refers to the original IP Multicast specification. To distribute
traffic, the concept of a distribution tree is required. 


Using ASM, routers perform RPF to calculate which interface is 
topologically nearest to the top of the tree. Once calculated, a router
treats the RPF interface as the incoming multicast interface for a
group.


Using ASM, a system known as SPT, is used to appoint the multicast
source is the root of the distribution tree. To construct the tree, any
router which receives a join request from a node forwards the join
request from it's RPT interface to the next upstream router, until the 
join message reaches a router connected with the multicast source or a
router in a multicast forwarding state for the relevant source-group
pair; thus forming a distribution tree. See example in Figure 
\ref{fig:asm-spt}.

\begin{figure}[ht]
\label{fig:asm-spt}

\begin{center}

\ \xy<1cm,0cm>:
(1, 7.5)  *=(1,1) \txt{$S$}	  	*\frm{-}  ="src",
(1, -3)   *=(1,1) \txt{$H_{1}$}	*\frm{-}  ="dst1",
(7, -3)   *=(1,1) \txt{$H_{2}$}	*\frm{-}  ="dst2",
%
(4,  5)   *=(1,1) \txt{$R_{1}$} *\frm{o}   ="r1",
(8,  5)   *=(1,1) \txt{$R_{2}$} *\frm{o}   ="r2",
(2,  0)   *=(1,1) \txt{$R_{3}$} *\frm{o}   ="r3",
(6,  0)   *=(1,1) \txt{$R_{4}$} *\frm{o}   ="r4",
(10, 0)   *=(1,1) \txt{$R_{5}$} *\frm{o}   ="r5",
%
"src" !DR ; "r1" !UC **\dir{~} ,
%
"r1" !CC ; "r2" !CC **\dir{~} ,
"r3" !CC ; "r4" !CC **\dir{~} ,
"r4" !CC ; "r5" !CC **\dir{~} ,
"r1" !CC ; "r3" !CC **\dir{~} ,
"r1" !CC ; "r4" !CC **\dir{~} ,
"r2" !CC ; "r5" !CC **\dir{~} ,
%
"src" !DR; "r1"   !CC **[thicker]\dir{=} ,
"r1"  !CC; "r3"   !CC **[thicker]\dir{=} ,
"r1"  !CC; "r4"   !CC **[thicker]\dir{=} ,
"r3"  !CC; "dst1" !UC **[thicker]\dir{=} *\dir3{>},
"r4"  !CC; "dst2" !UC **[thicker]\dir{=} *\dir3{>},
%
\POS (13,5) *\txt{Legend: \\ \\ \~{} L1 Link \\ = IP Multicast Link \\ $R_{n}$ Router \\ S Source \\ $H_{n}$ Host} *++\frm{-} !UL ="legend"
\endxy

\end{center}

\caption{Creating a SPT Tree in ASM}

\end{figure}


Numerous ASM routing protocols exist to aid the creating of a
distribution tree. One such protocol is PIM, which comes in two 
flavours, Sparse Mode (PIM-SM) and Dense Mode (PIM-DM). With PIM, a
distribution tree consists of a SPT from the source to a router acting 
as a Rendezvous Point. A complete distribution tree is then formed,
using the RP as the root.


The other routers on the network will join the distribution tree of a 
particular group under the same conditions as when creating a RPT. The 
key contrast is that only the router serving as the RP is aware of the 
true source address. All other routers treat the RP as the source node. 
See example in Figure \ref{fig:asm-rpt}.

\begin{figure}[ht]
\label{fig:asm-rpt}

\begin{center}

\ \xy<1cm,0cm>:
(1, 7.5)  *=(1,1) \txt{$S$}	  	*\frm{-}  ="src",
(1, -3)   *=(1,1) \txt{$H_{1}$}	*\frm{-}  ="dst1",
(7, -3)   *=(1,1) \txt{$H_{2}$}	*\frm{-}  ="dst2",
%
(4,  5)   *=(1,1) \txt{$R_{1}$} *\frm{o}   ="r1",
(8,  5)   *=(1,1) \txt{$R_{2}$} *\frm{o}   ="r2",
(2,  0)   *=(1,1) \txt{$R_{3}$} *\frm{o}   ="r3",
(6,  0)   *=(1,1) \txt{$R_{4}$} *\frm{o}   ="r4",
(10, 0)   *=(1,1) \txt{$R_{5}$} *\frm{o}   ="r5",
%
"src" !DR ; "r1" !UC **\dir{~} ,
%
"r1" !CC ; "r2" !CC **\dir{~} ,
"r3" !CC ; "r4" !CC **\dir{~} ,
"r4" !CC ; "r5" !CC **\dir{~} ,
"r1" !CC ; "r3" !CC **\dir{~} ,
"r1" !CC ; "r4" !CC **\dir{~} ,
"r2" !CC ; "r5" !CC **\dir{~} ,
%
"src" !DR; "r1"   !CC **[thicker]\dir{=} ,
"r1"  !CC; "r2"   !CC **[thicker]\dir{=} ,
"r2"  !CC; "r5"   !CC **[thicker]\dir{=} ,
"r5"  !CC; "r4"   !CC **[thicker]\dir{=} ,
"r4"  !CC; "r3"   !CC **[thicker]\dir{=} ,
"r3"  !CC; "dst1" !UC **[thicker]\dir{=} *\dir3{>},
"r4"  !CC; "dst2" !UC **[thicker]\dir{=} *\dir3{>},
%
\POS (8.75,5.75) !UR *\txt{RP} ,
%
\POS (13,5) *\txt{Legend: \\ \\ \~{} L1 Link \\ = IP Multicast Link \\ $R_{n}$ Router \\ S Source \\ $H_{n}$ Host \\ RP Rendezvous Point} *++\frm{-} !UL ="legend"
\endxy

\end{center}

\caption{Creating a RPT Tree in ASM}

\end{figure}


In order to use ASM properly, statically assigned Multicast IP numbers
are required. Since these IP numbers are generally unavailable,
organisation with public AS numbers automatically have a portion of
233.0.0.0/8 IP block GLOB range reserved for them, in accordance with 
RFC 3180.

\subsubsection{Source Specific Multicast (SSM)}
\label{sec:ssm}

SSM is a proposed standard to replace ASM multicast models. The key
difference is the lack of many-to-many multicast delivery, primarily due
to the fewer uses it has compared to the one-to-many multicast delivery
model. As a result, the functionality for source discovery has been
moved from the router to the host. Thus, RPTs and RPs are no longer
required in SSM, considerably reducing the complexity involved.


An additional by product of SSM is that the Multicast IPs are not 
required to be globally unique, making the 232.0.0.0/8 SSM IP block
reusable within different organisations.

\subsubsection{Bi-directional Protocol Independent Multicast (PIM-Bidir)}
\label{sec:pim-bidir}

PIM-Bidir is derived from the PIM suite of routing protocols. It is
becoming the preferred method of many-to-many multicasting compared to
ASM and is expected to compliment SSM based one-to-many multicast. PIM
Bidir has proven to scale to an arbitrary number of sources elegantly
without incurring overhead. 


PIM-Bidir creates a bidirectional shared distribution tree as
performing in other PIM variants, using the RP as the root of the group.
A network IP address is used as a key, allowing the routers on a network
to establish a loop-free spanning tree rooted to that IP address.


In order to avoid multicast packet loops, PIM-bidir uses a new mechanism
called a Designated Forwarder (DF), which is elected by discovering the
router with the nearest Unicast routing metric to the RP.

\subsection{Group Management}

While there are a number of group management protocol to control IP
multicast group membership, IGMP is the one primarily used. To join a
multicast group, a host transmits an IGMP report to a well-known
multicast group. All IGMP enabled routers on the network are listening to
this multicast group. Then the appropriate router with a direct link to
the host in question joins a multicast group using ASM or SSM based
multicast using the appropriate routing protocol.


In order to conserve bandwidth, and in order to support the preference
of L2 switches over L2 hubs (primarily due to unique collision domains
created for every port on a switch), IGMP Snooping is an operational
mode to allow switches to be aware of IGMP messages and to switch or
disregard IGMP messages in a similar manner to traditional Unicast
traffic.

\subsection{Practical Uses}

Multicast has numerous practical activities, where efficient use of
bandwidth is preferred. Applications such as selective broadcast
categorized information (such as stock values), multimedia systems (such
as audio and video streaming, IPTV), audio and video conference calling, 
as well as computer system protocols of interest to many, such as
authentication and configuration protocols where replication is
used (eg LDAP, NFS, etc).

\subsection{Common Multicast Issues and Problems}

\subsubsection{Layer 3 to Layer 2 Mapping}

On a L2 Ethernet network, all devices have a unique 48-bit MAC address
used as an identifier. IP Multicast IP blocks are Class D IPs and are 
thus do not correspond with a single end host. The MAC address range
assigned for L2 Multicast is only 24 bits long, thus resulting in a 1:32
overlap, where there are 32 possible Multicast MAC addresses mapped to a 
single multicast IP address.
overlap

\subsubsection{Network Design Flaws}

A badly structured layer 2 network design can experience numerous issues
when deploying multicast. Such issues can include symptoms such as: 
additional CPU load on switches when IGMP Snooping is enabled; 
Overlapping MAC addresses causing IGMP Snooping information for 
different groups to be merged; network flooded with IGMP Snooping 
messages as topology changes and networking devices re-learn network 
structure; inability to restrict groups by network given the large
number of active ports on the network.

Similarly, a badly structured L3 network can also experience adverse 
behaviour, such as routing instabilities which cause PIM to change it's
forwarding state.

\subsection{Further Information}

Further information about Multicast and Interdomain Multicast can be
found in Interdomain Multicast by Brian Edwards, Leonard Guiliano and
Brian Wright; and Developing IP Multicast Networks: Volume I by Beau
Williamson.
