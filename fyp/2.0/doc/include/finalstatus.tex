% $Id$

By the deadline of the demonstration of this project, the following
features from the original specification were fully implemented and
integrated:

\begin{itemize}

\item Client / Server Model

\item Portable Code

\item IPv4 and IPv6 Networking Sockets

\end{itemize}

Additionally, the following features have been partially implemented,
but not extensively tested or integrated into the client or server:

\begin{itemize}

\item XMPP XML Message Formatting

\end{itemize}


Due to various unpredicted issues during the course of the project, the
following features from the original specification were removed from the
intended deliverable:

\begin{itemize}

\item Unicode Formatting

\item OpenPGP (GPG) Cryptography

\item IP Multicast Support

\end{itemize}


\subsection{Unicode Formatting}

Created currently unresolved architectural issues. Although
the client can be informed if input should expect UTF-8 input. 
Messages between a client and server can also be encoded in and
decoded from UTF-8. However, there is no set method of checking if a
recipient's terminal is Unicode aware, or even Unicode enabled. Such
information is not stored in the termcap information.

One possible solution could be a third component, a user
agent, which is used to transmit some basic information to a daemon
process, user availability such as whether a user is able to receive 
Unicode formatted messages.

\subsection{OpenPGP (GPG) Cryptography}

Developing OpenPGP functionality using the GnuPG software
suite was delayed due to two independent issues.

Documentation for the GnuPG Made Easy (GPGME) wrapper of
functions (which is recommended library for adding GnuPG encryption,
decryption and digital signatures to MUA clients), is virtually not
existent. As a result, integrating GnuPG functionality would take
significantly longer than initially allocated.


A more fundamental issue with integrating GnuPG cryptography 
is how a recipient decodes a message. Due to the architectural 
design of the system, the daemon runs as a process forked in the 
background, delivering messages only when they arrive at a server. 
Since the recipient does not have an active client awaiting messages, 
a recipient cannot be prompted to enter a pass-phrase to decode any 
messages delivered. 

One possible solution would be to employ the GPG Agent, a 
tool found in the GnuPG development branch, version 1.9. The GPG
Agent would work similarly to the suggested Unicode solution. When a
user runs the GPG agent, preferably at login, they enter their
pass-phrase. The corresponding daemon process is then informed that
this user is using the GPG Agent. During this session, any encrypted 
messages transmitted to a user running the ssh agent is automatically
and securely decrypted. Due to the work involved in adding a third
agent component to the system architecture, combined with the
warnings that the GPG Agent is not ready for production usage yet, it
was decided that development on GnuPG encryption would be halted.

\subsection{IP Multicast Sockets}

Although there are pre-existing routines in APR to create and
manage IP multicast connections, testing multicast connections would
prove problematic. 

Multicasting across a simple network consisting of one or 
more unmanaged network switches daisy chained together is trivial. 
Multicast tests across a sophisticated network using managed
switches, IGMP snooping and vlans is more useful. Finally, enquiries
were made during the course of development to find a suitable testing
environment for IP multicasting across dynamic multicast routing. One
such test network would be the INEX\footnote{The Internet Neutral
Exchange, the sole ISP hub in Ireland, consisting of more
25 ISP, hosting and content organisation}, where 3 ISP members 
have currently deployed IP multicast to the edges of their networks.


However, IP multicast is not properly supported across the 
TCD network, primarily due to the design of the production backbone in 
addition to a lack of layer 3 subnetting. Therefore, there were two 
options, either submit an official request for IP multicast to be 
deployed across the core backbone and parts of the distribution layer, 
to allow testing. Alternatively, a multicast tunnel could be established
to HEAnet, who have a functioning, multicast enabled network, are the 
upstream connectivity providers to TCD. 


Queries to HEAnet management regarding the feasibility of a deploying
a multicast tunnel resulted in the agreement that a request to deploy IP
multicast across the TCD backbone should be submitted first. If denied,
a tunnel could then be established. 


After meeting with the TCD network designer and a member of the TCD 
network management team, multicast tests were encouraged, although 
there was concern over high probability of adverse behaviour occurring, 
which would affect network performance. Subsequent to this meeting, 
IP multicast functionality was re-prioritised to a non compulsory
feature, due to the proposed timing of any such tests, accompanied by 
the volume of users potentially affected; any adverse network behaviour 
would be viewed as a widespread DoS. 
