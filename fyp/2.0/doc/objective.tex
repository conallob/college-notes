Since John Tobin never finished implementing his \emph{oi} messaging
system, I decided to review the state of his work and explore ways to
build upon his work, in addition to adding additional functionality.


As of May 2002, the \emph{oi} client and server were functional, but
were primarily running on Solaris Unix. There were unresolved
compilation issues preventing it being ported to Linux. As a result, it
was unable to demonstrate messages in a distributed enviroment.

\subsection{Requirements}

\subsubsection{Interface Simplicity}

There is no need to provide a complete GUI interface to a user. Most
Unix users are capable of or restricted to command line interfaces via
an interactive shell. Therefore, the client should comply with the 
Unix de facto CLI standard, input is received via \verb!stdin!, output
is sent to \verb!stdout! and errors are sent to \verb!stderr!. By
following these de facto standards, it gives a user much more power, by
availing of the 
\footnote{generally referred to as the pipe - \verb!|!}{I/O redirection}

\subsubsection{Networking}

Building upon the existing software model, internet sockets should be 
primary IPC used to communicate between nodes. In order to be modern,
IPv6 sockets should be supported in addition to IPv4 sockets.

\subsubsection{Portability}

In an effort to resolve one of the key issues experienced in 2002, 
portablity to allow compilation and execution on a wide range of Unix
and Unix like platforms.

\subsubsection{Cryptography}

Due to the potential for sensitive messages being sent, some form of
cryptography is required to ensure security of the message contents.

\subsubsection{Unicode}

Many modern Unix and Linux systems now support Unicode (UTF-8) input,
allowing a user to input text using Latin, Hebrew, Arabic, Cyrillic
\footnote{The full list of supported alphabets and character sets is
available at http://www.unicode.org/charts/}{and many more}
alphabets and glyphs. 

\subsubsection{IP Multicast}

\footnote{A broad overview to mutlicast is explained later in Appendix
Bla}{Multicast}, a previously seldom used networking feature exists 
between unicast and broadcast. For years, there were few uses for
multicast connectivity, however there is now a growing number of uses,
making it much more popular. 


Many of these uses involve complex applications which are not suited to
perform basic multicast functionality tests. A simple messaging system,
such as \emph{oi} is a much better test application before attempting 
larger tests.
