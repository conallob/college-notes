Since the release of Version 1 of AT\&T Unix upon the world 
\footnote{http://www.faqs.org/faqs/unix-faq/faq/part6/section-2.html}{in 1971}, 
there has been the \emph{write} tool, which allows multiple users on a Unix 
system to communicate with each other.


\emph{write} is quite simple and delivers a message one line at a time,
resulting in the recipient of a \emph{write} message waiting while the
sender composes the next line of his or her message.


Due ot this behaviour, implementing a tool which buffers the message
prior to sending is a logical progression for improved communication
between users. Since this is a logical progression, I believe this
enhancement has been implemented by many people, however I am mainly
aware of the timeline of \emph{hey} from Trinity College Dublin in 
the early 1990s.


In 199X, \emph{hey} was created as a shell script by Adrian Colley. In 
1993/1994, Paul McGaley translated \emph{hey} to Perl and enhanced it's 
usability until 1996. Paul's Perl version of \emph{hey} resurfaced in 
Dublin City University's RedBrick Society in 1998, where it was enhanced
further by Colin Whittaker and others to interpret enviromental variable 
settings and accept additional command line flags.


Due to it's popularity in RedBrick, Cian Synnott begun work in 1998/1999
on a version written in C, known as \emph{c-hey}. \emph{c-hey} has since
recieved many contributions and additional features such as user
definible formatting and at the time of writing, is currently maintained 
by Colm MacC\'{a}rthaigh.


Meanwile in 199X, back in Trinity College, Aidan Kehoe began work on his
own rewrite of \emph{hey} in C. Over the following years, it inherited 
and shared many feaures with RedBrick's c-hey, including ANSI colour
formatting.


Finally, in 2000/2001, the idea of making hey distributed across a
network was formed by John Tobin in Trinity College. He went on to 
christened this idea \emph{oi} and work on it as his final year project
in 2002. 
